\chapter{Artículos Periodísticos Utilizados para el análisis cualitativo y cuantitativo }
\label{sec:apendiceA}

En este apéndice se muestran los tres artículos periodísticos utilizados para realizar los análisis cuantitativos y cualitativos del diseño de la evaluación.

%-------------------------------------------------------------------
\section{Artículo Tecnológico}
%-------------------------------------------------------------------
\label{cap:sec:articulotecnologico}
Los fósiles de una escena primitiva, hallados con la precisión de una fotografía, demuestran que los seres humanos se comieron a los últimos grandes mamíferos que quedaban en América después de la última glaciación. La evidencia, que figura en un trabajo publicado por Science Advances, fue analizada por un equipo de arqueólogos argentinos del Consejo Nacional de Investigaciones Científicas y Técnicas (CONICET) en la Universidad Nacional del Centro de la provincia de Buenos Aires (UNICEN) junto a investigadores estadounidenses.

El clima contra la depredación humana es (ahora se sabe) una falsa controversia científica respecto a la extinción de los megamamíferos en Sudamérica y en el Mundo. Las sucesivas evidencias han enfatizado una causa sobre la otra, pero en conjunto reflejan que ambas fueron determinantes en la desaparición de los grandes animales del Pleistoceno.

Se trata de un proceso que en América se inició en el deshielo y que el apetito humano probablemente sólo aceleró. ``El clima jugó un rol también. Se extinguieron grandes animales en el mundo, no solamente acá aunque se extinguieron más en Sudamérica, también lo hicieron en Norteamérica y Europa. Entonces la discusión es: el clima y algo más. Este algo más creemos que son los seres humanos'', aclara el director de la investigación Gustavo Politis, sentado en su oficina del área de Investigaciones Arqueológicas y Paleontológicas del Cuaternario Pampeano (INCUAPA), a pocos kilómetros del sitio del hallazgo. Sin embargo, agrega una advertencia para quienes cargan las culpas sobre los seres humanos. ``En Sudamérica hay al menos 30 especies de megamamíferos que se han extinguido. Las que han sido cazadas son 5, 6 no más. No se puede explicar toda la extinción por la acción del hombre''. La caza no es el único daño que podríamos haber hecho en el pasado. ``También puede haber pasado que los seres humanos hayan hecho disrupciones en el ambiente como la introducción de nuevos parásitos o quemazones en los campos. Si el fuego produjo quemazón en poblaciones de animales con bajas tasas de reproducción, había un clima desfavorable y encima aparecieron seres humanos que los depredaron, los extinguieron'' concluye Politis.

La imagen que este reciente descubrimiento permite evocar ocurre hace 12.600 años a la vera de un pantano (hoy convertido en arroyo) en la llanura pampeana argentina. Quedan pocos gigantes para cazar. Hace un poco más de frío, hay menos humedad que en la actualidad y el mar está más lejos. Sobre los pastizales se erige quejoso un perezoso de cuatro metros de alto (como la estatura del oso y el madroño) y unas tres toneladas (el equivalente a cinco toros de lidia). Un grupo de cazadores armados con lanzas y proyectiles de piedra lo ataca. El grotesco animal, cuyo nombre científico es Megaterio, no logra ganar la batalla y acaba convertido en cena, abrigo y herramientas. Para saber esto, los investigadores desenterraron y examinaron fósiles durante 18 años en un campo de hacienda al que llegaron gracias a la suerte de un agricultor que, operado de la cadera, tuvo que dejar un tiempo el caballo y recorrer el campo andando. La caminata lenta y larga lo sorprendió con un fémur inmenso y ennegrecido asomando de la tierra.

Esa oscuridad se convertiría en la clave para desentramar la prehistórica escena. ``Es un ambiente con mucha materia orgánica porque era un antiguo pantano'', describe Politis. En esas condiciones, explica, es muy probable que se produzca la reacción de Maillard'sobre el fósil; cuando las moléculas de colágeno -que permiten conocer la antigüedad de un hueso- se juntan con las de los ácidos húmicos y fúlvicos -esos que forman el humus, la tierra negra fértil que aman los jardineros. ``Para separar eso hay que hacer un tratamiento especial porque si no, se datan las dos cosas juntas'', advierte Politis, licenciado en Antropología y doctor en Ciencias Naturales. ``Digamos que te rejuvenece la edad del fósil'', agrega en la misma oficina, y con iguales títulos académicos, el segundo autor del trabajo Pablo Messineo. ``El problema que teníamos siempre era que la datación nos daba entre 7.500 y 8.000 años atrás. Es una edad muy reciente para lo que es la extinción de los megamamíferos en Sudamérica, estimada en 12.000 años''. Hasta que llegaron a Thomas Stafford y su laboratorio. Este investigador desarrolló a principios de los '80 un método único que permite separar el colágeno del resto de la materia orgánica. ``Al separar el colágeno y datarlo te asegurás de que esa contaminación no exista más. Lo que hicimos ahora fue separarlos y datar ambos. Entonces, el colágeno nos dio 12.600 años y lo demás, 9.000. Así confirmamos que los ácidos estaban contaminando la muestra'', explica Messineo.

El paso del tiempo, el clima, la erosión, pueden tergiversar la historia. ``Si hay varios eventos a lo largo del tiempo, se pueden mezclar y es difícil separarlos. Acá hay uno solo; no pasó nada antes ni después. Tenemos una foto arqueológica de ese momento. Hay pocas cosas pero muy coherentes entre sí'', asegura el director del equipo de investigación. ``Una de las más interesantes es un artefacto que es una especie de cuchillo de piedra que se les rompió y tenemos los dos pedazos. Uno lo encontramos en 2003 y otro ahora. Estaban a 3 metros. Cuando los juntamos nos dimos cuenta de que uno lo habían tirado cuando se les rompió y al otro lo siguieron reactivando, tiene el filo más usado'', destaca Messineo.

``Es una conducta esperable; gente que está carneando, cuando el filo se les agota lo reactivan, cuando lo reactivan se puede romper y con el pedazo que les conviene más, siguen cortando. Es como llegar a una escena ni bien ha ocurrido. Están los huesos, las herramientas, solo falta la gente'', detalla con fascinación Politis, dinstinguido por su trayectoria por el Estado argentino. La evidencia de que los humanos se comieron al perezoso gigante es contundente. ``Encontramos marcas de corte en una costilla, eso indica que sin dudas que los grupos humanos lo carnearon. Después encontramos dos instrumentos hechos con las costillas del megaterio. Es otro indicio de que no es solamente la asociación de los artefactos de piedra con los huesos sino que los tipos estuvieron ahí procesando al animal. Eso hoy es indiscutible'', afirma.

Las dataciones de este estudio, en el que también trabajó Emily Lindsey, investigadora del Museo Rancho La Brea de Los Ángeles, Estados Unidos, permiten saber que los seres humanos y los grandes mamíferos coexistieron durante unos 1.500 años y arrojan sospechas sobre otras investigaciones. ``Hay que re-datar otros hallazgos también. Ahora sospechamos de las dataciones de los otros sitios porque estaban hechos con métodos más convencionales'', advierte Politis. Si el presupuesto (por el momento suspendido) lo permite, esas nuevas dataciones podrán descontaminarse en laboratorios propios ya que la UNICEN y el CONICET están trabajando en el desarrollo local de esa tecnología disponible, por ahora, en muy pocos laboratorios de Estados Unidos y Europa.


%-------------------------------------------------------------------
\section{Artículo Deportivo}
%-------------------------------------------------------------------
\label{cap:sec:articulodeportivo}

Las tres caídas consecutivas en los cuartos de final de la Champions habían dejado tocado al Barcelona, sobre todo la temporada pasada, cuando los muchachos de Valverde se derrumbaron por 3-0 ante la Roma. Los directivos apuntaron al técnico, actitud que alertó al vestuario. El grupo, en la intimidad, había despojado de cualquier responsabilidad a su entrenador, un tipo al que los jugadores reconocen y respetan, esencialmente por su mano izquierda para gestionar el camerino. Sabían que se habían quedado sin gasolina. Luis Suárez, uno de los pesos pesados, reconoció públicamente que se había equivocado ante el Leganés, en la previa de la debacle azulgrana en Roma. La herida se acentuó cuando el Madrid levantó su tercera Champions seguida. Messi no lo toleró: ``Queremos que vuelva al Camp Nou la copa linda y deseada''.

Reconocido el error, para esta campaña la consiga era clara: había que confiar más en las rotaciones del Valverde. Y esta vez, cuando llegaron los cuartos de la Champions ante el United, nadie protestó. El técnico borró a todas sus estrellas del duelo frente al Huesca en la víspera de la vuelta ante el equipo de Manchester. Funcionó. Los suplentes empataron en El Alcoraz y los titulares barrieron al conjunto de Ole Gunnar Solskjaer. Sin embargo, en la previa de las semifinales ante el Liverpool, Valverde apostó por los mejores contra la Real y viajarán todos, salvo Rakitic, a Vitoria. El Barça quiere llegar a la eliminatoria contra el equipo de Klopp, el martes 1 de mayo en el Camp Nou, con la Liga en el Museo del Camp Nou. El problema es que no es lo mismo salir campeón el miércoles —necesita ganar hoy al Alavés (21.30, beIN LaLiga) y que mañana pierda el Atlético con el Valencia— que el sábado, día en que el Levante visita el Camp Nou, antes del encuentro de Champions. ¿Qué alineación sacará el técnico ante el equipo granota si se juega el título y el miércoles aguarda el Liverpool?

``Nos ponemos estupendos para ver cuándo vamos a ganar… Lo que queremos es ganar cuanto antes, pero la gente lo ve desde fuera y piensa que todo es sencillo'', resolvió Valverde. ``Cuando se gana la Liga al final parece que la alegría es superior, pero cuando tienes una ventaja sobre el segundo parece un título previsible'', completó el Txingurri. ``Ojalá pudiéramos ganar siempre con margen''.

Valverde dejó caer que habrá rotaciones en Vitoria. ``Desde luego, es muy posible que haya cambios'', advirtió. El técnico lleva perfectamente la cuenta de los minutos de sus jugadores, y los titulares están más descansados que la última campaña. ``Físicamente los veo bien'', subrayó; ``es normal que a estas alturas de la temporada los equipos acusemos el paso del tiempo, pero nos alimentamos de las sensaciones que tenemos. Los retos eliminan cualquier cansancio, cuando hay objetivos las piernas están frescas''. Valverde confía en la ambición de sus futbolistas, pero, por las dudas, los quiere descansados ante el Liverpool.


%-------------------------------------------------------------------
\section{Artículo Político}
%-------------------------------------------------------------------
\label{cap:sec:articulopolitico}

El PP ha criticado este martes los ataques recibidos por el líder de Ciudadanos durante el debate electoral. ``Rivera le hizo el trabajo sucio a Sánchez'', se ha quejado el secretario general de los populares, Teodoro García Egea. El candidato de Cs salió al choque tanto contra Pedro Sánchez como contra Casado, mientras este evitaba el enfrentamiento con él. Solo una vez –cuando Rivera sugirió que podían pactar con el PNV como ya habían hecho en el pasado- entró Casado en el cuerpo a cuerpo: ``Ni sus votantes ni los míos entienden lo que está diciendo, Usted no es mi adversario. Yo no voy a pactar ni con el PNV ni con Sánchez. En materia de pactos soy más creíble que usted''.

Todos los partidos se arrogan este martes la victoria del debate en TVE, a pocas horas del segundo asalto en Atresmedia. Fuentes del PP indican que Casado hizo “el debate que quería hacer”, con un ``tono moderado y presidenciable''. El candidato popular quiere trata de recuperar un perfil centrado en la última semana de campaña y dejar atrás las graves acusaciones y descalificativos empleados contra Sánchez, como cuando le acusó de preferir ``las manos manchadas de sangre''. Los populares dudan sobre la estrategia. Algunos dirigentes consultados por EL PAÍS creen que con el tono moderado del lunes, Casado puede decantar a algunos indecisos que se debaten entre votar al PP o a Ciudadanos. Otros señalan que el líder estuvo excesivamente plano y dejó que Rivera acaparara protagonismo. Fuentes de la dirección indican, en cualquier caso, que el formato de esta noche, con preguntas y repreguntas, da pie a intervenciones algo más atrevidas.

``Rivera todavía no se ha enterado de que el enemigo se llama Pedro Sánchez'', ha subrayado este martes el secretario general del PP. Los populares quieren unir fuerzas (y votos) en torno al enemigo común, ya que su única oportunidad de gobernar es sumando los escaños de lo que el presidente llamó el lunes, insistentemente, ``las derechas'', incluyendo a Vox.

Pero Rivera salió a la cita a proyectarse como alternativa a Sánchez y eso implicaba también tratar de noquear a Casado, con el que se disputa el liderazgo del bloque de centroderecha. Los suyos están convencidos de que lo logró, ante un Casado demasiado contenido.

``Todos los españoles vieron ayer un ganador claro, el único capaz de demostrar que es una alternativa a Pedro Sánchez, el único que lleva mucho años luchando por la igualdad de todos los españoles, el único con la fuerza y la valentía necesarias para hacer frente a los grandes retos de España'', ha enfatizado hoy sobre Rivera Inés Arrimadas, candidata de Cs por Barcelona. La dirección de Ciudadanos considera que el de ayer fue un ``punto de inflexión'' en la campaña electoral en el que su candidato salió beneficiado. Rivera, según fuentes de la cúpula, volverá a salir hoy a la ofensiva y a intentar liderar el segundo debate.

En la izquierda, el PSOE ha cargado contra los ``exabruptos'' de la derecha y ha reivindicado que el candidato socialista fue el único en hacer un debate en positivo. ``Creo que el tono de los candidatos de derechas en el debate fue el mismo tono que hemos estado viendo durante toda la campaña: uno donde han primado más los insultos, los exabruptos, con pocas propuestas y poco proyecto de país'', ha defendido la candidata del PSOE por Barcelona, Meritxell Batet. Sánchez puso sobre la mesa, a juicio de los socialistas, ``un proyecto de país, muchas propuestas e hizo un listado exhaustivo'' de la labor del Gobierno socialista en sus diez meses al frente del Ejecutivo.

Por su parte, Unidas Podemos ha destacado que el candidato del PSOE no contestó ayer a las reiteradas preguntas de Pablo Iglesias sobre si pactará o no con Ciudadanos tras el 28 de abril, lo que para la formación morada es revelador de que esa es la intención de los socialistas. El secretario de Organización de Podemos, Pablo Echenique, se ha dirigido este martes a los indecisos y a los votantes socialistas para advertirles de que su voto el 28 de abril puede acabar ``en un Gobierno de derechas en el que Albert Rivera sea ministro''.

Ese es el punto central de la campaña de Podemos, el de alertar sobre un posible pacto del PSOE con Ciudadanos, aunque la ejecutiva del partido de Albert Rivera ha aprobado cerrar la puerta a ese acuerdo, como el candidato naranja recordó también ayer durante el debate electoral. Su apuesta es un pacto con el PP para el que ayer volvió a tender la mano a Casado, sin respuesta. En el segundo asalto, el debate de esta noche, los pactos poselectorales volverán a dar previsiblemente momentos de choque entre los candidatos.