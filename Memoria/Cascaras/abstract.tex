\section*{Abstract}

Today, 1\% of the Spanish population suffers from some kind of cognitive disability. It is very difficult for these people to perform simple tasks such as reading a book or doing the daily shopping, due to their difficulties to understand common concepts. The easiest solution would be to look up for words in a dictionary, but this solution is not appropriate for these people because the definitions and concepts used in them include complex concepts and they are also hard to understand.

In order to help these people with their inability to understand certain words, we have developed a web application in which they can look up a word and obtain a definition relating these difficult words with easier ones they are likely to know and understand. This has been accomplished by using similes and metaphors related to those words. The fact that it is a web application makes it more handy for everyone with an Internet connection.

We have developed the application through a User-centered design, taking into account the opinions and advices of experts. The aim of this approach has been to develop an application adapted to the real needs of the target users.

This application is composed of a few web services that can obtain related words, similes and metaphors. Also, in order to  make it easier to use for the user, we have implemented two more web services: one of them finds pictograms and the other one finds examples and a description of the word they look for.

The application has been tested by users with cognitive impairments. After being tested by these users, we have concluded that this application is useful for people who have mild disabilities but further adaptations have to be carried out in order to make it useful for people with medium or severe impairments.


\section*{Keywords}

\noindent Cognitive disability, Inclusion, Easy words, Complex words, Rhetorical figures, Metaphor, Simile, Web service, User-centered design, Pictograms



