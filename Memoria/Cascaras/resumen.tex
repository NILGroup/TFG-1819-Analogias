\chapter*{Resumen}

Actualmente, en España se calcula que aproximadamente el 1\% de la población sufre algún tipo de discapacidad intelectual. Para estas personas, la realización de tareas sencillas como ir a comprar o leer un libro puede ser complicada, ya que tienen dificultad de comprensión de términos cotidianos. Una posible solución para entender el concepto es la consulta del mismo en un diccionario. Pero no es una solución viable para estos colectivos, ya que las definiciones ofrecidas suelen tener un lenguaje complicado y difícil de entender.

Otra de las opciones que existen actualmente, es la traducción de texto a pictogramas. Existen aplicaciones que introduciendo un texto lo traducen automáticamente a pictogramas facilitando mucho la comprensión para estas personas. Sin embargo, esta solución es poco efectiva cuando las palabras son polisémicas o su significado es demasiado complejo.

Para ayudar a estas personas con sus problemas de comprensión, se ha implementado una aplicación que dada una palabra compleja, genere metáforas y símiles utilizando palabras fáciles(entendiendo como palabras fáciles, las listas de formas más comunes del castellano según la RAE). De esta manera puede hacerse una idea precisa del significado del concepto buscado. Además, dicha aplicación esta al alcance de cualquier usuario que disponga de un dispositivo con conexión a internet, por lo que puede ser usada en cualquier momento.

Para la creación de la interfaz, inicialmente se presentó un prototipo a varios expertos para que esta fuese lo más fácil posible de usar por los usuarios finales. Posteriormente, se introdujeron todas las modificaciones sugeridas en esta evaluación preliminar y una vez terminada la aplicación se realizó una evaluación final.

La evaluación final consistió en que inicialmente los usuarios introducían palabras establecidas por los autores de la aplicación para posteriormente hacer un uso libre de la misma. De esta evaluación se pudo concluir que el trabajo es útil para personas cuyo grado de discapacidad no fuese muy elevado, pero queda trabajo por hacer si queremos que la aplicación sea útil para personas con un grado de discapacidad mayor.


\section*{Palabras clave}
   
\noindent Discapacidad cognitiva\\
\noindent Inclusión\\
\noindent Palabras fáciles\\
\noindent Palabras complejas\\
\noindent Figuras Retóricas\\
\noindent Metáfora\\
\noindent Símil\\
\noindent Servicios Web\\
\noindent Diseño centrado en el usuario\\
\noindent Pictogramas\\

   


