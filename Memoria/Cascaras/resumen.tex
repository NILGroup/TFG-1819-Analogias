\section*{Resumen}

Actualmente, en España se calcula que aproximadamente el 1\% de la población sufre algún tipo de discapacidad intelectual. Para estas personas, la realización de tareas tan sencillas y cotidianas como ir a comprar o leer un libro pueden resultar muy complicadas, ya que tienen dificultad de comprensión para comprender muchos de los términos empleados en el día a día. Una posible solución para entender un concepto es la consulta del mismo en un diccionario. Pero esta no es una solución viable para estos colectivos, ya que las definiciones ofrecidas suelen tener un lenguaje complicado y difícil de entender para ellos.

Para ayudar a estas personas con sus problemas de comprensión, se ha implementado una aplicación web que dada una palabra la define en base a otras palabras relacionadas con ella más fáciles y que el usuario conoce. Para dar estas definiciones se emplearon metáforas y símiles. Al tratarse de una aplicación web, está al alcance de cualquier usuario que disponga de un dispositivo con conexión a internet.

Para diseñar la aplicación se ha hecho un diseño centrado en el usuario, contando con las opiniones de expertos. Se quiere conseguir una aplicación que se adapte a las necesidades reales del colectivo objetivo.

La aplicación está compuesta por varios servicios web, para obtener tanto las palabras relacionadas como para generar los símiles y las metáforas. Además, para facilitar la comprensión del usuario, se han implementado varios servicios web complementarios. Uno que obtiene el pictograma asociado a una palabra, y otro que proporciona una descripción y un ejemplo.

La aplicación ha sido evaluada por usuario finales. De esta evaluación se pudo concluir que el trabajo es útil para personas cuyo grado de discapacidad no fuese muy elevado, y que queda trabajo por hacer si queremos que la aplicación sea útil para personas con un grado de discapacidad mayor.


\section*{Palabras clave}
   
\noindent Discapacidad cognitiva, Inclusión, Palabras fáciles, Palabras complejas, Figuras Retóricas, Metáfora, Símil, Servicios Web, Diseño centrado en el usuario, Pictogramas

   


