% ----------------------------------------------------------------------
%\part{title}
%                            TFMTesis.tex
%
%----------------------------------------------------------------------
%
% Este fichero contiene el "documento maestro" del documento. Lo único
% que hace es configurar el entorno LaTeX e incluir los ficheros .tex
% que contienen cada sección.
%
%----------------------------------------------------------------------
%
% Los ficheros necesarios para este documento son:
%
%       TeXiS/* : ficheros de la plantilla TeXiS.
%       Cascaras/* : ficheros con las partes del documento que no
%          son capítulos ni apéndices (portada, agradecimientos, etc.)
%       Capitulos/*.tex : capítulos de la tesis
%       Apendices/*.tex: apéndices de la tesis
%       constantes.tex: constantes LaTeX
%       config.tex : configuración de la "compilación" del documento
%       guionado.tex : palabras con guiones
%
% Para la bibliografía, además, se necesitan:
%
%       *.bib : ficheros con la información de las referencias
%
% ---------------------------------------------------------------------

\documentclass[11pt,a4paper,twoside]{book}

%
% Definimos  el   comando  \compilaCapitulo,  que   luego  se  utiliza
% (opcionalmente) en config.tex. Quedaría  mejor si también se definiera
% en  ese fichero,  pero por  el modo  en el  que funciona  eso  no es
% posible. Puedes consultar la documentación de ese fichero para tener
% más  información. Definimos también  \compilaApendice, que  tiene el
% mismo  cometido, pero  que se  utiliza para  compilar  únicamente un
% apéndice.
%
%
% Si  queremos   compilar  solo   una  parte  del   documento  podemos
% especificar mediante  \includeonly{...} qué ficheros  son los únicos
% que queremos  que se incluyan.  Esto  es útil por  ejemplo para sólo
% compilar un capítulo.
%
% El problema es que todos aquellos  ficheros que NO estén en la lista
% NO   se  incluirán...  y   eso  también   afecta  a   ficheros  de
% la plantilla...
%
% Total,  que definimos  una constante  con los  ficheros  que siempre
% vamos a querer compilar  (aquellos relacionados con configuración) y
% luego definimos \compilaCapitulo.
\newcommand{\ficherosBasicosTeXiS}{%
TeXiS/TeXiS_pream,TeXiS/TeXiS_cab,TeXiS/TeXiS_bib,TeXiS/TeXiS_cover%
}
\newcommand{\ficherosBasicosTexto}{%
constantes,guionado,Cascaras/bibliografia,config%
}
\newcommand{\compilaCapitulo}[1]{%
\includeonly{\ficherosBasicosTeXiS,\ficherosBasicosTexto,Capitulos/#1}%
}

\newcommand{\compilaApendice}[1]{%
\includeonly{\ficherosBasicosTeXiS,\ficherosBasicosTexto,Apendices/#1}%
}

%- - - - - - - - - - - - - - - - - - - - - - - - - - - - - - - - - - -
%            Preámbulo del documento. Configuraciones varias
%- - - - - - - - - - - - - - - - - - - - - - - - - - - - - - - - - - -

% Define  el  tipo  de  compilación que  estamos  haciendo.   Contiene
% definiciones  de  constantes que  cambian  el  comportamiento de  la
% compilación. Debe incluirse antes del paquete TeXiS/TeXiS.sty
%---------------------------------------------------------------------
%
%                          config.tex
%
%---------------------------------------------------------------------
%
% Contiene la  definición de constantes  que determinan el modo  en el
% que se compilará el documento.
%
%---------------------------------------------------------------------
%
% En concreto, podemos  indicar si queremos "modo release",  en el que
% no  aparecerán  los  comentarios  (creados  mediante  \com{Texto}  o
% \comp{Texto}) ni los "por  hacer" (creados mediante \todo{Texto}), y
% sí aparecerán los índices. El modo "debug" (o mejor dicho en modo no
% "release" muestra los índices  (construirlos lleva tiempo y son poco
% útiles  salvo  para   la  versión  final),  pero  sí   el  resto  de
% anotaciones.
%
% Si se compila con LaTeX (no  con pdflatex) en modo Debug, también se
% muestran en una esquina de cada página las entradas (en el índice de
% palabras) que referencian  a dicha página (consulta TeXiS_pream.tex,
% en la parte referente a show).
%
% El soporte para  el índice de palabras en  TeXiS es embrionario, por
% lo  que no  asumas que  esto funcionará  correctamente.  Consulta la
% documentación al respecto en TeXiS_pream.tex.
%
%
% También  aquí configuramos  si queremos  o  no que  se incluyan  los
% acrónimos  en el  documento final  en la  versión release.  Para eso
% define (o no) la constante \acronimosEnRelease.
%
% Utilizando \compilaCapitulo{nombre}  podemos también especificar qué
% capítulo(s) queremos que se compilen. Si no se pone nada, se compila
% el documento  completo.  Si se pone, por  ejemplo, 01Introduccion se
% compilará únicamente el fichero Capitulos/01Introduccion.tex
%
% Para compilar varios  capítulos, se separan sus nombres  con comas y
% no se ponen espacios de separación.
%
% En realidad  la macro \compilaCapitulo  está definida en  el fichero
% principal tesis.tex.
%
%---------------------------------------------------------------------


% Comentar la línea si no se compila en modo release.
% TeXiS hará el resto.
% ¡¡¡Si cambias esto, haz un make clean antes de recompilar!!!
\def\release{1}


% Descomentar la linea si se quieren incluir los
% acrónimos en modo release (en modo debug
% no se incluirán nunca).
% ¡¡¡Si cambias esto, haz un make clean antes de recompilar!!!
%\def\acronimosEnRelease{1}


% Descomentar la línea para establecer el capítulo que queremos
% compilar

% \compilaCapitulo{01Introduccion}
% \compilaCapitulo{02EstructuraYGeneracion}
% \compilaCapitulo{03Edicion}
% \compilaCapitulo{04Imagenes}
% \compilaCapitulo{05Bibliografia}
% \compilaCapitulo{06Makefile}

% \compilaApendice{01AsiSeHizo}

% Variable local para emacs, para  que encuentre el fichero maestro de
% compilación y funcionen mejor algunas teclas rápidas de AucTeX
%%%
%%% Local Variables:
%%% mode: latex
%%% TeX-master: "./Tesis.tex"
%%% End:


% Paquete de la plantilla
\usepackage{TeXiS/TeXiS}

% Incluimos el fichero con comandos de constantes
%---------------------------------------------------------------------
%
%                          constantes.tex
%
%---------------------------------------------------------------------
%
% Fichero que  declara nuevos comandos LaTeX  sencillos realizados por
% comodidad en la escritura de determinadas palabras
%
%---------------------------------------------------------------------

%%%%%%%%%%%%%%%%%%%%%%%%%%%%%%%%%%%%%%%%%%%%%%%%%%%%%%%%%%%%%%%%%%%%%%
% Comando: 
%
%       \titulo
%
% Resultado: 
%
% Escribe el título del documento.
%%%%%%%%%%%%%%%%%%%%%%%%%%%%%%%%%%%%%%%%%%%%%%%%%%%%%%%%%%%%%%%%%%%%%%
\def\titulo{\textsc{TeXiS}: Una plantilla de \LaTeX\
  para Tesis y otros documentos}

%%%%%%%%%%%%%%%%%%%%%%%%%%%%%%%%%%%%%%%%%%%%%%%%%%%%%%%%%%%%%%%%%%%%%%
% Comando: 
%
%       \autor
%
% Resultado: 
%
% Escribe el autor del documento.
%%%%%%%%%%%%%%%%%%%%%%%%%%%%%%%%%%%%%%%%%%%%%%%%%%%%%%%%%%%%%%%%%%%%%%
\def\autor{Marco Antonio y Pedro Pablo G\'omez Mart\'in}

% Variable local para emacs, para  que encuentre el fichero maestro de
% compilación y funcionen mejor algunas teclas rápidas de AucTeX

%%%
%%% Local Variables:
%%% mode: latex
%%% TeX-master: "tesis.tex"
%%% End:


% Sacamos en el log de la compilación el copyright
%\typeout{Copyright Marco Antonio and Pedro Pablo Gomez Martin}

%
% "Metadatos" para el PDF
\usepackage{algorithmic}
%
\ifpdf\hypersetup{%
    pdftitle = {\titulo},
    pdfsubject = {Plantilla de Tesis},
    pdfkeywords = {Plantilla, LaTeX, tesis, trabajo de
      investigación, trabajo de Master},
    pdfauthor = {\textcopyright\ \autor},
    pdfcreator = {\LaTeX\ con el paquete \flqq hyperref\frqq},
    pdfproducer = {pdfeTeX-0.\the\pdftexversion\pdftexrevision},
    }
    \pdfinfo{/CreationDate (\today)}
\fi

% Para el entrecomillado
\usepackage[spanish]{babel}
\usepackage{multirow} 
\usepackage{longtable}
\usepackage{verbatim}
%- - - - - - - - - - - - - - - - - - - - - - - - - - - - - - - - - - -
%                        Documento
%- - - - - - - - - - - - - - - - - - - - - - - - - - - - - - - - - - -
\begin{document}

% Incluimos el  fichero de definición de guionado  de algunas palabras
% que LaTeX no ha dividido como debería
%----------------------------------------------------------------
%
%                          guionado.tex
%
%----------------------------------------------------------------
%
% Fichero con algunas divisiones de palabras que LaTeX no
% hace correctamente si no se le da alguna ayuda.
%
%----------------------------------------------------------------

\hyphenation{
% a
abs-trac-to
abs-trac-tos
abs-trac-ta
abs-trac-tas
ac-tua-do-res
a-gra-de-ci-mien-tos
ana-li-za-dor
an-te-rio-res
an-te-rior-men-te
apa-rien-cia
a-pro-pia-do
a-pro-pia-dos
a-pro-pia-da
a-pro-pia-das
a-pro-ve-cha-mien-to
a-que-llo
a-que-llos
a-que-lla
a-que-llas
a-sig-na-tu-ra
a-sig-na-tu-ras
a-so-cia-da
a-so-cia-das
a-so-cia-do
a-so-cia-dos
au-to-ma-ti-za-do
% b
batch
bi-blio-gra-fía
bi-blio-grá-fi-cas
bien
bo-rra-dor
boo-l-ean-expr
% c
ca-be-ce-ra
call-me-thod-ins-truc-tion
cas-te-lla-no
cir-cuns-tan-cia
cir-cuns-tan-cias
co-he-ren-te
co-he-ren-tes
co-he-ren-cia
co-li-bri
co-men-ta-rio
co-mer-cia-les
co-no-ci-mien-to
cons-cien-te
con-si-de-ra-ba
con-si-de-ra-mos
con-si-de-rar-se
cons-tan-te
cons-trucción
cons-tru-ye
cons-tru-ir-se
con-tro-le
co-rrec-ta-men-te
co-rres-pon-den
co-rres-pon-dien-te
co-rres-pon-dien-tes
co-ti-dia-na
co-ti-dia-no
crean
cris-ta-li-zan
cu-rri-cu-la
cu-rri-cu-lum
cu-rri-cu-lar
cu-rri-cu-la-res
% d
de-di-ca-do
de-di-ca-dos
de-di-ca-da
de-di-ca-das
de-rro-te-ro
de-rro-te-ros
de-sa-rro-llo
de-sa-rro-llos
de-sa-rro-lla-do
de-sa-rro-lla-dos
de-sa-rro-lla-da
de-sa-rro-lla-das
de-sa-rro-lla-dor
de-sa-rro-llar
des-cri-bi-re-mos
des-crip-ción
des-crip-cio-nes
des-cri-to
des-pués
de-ta-lla-do
de-ta-lla-dos
de-ta-lla-da
de-ta-lla-das
di-a-gra-ma
di-a-gra-mas
di-se-ños
dis-po-ner
dis-po-ni-bi-li-dad
do-cu-men-ta-da
do-cu-men-to
do-cu-men-tos
% e
edi-ta-do
e-du-ca-ti-vo
e-du-ca-ti-vos
e-du-ca-ti-va
e-du-ca-ti-vas
e-la-bo-ra-do
e-la-bo-ra-dos
e-la-bo-ra-da
e-la-bo-ra-das
es-co-llo
es-co-llos
es-tu-dia-do
es-tu-dia-dos
es-tu-dia-da
es-tu-dia-das
es-tu-dian-te
e-va-lua-cio-nes
e-va-lua-do-res
exis-ten-tes
exhaus-ti-va
ex-pe-rien-cia
ex-pe-rien-cias
% f
for-ma-li-za-do
% g
ge-ne-ra-ción
ge-ne-ra-dor
ge-ne-ra-do-res
ge-ne-ran
% h
he-rra-mien-ta
he-rra-mien-tas
% i
i-dio-ma
i-dio-mas
im-pres-cin-di-ble
im-pres-cin-di-bles
in-de-xa-do
in-de-xa-dos
in-de-xa-da
in-de-xa-das
in-di-vi-dual
in-fe-ren-cia
in-fe-ren-cias
in-for-ma-ti-ca
in-gre-dien-te
in-gre-dien-tes
in-me-dia-ta-men-te
ins-ta-la-do
ins-tan-cias
% j
% k
% l
len-gua-je
li-be-ra-to-rio
li-be-ra-to-rios
li-be-ra-to-ria
li-be-ra-to-rias
li-mi-ta-do
li-te-ra-rio
li-te-ra-rios
li-te-ra-ria
li-te-ra-rias
lo-tes
% m
ma-ne-ra
ma-nual
mas-que-ra-de
ma-yor
me-mo-ria
mi-nis-te-rio
mi-nis-te-rios
mo-de-lo
mo-de-los
mo-de-la-do
mo-du-la-ri-dad
mo-vi-mien-to
% n
na-tu-ral
ni-vel
nues-tro
% o
obs-tan-te
o-rien-ta-do
o-rien-ta-dos
o-rien-ta-da
o-rien-ta-das
% p
pa-ra-le-lo
pa-ra-le-la
par-ti-cu-lar
par-ti-cu-lar-men-te
pe-da-gó-gi-ca
pe-da-gó-gi-cas
pe-da-gó-gi-co
pe-da-gó-gi-cos
pe-rio-di-ci-dad
per-so-na-je
plan-te-a-mien-to
plan-te-a-mien-tos
po-si-ción
pre-fe-ren-cia
pre-fe-ren-cias
pres-cin-di-ble
pres-cin-di-bles
pri-me-ra
pro-ble-ma
pro-ble-mas
pró-xi-mo
pu-bli-ca-cio-nes
pu-bli-ca-do
% q
% r
rá-pi-da
rá-pi-do
ra-zo-na-mien-to
ra-zo-na-mien-tos
re-a-li-zan-do
re-fe-ren-cia
re-fe-ren-cias
re-fe-ren-cia-da
re-fe-ren-cian
re-le-van-tes
re-pre-sen-ta-do
re-pre-sen-ta-dos
re-pre-sen-ta-da
re-pre-sen-ta-das
re-pre-sen-tar-lo
re-qui-si-to
re-qui-si-tos
res-pon-der
res-pon-sa-ble
% s
se-pa-ra-do
si-guien-do
si-guien-te
si-guien-tes
si-guie-ron
si-mi-lar
si-mi-la-res
si-tua-ción
% t
tem-pe-ra-ments
te-ner
trans-fe-ren-cia
trans-fe-ren-cias
% u
u-sua-rio
Unreal-Ed
% v
va-lor
va-lo-res
va-rian-te
ver-da-de-ro
ver-da-de-ros
ver-da-de-ra
ver-da-de-ras
ver-da-de-ra-men-te
ve-ri-fi-ca
% w
% x
% y
% z
}
% Variable local para emacs, para que encuentre el fichero
% maestro de compilación
%%%
%%% Local Variables:
%%% mode: latex
%%% TeX-master: "./Tesis.tex"
%%% End:


% Marcamos  el inicio  del  documento para  la  numeración de  páginas
% (usando números romanos para esta primera fase).
\frontmatter
\pagestyle{empty}

%---------------------------------------------------------------------
%
%                          configCover.tex
%
%---------------------------------------------------------------------
%
% cover.tex
% Copyright 2009 Marco Antonio Gomez-Martin, Pedro Pablo Gomez-Martin
%
% This file belongs to the TeXiS manual, a LaTeX template for writting
% Thesis and other documents. The complete last TeXiS package can
% be obtained from http://gaia.fdi.ucm.es/projects/texis/
%
% Although the TeXiS template itself is distributed under the 
% conditions of the LaTeX Project Public License
% (http://www.latex-project.org/lppl.txt), the manual content
% uses the CC-BY-SA license that stays that you are free:
%
%    - to share & to copy, distribute and transmit the work
%    - to remix and to adapt the work
%
% under the following conditions:
%
%    - Attribution: you must attribute the work in the manner
%      specified by the author or licensor (but not in any way that
%      suggests that they endorse you or your use of the work).
%    - Share Alike: if you alter, transform, or build upon this
%      work, you may distribute the resulting work only under the
%      same, similar or a compatible license.
%
% The complete license is available in
% http://creativecommons.org/licenses/by-sa/3.0/legalcode
%
%---------------------------------------------------------------------
%
% Fichero que contiene la configuración de la portada y de la 
% primera hoja del documento.
%
%---------------------------------------------------------------------


% Pueden configurarse todos los elementos del contenido de la portada
% utilizando comandos.

%%%%%%%%%%%%%%%%%%%%%%%%%%%%%%%%%%%%%%%%%%%%%%%%%%%%%%%%%%%%%%%%%%%%%%
% Título del documento:
% \tituloPortada{titulo}
% Nota:
% Si no se define se utiliza el del \titulo. Este comando permite
% cambiar el título de forma que se especifiquen dónde se quieren
% los retornos de carro cuando se utilizan fuentes grandes.
%%%%%%%%%%%%%%%%%%%%%%%%%%%%%%%%%%%%%%%%%%%%%%%%%%%%%%%%%%%%%%%%%%%%%%
\tituloPortada{%
Mejora de la Comprensión Lectora para la Inclusión mediante Figuras Retóricas
}

%%%%%%%%%%%%%%%%%%%%%%%%%%%%%%%%%%%%%%%%%%%%%%%%%%%%%%%%%%%%%%%%%%%%%%
% Autor del documento:
% \autorPortada{Nombre}
% Se utiliza en la portada y en el valor por defecto del
% primer subtítulo de la segunda portada.
%%%%%%%%%%%%%%%%%%%%%%%%%%%%%%%%%%%%%%%%%%%%%%%%%%%%%%%%%%%%%%%%%%%%%%
\autorPortada{Irene Martín Berlanga\\Pablo García Hernández}

%%%%%%%%%%%%%%%%%%%%%%%%%%%%%%%%%%%%%%%%%%%%%%%%%%%%%%%%%%%%%%%%%%%%%%
% Fecha de publicación:
% \fechaPublicacion{Fecha}
% Puede ser vacío. Aparece en la última línea de ambas portadas
%%%%%%%%%%%%%%%%%%%%%%%%%%%%%%%%%%%%%%%%%%%%%%%%%%%%%%%%%%%%%%%%%%%%%%
\fechaPublicacion{\today}

%%%%%%%%%%%%%%%%%%%%%%%%%%%%%%%%%%%%%%%%%%%%%%%%%%%%%%%%%%%%%%%%%%%%%%
% Imagen de la portada (y escala)
% \imagenPortada{Fichero}
% \escalaImagenPortada{Numero}
% Si no se especifica, se utiliza la imagen TODO.pdf
%%%%%%%%%%%%%%%%%%%%%%%%%%%%%%%%%%%%%%%%%%%%%%%%%%%%%%%%%%%%%%%%%%%%%%
\imagenPortada{Imagenes/Vectorial/escudoUCM}
\escalaImagenPortada{.2}

%%%%%%%%%%%%%%%%%%%%%%%%%%%%%%%%%%%%%%%%%%%%%%%%%%%%%%%%%%%%%%%%%%%%%%
% Tipo de documento.
% \tipoDocumento{Tipo}
% Para el texto justo debajo del escudo.
% Si no se indica, se utiliza "TESIS DOCTORAL".
%%%%%%%%%%%%%%%%%%%%%%%%%%%%%%%%%%%%%%%%%%%%%%%%%%%%%%%%%%%%%%%%%%%%%%
\tipoDocumento{Trabajo de Fin de Grado}

%%%%%%%%%%%%%%%%%%%%%%%%%%%%%%%%%%%%%%%%%%%%%%%%%%%%%%%%%%%%%%%%%%%%%%
% Institución/departamento asociado al documento.
% \institucion{Nombre}
% Puede tener varias líneas. Se utiliza en las dos portadas.
% Si no se indica aparecerá vacío.
%%%%%%%%%%%%%%%%%%%%%%%%%%%%%%%%%%%%%%%%%%%%%%%%%%%%%%%%%%%%%%%%%%%%%%
\institucion{%
Grado en Ingeniería de Software\\[0.2em]
Facultad de Informática\\[0.2em]
Universidad Complutense de Madrid
}

%%%%%%%%%%%%%%%%%%%%%%%%%%%%%%%%%%%%%%%%%%%%%%%%%%%%%%%%%%%%%%%%%%%%%%
% Director del trabajo.
% \directorPortada{Nombre}
% Se utiliza para el valor por defecto del segundo subtítulo, donde
% se indica quién es el director del trabajo.
% Si se fuerza un subtítulo distinto, no hace falta definirlo.
%%%%%%%%%%%%%%%%%%%%%%%%%%%%%%%%%%%%%%%%%%%%%%%%%%%%%%%%%%%%%%%%%%%%%%
\directorPortada{Virginia Francisco Gilmartín\\Gonzalo Rubén Mendez Pozo}

%%%%%%%%%%%%%%%%%%%%%%%%%%%%%%%%%%%%%%%%%%%%%%%%%%%%%%%%%%%%%%%%%%%%%%
% Texto del primer subtítulo de la segunda portada.
% \textoPrimerSubtituloPortada{Texto}
% Para configurar el primer "texto libre" de la segunda portada.
% Si no se especifica se indica "Memoria que presenta para optar al
% título de Doctor en Informática" seguido del \autorPortada.
%%%%%%%%%%%%%%%%%%%%%%%%%%%%%%%%%%%%%%%%%%%%%%%%%%%%%%%%%%%%%%%%%%%%%%
\textoPrimerSubtituloPortada{%
\textbf{Trabajo de Fin de Grado en Ingeniería de Software}  \\ [0.3em]
\textbf{Departamento de Ingeniería de Software e Inteligencia Artificial} \\ [0.3em]
}

%%%%%%%%%%%%%%%%%%%%%%%%%%%%%%%%%%%%%%%%%%%%%%%%%%%%%%%%%%%%%%%%%%%%%%
% Texto del segundo subtítulo de la segunda portada.
% \textoSegundoSubtituloPortada{Texto}
% Para configurar el segundo "texto libre" de la segunda portada.
% Si no se especifica se indica "Dirigida por el Doctor" seguido
% del \directorPortada.
%%%%%%%%%%%%%%%%%%%%%%%%%%%%%%%%%%%%%%%%%%%%%%%%%%%%%%%%%%%%%%%%%%%%%%


%%%%%%%%%%%%%%%%%%%%%%%%%%%%%%%%%%%%%%%%%%%%%%%%%%%%%%%%%%%%%%%%%%%%%%
% \explicacionDobleCara
% Si se utiliza, se aclara que el documento está preparado para la
% impresión a doble cara.
%%%%%%%%%%%%%%%%%%%%%%%%%%%%%%%%%%%%%%%%%%%%%%%%%%%%%%%%%%%%%%%%%%%%%%
\explicacionDobleCara

%%%%%%%%%%%%%%%%%%%%%%%%%%%%%%%%%%%%%%%%%%%%%%%%%%%%%%%%%%%%%%%%%%%%%%
% \isbn
% Si se utiliza, aparecerá el ISBN detrás de la segunda portada.
%%%%%%%%%%%%%%%%%%%%%%%%%%%%%%%%%%%%%%%%%%%%%%%%%%%%%%%%%%%%%%%%%%%%%%
%\isbn{978-84-692-7109-4}


%%%%%%%%%%%%%%%%%%%%%%%%%%%%%%%%%%%%%%%%%%%%%%%%%%%%%%%%%%%%%%%%%%%%%%
% \copyrightInfo
% Si se utiliza, aparecerá información de los derechos de copyright
% detrás de la segunda portada.
%%%%%%%%%%%%%%%%%%%%%%%%%%%%%%%%%%%%%%%%%%%%%%%%%%%%%%%%%%%%%%%%%%%%%%
\copyrightInfo{\autor}


%%
%% Creamos las portadas
%%
\makeCover

% Variable local para emacs, para que encuentre el fichero
% maestro de compilación
%%%
%%% Local Variables:
%%% mode: latex
%%% TeX-master: "../Tesis.tex"
%%% End:

%\chapter*{Autorización de difusión}

   
Los abajo firmantes, matriculados en el Grado de Ingeniería de Software de la Facultad de Informática, autoriza a la Universidad Complutense de Madrid (UCM) a difundir y utilizar con fines académicos, no comerciales y mencionando expresamente a sus autores el presente Trabajo de Fin de Grado: ``Mejora de la Comprensión Lectora mediante Analogías para la Inclusión'', realizado durante el curso académico 2018-2019 bajo la dirección de Virginia Francisco Gilmartín y Gonzalo Rubén Mendez Pozo en el Departamento de Ingeniería de Software e Inteligencia Artificial, y a la Biblioteca de la UCM a depositarlo en el Archivo Institucional E-Prints Complutense con el objeto de incrementar la difusión, uso e impacto del trabajo en Internet y garantizar su preservación y acceso a largo plazo.

\vspace{5cm}

% +--------------------------------------------------------------------+
% | On the line below, replace "Enter Your Name" with your name
% | Use the same form of your name as it appears on your title page.
% | Use mixed case, for example, Lori Goetsch.
% +--------------------------------------------------------------------+
\begin{center}
	\large Nombre Del Alumno\\
	
	\vspace{0.5cm}
	
	% +--------------------------------------------------------------------+
	% | On the line below, replace Fecha
	% |
	% +--------------------------------------------------------------------+
	
	\today\\
	
\end{center}

%% +--------------------------------------------------------------------+
% | Dedication Page (Optional)
% +--------------------------------------------------------------------+

\chapter*{Dedicatoria}


Texto de la dedicatoria...
% +--------------------------------------------------------------------+
% | Acknowledgements Page (Optional)                                   |
% +--------------------------------------------------------------------+

\chapter*{Agradecimientos}

Queremos agradecer el apoyo recibido por parte de nuestros directores Virginia y Gonzalo, por su paciencia y sus ánimos cuando más lo necesitábamos.

Agradecer también a nuestros amigos, familiares y a nuestras parejas, por estar siempre a nuestro lado, por aguantar nuestros malos días, por asesorarnos para mejorar este trabajo.
No tenemos suficientes palabras de agradecimiento para expresar todo lo que sentimos.

Tenemos muchísima suerte de teneros.
 












\chapter*{Resumen}

Actualmente, en España se calcula que aproximadamente el 1\% de la población sufre algún tipo de discapacidad intelectual. Para estas personas, la realización de tareas tan sencillas y cotidianas como ir a comprar o leer un libro pueden resultar muy complicadas, ya que tienen dificultad de comprensión para comprender muchos de los términos empleados en el día a día. Una posible solución para entender un concepto es la consulta del mismo en un diccionario. Pero esta no es una solución viable para estos colectivos, ya que las definiciones ofrecidas suelen tener un lenguaje complicado y difícil de entender para ellos.

Para ayudar a estas personas con sus problemas de comprensión, se ha implementado una aplicación web que dada una palabra la define en base a otras palabras relacionadas con ella más fáciles y que el usuario conoce. Para dar estas definiciones se emplearon metáforas y símiles. Al tratarse de una aplicación web, está al alcance de cualquier usuario que disponga de un dispositivo con conexión a internet.

Para diseñar la aplicación se ha hecho un diseño centrado en el usuario, contando con las opiniones de expertos. Se quiere conseguir una aplicación que se adapte a las necesidades reales del colectivo objetivo.

La aplicación está compuesta por varios servicios web, para obtener tanto las palabras relacionadas como para generar los símiles y las metáforas. Además, para facilitar la comprensión del usuario, se han implementado varios servicios web complementarios. Uno que obtiene el pictograma asociado a una palabra, y otro que proporciona una descripción y un ejemplo.

La aplicación ha sido evaluada por usuario finales. De esta evaluación se pudo concluir que el trabajo es útil para personas cuyo grado de discapacidad no fuese muy elevado, y que queda trabajo por hacer si queremos que la aplicación sea útil para personas con un grado de discapacidad mayor.


\section*{Palabras clave}
   
\noindent Discapacidad cognitiva\\
\noindent Inclusión\\
\noindent Palabras fáciles\\
\noindent Palabras complejas\\
\noindent Figuras Retóricas\\
\noindent Metáfora\\
\noindent Símil\\
\noindent Servicios Web\\
\noindent Diseño centrado en el usuario\\
\noindent Pictogramas\\

   



\begin{otherlanguage}{english}
\chapter*{Abstract}

Abstract in English.


\section*{Keywords}

\noindent 10 keywords max., separated by commas.




% Si el trabajo se escribe en inglés, comentar esta línea y descomentar
% otra igual que hay justo antes de \end{document}
\end{otherlanguage}

\ifx\generatoc\undefined
\else
%---------------------------------------------------------------------
%
%                          TeXiS_toc.tex
%
%---------------------------------------------------------------------
%
% TeXiS_toc.tex
% Copyright 2009 Marco Antonio Gomez-Martin, Pedro Pablo Gomez-Martin
%
% This file belongs to TeXiS, a LaTeX template for writting
% Thesis and other documents. The complete last TeXiS package can
% be obtained from http://gaia.fdi.ucm.es/projects/texis/
%
% This work may be distributed and/or modified under the
% conditions of the LaTeX Project Public License, either version 1.3
% of this license or (at your option) any later version.
% The latest version of this license is in
%   http://www.latex-project.org/lppl.txt
% and version 1.3 or later is part of all distributions of LaTeX
% version 2005/12/01 or later.
%
% This work has the LPPL maintenance status `maintained'.
% 
% The Current Maintainers of this work are Marco Antonio Gomez-Martin
% and Pedro Pablo Gomez-Martin
%
%---------------------------------------------------------------------
%
% Contiene  los  comandos  para  generar los  índices  del  documento,
% entendiendo por índices las tablas de contenidos.
%
% Genera  el  índice normal  ("tabla  de  contenidos"),  el índice  de
% figuras y el de tablas. También  crea "marcadores" en el caso de que
% se esté compilando con pdflatex para que aparezcan en el PDF.
%
%---------------------------------------------------------------------


% Primero un poquito de configuración...


% Pedimos que inserte todos los epígrafes hasta el nivel \subsection en
% la tabla de contenidos.
\setcounter{tocdepth}{2} 

% Le  pedimos  que nos  numere  todos  los  epígrafes hasta  el  nivel
% \subsubsection en el cuerpo del documento.
\setcounter{secnumdepth}{3} 


% Creamos los diferentes índices.

% Lo primero un  poco de trabajo en los marcadores  del PDF. No quiero
% que  salga una  entrada  por cada  índice  a nivel  0...  si no  que
% aparezca un marcador "Índices", que  tenga dentro los otros tipos de
% índices.  Total, que creamos el marcador "Índices".
% Antes de  la creación  de los índices,  se añaden los  marcadores de
% nivel 1.

\ifpdf
   \pdfbookmark{Índices}{indices}
\fi

% Tabla de contenidos.
%
% La  inclusión  de '\tableofcontents'  significa  que  en la  primera
% pasada  de  LaTeX  se  crea   un  fichero  con  extensión  .toc  con
% información sobre la tabla de contenidos (es conceptualmente similar
% al  .bbl de  BibTeX, creo).  En la  segunda ejecución  de  LaTeX ese
% documento se utiliza para  generar la verdadera página de contenidos
% usando la  información sobre los  capítulos y demás guardadas  en el
% .toc
\ifpdf
   \pdfbookmark[1]{Tabla de Contenidos}{tabla de contenidos}
\fi

\cabeceraEspecial{\'Indice}

\tableofcontents

\newpage 

% Índice de figuras
%
% La idea es semejante que para  el .toc del índice, pero ahora se usa
% extensión .lof (List Of Figures) con la información de las figuras.

\ifpdf
   \pdfbookmark[1]{Índice de figuras}{indice de figuras}
\fi

\cabeceraEspecial{\'Indice de figuras}

\listoffigures

\newpage

% Índice de tablas
% Como antes, pero ahora .lot (List Of Tables)

\ifpdf
   \pdfbookmark[1]{Índice de tablas}{indice de tablas}
\fi

\cabeceraEspecial{\'Indice de tablas}

\listoftables

\newpage

% Índice de listados
% Como antes, pero ahora .lot (List Of Tables)

\ifpdf
\pdfbookmark[1]{Índice de listados}{indice de listados}
\fi

\cabeceraEspecial{\'Indice de listados}

\lstlistoflistings

\newpage


% Variable local para emacs, para  que encuentre el fichero maestro de
% compilación y funcionen mejor algunas teclas rápidas de AucTeX

%%%
%%% Local Variables:
%%% mode: latex
%%% TeX-master: "../Tesis.tex"
%%% End:

\fi

% Marcamos el  comienzo de  los capítulos (para  la numeración  de las
% páginas) y ponemos la cabecera normal
\mainmatter

\pagestyle{fancy}
\restauraCabecera

%%%%%%%%%%%%%%%%%%%%%%%%%%%%%%%%%%%%%%%%%%%%%%%%%%%%%%%%%%%%%%%%%%%%%%%%%%%
% Si el TFM se escribe en ingles, comentar las siguientes líneas 
% porque no hace falta incluir nuevamente la Introducción en inglés
\begin{otherlanguage}{english}
\chapter{Introduction}
\label{cap:introduction}

\section{Motivation}
\label{sec:motivation}

\section{Goals}
\label{sec:goals}

\section{Project management methodology}
\label{sec:project_management}

\section{Memory structure}
\label{sec:memory_structure}













\end{otherlanguage}
\addtocounter{chapter}{-1} 
%%%%%%%%%%%%%%%%%%%%%%%%%%%%%%%%%%%%%%%%%%%%%%%%%%%%%%%%%%%%%%%%%%%%%%%%%%%

\chapter{Introducción}
\label{cap:introduccion}

Actualmente, si pensamos en el día a día de cualquier persona en nuestra sociedad, creemos que no tiene ningún incoveniente para realizar cualquier acción. Es más, podemos pensar que hoy en día existen grandes avances que hacen que la vida resulte más fácil.
Pero esto no es del todo cierto, ya que existen ciertos colectivos como pueden ser inmigrantes, personas con alguna discapacidad cognitiva, ancianos, analfabetos funcionales, etc... que para poder realizar cualquier acción cotidiana les supone un gran esfuerzo o incluso no pueden realizarla y estos avances no están pensados para ellos.

En especial, este trabajo se va a centrar en personas con discapacidad cognitiva y en la sección 1.1 se explicará con mayor detalle los problemas a los que se tienen que enfrentar, que solución queremos aportar y en que ayudaría dicha solución.
Para poder ofrecer la mejor solución posible se necesita desgranar el problema e ir formando dicha solución en función de sus necesidades y limitaciones cumpliendo pequeños objetivos como puede ser la distinción entre palabras fáciles y palabras difíciles, o como obtener los términos relacionados de un concepto específico. En la sección 1.2 se explicarán cuales han sido estos objetivos así como la forma en la que se llevarán a cabo.

Por último, en la sección 1.3 vendrá explicada la estructura de dicho documento explicando en cada capítulo lo que se contará de una manera breve.


%-------------------------------------------------------------------
\section{Motivación}
%-------------------------------------------------------------------
\label{cap:sec:motivacion}

El español, hoy en día es la segunda lengua más hablada del mundo y actualmente más de 90000 palabras forman el castellano. 
Se trata de una lengua con multitud de términos, y que dependiendo del contexto en el que se encuentren, pueden tener múltiples significados. Por ejemplo, la palabra gato puede hacer referencia a un gato de animal o un gato como herramienta para elevar un coche.
Si esto puede suponer una complicación para cualquier persona, para ciertos colectivos de la sociedad afectados por algún trastorno cognitivo lo es aún mucho más, afectándoles en su vida cotidiana, profesional o personal. Por ejemplo, el simple hecho de leer un periódico es una acción bastante difícil para ellos ya que muchas palabras no saben lo que significan. Otro ejemplo podría ser leer un manual de instrucciones de una función más técnica, donde se encuentran en la misma situación de no poder entender ciertos conceptos.

Una de las soluciones que se podrían pensar en un primer momento, es buscar su significado en un diccionario. Pero esto no les sirve, puesto que las definiciones que aparecen en muchos casos no utilizan términos o frases sencillas. Por ejemplo, la palabra computadora tiene la siguiente definición en el diccionario: 
\textit{Máquina electrónica que, mediante determinados programas, permite almacenar y tratar información, y resolver problemas de diversa índole}\footnote{https://dle.rae.es/?id=A4hIGQC}. 
Esta definición puede ser bastante complicada de entender por una persona con discapacidad cognitiva ya que utiliza términos más técnicos y es una frase bastante larga, por lo que una solución que se ha pensado para poder solventar dicho problema, es ofrecer un resultado más sencillo basado en figuras retóricas.

Dentro de las figuras retóricas, se pueden encontrar múltitud de tipos pero para este trabajo se utilizarán únicamente la metáfora, la analogía y el símil.
Por ejemplo, para el mismo concepto (computadora) se podría obtener los siguientes resultados:
\begin{itemize}
	\item Haciendo uso de las metáforas: \textit{Una computadora es un ordenador}. 
	\item Haciendo uso de las analogías: \textit{Una computadora es como una máquina}.
	\item Haciendo uso de los símiles: \textit{Una computadora es fuerte como una piedra}.
\end{itemize}

Gracias al uso de las figuras retóricas, se puede obtener conceptos mucho más sencillos y que facilitan el entendimiento del concepto al usuario. En estas definiciones se utilizan frases cortas y conceptos sencillos de entender. Si a esto le sumamos que se añada un pictograma para la palabra ordenador, máquina y piedra así como al propio concepto de computadora, ayudaría todavía más al fácil entendimiento del concepto.
Por ello, nosotros crearemos una aplicación que dada una palabra compleja devuelva los resultados mediante comparaciones con conceptos más sencillos y haciendo uso de las figuras retóricas.
Esta aplicación, ayudaría a que cualquier palabra que les resulte complicada de entender, puedan buscarla en el momento y así ayudarles en todos los ámbitos de su vida cotidiana. Así, a la hora de poder leer un períodico, un libro o un manual técnico no les resultaría un impedimento para ellos.








%-------------------------------------------------------------------
\section{Objetivos}
%-------------------------------------------------------------------
\label{cap:sec:objetivos}

El objetivo principal de este Trabajo de Fin de Grado es crear una aplicación web basada en servicios que dada una palabra compleja para el usuario devuelva una definición de dicha palabra mediante símiles, analogías o metáforas que empleen palabras más sencillas y conocidas para el usuario. 
Para poder obtener esta definición, habrá que estudiar como obtener los términos relacionados de un concepto, así como distinguir de estos resultados cuales son palabras fáciles y dificiles. 
También habrá que saber que tipos de figuras retóricas existen y cuál utilizar según la relación entre el concepto inicial y los conceptos fáciles, de esta forma se podrá facilitar al usuario final un resultado claro y correcto.

La aplicación estará construida con servicios web que doten de funcionalidad a la aplicación y que sean reutilizables en otras aplicaciones, haciendo así que se puedan adaptar a las distintas necesidades de los usuarios finales.
Los servicios web desarrollados estarán disponibles en una API pública para que todo el mundo pueda utilizarlos y puedan servir para que otros desarrolladores integren nuestros servicios en sus aplicaciones.

La aplicación se construirá de manera incremental, añadiéndo valor al producto poco a poco. De este modo se podrá testear las distintas hipótesis de trabajo poco a poco y realizar modificaciones de una manera simple para así conseguir una aplicación que se adecúe a las necesidades de los usuarios.

Una vez que se disponga de lo comentado anteriormente, se deberá crear una interfaz. Siempre es necesario que el diseño de la interfaz esté centrada en el usuario, y para ello se debe obtener la mayor información posible sobre el usuario final. De esta forma se realiza un diseño basado en sus necesidades y se obtiene una mayor satisfacción de este al utilizar la aplicación, reduce el tiempo de desarrollo, etc...
Como dicho trabajo está enfocado para personas con discapacidad cognitiva, se debe realizar un diseño que se adapte aún más a sus necesidades y limitaciones.

Por último, no se deben olvidar los objetivos académicos de este trabajo, como puede ser poner en práctica los conocimientos adquiridos durante el Grado y ampliar nuestros conocimientos en distintas áreas. 

Alcanzando los objetivos anteriormente descritos, se conseguirá obtener un producto de calidad, con una gran utilidad tanto social como académica, que pueda ayudar a muchos usuarios a aprender ciertos conceptos de nuestro idioma de una manera más sencilla.
	
	
%-------------------------------------------------------------------
\section{Estructura de la memoria}
%-------------------------------------------------------------------
\label{cap:sec:estructuramemoria}


En el \textbf{capítulo dos} se presenta el Estado de la Cuestión, en el que se explicará que es la Lectura Fácil y como se aplica y se introducirán los conceptos de Procesamiento del Lenguaje Natural (PLN) y algunas herramientas que sirven para PLN, además se hablará de figuras retóricas y servicios web, en especial qué son, su arquitectura y las ventajas y desventajas de su uso.


En el \textbf{capítulo tres} se explicarán las herramientas utilizadas para la creación de este trabajo, como pueden ser Django para el desarrollo de la aplicación y SpaCy para el etiquetado de palabras, donde se explicará que son ambas herramientas, para que se utilizan y sus características principales.

En el \textbf{capitulo cuatro} queda detallado como ha sido la gestión del proyecto, que herramientas se han utilizado para la asignación de tareas, como se ha realizado la distinción de si una tarea hace referencia al código o a la memoria y el tratamiento de cada una de ellas, las reuniones establecidas con los directores así como la utilización de un repositorio para el control de versiones de dicho trabajo.

En el \textbf{capitulo cinco} se explicarán los Servicios Web implementados por los integrantes del trabajo para dotar de funcionalidad a la aplicación.

El \textbf{capítulo seis} está enfocado en el diseño de la aplicación, se explicará el proceso de diseño de la interfaz de la aplicación, desde la creación de los primeros bocetos hasta la implementación del diseño final pasando por la evaluación de los prototipos por parte de los expertos.
 
En el \textbf{capítulo siete} se describe el trabajo realizado por cada uno de los autores de dicho trabajo.

\chapter{Estado de la Cuestión}
\label{cap:estadoDeLaCuestion}



Actualmente, los servicios web son una tecnología utilizada a diario por la mayoría de la población ya sea para la traducción de texto o la consulta del tiempo.

Debido a esto, para el desarrollo de la aplicación utilizaremos esta tecnología. Además, las definiciones generadas deben de ser símiles, metáforas o analogías ya que mediante estas figuras retóricas, se puede comparar el término que se quiere definir con otro concepto más cotidiano para facilitar su comprensión. Estas definiciones deben de estar en un formato que sea comprensible por el mayor número de gente posible para ello se empleará una adaptación llamada ``lectura fácil'' que tiene como objetivo que los texto estén en un formato fácil de leer. Para crear estas definiciones se utilizará Inteligencia Artificial para buscar términos relacionados o sinónimos más sencillos de la palabra que se quiere definir, concretamente la rama del Procesamiento del Lenguaje Natural.

%-------------------------------------------------------------------
\section{Lectura Fácil}
%-------------------------------------------------------------------
\label{cap:sec:lecturafacil}

Se llama lectura fácil a aquellos contenidos que han sido resumidos y reescritos con lenguaje sencillo y claro, de forma que puedan ser entendidos por personas con discapacidad cognitiva o discapacidad intelectual. Es decir, es la adaptación de textos, ilustraciones y maquetaciones que permite una mejor lectura y comprensión.
Este trabajo se va a centrar en la lectura fácil aplicada a textos.

La lectura fácil surgió en Suecia en el año 1968, donde se editó el primer libro en la Agencia de Educación en el marco de un proyecto experimental. A continuación, en 1976, se creó en el Ministerio de Justicia un grupo de trabajo para conseguir textos legales más claros.
En 1984 nació el primer periódico en lectura fácil, titulado "8 páginas", que tres años más tarde, en 1987, se publicó de forma permanente en papel hasta que empezó a editarse en la web. 
En el año 2013, en México se produce la primera sentencia judicial en lectura fácil\footnote{https://dilofacil.wordpress.com/2013/12/04/el-origen-de-la-lectura-facil/}. En la actualidad, podemos distinguir los documentos en lectura fácil gracias al logo de la Figura \ref{fig:lecturafacil}.

	\figura{Bitmap/Capitulo2/lecturaFacil}{width=.3\textwidth}{fig:lecturafacil}{Logo Lectura Fácil} 
	
	
Los documentos escritos en Lectura Fácil \citep{wiki:lecturafacil} son documentos de todo tipo que siguen las directrices internacionales de la IFLA\footnote{International Federation of Library Associations and Institutions} y de la Inclusion Europe\footnote{Una asociación de personas con discapacidad intelectual y sus familias en Europa} en cuanto al contenido y la forma.
Algunas pautas a seguir para escribir correctamente un texto en Lectura Fácil son \citep{GarciaMunoz2012LecturaFacil}:


\begin{itemize}
	\item Evitar mayúsculas fuera de la norma, es decir, escribir en mayúsculas sólo cuando lo dicten las reglas ortográficas, como por ejemplo, después de un punto o la primera letra de los nombres propios.
	\item Deben evitarse el punto y seguido, el punto y coma y los puntos suspensivos. El punto y aparte hará la función del punto y seguido.
	\item Evitar corchetes y signos ortográficos poco habituales, como por ejemplo: \%, \& y /.
	\item Evitar frases superiores a 60 carácteres y utilizar oraciones simples. Por ejemplo, la oración \textit{Caperucita ha ido a casa de su abuela y ha desayunado con ella} es mejor dividirla en dos oraciones simples:\textit{ Caperucita ha ido a casa de su abuela} y  \textit{Caperucita ha desayunado con ella}.
	\item Evitar tiempos verbales como: futuro, subjuntivo, condicional y formas compuestas.
	\item Utilizar palabras cortas y de sílabas poco complejas. 
	Por ejemplo: casa, gato, comer o mano.
	\item Evitar abreviaturas, acrónimos y siglas.
	\item Alinear el texto a la izquierda.
	\item Incluir imágenes y pictogramas a la izquierda y su texto vinculado a la derecha.
	\item Evitar la saturación de texto e imágenes.
	\item Utilizar uno o dos tipos de letra como mucho.
	\item Tamaño de letra entre 12 y 16 puntos.
	\item Si el documento está paginado, incluir la paginación claramente y reforzar el mensaje de que la información continúa en la página siguiente.
\end{itemize}

Debemos también hacer hincapié en la distinción entre palabras fáciles y complejas, puesto que son de gran importancia para la lectura fácil. 
Las palabras complejas son aquellas que no se utilizan a menudo en nuestra sociedad, como por ejemplo: melifluo o inefable. Es por ello que este tipo de palabras deben estar totalmente descartadas en la lectura fácil, y en su lugar debemos introducir palabras fáciles, que son aquellas que se utilizan asiduamente. La RAE dispone de un documento con las mil palabras más usadas\footnote{http://corpus.rae.es/lfrecuencias.html}.

%-------------------------------------------------------------------



%-------------------------------------------------------------------
\section{Figuras retóricas}
%-------------------------------------------------------------------
\label{cap:sec:figurasretoricas}

Las figuras literarias (o retóricas) son formas no convencionales de utilizar las palabras, de manera que, aunque se emplean con sus acepciones habituales, se acompañan de algunas particularidades fónicas, gramaticales o semánticas, que las alejan de ese uso habitual, por lo que terminan por resultar especialmente expresivas. 
La metáfora, el símil y la analogía se basan en la comparación de dos conceptos: de origen (o tenor), que es el término literal (al que la metáfora se refiere) y el de destino (o vehículo), que es el término figurado. La relación que hay entre el tenor y el vehículo se denomina fundamento. Por ejemplo, en la metáfora \textit{Tus ojos son dos luceros}, \textit{ojos} es el tenor, \textit{luceros} es el vehículo y el fundamento es la belleza de los ojos \citep{GalianaYCasas1994}.


En este trabajo vamos a trabajar con tres tipos de figuras retóricas \citep{TFMPaloma2017}: 
\begin{itemize}
	\item Metáfora: Utiliza el desplazamiento de características similares entre dos conceptos con fines estéticos o retóricos. Por ejemplo, cuando una tarea es muy fácil de realizar, se dice que es un regalo.
	
	\item Símil: Realiza una comparación entre dos términos usando conectores (por ejemplo, como, cual, que, o verbos).
	Por ejemplo, cuando nos referimos a una persona que es muy corpulenta, se dice: ``es como un oso'', ya que los osos son muy grandes.
	
	\item Analogía: Es la comparación entre varios conceptos, indicando las características que permiten dicha relación. En la retórica, una analogía es una comparación textual que resalta alguna de las similitudes semánticas entre los conceptos protagonistas de dicha comparación. Por ejemplo: \textit{Sus ojos son azules como el mar}.
	
	
	
\end{itemize}

%-------------------------------------------------------------------
\section{Servicios Web}
%-------------------------------------------------------------------
\label{cap:sec:serviciosweb}

Para definir el concepto de servicio Web de la forma más simple posible, se podría decir que es una tecnología que utiliza un conjunto de protocolos para intercambiar datos entre aplicaciones, sin importar el lenguaje de programación en el cual estén programadas o ejecutadas en cualquier tipo de plataforma \citep{wiki:w3c2004}. Según el W3C(\textit{World Wide Web Consortium})\footnote{https://www.w3.org/}, un servicio web es un sistema software diseñado para soportar la interacción máquina-a-máquina, a través de una red, de forma interoperable.




Las principales características de un servicio web son \citep{TorresJoaquin2017SC}:



\begin{itemize}
	\item Accesible a través de la Web. Para ello debe utilizar protocolos de transporte estándares como HTTP, y codificar los mensajes en un lenguaje estándar que pueda ser accesible por cualquier cliente que quiera utilizar el servicio. 
	\item Contiene una descripción de sí mismo. De esta forma, una aplicación web podrá saber cual es la función de un determinado Servicio Web, y cuál es su interfaz, de manera que pueda ser utilizado de forma automática por cualquier aplicación, sin la intervención del usuario.
	\item Debe ser localizado. Deberemos tener algún mecanismo que nos permita encontrar un Servicio Web que realice una determinada función. De esta forma tendremos la posibilidad de que una aplicación localice el servicio que necesite de forma automática, sin tener que conocerlo previamente el usuario.
\end{itemize}

Por otro lado, los servicios web pueden definirse tanto a nivel conceptual como a nivel técnico, es por ello que mediante este último podemos diferenciar dos tipos distintos de servicio web \citep{TorresJoaquin2017SC}:
\begin{itemize}
	\item Servicios web SOAP  \textit({Simple Object Access Protocol}): es un protocolo basado en XML para el intercambio de información entre ordenadores. Normalmente utilizaremos SOAP para conectarnos a un servicio e invocar métodos remotos 
	\footnote{https://www.ibm.com/support/knowledgecenter/es}. Los mensajes SOAP tienen el siguiente formato representado en la figura \ref{fig:SOAP}:

	
	\begin{itemize}
		\item <Envelope>: Elemento raíz de cada mensaje SOAP y contiene dos elementos: <Header> que es opcional y <Body> que es obligatorio, ambos elementos los describiremos a continuación.
		\item <Header>: Es un elemento que se utiliza para indicar información acerca de los mensajes SOAP.
		\item <Body>: Elemento que contiene información dirigida al destinatario del mensaje.
		\item <Fault>: Elemento en el que se notifican los errores.
			
	\end{itemize}

	\item Servicios Web RESTful: es un protocolo que suele integrar mejor con HTTP que los servicios basado en SOAP, ya que no requieren mensajes XML. Cada petición del cliente debe contener toda la información necesaria para entender la petición, y no puede aprovecharse de ningún contexto almacenado en el servidor.
	
\end{itemize}
\figura{Bitmap/Capitulo2/SOAP}{width=.9\textwidth}{fig:SOAP}{Estructura de un mensaje SOAP}


\subsection{Arquitectura Servicios Web}
\label{cap:subsec:arquitecturaserviciosweb}

Hay que distinguir tres partes fundamentales en los servicios web \citep{TorresJoaquin2017SC}:
\begin{itemize}
	\item El proveedor: Es la aplicación que implementa el servicio y lo hace accesible desde Internet.
	\item El solicitante: Cualquier cliente que necesite utilizar el servicio web.
	\item El publicador: Se refiere al repositorio centralizado en el que se encuentra la información de la funcionalidad disponible y como se utiliza.
	
\end{itemize}
Por otro lado, los servicios web se componen de varias capas \footnote{https://diego.com.es/introduccion-a-los-web-services}:
\begin{itemize}
	\item Service Discovery: Responsable de centralizar los servicios web en un directorio común, de esta forma es más sencillo buscar y publicar.
	\item Service Description: Como ya hemos comentado con anterioridad, los servicios web se pueden definir así mismos, por lo que una vez que los localicemos el Service Description nos dará la información para saber que operaciones soporta y como activarlo.
	\item Service Invocatio: Invocar a un Web Service implica pasar mensajes entre el cliente y el servidor. Por ejemplo, si utilizamos SOAP  \textit({Simple Object Access Protocol}), el Service Invocation especifica cómo deberíamos formatear los mensajes request para el servidor, y cómo el servidor debería formatear sus mensajes de respuesta.
	\item Transport: Todos los mensajes han de ser transmitidos de alguna forma entre el servidor y el cliente. El protocolo elegido para ello es HTTP  \textit({(HyperText Transfer Protocol)}). 
\end{itemize}



\subsection{Ventajas de los  Servicios Web}
\label{cap:subsec:ventajasserviciosweb}

	Las principales ventajas del uso de los servicios web son las siguientes \citep{doctorado2005}:
\begin{itemize}
	\item Permiten la integración “justo-a-tiempo”:  Esto significa que los solicitantes, los proveedores y los agentes actúan en conjunto para crear sistemas que son auto-configurables, adaptativos y robustos.
	\item Reducen la complejidad por medio del encapsulamiento: Un solicitante de servicio no sabe cómo fue implementado el servicio por parte del proveedor, y éste a su vez, no sabe cómo utiliza el cliente el servicio. Estos detalles se encapsulan en los solicitantes y proveedores. El encapsulamiento es crucial para reducir la complejidad.
	\item Promueven la interoperabilidad: La interacción entre un proveedor y un solicitante de servicio está diseñada para que sea completamente independiente de la plataforma y el lenguaje. 
	\item Abren la puerta a nuevas oportunidades de negocio: Los servicios web facilitan la interacción con socios de negocios, al poder compartir servicios internos con un alto grado de integración.
	\item Disminuyen el tiempo de desarrollo de las aplicaciones: Gracias a la filosofía de orientación a objetos que utilizan, el desarrollo se convierte más bien en una labor de composición.
	\item Fomentan los estándares y protocolos basados en texto, que hacen más fácil acceder a su contenido y entender su funcionamiento.
\end{itemize}


\subsection{Desventajas de los  Servicios Web}
\label{cap:subsec:desventajasserviciosweb}
	El uso de servicios web también tiene algunas desventajas \footnote{http://fabioalfarocc.blogspot.com/2012/08/ventajas-y-desventajas-del-soap.html}:
\begin{itemize}
	\item Al apoyarse en HTTP, pueden esquivar medidas de seguridad basadas en firewall cuyas reglas tratan de bloquear.
	\item Existe poca información de servicios web para algunos lenguajes de programación.
	\item Dependen de la disponibilidad de servidores y comunicaciones.
\end{itemize}

\section{Procesamiento del Lenguaje Natural}
%-------------------------------------------------------------------
\label{cap:sec:lenguajenatural}
El Procesamiento del Lenguaje Natural (PLN) es una rama de la Inteligencia Artificial que se encarga de investigar la manera de comunicar máquinas con personas mediante el uso del lenguaje natural (entendiendo como lenguaje natural el idioma usado con fines de comunicación por humanos, ya sea hablado o escrito, como pueden ser el español, el ruso o el inglés). 
El Procesamiento del Lenguaje Natural se ayuda de las redes semánticas, puesto que estas son una forma de representación del conocimiento en la que los conceptos que componen el mundo y sus relaciones se representan mediante un grafo. Se utilizan para representar mapas conceptuales y mentales \citep{wiki:redSemantica2018}.
Los nodos están representados por el elemento lingüístico, y  la relación entre los nodos sería la arista. Podemos ver un ejemplo en la Figura \ref{fig:red}.
\figura{Bitmap/Capitulo2/redSemantica}{width=.9\textwidth}{fig:red}{Ejemplo Red Semántica}
Existen principalmente tres tipos de redes semánticas\footnote{http://elies.rediris.es/elies9/4-3-2.htm}:
\begin{itemize}
	\item Redes de Marcos: los enlaces de unión de los nodos son parte del propio nodo, es decir, se encuentran organizados jerárquicamente, según un número de criterios estrictos, como por ejemplo la similitud entre nodos.
	\item Redes IS-A: los enlaces entre los nodos están etiquetados con una relación entre ambos. Es el tipo que habitualmente se utiliza junto con las Redes de Marcos.
	\item Grafos Conceptuales: existen dos tipos de nodos: nodos de conceptos, los cuáles representan una entidad, un estado o un proceso y los nodos de relaciones, que indican como se relacionan los nodos de concepto. En este tipo de red semántica no existen enlaces entre los nodos con una etiqueta, sino que son los propios nodos los que tienen el significado. 
\end{itemize}

Para el trabajo que queremos realizar, existen varias aplicaciones web que actúan como redes semánticas y son capaces de procesar el Lenguaje Natural.  A continuación, hablaremos de algunas de ellas.

\subsection{ConceptNet} 
\label{cap:subsec:concepnet}

Es una red semántica creada por el MIT \textit{(Massachusetts Institute of Technology)} en 1999, diseñada para ayudar a los ordenadores a entender el significado de las palabras. Está disponible en múltiples idiomas, como el español, el inglés o el chino. ConceptNet dispone de una aplicación web\footnote{http://conceptnet.io/}, donde seleccionas el idioma deseado y añades la palabra a buscar. En la Figura  \ref{fig:busquedaConcepnet}, podemos ver que si añadimos la palabra chaqueta nos devuelve sus sinónimos, términos relacionados, símbolos, etc...
\figura{Bitmap/Capitulo2/busquedaConcepnet}{width=.9\textwidth}{fig:busquedaConcepnet}{Resultados de ConcepNet para la palabra chaqueta} 
\figura{Bitmap/Capitulo2/json}{width=.9\textwidth}{fig:json}{JSON devuelto por la API de ConceptNet para la palabra chaqueta}
Por otro lado, ConcepNet dispone de un servicio web\footnote{http://api.conceptnet.io} que devuelve los resultados en formato JSON. Siguiendo con el mismo ejemplo anterior, podemos ver en la Figura  \ref{fig:json} los resultados en dicho formato para la palabra chaqueta. Este consta de cuatro campos\footnote{principales https://github.com/commonsense/conceptnet5/wiki/AP}:
\begin{itemize}
	\item @context: URL enlazada a un archivo de información del JSON para comprender la API. También puede contener comentarios que pueden ser útiles para el usuario.
	\item @id: concepto que se ha buscado y su idioma. En nuestro caso, aparece de la siguiente manera: \textit{/c/es/chaqueta}, donde  \textit{c} significa que es un concepto o término,  \textit{es} indica el lenguaje, en este caso, el español y por último \textit{chaqueta} que es la palabra buscada.
	\item edges: representa una estructura de datos devueltos por Conceptnet compuesta por:
	\begin{itemize}
		\item @id: describe el tipo de relación que existe entre la palabra introducida y la devuelta, en nuestro caso \textit{/a/[/r/Synonym/,/c/es/chaqueta/n/,/c/es/americana/]}, nos indica que la palabra \textit{americana} es un sinónimo de \textit{chaqueta}.
		\item @type: define el tipo del id, es decir, si es una relación (edge) o un término (nodo).
		\item dataset: URI que representa el conjunto de datos creado.
		\item edges: representa una estructura de datos devueltos por ConceptNet compuesta por:
		\item end: nodo destino, que a su vez se compone de:	
		\begin{itemize}
			\item @id: coincide con la palabra del id anterior.
			\item @type: define el tipo de id, como se ha explicado anteriormente.
			\item label: frase más completa, donde adquiera significado la palabra obtenida.
			\item language: lenguaje en el que está la palabra devuelta de la consulta.
			\item term: enlace a una versión mas general del propio término. Normalmente, suele coincidir con la URI.			
		\end{itemize}
		\item license: aporta información sobre como debe usarse la información proporcionada por conceptnet.
		\item rel: describe la relación que hay entre la palabra origen y destino, dentro del cual hay tres campos: @id, @type y label, descritos anteriormente.
		\item sources: indica por qué ConceptNet guarda esa información, este campo como los anteriores, es un objeto que tiene su propio id y un campo @type, A parte, hay un campo \textit{contributor}, en el que aparece la fuente por la que se ha obtenido ese resultado y por último un campo \textit{process} indicando si la palabra se ha añadido mediante un proceso automático.
		\item start: describe el nodo origen, es decir, la palabra que hemos introducido en ConceptNet para que haga la consulta, este campo esta compuesto por elementos ya descritos como son: @id, @type, label, language y term.
		\item surfaceText: indica de que frase del lenguaje natural se han extraido los datos que estan guardados en conceptnet. En nuestro caso es nulo.
		\item weight: indica la fiabilidad de la información guardada en conceptnet, siendo normal que su valor sea 1.0. Cuanto mayor sea este valor, más fiables serán.
	\end{itemize}
	\item view: describe la longitud de la lista de paginación, es un objeto con un id propio, y además, aparecen los campos \textit{firstPage} que tiene como valor un enlace a la primera pagina de los resultados obtenidos, y \textit{nextPage} que tiene un enlace a la siguiente página de la lista.
\end{itemize}


\subsection{Thesaurus}
\label{cap:subsec:thesaurus}
Es una aplicación web\footnote{https://www.thesaurus.com/} que se autodefine como principal diccionario de sinónimos de la web. Su nombre proviene de la palabra griega Tesauro, cuyo significado es \textit{Trabajo de referencia que enumera las palabras agrupadas según la similitud del significado}. Esta página ofrece la posibilidad de introducir una palabra para poder conocer sus sinónimos, pero solamente devuelve resultados en inglés. Aparte del listado de sinónimos,  Thesaurus te dice que tipo de palabra es y una definición de la misma.
Esta aplicación proporciona una API tipo RESTful\footnote{http://thesaurus.altervista.org/} que obtiene los sinónimos de una palabra y devuelve los resultados en formato XML o JSON. El contenido de la respuesta es una lista y cada elemento de esta lista contiene un par de elementos: categoría y sinónimos. Este último a su vez contiene una lista de sinónimos separados por el carácter |. 
Podemos ver en la Figura \ref{fig:xmlthesaurus} un ejemplo de como sería el resultado de una petición en formato XML:


\definecolor{gray}{rgb}{0.4,0.4,0.4}
\definecolor{darkblue}{rgb}{0.0,0.0,0.6}
\definecolor{cyan}{rgb}{0.0,0.6,0.6}

\lstset{
	basicstyle=\ttfamily,
	columns=fullflexible,
	showstringspaces=false,
	commentstyle=\color{gray}\upshape
}

\lstdefinelanguage{XML}
{
	morestring=[b]",
	morestring=[s]{>}{<},
	morecomment=[s]{<?}{?>},
	stringstyle=\color{black},
	identifierstyle=\color{darkblue},
	keywordstyle=\color{cyan},
	morekeywords={xmlns,version,type}% list your attributes here
}


 y en la Figura \ref{fig:jsonthesaurus} en formato JSON.
\figura{Bitmap/Capitulo2/xmlthesaurus}{width=.9\textwidth}{fig:xmlthesaurus}{Ejemplo de salida de Thesaurus en formato XML}
\figura{Bitmap/Capitulo2/jsonthesaurus}{width=.9\textwidth}{fig:jsonthesaurus}{Ejemplo de salida de Thesaurus en formato JSON}




\lstset{language=XML}
\begin{lstlisting}
<response> 
<list>
	<category>(noun)</category> 
	<synonyms>order | war (antonym) </synonyms>
 </list>
 <list>
 	<category>(noun)</category> 
 	<synonyms> harmony | concord | concordance </synonyms>
 </list>
 <list>
 	<category>(noun)</category> 
 	<synonyms>public security | security </synonyms>
 </list>
 <list>
 	<category>(noun)</category> 
 	<synonyms> tratado de paz | pacificacion | acuerdo | pact | accord </synonyms>
 </list>
\end{lstlisting}

\subsection{Thesaurus Rex}
\label{cap:subsec:thesaurusrex}

Thesaurus Rex\footnote{http://ngrams.ucd.ie/therex3/} es una red semántica. La aplicación permite introducir una palabra y devuelve palabras relacionados, o bien puedes introducir un par de palabras (separadas por el operador  \& ) y devolverá las categorías que comparten ambos términos.
Thesaurus Rex solo admite palabras en inglés. En la Figura \ref{fig:thesaurusrex} podemos ver los resultados para la palabra  \textit{house}.
La aplicación devuelve estos resultados en formato XML, y este como se puede ver en la Figura \ref{fig:xmlthesaurusrex} los divide en distintos campos: \textit{Categories}, \textit{Modifiers} y \textit{CategoryHeads}.


\figura{Bitmap/Capitulo2/thesaurusrex}{width=.9\textwidth}{fig:thesaurusrex}{Resultados búsqueda Thesaurus Rex con la palabra \textit{house}}
\figura{Bitmap/Capitulo2/xmlthesaurusrex}{width=.9\textwidth}{fig:xmlthesaurusrex}{Ejemplos formato XML Thesaurus Rex}

\subsection{Metaphor Magnet}
\label{cap:subsec:metaphormagnet}
Otra aplicación web \footnote{http://ngrams.ucd.ie/metaphor-magnet-acl/}, en la cuál introduces una palabra pero esta vez devuelve metáforas. El objetivo \citep{VealeT2012} de dicha aplicación es encontrar metáforas comunes en los n-gramas de Google y utilizar estos mapeos para interpretar metáforas, pero esta aplicación está limitada ya que solamente se puede buscar palalabras en inglés como podemos observar en la imagen \ref{fig:metaphormagnet} introduciendo la palabra  \textit{house}.
Esta consulta devuelve un fichero XML como el expuesto en la figura \ref{fig:xmlmetaphormagnet} que puede ser utilizado para otras aplicaciones de Procesamiento de Lenguaje Natural.
\figura{Bitmap/Capitulo2/metaphormagnet}{width=.9\textwidth}{fig:metaphormagnet}{Resultados búsqueda Metaphor Magnet con la palabra \textit{house}}
\figura{Bitmap/Capitulo2/xmlmetaphormagnet}{width=.9\textwidth}{fig:xmlmetaphormagnet}{Ejemplos formato XML Metaphor Magnet}



\subsection{Wordnet}
\label{cap:subsec:wordnet}
Es una base de datos léxica solamente en inglés donde agrupa las palabras en función de su significado \citep{wordnet2010}. Utiliza distintos tipos de palabras tales como sustantivos, verbos, adjetivos y adverbios ignorando preposiciones, determinantes y otras palabras funcionales. Agrupando estos tipos en conjuntos de sinónimos cognitivos o también llamados \textit{synsets}. Estos, aparte contienen una breve definición y en muchas ocasiones oraciones cortas que explican el significado. En el caso de que la palabra tenga distintos significados se representará en tantos \textit{synsets} como significados haya. Podemos ver un ejemplo en la figura \ref{fig:wordnet} realizando una búsqueda con la palabra \textit{house}.
WordNet dispone de una API HTTP donde permite buscar información en la base de datos de WordNet mediante métodos GET y POST a través de la URL raíz \textit{/graph/v1}.
\figura{Bitmap/Capitulo2/wordnet}{width=.9\textwidth}{fig:wordnet}{Resultados búsqueda Wordnet con la palabra \textit{house}}


\subsection{EuroWordnet}
\label{cap:subsec:eurowordnet}
Es una base de datos multilingüe que se basa en relaciones semánticas básicas entre palabras. Es muy parecido a WordNet pero en vez de usar \textit{synsets} ellos lo llaman \textit{WordNet}. Cada \textit{WordNet} representa un sistema único de lexicalizaión y ádemas estos están conectados con el Índice Inter-Lingual (ILI) \citep{wiki:eurowordnet2017}.
Podemos ver en la figura \ref{fig:eurowordnet} un ejemplo de búsqueda con la palabra  \textit{casa}, donde la aplicación devuelve los sinónimos, una pequeña definición así como un ejemplo para que el usuario pueda entender mejor el concepto.
\figura{Bitmap/Capitulo2/eurowordnet}{width=.9\textwidth}{fig:eurowordnet}{Resultados búsqueda EuroWordnet con la palabra \textit{casa}}

\section{Conclusiones}



\chapter{Herramientas Utilizadas}
\label{cap:herramientas}

En este capítulo se van a explicar las herramientas utilizadas para el desarrollo de este trabajo. En el apartado 3.1 se explicará Django que es el \textit{framework} utilizado para el desarrollo de la aplicación y en el apartado 3.2 se explicará SpaCy, que es la herramienta que se ha utilizado para la clasificación semántica de las palabras fáciles. Las herramientas y recursos que se han usado para la gestión del proyecto se explicarán en un capítulo aparte.


%-------------------------------------------------------------------
\section{Django}
%-------------------------------------------------------------------
\label{cap:sec:django}
Para construir un servicio web se necesita una manera de gestionar los elementos propios de estos servicios, como pueden ser el procesamiento de formularios y el mapeo de urls, para satisfacer estas necesidades, se requieren \textit{frameworks} web que son estructuras que contienen los componentes necesarios para el desarrollo de aplicaciones web \footnote{https://tutorial.djangogirls.org/es/django/}.
Django es un \textit{framework} de alto nivel que permite el desarrollo rápido de sitios web seguros y mantenibles y se basa en el patrón MVC\footnote{https://docs.djangoproject.com/en/2.0/}. Fue desarrollado entre los años 2003 y 2005 por un grupo de programadores que se encargaban de crear y mantener sitios web de periódicos. 
Es gratuito y de código abierto y dispone de una gran documentación actualizada así como muchas opciones de soporte gratuito y de pago. 

Algunas de las razones por las que se ha elegido este \textit{framework} han sido las siguientes\footnote{https://openwebinars.net/blog/que-es-django-y-por-que-usarlo/}:

\begin{itemize}
	\item Seguridad: Implementa por defecto algunas medidas de seguridad para evitar SQL Injection(adición de consultas SQL malignas que puedan alterar la base de datos) o Clickjacking(Que el usuario haga click en un enlace oculto, para obtener información del mismo sin su permiso o incluso tomar el control de su ordenador) por JavaScript.
	
	\item Escalabilidad: Se puede pasar de una aplicación sencilla a otra más compleja rápidamente, ya que es muy fácil añadir nuevos módulos al \textit{framework}.
	
	\item Fácil acceso a bases de datos: Mediante ORM(Object Relational Mapper), que es una biblioteca para el acceso de datos, generando clases a partir de las tablas de la base de datos para poder realizar las consultas. Django tiene su propio ORM, con el que se pueden hacer consultas de manera muy intuitiva.
	
	\item Además, es muy popular por lo que para resolver cualquier problema que surja, como se ha explicado anteriormente, hay mucha documentación disponible y muchos hilos en foros de programación en donde encontrar posibles soluciones.
	
\end{itemize}



\section{SpaCy}
\label{cap:sec:spacy}
Cuando se obtuvieron las palabras fáciles de la RAE solo se necesitaban adverbios, nombres, verbos y adjetivos, por lo que se requería un análisis semántico de cada una de ellas para obtener solo las palabras que pertenecían a uno de estos grupos. Para ello se utilizó SpaCy. 
SpaCy\footnote{https://spacy.io/} es una biblioteca de código abierto para el Procesamiento del Lenguaje Natural en Python. Soporta más de 34 idiomas, entre ellos el español.



\figura{Bitmap/Capitulo3/spacy}{width=.8\textwidth}{fig:spacy}{Ejemplo de clasificación de palabras}

El analizador sintáctico de Spacy, según un estudio realizado en 2015 por la universidad Emory\footnote{https://www.aclweb.org/anthology/P15-1038}, es el más rápido y según Medium Corporation tiene un índice de acierto mucho mayor que el analizador sintáctico de NLTK \footnote{https://medium.com/@pemagrg/private-nltk-vs-spacy-3926b3674ee4}. Para utilizar en analizador, hay que importar la biblioteca de idioma correspondiente (en este caso español: ``es\_core\_news\_sm'') y pasar como parámetro las palabras que se deseen clasificar. El resultado para cada palabra introducida en el analizador, serán una serie de etiquetas tal y como se pueden apreciar en la tabla \ref{fig:spacy}, que es un ejemplo de las etiquetas generadas para una frase en lengua inglesa y que se explica a continuación:
\begin{itemize}
	\item La columna \textit{text} indica cual es la palabra que se ha procesado. En el caso de la primera fila es la palabra \textit{Apple}.
	\item La columna \textit{lemma} indica la forma base de la palabra procesada. En la segunda fila, \textit{be} indica el verbo en infinitivo que se ha analizado.
	\item La columna \textit{pos} es la etiqueta asignada a dicha palabra. Indicando lo siguiente:
		\begin{itemize}
			\item PROPN: Nombre propio. Por ejemplo \textit{Apple}
			\item VERB: Verbo. Como \textit{is} que es una forma del verbo \textit{to be}.
			\item ADP: Preposicion. Por ejemplo \textit{at}.
		\end{itemize}
	
	\item La columna \textit{tag} indica cual es la palabra que se ha procesado con más detalle.
	\item La columna \textit{dep} indica la dependencia sintáctica de la palabra en la frase. Por ejemplo en la segunda frase \textit{is} es un verbo auxiliar.
	\item La columna \textit{shape} indica la apariencia de la palabra procesada, es decir, si está en mayúsculas o si tiene algún signo de puntuación. Por ejemplo en la última columna, la palabra \textit{U.K.} tiene un \textit{shape} X.X. ya que está formado por dos letras mayúsculas y dos puntos.
	\item La columna \textit{alpha} es un valor booleano que tendrá el valor \textit{True} si la palabra es un carácter alfanumérico y \textit{False} si no lo és. Por ejemplo: \textit{buying} tiene la etiqueta \textit{alpha} a \textit{True} ya que está formada por caracteres alfanuméricos y \textit{U.K.} la tiene a \textit{False} ya que los ``.'' no se consideran caracteres alfanuméricos.
	\item La columna \textit{stop} es un valor booleano que tendrá el valor \textit{True} si la palabra forma parte de las palabras más comunes del lenguaje en el que se encuentra \textit{False} si no lo és. Por ejemplo: \textit{is} que es una forma del verbo \textit{to be} tiene esta etiqueta a \textit{True} ya que es muy utilizado en el lenguaje inglés.
\end{itemize}

Se ha elegido ya que ha sido el que mejor resultado ha dado. Inicialmente se utilizó el POS-tagger (clasificador sintáctico de palabras) de NLTK, pero su índice de acierto no era muy bueno, por lo que se descartó su uso. A continuación se probó con SpaCy y ,además de ser más rápido, su índice de acierto ha sido prácticamente del 100\% ,por lo que fue lo finalmente utilizado.


\chapter{Aprende Fácil}
\label{cap:aprendeFacil}




%-------------------------------------------------------------------
\section{Diseño de la Interfaz}
%-------------------------------------------------------------------
\label{cap:sec:disenioInterfaz}
Como ya se ha explicado en capítulos anteriores, está aplicación está creada por y para personas que tienen alguna discapacidad cognitiva, por lo que el diseño de la Interfaz debe estar centrado para este tipo de usuarios, es decir, debe ser lo más sencilla posible, su uso no debe llevar a confusión ni debe hacer sentir al usuario frustrado al usarla. En definitiva, debe cumplir con las Ocho Reglas de Oro del diseño de interfaces, pero simplificándola aún más de lo que podría ser un diseño para otro tipo de aplicación destinada para otro tipo de usuario. 
Las Ocho Reglas de Oro del diseño de interfaces consisten en que el diseño tenga:
\begin{itemize} 
	\item Consistencia: La funcionalidad debe ser similar a otras aplicaciones que el usuario está acostumbrado a utilizar. En cuanto a la interfaz debe tener los mismos colores, iconos, formas, botones, mensajes de aviso... Por ejemplo, si el usuario está acostumbrado a que el botón de eliminar o cancelar sea rojo, no debemos añadirle uno de color verde.
	\item Usabilidad Universal: Debemos tener en cuenta las necesidades de los distintos tipos de usuario, como por ejemplo, atajos de teclado para un usuario experto o filtros de color para usuarios con deficiencias visuales.
	\item Retroalimentación activa: Por cada acción debe existir una retroalimentación legible y razonable por parte de la aplicación. Por ejemplo, si el usuario quiere guardar los datos obtenidos de la búsqueda, la aplicación debe informarle de si han sido guardados o no.
	\item Diálogos para conducir la finalización: El usuario debe saber en que paso se encuentra en cada momento. Por ejemplo, en un proceso de compra que conlleva varios pasos hasta la finalización de la misma, se le debe informar donde se encuentra y cuánto le queda para terminar.
	\item Prevención de errores: La interfaz debe ayudar al usuario a no cometer errores serios, y en caso de cometerlos se le debe dar una solución lo más clara y sencilla posible. Por ejemplo, deshabilitando opciones o indicando en un formulario el campo en el cual se ha producido el error sin perder la información ya escrita.
	\item Deshacer acciones fácilmente: Se debe dar al usuario la capacidad de poder deshacer o revertir acciones de una manera sencilla. 
	\item Sensación de control: Hay que dar al usuario la sensación de que tiene en todo momento el control de la aplicación, añadiendo contenidos fáciles de encontrar y de esta forma no causarle ansiedad o frustración por utilizar nuestra aplicación.
	\item Reducir la carga de memoria a corto plazo: La interfaz debe ser lo más sencilla posible y con una jerarquía de información evidente, es decir, hay que minimizar la cantidad de elementos a memorizar por el usuario.
\end{itemize}
Teniendo en cuenta estas ocho reglas, la creación del diseño de la Interfaz se ha realizado en dos iteracciones distintas. Una primera iteracción competitiva entre los integrantes de dicho trabajo y una segunda iteracción con expertos del Colegio Estudio3 Afanias. 

%-------------------------------------------------------------------
\subsection{Primera Iteración: Iteración Competitiva}
%-------------------------------------------------------------------
\label{cap:subsec:iteracionCompetitiva}

Esta primera iteración se trata de realizar los distintos prototipos de diseño de la aplicación. Por ello, cada integrante del grupo ha realizado cuatro diseños distintos. En la realización de estos, los integrantes no podían hablar entre ellos ni comentar las diferentes ideas que tenían para la implementación, de esta forma se consigue que los diseños sean totalmente dispares, que las ideas de uno no provoquen la modificación del diseño de otro.
Una vez que los prototipos estaban terminados, los integrantes de este trabajo se juntaron para hacer una puesta en común y analizar los resultados. A continuación, se explicarán los resultados de los análisis indicando las semejanzas y diferencias que se encontraron.

Como se ha comentado anteriormente, cada integrante realizó cuatro prototipos distintos:
\begin{itemize}
	\item Un primer diseño mostrando únicamente el resultado que más se suele utilizar en la sociedad.
	\item Un segundo diseño mostrando el resultado más utilizado pero añadiendo una definición del concepto y/o un ejemplo.
	\item Un tercer diseño mostrando todos los resultados, pero esta vez sin añadir la definición ni el ejemplo.
	\item Y por útimo, un cuarto diseño mostrando todos los resultados y añadiendo pictos que facilitan aún más la comprensión del concepto.
	
\end{itemize}

Por lo general, los cuatro prototipos son bastantes similares, como se puede ver en la Figura \ref{fig:mockup1pablo} los resultados para la palabra vehículo se muestran en el siguiente orden: primero varios sinónimos e hiperónimos (\textit{Un vehículo es una máquina} o \textit{Un vehículo es un transporte}), después los hipónimos (\textit{Un vehículo es como un taxi}) y finalmente los hipónimos pero añadiendo un adjetivo que representa una característica del resultado devuelto (\textit{Un vehículo es rápido como un caballo}). Por otro lado, en la Figura \ref{fig:mockup1irene_vInicial} los resultados para la palabra portero se devuelven en un orden distinto y no muestra varios sinónimos, si no que en primer lugar aparecen los hipónimos junto con el adjetivo que le caracteriza (\textit{Un portero es fuerte como un gorila}), después se muestran los sinónimos e hiperónimos (\textit{Un portero es un vigilante}) y por último aparecen los hipónimos sin el adjetivo (\textit{Un portero es como un guardia}). Otra diferencia que existen entre ellos, es el diseño de colores y formas, y en cuánto a las similitudes ambos disponen de un campo de texto para introducir una palabra, un botón de búsqueda y se muestran los resultados en forma de lista.

En cuanto al segundo diseño, en la Figura \ref{fig:mockup2pablo} se puede ver que los resultados para la palabra vehículo aparecen de la misma forma que en el diseño anterior incluyendo un ejemplo (\textit{Él necesita un coche para ir a trabajar}) y en la Figura \ref{fig:mockup2irene} se muestra el resultado más común para la palabra portero pero añadiendo una definición (\textit{Un portero es alguien que protege una entrada}). Esta sería la gran diferencia entre ambos prototipos, proporcionando al usuario más opciones para elegir, el diseño de colores y formas también son distintos pero los elementos que disponen la vista son iguales, es decir, existe un buscador del concepto, un botón y los resultados se muestran en forma de lista.

Respecto al tercer diseño, donde se muestran todos los resultados pero sin añadir definición ni ejemplo, se puede ver en la Figura \ref{fig:mockup3pablo} que se muestran en dos listas distintas añadiendo un título aclaratorio (\textit{Un vehículo es...} y \textit{O también puede ser...}) y en cambio como se puede ver en la Figura \ref{fig:mockup3irene} se muestran en listas identificando cada resultado con un número delante.

Y por último, para el cuarto diseño, en la Figura \ref{fig:mockup4pablo} se muestra los resultados como en los anteriores diseños pero añadiendo un picto que hace referencia al concepto según en que contexto se utilice. Por ejemplo, un vehículo puede hacer referencia a un coche o puede ser un medio para llegar o lograr un fin, en cambio, en la Figura \ref{fig:mockup4irene} se han añadido los pictos del concepto portero según el contexto que se utilice (portero de discoteca o portero de jugador) y además se han añadido pictos a cada resultado. Por ejemplo, en el resultado \textit{Un portero es fuerte como un gorila}, se añade un picto a la palabra gorila. 
Tanto en el prototipo diseñado por Pablo como en el de Irene se han utilizado los pictos de ARASAAC\footnote{http://www.arasaac.org/}.

Finalmente, tras realizar dicho análisis, se decidió que para la evaluación con los expertos, que se explicará en el siguiente apartado, utilizaríamos el diseño de colores y formas de los prototipos de Irene pero modificando ciertos aspectos y añadiendo ideas de los prototipos de Pablo. Por ejemplo, se hicieron tres versiones del segundo diseño para añadir tanto la definición como el ejemplo, en la Figura \ref{fig:mockup2_v1_irene} podemos ver que al prototipo original se le añadió otro botón para poder ver el ejemplo, de esta forma el usuario puede elegir entre ver una cosa u otra, pero no ambas a la vez. En la Figura \ref{fig:mockup2_v2_irene} se quitarón los botones y tanto la definición como el ejemplo aparecen justo debajo de los resultados y por último en la Figura \ref{fig:mockup2_v3_irene} existe un único botón en el cuál si el usuario pincha aparecen la definición y el ejemplo.
Otra modificación que se realizó fue que los resultados deben aparecer en el siguiente orden y que se puede ver en la Figura \ref{fig:mockup1irene}:

\begin{itemize} 
	\item Primero: los sinónimos e hiperónimos, como por ejemplo \textit{Un portero es un vigilante}.
	\item Segundo: los hipónimos, como por ejemplo \textit{Un portero es como un guardia}.
	\item Tercero: añadiendo un adjetivo que permita realizar una comparación con un hipónimo, por ejemplo \textit{Un portero es grande como un oso}.
\end{itemize}

Y por último, como se puede ver en la Figura \ref{fig:mockup4_vFinal_irene} se añadieron pictos también a los adjetivos que caracterizan a la palabra, por ejemplo en la frase \textit{Un portero es rápido como un caballo}, se han añadido pictos a rápido y a caballo.



 	\figura{Bitmap/Mockups/mockup1_pablo}{width=.9\textwidth}{fig:mockup1pablo}{Prototipo de Pablo mostrando resultado más común para la palabra vehículo}
 	
 	\figura{Bitmap/Mockups/mockup1_irene_inicial.png}{width=.9\textwidth}{fig:mockup1irene_vInicial}{Prototipo de Irene mostrando resultado más común para la palabra portero}
	\figura{Bitmap/Mockups/mockup1_irene.png}{width=.9\textwidth}{fig:mockup1irene}{Prototipo de Irene mostrando resultado más común para la palabra portero pero cambiando el orden}
	
	\figura{Bitmap/Mockups/mockup2_pablo}{width=.9\textwidth}{fig:mockup2pablo}{Prototipo de Pablo mostrando resultado más común para la palabra vehículo y un ejemplo} 
	\figura{Bitmap/Mockups/mockup2_irene}{width=.9\textwidth}{fig:mockup2irene}{Prototipo de Irene mostrando resultado más común para la palabra portero y una definición}
	
	\figura{Bitmap/Mockups/mockup3_pablo}{width=.9\textwidth}{fig:mockup3pablo}{Prototipo de Pablo mostrando todos los resultados para la palabra vehículo} 
	\figura{Bitmap/Mockups/mockup3_irene}{width=.9\textwidth}{fig:mockup3irene}{Prototipo de Irene mostrando todos los resultados para la palabra portero} 
	
	\figura{Bitmap/Mockups/mockup4_pablo}{width=.9\textwidth}{fig:mockup4pablo}{Prototipo de Pablo mostrando todos los resultados para la palabra vehículo junto con pictos} 
	\figura{Bitmap/Mockups/mockup4_irene}{width=.9\textwidth}{fig:mockup4irene}{Prototipo de Irene mostrando todos los resultados para la palabra portero junto con pictos} 
	
	
	\figura{Bitmap/Mockups/mockup2_v1_irene}{width=.9\textwidth}{fig:mockup2_v1_irene}{Prototipo Versión 1 mostrando resultado más común para la palabra portero junto con un ejemplo y una definición}
	\figura{Bitmap/Mockups/mockup2_v2_irene}{width=.9\textwidth}{fig:mockup2_v2_irene}{Prototipo Versión 2 mostrando resultado más común para la palabra portero junto con un ejemplo y una definición}
	\figura{Bitmap/Mockups/mockup2_v3_irene}{width=.9\textwidth}{fig:mockup2_v3_irene}{Prototipo Versión 3 mostrando resultado más común para la palabra portero junto con un ejemplo y una definición}
	\figura{Bitmap/Mockups/mockup4_vFinal_irene}{width=.9\textwidth}{fig:mockup4_vFinal_irene}{Prototipo Final  mostrando todos los resultados para la palabra portero junto con pictos}
	
	
	 
%-------------------------------------------------------------------
\subsection{Segunda Iteración: Evaluación con Expertos}
%-------------------------------------------------------------------
\label{cap:subsec:evaluacionExpertos}

Una vez realizadas las pequeñas modificaciones de los prototipos que nuestros directores consideraron que serían mejoras, tuvimos una reunión con la directora y los profesores del colegio Estudio3 Afanias situado en la Comunidad de Madrid. Este colegio se basa en la educación especial para la discapacidad intelectual, trastornos generalizados del desarrollo, autismo, EBO Infantil y ayuda con la transición a la vida adulta.
La reunión tuvo lugar el día 26 de Marzo de 2019 y acudimos los dos integrantes de este trabajo junto con los directores.
Una vez expuestos los distintos diseños, la primera impresión que tuvieron fue bastante optimista y nos dieron su opinión de lo que ellos creían que se debería añadir, eliminar o modificar para que su funcionalidad sea lo más amplia y sencilla posible tanto para su uso al impartir clase, como para el uso propio de los alumnos. 
Las conclusiones principales que obtuvimos de está evaluación han sido:
\begin{itemize} 
	\item Modificar el color amarillo-mostaza por un color más oscuro que contraste más con el fondo blanco.
	\item El tipo de letra debe ser Arial o Script, ya que son las letras con las que los alumnos están familiarizados y las que mejor entienden.
	\item Añadir un reproductor que lea la frase, haciendo así que en caso de no entender el resultado escrito, pueda ayudarles la voz.
	\item Incluir la opción de poder ver un video por si los pictos no aclaran al usuario a entender el concepto.
	\item Una parte configurable, donde se tenga en cuenta:
	\begin{itemize}
		\item La búsqueda puede realizarse en tres niveles: sencillo, medio y amplio. El nivel sencillo sería realizando la búsqueda de las palabras fáciles con las 1000 palabras de la RAE, el medio con las 5000 palabras de la RAE y el amplio con las 10000 palabras de la RAE. 
		\item Dar la opción de poder introducir mayúsculas para realizar la búsqueda.
		\item Que el usuario elija si quiere que aparezca la definición y el ejemplo o no.
		\item Poder elegir si deben aparecer los pictos o no. En caso de que el usuario decida que si deben aparecer, estos deben situarse debajo de la palabra y no al lado, ya que los expertos comentaron que esto puede llevar a confusión a los alumnos pensando que sería otra palabra más para leer. Y además, debería aparecer toda la frase traducida a pictos.
	\end{itemize}
\end{itemize}

\chapter{Trabajo Realizado}
\label{cap:TrabajoRealizado}


En este capítulo se describe el trabajo realizado por cada uno de los autores de este proyecto.


\section{Trabajo realizado por Irene}
\label{cap:sec:trabajo_Irene}

Al comenzar el proyecto, la primera tarea a realizar consistió en investigar las bibliotecas que se podrían utilizar para el procesado de las palabras. Al principio se encontró una biblioteca para el procesado de texto en Python llamada \textit{NLTK}, pero se pudo comprobar que las etiquetas que asignaba a las palabras no eran del todo correctas. Se decidió buscar otra biblioteca distinta y se recurrió a Spacy. Con esta librería ya se pudo etiquetar correctamente todas las palabras, lo que dio lugar a un primer diseño de un programa utilizando Jupyter. A continuación, investigué qué tecnologías utilizar para la realización del prototipo tecnológico. Encontramos como entorno de desarrollo \textit{Pycharm} y como framework de desarrollo web \textit{Django}. 

Una vez seleccionadas estas tecnologías, se investigó cómo poder sacarles el mejor partido y comenzamos a desarrollar el primer el prototipo tecnológico.
En esta tarea, mi cometido principal fue conectar las vistas html con la lógica en Python. A continuación tuvimos que aprender cómo se implementa un formulario web y como se hacía una redirección a la vista. Una vez tuvimos claro cómo realizar esta tarea, integramos el código desarrollado inicialmente en Jupyter en nuestro servicio web, finalizando el prototipo tecnológico.

En cuanto a la memoria, dividimos su elaboración en dos partes iguales entre los dos miembros del grupo, intentando que ambos tuviésemos asignada una parte en todos los capítulos de la misma, por lo que ambos redactamos tanto una parte de la introducción (en la que me correspondió redactar la motivación) como el estado de la cuestión (en este caso me correspondió el apartado de Lectura Fácil y Procesamiento del Lenguaje Natural).

La investigación de como funcionaba ConceptNet y su API la hicimos conjuntamente. A partir de aquí se realizó un prototipo con esta red semántica que contenía dos servicios web uno para los sinónimos y otro para los términos relacionados. 

Por otro lado se obtuvieron ficheros con las 1.000, 5.000 y 10.000 palabras más usadas según la RAE, se hizo un filtrado de estas palabras para quedarnos solo con adjetivos, verbos y nombres utilizando SpaCy.

Tras la realización del prototipo y hacer las pruebas correspondientes nos dimos cuenta que los resultados obtenidos no eran válidos por lo que se buscó otra red semántica.

Se decidió probar con WordNet, inicialmente tuvimos muchos problemas para poder obtener los recursos, ya que no tienen una API, sino que es una base de datos que te debes descargar.


Con los recursos descargados, estudiamos como estaban estructurados para poder utilizarlos según nuestros intereses y nos pusimos a programar el prototipo.

Nos descargamos los ficheros de la base de datos e investigamos como funcionaban para poder utilizarlos. Yo empecé a diseñar la vista del prototipo. Y después, nos pusimos los dos con la parte del backend.
La implementación de todos los servicios web se realizó conjuntamente entre mi compañero y yo.


Después, ya empezamos a trabajar en lo que sería el proyecto propiamente dicho, empezando por diseñar la vista. Para ello realizamos una iteración competitiva tanto Pablo como yo, para ver con que diseño nos quedábamos finalmente.

Se decidió que mi diseño de la vista era el que se va a implementar, añadiendo algunos aspectos que había añadido Pablo en su diseño. Por lo que adapté mi vista añadiendo dichos cambios para utilizarla en el proyecto que se desarrollaría finalmente.

A continuación, desarrollé la parte de la lógica de la vista de la aplicación, ya que aparte de la lógica de negocio desarrollada en Python, para mostrar los datos correctamente utilizamos jQuery con peticiones ajax para tratar los datos. Esta parte me llevó varios días ya que se presentaron varias dificultades.

Al terminar esta tarea, nos reunimos Pablo y yo para desplegar el proyecto en el contenedor que nos proporcionó el grupo NIL de la facultad de informática. 

Cuando subimos el proyecto al contenedor, nos dimos cuenta de que  había varios problemas, por lo que estuvimos varios días solucionándolos para que funcionase correctamente de cara a la evaluación que teníamos por delante en pocos días.

Para la evaluación de la aplicación, creamos entre Pablo y yo un formulario de Google Forms. Y una vez realizada, se transcribió a la memoria también entre los dos.

En todo momento, a la vez que se realizaba la implementación de código (ya sea backend o fronted) se iba realizando la memoria. Siempre dividida a partes iguales entre mi compañero y yo. 


\section{Trabajo realizado por Pablo}
\label{cap:sec:trabajo_Pablo}

Al igual que mi compañera, lo primero que hicimos fue investigar como podíamos etiquetar las palabras, encontramos la librería NLTK de Python para hacerlo, pero tras un primer intento nos dimos cuenta de que muchas palabras no estaban etiquetadas como deberían por lo que decidimos buscar alternativas, indagando un poco encontramos Spacy, la probamos y obtuvimos unos resultados mucho mejores que con NLTK por lo que decidimos utilizar esta última (todo esto lo hicimos desde el Jupyter). 

Cuando terminamos de etiquetar las palabras nos pusimos a investigar herramientas para el desarrollo del prototipo tecnológico y nos decantamos por utilizar Django como framework integrado en Pycharm, que es el entorno de desarrollo.

A continuación empezamos el desarrollo del prototipo tecnológico primero investigando como se utilizaban estas tecnologías(implementar formularios, hacer las redirecciones a vista...). Para finalizar migramos lo hecho desde Jupyter a nuestro servicio web.

Irene y yo nos dividimos la redacción de la memoria de tal manera que los dos hicimos tanto la parte de la introducción como del estado de la cuestión, de la introducción a mí me tocó la parte de los objetivos y del estado de la cuestión la parte de figuras retóricas y servicios web.

La investigación de como funcionaba ConceptNet y su API la hicimos de manera conjunta.

Tras investigar el funcionamiento de esta red semántica, empezamos a implementar un prototipo para probar si es funcional.

Mientras estábamos trabajando en esto, avanzamos en la redacción de la memoria, dividiéndonos el trabajo equitativamente.

Cuando terminamos la implementación del prototipo, se lo presentamos a los directores del TFG y nos dimos cuenta de que no iba a ser útil por lo que empezamos a trabajar en otra red semántica: WordNet.

Investigamos como funcionaba esta red semántica, estuvimos varios días con esto, hasta que descubrimos para acceder a los datos no hay una API pública sino que es una base de datos que hay que descargarse.

Encontramos varios sitios que hacían uso de la misma y para poder descargarse los recursos había que registrarse. Me registré en varios de ellos pero no nos servían ya que no estaban en castellano. Finalmente tras varios días encontramos los recursos necesarios en la página web de la universidad de Princeton.

Con los recursos descargados, estudiamos como estaban estructurados para poder utilizarlos según nuestros intereses y nos pusimos a programar el prototipo.

Mientras Irene empezaba a diseñar la vista, yo empecé a desarrollar la lógica interna para integrarla con los recursos descargados.

Después nos pusimos los dos con la parte de la lógica del prototipo para terminarlo cuanto antes.

A continuación, empecé a diseñar una prueba estadística para comprobar de manera empírica que red semántica ofrece mejores resultados. Realicé la prueba y la puse en común con Irene para analizar los resultados obtenidos.

Con los resultados de las pruebas y tras enseñar a los directores del TFG los resultados que ofrecía, decidimos que con estos recursos se desarrollaría la aplicación. 

Empezamos a realizar la aplicación final. Me puse a trabajar en mi prototipo visual para la iteración competitiva con Irene para ver como sería el diseño visual.

A continuación, empecé a desarrollar la base de datos de pictogramas para integrarlos en nuestra aplicación. Me descargué los pictogramas e implementé un script que relacionaba los pictogramas descargados con los recursos de WordNet, utilizando la API pública de ARASAAC, que relacionaba sus pictogramas con los recursos de WordNet mediante el offset. Sin embargo, estos offset correspondían a una versión de WordNet que no era la que estábamos utilizando, por lo que tuve que utilizar una API de la universidad de Princeton para traducir dichos offset a los que utiliza la versión que trabajamos nosotros, que es la 3.0.

Cuando acabé la base de datos de pictogramas, Irene y yo subimos la aplicación que teníamos implementada en local al servidor ofrecido por el grupo NIL de la facultad de informática para que pudiese estar disponible para todo el mundo. 

Cuando subimos el proyecto, nos dimos cuenta de que en el servidor había varios problemas, por lo que estuvimos varios días solucionándolos para que funcionase correctamente de cara a la evaluación que teníamos por delante en pocos días.


Una vez subido y con los errores corregidos, seguimos avanzando en el desarrollo de la memoria y comenzamos a diseñar un formulario para realizar la evaluación de la aplicación con usuarios finales terminando el desarrollo del proyecto.





%\include{Capitulos/Capitulo4}
%\include{Capitulos/Capitulo5}
\chapter{Conclusiones y Trabajo Futuro}
\label{cap:conclusionesyFuturo}

 

\section{Conclusiones}
\label{cap:sec:conclusiones}

Tras finalizar el proyecto y explicar todo lo recogido en esta memoria, se han podido recabar una serie de conclusiones atendiendo a los objetivos planteados y como se han conseguido.

Como se ha explicado desde el inicio de este trabajo, existen ciertos colectivos en nuestra sociedad que sufren de discapacidad cognitiva, haciendo que el significado de ciertas palabras no puedan ser entendidos. Esto les afecta directamente en su vida personal y profesional, ya que supone una gran limitación para ellos.
El objetivo principal del presente trabajo, era crear una aplicación web basada en servicios web que dada una palabra compleja para el usuario, devuelva mediante el uso de figuras retóricas (metáforas, símiles y analogías) un concepto mucho más sencillo.

Definir el concepto de sencillo es bastante relativo según para que persona, ya que no existe un grado definido de que es una palabra fácil. Por lo que, para la realización de este trabajo se tomó como referencia los listados facilitados por la RAE de las 1.000, 5.000 y 10.000 palabras más usadas del castellano.

La implementación de cada servicio web se ha hecho de manera modular y están accesibles en una API pública, para que de esta forma puedan ser reutilizados en futuras aplicaciones por distintos desarrolladores.

 Respecto a la creación de la interfaz, se han utilizado colores que resaltan sobre fondos claros, así como intentando que esta fuese lo más juvenil posible.
 En especial, hay que recalcar que gracias a la opinión de expertos que trabajan en centros dedicados a la educación de este colectivo, nos han ayudado a que elementos deberían aparecer y de que manera, así como se deberían mostrar los resultados, entre otros aspectos. 

Para crear una aplicación que fuese aún más útil, se añadieron una serie de opciones configurables, consiguiendo de esta forma que los usuarios pudieran decidir como quieren obtener los resultados. Estas opciones configurables son: 
\begin{itemize}
	\item Poder convertir todo el texto de la página en mayúsculas: Esto facilita la lectura puesto que están familiarizados a leer en mayúscula y es como aprenden a escribir.
    \item Poder mostrar pictogramas o no: Gracias a la utilización de pictogramas, los conceptos pueden ser entendidos perfectamente.
    \item Poder mostrar una definición y ejemplo del resultado: Por si el pictograma no terminara de ayudar al usuario a comprender el concepto, decidimos añadir una definición y un ejemplo para ayudarles aún más en su comprensión.
\end{itemize}

	Cabe destacar que la aplicación se construyó desde cero, sin hacer uso de plantillas externas.
	La aplicación web ha sido sometida a una evaluación con usuarios finales, donde pudimos probar tanto su funcionalidad como su interfaz. Tras esta evaluación pudimos comprobar que la aplicación solamente es útil para personas con problemas semánticos no muy altos y surgieron bastantes aspectos que mejorar.
	
	 Este trabajo ha sido una oportunidad para aplicar todos los conocimientos adquiridos en distintas asignaturas cursadas durante la carrera a un proyecto grande y de impacto real. Entre estas asignaturas cabe destacar:
 	
 	\begin{itemize}
 		\item \textbf{Estructura de Datos y Algoritmos }para ayudarnos a pensar en el código de una manera eficiente y estructurada, así como a elegir las estructuras de datos más adecuadas.
 		\item \textbf{Fundamentos de la Programación} y \textbf{Tecnología de la Programación} para las base de la lógica interna de la aplicación.
 		\item \textbf{Aplicaciones Web} e \textbf{Interfaces de Usuario} para obtener los conocimientos suficientes de HTML, CSS, Bootstrap, jQuery y JavaScript para desarrollar la aplicación así como el funcionamiento básico de una aplicación web y las reglas básicas que debe contener una interfaz.
 		\item \textbf{Base de Datos} y \textbf{Ampliación de Base de Datos} para el tratamiento y gestión de múltiples datos.
 		\item \textbf{Gestión de Proyectos Software} para la gestión adecuada de la aplicación.
 		\item \textbf{Administración de Sistemas y Redes} que nos ayudo a la configuración del servidor así como al despliegue de la aplicación en el mismo.
 		\item \textbf{Ingeniería del Software} y \textbf{Modelado del Software} para la realización de la parte de modelado y los respectivos diagramas.
 		\item \textbf{Ética, Legislación y Profesión} para obtener los conocimientos sobre licencias, tanto para nuestro propio código como para tratar y utilizar el código de terceros.
 	\end{itemize}

	Por otro lado, también hemos aprendido bastantes cosas nuevas, como puede ser Django, Python, Latex, configurar un servidor y Procesamiento de Lenguaje Natural. 
	Ambos integrantes del equipo hemos aprendido sobre las discapacidades cognitivas existentes, y gracias a la evaluación de la aplicación pudimos convivir unas horas con ellos, haciendo que fuese una experiencia muy enriquecedora en lo personal ya que pudimos comprobar las dificultades que afrontan en su día a día y como se enfrentan a ellas con positivismo y naturalidad.

	En definitiva, de los objetivos que se plantearon en la sección \ref{cap:sec:objetivos} se han cumplido la gran mayoría, y los que no se pudieron cumplir se explicarán en la siguiente apartado.
	
	

\section{Trabajo Futuro}
\label{cap:sec:TrabajoFuturo}

Como se ha comentado en el apartado anterior, no se han podido cumplir todos los objetivos marcados en un primer momento. El motivo principal de esto es por falta de tiempo, ya que al ser un proyecto tan grande y al tener que cubrir el mayor número de necesidades posible, se convierte en un trabajo aún mayor.
Por ello, todos los aspectos que no se han llegado a suplir, tanto los fijados en un primer momento como los sugeridos en la evaluación final se dejarán aquí reflejados como trabajo futuro. 

\begin{itemize}
	\item Mostrar resultado mediante analogías: Al igual que nuestra aplicación muestra los resultados mediante metáforas y símiles, también deberían mostrarse mediante analogías. Por ello, habría que buscar la información necesaria para saber como enlazar el concepto buscado y el resultado mostrado y ver que características comparten ambos.
	\item Añadir reconocimiento por voz: En cada elemento de la interfaz debería haber una descripción auditiva del elemento seleccionado para facilitar aún más su uso.
	\item Video explicativo del pictograma: Al lado de cada pictograma debería aparecer un botón que al ser pulsado mostrara un video explicativo del concepto.
	\item Mostrar primero significado más utilizado: Una vez que el usuario haya buscado una palabra y se muestren los resultados en fichas, dentro de cada una debería aparecer primero el concepto más relacionado con el término buscado.
	\item Mostrar pictograma con la acepción correcta: En vez de realizar la búsqueda del pictograma de una metáfora o un símil, debería buscarse el pictograma de la relación existente entre el resultado y el concepto buscado.
	\item Poder buscar palabras más sencillas y obtención de resultados más sencillos: En vez de comparar los resultados obtenidos con las 1.000, 5.000 y 10.000 palabras de la RAE, se debería encontrar la manera de poder obtener resultados aún más sencillos para conceptos no tan complejos.
	\item Implementación de la aplicación móvil: Sería bastante útil que los usuarios pudieran tener en su teléfono la aplicación para poder acceder a ella siempre que quisieran.
	
	
	
\end{itemize}


En un futuro próximo nos gustaría poder seguir con este trabajo, ya que nos hemos involucrado a nivel personal con el y nos gustaría que en un futuro pueda ser una aplicación usable por cualquier persona.



%%%%%%%%%%%%%%%%%%%%%%%%%%%%%%%%%%%%%%%%%%%%%%%%%%%%%%%%%%%%%%%%%%%%%%%%%%%
% Si el TFM se escribe en inglés, comentar las siguientes líneas 
% porque no es necesario incluir nuevamente las Conclusiones en inglés
\setcounter{chapter}{\thechapter-1} 
\begin{otherlanguage}{english}
\chapter{Conclusions and Future Work}
\label{cap:conclusions_futureWork}

\section{Conclusions}
\label{sec:conclusions}

After completing the project, some conclusions have been obtained in relation to the objectives initially proposed.

The main goal of this work was to create a web application, based on web services, in which difficult words could be converted into easier words by using  rhetorical figures (metaphors, similes and analogies).

It is difficult to define what an easy concept means because is not the same for everyone. To do so, we have taken the RAE's 1.000, 5.000 and 10.000 most used words in the Spanish language.

The implementation of each web service has been made in a modular way and they are available in a public API, so they can be used by other developers at any point.

When we created our application interface, we used dark colours that stand out on plain backgrounds in order to make reading easy for the users. We consulted with professionals with vast experience in this field so they could give us advice on how to display the results for the words searched.

To make our application more useful we added some customizable settings for the user, such as:

\begin{itemize}
	\item To be able to convert the whole text into capital letter: this makes reading easier because that is how they learn to read and write.
	\item To be able to show pictograms if needed: using pictograms makes it easy to understand certain concepts.
	\item To be able to show definitons and examples for the results: if showing pictograms is not enough to understand a concept, the application can show them an example and definition.
\end{itemize}

We built this application from scratch, not using any external templates. This web application was tested by target users so we could prove its functionality and its interface. As we concluded these tests, we realized the application is useful for people with mild disabilities but it needs to be improved in order to make it useful for people with severe disabilities.

This project has been an opportunity to apply all the knowledge acquired
in different subjects taken during the degree to a large project with real impact. Among all of these subjects, we can highlight:

\begin{itemize}
	\item Data Structures and Algorithms, that helped us to acquire a structured and efficient way of thinking when programming.
	\item Foundations of the Programming and Technology of the Programming, which helped us understand the internal logic of the applications.
	\item Web Applications and User Interfaces, that helped us to acquire enough knowledge about html, css, Bootstrap, jQuery and javascript to develop the application and the basic rules about interface design.
	\item Databases and Extension of Databases to manage big volumes of data.
	\item Software Project Management, for the correct management of our application.
	\item System and Networks Management, that helped us to the configuration of our server.
	\item Software Engineering and Software Modeling, that helped us to design the application.
	\item Ethics, Legislation and Profession, where we learned about software licenses, both to protect our code and to know how to correctly use free software developed by third parties.
\end{itemize}

We have also acquired knowledge about some new things such us Django, Python, Latex and Natural Laguage Processing. We have learned about cognitive disabilities and we have got to spend some time with target users, which made this experience a really valuable one. We have realized the problems they face and the way they surpass them.

We consider we have achieved most of our main goals (section \ref{sec:goals}). The ones which could not be achieve will be explained in the next section.


\section{Future Work}
\label{sec:future_work}

We couldn't achieve every goal we aimed to at the beginning of the project. The main reason is the lack of time and the fact that we needed to cover so many fields. The items that can be developed as future work are:

\begin{itemize}
	\item To show results with analogies: Our application shows results with metaphors and similes. We think it would be useful for the user to show them analogies too. For that purpose, we would need to search information about how to link the concept and the result and which properties they share.
	
	\item To add voice synthesis: We should add an audio description of the selected item to make the application easier for the user.
	
	\item Explanatory video of each pictogram: a button could be added that, when clicked, shows an explanatory video.
	
	\item To show the most relevant result first: When the user has searched for a word, the most significant results should be shown first. This has been difficult to achieve, since the external resources used to mine for words do not provide any information that allows to decide what concepts are more relevant.
	
	\item The pictograms should be displayed with the correct meaning: in case of polysemic words, the pictogram should be shown with the correct relation between the result and the searched concept.
	
	\item We should be able to look for easier words and easier results: we need to find a way to obtain easier results rather than using RAE's 1.000, 5.000 and 10.000 lists of easy words.
	
	\item Develop a smartphone application: It would be useful for the user to be able to have this application in their smartphone instead of as a web application.
	
\end{itemize}

In the near future, we would like to keep working in this project, as during this year we have been involved in its development at a personal level, so we would like it to be an application that could be used by everyone.





\end{otherlanguage}
%%%%%%%%%%%%%%%%%%%%%%%%%%%%%%%%%%%%%%%%%%%%%%%%%%%%%%%%%%%%%%%%%%%%%%%%%%%


% Apéndices
\appendix
\chapter{Artículos Periodísticos Utilizados para el análisis cualitativo y cuantitativo }
\label{sec:apendiceA}

En este apéndice se muestran los tres artículos periodísticos utilizados para realizar los análisis cuantitativos y cualitativos del diseño de la evaluación.

%-------------------------------------------------------------------
\section{Artículo Tecnológico}
%-------------------------------------------------------------------
\label{cap:sec:articulotecnologico}
Los fósiles de una escena primitiva, hallados con la precisión de una fotografía, demuestran que los seres humanos se comieron a los últimos grandes mamíferos que quedaban en América después de la última glaciación. La evidencia, que figura en un trabajo publicado por Science Advances, fue analizada por un equipo de arqueólogos argentinos del Consejo Nacional de Investigaciones Científicas y Técnicas (CONICET) en la Universidad Nacional del Centro de la provincia de Buenos Aires (UNICEN) junto a investigadores estadounidenses.

El clima contra la depredación humana es (ahora se sabe) una falsa controversia científica respecto a la extinción de los megamamíferos en Sudamérica y en el Mundo. Las sucesivas evidencias han enfatizado una causa sobre la otra, pero en conjunto reflejan que ambas fueron determinantes en la desaparición de los grandes animales del Pleistoceno.

Se trata de un proceso que en América se inició en el deshielo y que el apetito humano probablemente sólo aceleró. ``El clima jugó un rol también. Se extinguieron grandes animales en el mundo, no solamente acá aunque se extinguieron más en Sudamérica, también lo hicieron en Norteamérica y Europa. Entonces la discusión es: el clima y algo más. Este algo más creemos que son los seres humanos'', aclara el director de la investigación Gustavo Politis, sentado en su oficina del área de Investigaciones Arqueológicas y Paleontológicas del Cuaternario Pampeano (INCUAPA), a pocos kilómetros del sitio del hallazgo. Sin embargo, agrega una advertencia para quienes cargan las culpas sobre los seres humanos. ``En Sudamérica hay al menos 30 especies de megamamíferos que se han extinguido. Las que han sido cazadas son 5, 6 no más. No se puede explicar toda la extinción por la acción del hombre''. La caza no es el único daño que podríamos haber hecho en el pasado. ``También puede haber pasado que los seres humanos hayan hecho disrupciones en el ambiente como la introducción de nuevos parásitos o quemazones en los campos. Si el fuego produjo quemazón en poblaciones de animales con bajas tasas de reproducción, había un clima desfavorable y encima aparecieron seres humanos que los depredaron, los extinguieron'' concluye Politis.

La imagen que este reciente descubrimiento permite evocar ocurre hace 12.600 años a la vera de un pantano (hoy convertido en arroyo) en la llanura pampeana argentina. Quedan pocos gigantes para cazar. Hace un poco más de frío, hay menos humedad que en la actualidad y el mar está más lejos. Sobre los pastizales se erige quejoso un perezoso de cuatro metros de alto (como la estatura del oso y el madroño) y unas tres toneladas (el equivalente a cinco toros de lidia). Un grupo de cazadores armados con lanzas y proyectiles de piedra lo ataca. El grotesco animal, cuyo nombre científico es Megaterio, no logra ganar la batalla y acaba convertido en cena, abrigo y herramientas. Para saber esto, los investigadores desenterraron y examinaron fósiles durante 18 años en un campo de hacienda al que llegaron gracias a la suerte de un agricultor que, operado de la cadera, tuvo que dejar un tiempo el caballo y recorrer el campo andando. La caminata lenta y larga lo sorprendió con un fémur inmenso y ennegrecido asomando de la tierra.

Esa oscuridad se convertiría en la clave para desentramar la prehistórica escena. ``Es un ambiente con mucha materia orgánica porque era un antiguo pantano'', describe Politis. En esas condiciones, explica, es muy probable que se produzca la reacción de Maillard'sobre el fósil; cuando las moléculas de colágeno -que permiten conocer la antigüedad de un hueso- se juntan con las de los ácidos húmicos y fúlvicos -esos que forman el humus, la tierra negra fértil que aman los jardineros. ``Para separar eso hay que hacer un tratamiento especial porque si no, se datan las dos cosas juntas'', advierte Politis, licenciado en Antropología y doctor en Ciencias Naturales. ``Digamos que te rejuvenece la edad del fósil'', agrega en la misma oficina, y con iguales títulos académicos, el segundo autor del trabajo Pablo Messineo. ``El problema que teníamos siempre era que la datación nos daba entre 7.500 y 8.000 años atrás. Es una edad muy reciente para lo que es la extinción de los megamamíferos en Sudamérica, estimada en 12.000 años''. Hasta que llegaron a Thomas Stafford y su laboratorio. Este investigador desarrolló a principios de los '80 un método único que permite separar el colágeno del resto de la materia orgánica. ``Al separar el colágeno y datarlo te asegurás de que esa contaminación no exista más. Lo que hicimos ahora fue separarlos y datar ambos. Entonces, el colágeno nos dio 12.600 años y lo demás, 9.000. Así confirmamos que los ácidos estaban contaminando la muestra'', explica Messineo.

El paso del tiempo, el clima, la erosión, pueden tergiversar la historia. ``Si hay varios eventos a lo largo del tiempo, se pueden mezclar y es difícil separarlos. Acá hay uno solo; no pasó nada antes ni después. Tenemos una foto arqueológica de ese momento. Hay pocas cosas pero muy coherentes entre sí'', asegura el director del equipo de investigación. ``Una de las más interesantes es un artefacto que es una especie de cuchillo de piedra que se les rompió y tenemos los dos pedazos. Uno lo encontramos en 2003 y otro ahora. Estaban a 3 metros. Cuando los juntamos nos dimos cuenta de que uno lo habían tirado cuando se les rompió y al otro lo siguieron reactivando, tiene el filo más usado'', destaca Messineo.

``Es una conducta esperable; gente que está carneando, cuando el filo se les agota lo reactivan, cuando lo reactivan se puede romper y con el pedazo que les conviene más, siguen cortando. Es como llegar a una escena ni bien ha ocurrido. Están los huesos, las herramientas, solo falta la gente'', detalla con fascinación Politis, dinstinguido por su trayectoria por el Estado argentino. La evidencia de que los humanos se comieron al perezoso gigante es contundente. ``Encontramos marcas de corte en una costilla, eso indica que sin dudas que los grupos humanos lo carnearon. Después encontramos dos instrumentos hechos con las costillas del megaterio. Es otro indicio de que no es solamente la asociación de los artefactos de piedra con los huesos sino que los tipos estuvieron ahí procesando al animal. Eso hoy es indiscutible'', afirma.

Las dataciones de este estudio, en el que también trabajó Emily Lindsey, investigadora del Museo Rancho La Brea de Los Ángeles, Estados Unidos, permiten saber que los seres humanos y los grandes mamíferos coexistieron durante unos 1.500 años y arrojan sospechas sobre otras investigaciones. ``Hay que re-datar otros hallazgos también. Ahora sospechamos de las dataciones de los otros sitios porque estaban hechos con métodos más convencionales'', advierte Politis. Si el presupuesto (por el momento suspendido) lo permite, esas nuevas dataciones podrán descontaminarse en laboratorios propios ya que la UNICEN y el CONICET están trabajando en el desarrollo local de esa tecnología disponible, por ahora, en muy pocos laboratorios de Estados Unidos y Europa.


%-------------------------------------------------------------------
\section{Artículo Deportivo}
%-------------------------------------------------------------------
\label{cap:sec:articulodeportivo}

Las tres caídas consecutivas en los cuartos de final de la Champions habían dejado tocado al Barcelona, sobre todo la temporada pasada, cuando los muchachos de Valverde se derrumbaron por 3-0 ante la Roma. Los directivos apuntaron al técnico, actitud que alertó al vestuario. El grupo, en la intimidad, había despojado de cualquier responsabilidad a su entrenador, un tipo al que los jugadores reconocen y respetan, esencialmente por su mano izquierda para gestionar el camerino. Sabían que se habían quedado sin gasolina. Luis Suárez, uno de los pesos pesados, reconoció públicamente que se había equivocado ante el Leganés, en la previa de la debacle azulgrana en Roma. La herida se acentuó cuando el Madrid levantó su tercera Champions seguida. Messi no lo toleró: ``Queremos que vuelva al Camp Nou la copa linda y deseada''.

Reconocido el error, para esta campaña la consiga era clara: había que confiar más en las rotaciones del Valverde. Y esta vez, cuando llegaron los cuartos de la Champions ante el United, nadie protestó. El técnico borró a todas sus estrellas del duelo frente al Huesca en la víspera de la vuelta ante el equipo de Manchester. Funcionó. Los suplentes empataron en El Alcoraz y los titulares barrieron al conjunto de Ole Gunnar Solskjaer. Sin embargo, en la previa de las semifinales ante el Liverpool, Valverde apostó por los mejores contra la Real y viajarán todos, salvo Rakitic, a Vitoria. El Barça quiere llegar a la eliminatoria contra el equipo de Klopp, el martes 1 de mayo en el Camp Nou, con la Liga en el Museo del Camp Nou. El problema es que no es lo mismo salir campeón el miércoles —necesita ganar hoy al Alavés (21.30, beIN LaLiga) y que mañana pierda el Atlético con el Valencia— que el sábado, día en que el Levante visita el Camp Nou, antes del encuentro de Champions. ¿Qué alineación sacará el técnico ante el equipo granota si se juega el título y el miércoles aguarda el Liverpool?

``Nos ponemos estupendos para ver cuándo vamos a ganar… Lo que queremos es ganar cuanto antes, pero la gente lo ve desde fuera y piensa que todo es sencillo'', resolvió Valverde. ``Cuando se gana la Liga al final parece que la alegría es superior, pero cuando tienes una ventaja sobre el segundo parece un título previsible'', completó el Txingurri. ``Ojalá pudiéramos ganar siempre con margen''.

Valverde dejó caer que habrá rotaciones en Vitoria. ``Desde luego, es muy posible que haya cambios'', advirtió. El técnico lleva perfectamente la cuenta de los minutos de sus jugadores, y los titulares están más descansados que la última campaña. ``Físicamente los veo bien'', subrayó; ``es normal que a estas alturas de la temporada los equipos acusemos el paso del tiempo, pero nos alimentamos de las sensaciones que tenemos. Los retos eliminan cualquier cansancio, cuando hay objetivos las piernas están frescas''. Valverde confía en la ambición de sus futbolistas, pero, por las dudas, los quiere descansados ante el Liverpool.


%-------------------------------------------------------------------
\section{Artículo Político}
%-------------------------------------------------------------------
\label{cap:sec:articulopolitico}

El PP ha criticado este martes los ataques recibidos por el líder de Ciudadanos durante el debate electoral. ``Rivera le hizo el trabajo sucio a Sánchez'', se ha quejado el secretario general de los populares, Teodoro García Egea. El candidato de Cs salió al choque tanto contra Pedro Sánchez como contra Casado, mientras este evitaba el enfrentamiento con él. Solo una vez –cuando Rivera sugirió que podían pactar con el PNV como ya habían hecho en el pasado- entró Casado en el cuerpo a cuerpo: ``Ni sus votantes ni los míos entienden lo que está diciendo, Usted no es mi adversario. Yo no voy a pactar ni con el PNV ni con Sánchez. En materia de pactos soy más creíble que usted''.

Todos los partidos se arrogan este martes la victoria del debate en TVE, a pocas horas del segundo asalto en Atresmedia. Fuentes del PP indican que Casado hizo “el debate que quería hacer”, con un ``tono moderado y presidenciable''. El candidato popular quiere trata de recuperar un perfil centrado en la última semana de campaña y dejar atrás las graves acusaciones y descalificativos empleados contra Sánchez, como cuando le acusó de preferir ``las manos manchadas de sangre''. Los populares dudan sobre la estrategia. Algunos dirigentes consultados por EL PAÍS creen que con el tono moderado del lunes, Casado puede decantar a algunos indecisos que se debaten entre votar al PP o a Ciudadanos. Otros señalan que el líder estuvo excesivamente plano y dejó que Rivera acaparara protagonismo. Fuentes de la dirección indican, en cualquier caso, que el formato de esta noche, con preguntas y repreguntas, da pie a intervenciones algo más atrevidas.

``Rivera todavía no se ha enterado de que el enemigo se llama Pedro Sánchez'', ha subrayado este martes el secretario general del PP. Los populares quieren unir fuerzas (y votos) en torno al enemigo común, ya que su única oportunidad de gobernar es sumando los escaños de lo que el presidente llamó el lunes, insistentemente, ``las derechas'', incluyendo a Vox.

Pero Rivera salió a la cita a proyectarse como alternativa a Sánchez y eso implicaba también tratar de noquear a Casado, con el que se disputa el liderazgo del bloque de centroderecha. Los suyos están convencidos de que lo logró, ante un Casado demasiado contenido.

``Todos los españoles vieron ayer un ganador claro, el único capaz de demostrar que es una alternativa a Pedro Sánchez, el único que lleva mucho años luchando por la igualdad de todos los españoles, el único con la fuerza y la valentía necesarias para hacer frente a los grandes retos de España'', ha enfatizado hoy sobre Rivera Inés Arrimadas, candidata de Cs por Barcelona. La dirección de Ciudadanos considera que el de ayer fue un ``punto de inflexión'' en la campaña electoral en el que su candidato salió beneficiado. Rivera, según fuentes de la cúpula, volverá a salir hoy a la ofensiva y a intentar liderar el segundo debate.

En la izquierda, el PSOE ha cargado contra los ``exabruptos'' de la derecha y ha reivindicado que el candidato socialista fue el único en hacer un debate en positivo. ``Creo que el tono de los candidatos de derechas en el debate fue el mismo tono que hemos estado viendo durante toda la campaña: uno donde han primado más los insultos, los exabruptos, con pocas propuestas y poco proyecto de país'', ha defendido la candidata del PSOE por Barcelona, Meritxell Batet. Sánchez puso sobre la mesa, a juicio de los socialistas, ``un proyecto de país, muchas propuestas e hizo un listado exhaustivo'' de la labor del Gobierno socialista en sus diez meses al frente del Ejecutivo.

Por su parte, Unidas Podemos ha destacado que el candidato del PSOE no contestó ayer a las reiteradas preguntas de Pablo Iglesias sobre si pactará o no con Ciudadanos tras el 28 de abril, lo que para la formación morada es revelador de que esa es la intención de los socialistas. El secretario de Organización de Podemos, Pablo Echenique, se ha dirigido este martes a los indecisos y a los votantes socialistas para advertirles de que su voto el 28 de abril puede acabar ``en un Gobierno de derechas en el que Albert Rivera sea ministro''.

Ese es el punto central de la campaña de Podemos, el de alertar sobre un posible pacto del PSOE con Ciudadanos, aunque la ejecutiva del partido de Albert Rivera ha aprobado cerrar la puerta a ese acuerdo, como el candidato naranja recordó también ayer durante el debate electoral. Su apuesta es un pacto con el PP para el que ayer volvió a tender la mano a Casado, sin respuesta. En el segundo asalto, el debate de esta noche, los pactos poselectorales volverán a dar previsiblemente momentos de choque entre los candidatos.
%\chapter{Título}
\label{Appendix:Key2}

%\include{Apendices/appendixC}
%\include{...}
%\include{...}
%\include{...}
\backmatter

%
% Bibliografía
%
% Si el TFM se escribe en inglés, editar TeXiS/TeXiS_bib para cambiar el
% estilo de las referencias
%---------------------------------------------------------------------
%
%                      configBibliografia.tex
%
%---------------------------------------------------------------------
%
% bibliografia.tex
% Copyright 2009 Marco Antonio Gomez-Martin, Pedro Pablo Gomez-Martin
%
% This file belongs to the TeXiS manual, a LaTeX template for writting
% Thesis and other documents. The complete last TeXiS package can
% be obtained from http://gaia.fdi.ucm.es/projects/texis/
%
% Although the TeXiS template itself is distributed under the 
% conditions of the LaTeX Project Public License
% (http://www.latex-project.org/lppl.txt), the manual content
% uses the CC-BY-SA license that stays that you are free:
%
%    - to share & to copy, distribute and transmit the work
%    - to remix and to adapt the work
%
% under the following conditions:
%
%    - Attribution: you must attribute the work in the manner
%      specified by the author or licensor (but not in any way that
%      suggests that they endorse you or your use of the work).
%    - Share Alike: if you alter, transform, or build upon this
%      work, you may distribute the resulting work only under the
%      same, similar or a compatible license.
%
% The complete license is available in
% http://creativecommons.org/licenses/by-sa/3.0/legalcode
%
%---------------------------------------------------------------------
%
% Fichero  que  configura  los  parámetros  de  la  generación  de  la
% bibliografía.  Existen dos  parámetros configurables:  los ficheros
% .bib que se utilizan y la frase célebre que aparece justo antes de la
% primera referencia.
%
%---------------------------------------------------------------------


%%%%%%%%%%%%%%%%%%%%%%%%%%%%%%%%%%%%%%%%%%%%%%%%%%%%%%%%%%%%%%%%%%%%%%
% Definición de los ficheros .bib utilizados:
% \setBibFiles{<lista ficheros sin extension, separados por comas>}
% Nota:
% Es IMPORTANTE que los ficheros estén en la misma línea que
% el comando \setBibFiles. Si se desea utilizar varias líneas,
% terminarlas con una apertura de comentario.
%%%%%%%%%%%%%%%%%%%%%%%%%%%%%%%%%%%%%%%%%%%%%%%%%%%%%%%%%%%%%%%%%%%%%%
\setBibFiles{%
nuestros,latex,otros%
}

%%%%%%%%%%%%%%%%%%%%%%%%%%%%%%%%%%%%%%%%%%%%%%%%%%%%%%%%%%%%%%%%%%%%%%
% Definición de la frase célebre para el capítulo de la
% bibliografía. Dentro normalmente se querrá hacer uso del entorno
% \begin{FraseCelebre}, que contendrá a su vez otros dos entornos,
% un \begin{Frase} y un \begin{Fuente}.
%
% Nota:
% Si no se quiere cita, se puede eliminar su definición (en la
% macro setCitaBibliografia{} ).
%%%%%%%%%%%%%%%%%%%%%%%%%%%%%%%%%%%%%%%%%%%%%%%%%%%%%%%%%%%%%%%%%%%%%%
\setCitaBibliografia{
\begin{FraseCelebre}
\begin{Frase}
  Y así, del mucho leer y del poco dormir, se le secó el celebro de
  manera que vino a perder el juicio.
\end{Frase}
\begin{Fuente}
  Miguel de Cervantes Saavedra
\end{Fuente}
\end{FraseCelebre}
}

%%
%% Creamos la bibliografia
%%
\makeBib

% Variable local para emacs, para  que encuentre el fichero maestro de
% compilación y funcionen mejor algunas teclas rápidas de AucTeX

%%%
%%% Local Variables:
%%% mode: latex
%%% TeX-master: "../Tesis.tex"
%%% End:

%
% Índice de palabras
%

% Sólo  la   generamos  si  está   declarada  \generaindice.  Consulta
% TeXiS.sty para más información.

% En realidad, el soporte para la generación de índices de palabras
% en TeXiS no está documentada en el manual, porque no ha sido usada
% "en producción". Por tanto, el fichero que genera el índice
% *no* se incluye aquí (está comentado). Consulta la documentación
% en TeXiS_pream.tex para más información.
\ifx\generaindice\undefined
\else
%%---------------------------------------------------------------------
%
%                        TeXiS_indice.tex
%
%---------------------------------------------------------------------
%
% TeXiS_indice.tex
% Copyright 2009 Marco Antonio Gomez-Martin, Pedro Pablo Gomez-Martin
%
% This file belongs to TeXiS, a LaTeX template for writting
% Thesis and other documents. The complete last TeXiS package can
% be obtained from http://gaia.fdi.ucm.es/projects/texis/
%
% This work may be distributed and/or modified under the
% conditions of the LaTeX Project Public License, either version 1.3
% of this license or (at your option) any later version.
% The latest version of this license is in
%   http://www.latex-project.org/lppl.txt
% and version 1.3 or later is part of all distributions of LaTeX
% version 2005/12/01 or later.
%
% This work has the LPPL maintenance status `maintained'.
% 
% The Current Maintainers of this work are Marco Antonio Gomez-Martin
% and Pedro Pablo Gomez-Martin
%
%---------------------------------------------------------------------
%
% Contiene  los  comandos  para  generar  el índice  de  palabras  del
% documento.
%
%---------------------------------------------------------------------
%
% NOTA IMPORTANTE: el  soporte en TeXiS para el  índice de palabras es
% embrionario, y  de hecho  ni siquiera se  describe en el  manual. Se
% proporciona  una infraestructura  básica (sin  terminar)  para ello,
% pero  no ha  sido usada  "en producción".  De hecho,  a pesar  de la
% existencia de  este fichero, *no* se incluye  en Tesis.tex. Consulta
% la documentación en TeXiS_pream.tex para más información.
%
%---------------------------------------------------------------------


% Si se  va a generar  la tabla de  contenidos (el índice  habitual) y
% también vamos a  generar el índice de palabras  (ambas decisiones se
% toman en  función de  la definición  o no de  un par  de constantes,
% puedes consultar modo.tex para más información), entonces metemos en
% la tabla de contenidos una  entrada para marcar la página donde está
% el índice de palabras.

\ifx\generatoc\undefined
\else
   \addcontentsline{toc}{chapter}{\indexname}
\fi


% Generamos el índice
\printindex

% Variable local para emacs, para  que encuentre el fichero maestro de
% compilación y funcionen mejor algunas teclas rápidas de AucTeX

%%%
%%% Local Variables:
%%% mode: latex
%%% TeX-master: "./tesis.tex"
%%% End:

\fi

%
% Lista de acrónimos
%

% Sólo  lo  generamos  si  está declarada  \generaacronimos.  Consulta
% TeXiS.sty para más información.


\ifx\generaacronimos\undefined
\else
%---------------------------------------------------------------------
%
%                        TeXiS_acron.tex
%
%---------------------------------------------------------------------
%
% TeXiS_acron.tex
% Copyright 2009 Marco Antonio Gomez-Martin, Pedro Pablo Gomez-Martin
%
% This file belongs to TeXiS, a LaTeX template for writting
% Thesis and other documents. The complete last TeXiS package can
% be obtained from http://gaia.fdi.ucm.es/projects/texis/
%
% This work may be distributed and/or modified under the
% conditions of the LaTeX Project Public License, either version 1.3
% of this license or (at your option) any later version.
% The latest version of this license is in
%   http://www.latex-project.org/lppl.txt
% and version 1.3 or later is part of all distributions of LaTeX
% version 2005/12/01 or later.
%
% This work has the LPPL maintenance status `maintained'.
% 
% The Current Maintainers of this work are Marco Antonio Gomez-Martin
% and Pedro Pablo Gomez-Martin
%
%---------------------------------------------------------------------
%
% Contiene  los  comandos  para  generar  el listado de acrónimos
% documento.
%
%---------------------------------------------------------------------
%
% NOTA IMPORTANTE:  para que la  generación de acrónimos  funcione, al
% menos  debe  existir  un  acrónimo   en  el  documento.  Si  no,  la
% compilación  del   fichero  LaTeX  falla  con   un  error  "extraño"
% (indicando  que  quizá  falte  un \item).   Consulta  el  comentario
% referente al paquete glosstex en TeXiS_pream.tex.
%
%---------------------------------------------------------------------


% Redefinimos a español  el título de la lista  de acrónimos (Babel no
% lo hace por nosotros esta vez)

\def\listacronymname{Lista de acrónimos}

% Para el glosario:
% \def\glosarryname{Glosario}

% Si se  va a generar  la tabla de  contenidos (el índice  habitual) y
% también vamos a  generar la lista de acrónimos  (ambas decisiones se
% toman en  función de  la definición  o no de  un par  de constantes,
% puedes consultar config.tex  para más información), entonces metemos
% en la  tabla de contenidos una  entrada para marcar  la página donde
% está el índice de palabras.

\ifx\generatoc\undefined
\else
   \addcontentsline{toc}{chapter}{\listacronymname}
\fi


% Generamos la lista de acrónimos (en realidad el índice asociado a la
% lista "acr" de GlossTeX)

\printglosstex(acr)

% Variable local para emacs, para  que encuentre el fichero maestro de
% compilación y funcionen mejor algunas teclas rápidas de AucTeX

%%%
%%% Local Variables:
%%% mode: latex
%%% TeX-master: "../Tesis.tex"
%%% End:

\fi

%
% Final
%
%---------------------------------------------------------------------
%
%                      fin.tex
%
%---------------------------------------------------------------------
%
% fin.tex
% Copyright 2009 Marco Antonio Gomez-Martin, Pedro Pablo Gomez-Martin
%
% This file belongs to the TeXiS manual, a LaTeX template for writting
% Thesis and other documents. The complete last TeXiS package can
% be obtained from http://gaia.fdi.ucm.es/projects/texis/
%
% Although the TeXiS template itself is distributed under the 
% conditions of the LaTeX Project Public License
% (http://www.latex-project.org/lppl.txt), the manual content
% uses the CC-BY-SA license that stays that you are free:
%
%    - to share & to copy, distribute and transmit the work
%    - to remix and to adapt the work
%
% under the following conditions:
%
%    - Attribution: you must attribute the work in the manner
%      specified by the author or licensor (but not in any way that
%      suggests that they endorse you or your use of the work).
%    - Share Alike: if you alter, transform, or build upon this
%      work, you may distribute the resulting work only under the
%      same, similar or a compatible license.
%
% The complete license is available in
% http://creativecommons.org/licenses/by-sa/3.0/legalcode
%
%---------------------------------------------------------------------
%
% Contiene la última página
%
%---------------------------------------------------------------------


% Ponemos el marcador en el PDF
\ifpdf
   \pdfbookmark{Fin}{fin}
\fi

\thispagestyle{empty}\mbox{}

\vspace*{4cm}

\small

\hfill \emph{--¿Qué te parece desto, Sancho? -- Dijo Don Quijote --}

\hfill \emph{Bien podrán los encantadores quitarme la ventura,}

\hfill \emph{pero el esfuerzo y el ánimo, será imposible.}

\hfill 

\hfill \emph{Segunda parte del Ingenioso Caballero} 

\hfill \emph{Don Quijote de la Mancha}

\hfill \emph{Miguel de Cervantes}

\vfill%space*{4cm}

\hfill \emph{--Buena está -- dijo Sancho --; fírmela vuestra merced.}

\hfill \emph{--No es menester firmarla -- dijo Don Quijote--,}

\hfill \emph{sino solamente poner mi rúbrica.}

\hfill 

\hfill \emph{Primera parte del Ingenioso Caballero} 

\hfill \emph{Don Quijote de la Mancha}

\hfill \emph{Miguel de Cervantes}


\newpage
\thispagestyle{empty}\mbox{}

\newpage

% Variable local para emacs, para  que encuentre el fichero maestro de
% compilación y funcionen mejor algunas teclas rápidas de AucTeX

%%%
%%% Local Variables:
%%% mode: latex
%%% TeX-master: "../Tesis.tex"
%%% End:

%\end{otherlanguage}
\end{document}
