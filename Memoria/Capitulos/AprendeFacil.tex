\chapter{Aprende Fácil}
\label{cap:aprendeFacil}




%-------------------------------------------------------------------
\section{Diseño de la Aplicación}
%-------------------------------------------------------------------
\label{cap:sec:disenioInterfaz}
Como ya se ha explicado en capítulos anteriores, esta aplicación está creada por y para personas que tienen alguna discapacidad cognitiva, por lo que el diseño de la interfaz debe ser sencillo y su uso no debe llevar a confusión ni debe hacer sentir al usuario frustrado al usarla. En definitiva, debe cumplir con las Ocho Reglas de Oro del diseño de interfaces, pero simplificándola aún más de lo que podría ser un diseño para otro tipo de aplicación destinada para otro tipo de usuario. 
Las Ocho Reglas de Oro del diseño de interfaces consisten en que el diseño tenga \citep{ochoreglasdeoro}:
\begin{itemize} 
	\item Consistencia: La funcionalidad debe ser similar a otras aplicaciones que el usuario está acostumbrado a utilizar. En cuanto a la interfaz debe tener los mismos colores, iconos, formas, botones, mensajes de aviso... Por ejemplo, si el usuario está acostumbrado a que el botón de eliminar o cancelar sea rojo, no debemos añadirle uno de color verde.
	\item Usabilidad Universal: Debemos tener en cuenta las necesidades de los distintos tipos de usuario, como por ejemplo, atajos de teclado para un usuario experto o filtros de color para usuarios con deficiencias visuales.
	\item Retroalimentación activa: Por cada acción debe existir una retroalimentación legible y razonable por parte de la aplicación. Por ejemplo, si el usuario quiere guardar los datos obtenidos de la búsqueda, la aplicación debe informarle de si han sido guardados o no.
	\item Diálogos para conducir la finalización: El usuario debe saber en que paso se encuentra en cada momento. Por ejemplo, en un proceso de compra que conlleva varios pasos hasta la finalización de la misma, se le debe informar donde se encuentra y cuánto le queda para terminar.
	\item Prevención de errores: La interfaz debe ayudar al usuario a no cometer errores serios, y en caso de cometerlos se le debe dar una solución lo más clara y sencilla posible. Por ejemplo, deshabilitando opciones o indicando en un formulario el campo en el cual se ha producido el error sin perder la información ya escrita.
	\item Deshacer acciones fácilmente: Se debe dar al usuario la capacidad de poder deshacer o revertir acciones de una manera sencilla. 
	\item Sensación de control: Hay que dar al usuario la sensación de que tiene en todo momento el control de la aplicación, añadiendo contenidos fáciles de encontrar y de esta forma no causarle ansiedad o frustración por utilizar nuestra aplicación.
	\item Reducir la carga de memoria a corto plazo: La interfaz debe ser lo más sencilla posible y con una jerarquía de información evidente, es decir, hay que minimizar la cantidad de elementos a memorizar por el usuario.
\end{itemize}
Teniendo en cuenta estas ocho reglas, la creación del diseño de la Interfaz se ha realizado en dos iteraciones distintas. Una primera iteración competitiva entre los integrantes del grupo y una segunda iteración con expertos del Colegio Estudio3 Afanias. 

%-------------------------------------------------------------------
\subsection{Primera Iteración: Iteración Competitiva}
%-------------------------------------------------------------------
\label{cap:subsec:iteracionCompetitiva}

En esta primera iteración cada integrante del grupo ha realizado su propio diseño de la aplicación. En la realización de estos diseños, los integrantes no podían hablar entre ellos ni comentar las diferentes ideas que tenían para la implementación. De esta forma se consigue que los diseños sean totalmente dispares y que las ideas de uno no provoquen la modificación del diseño del otro y que surjan ideas distintas.
Se realizaron cuatro prototipos distintos ya que no teníamos claro si los usuarios finales iban a preferir que se mostrara un solo significado o todos, si mostrar la definición y el ejemplo les iba a ayudar o no y si añadir los pictogramas les ayudaría o no.
Los prototipos fueron los siguientes:
\begin{itemize}
	\item Prototipo 1: Se muestra un único significado, siendo este el más común e incluyendo las comparaciones.
	\item Prototipo 2: Se muestra el significado más común, junto con una definición tradicional del concepto e incluyendo las comparaciones.
	\item Prototipo 3: Se muestran todos los significados de la palabra buscada e incluyendo las comparaciones.
	\item Prototipo 4: Se muestran todos los significados de la palabra y se añaden pictogramas\footnote{Se entiende como pictograma un dibujo, imagen o figura que representa el significado de una palabra.} a las palabras, haciendo así que la comprensión del concepto sea mucho mas sencilla. Tanto en el prototipo diseñado por Pablo como en el de Irene se han utilizado los pictogramas de ARASAAC\footnote{http://www.arasaac.org/}.
	
\end{itemize}

En las Figuras \ref{fig:mockup1pablo}, \ref{fig:mockup2pablo}, \ref{fig:mockup3pablo} y \ref{fig:mockup4pablo} se muestran los prototipos creados por Pablo y en las Figuras \ref{fig:mockup1irene_vInicial}, \ref{fig:mockup2irene_vInicial},  \ref{fig:mockup3irene_vInicial} y \ref{fig:mockup4irene_vInicial} los prototipos creados por Irene.

Una vez que los prototipos estaban terminados, lnos juntamos para hacer una puesta en común y analizar los prototipos creados. 
Por lo general, los prototipos de los dos integrantes del grupo eran bastantes similares, las principales diferencias eran:

\begin{itemize}
	\item Ambos prototipos integran los mismos elementos: un campo de texto para introducir la palabra, un botón de búsqueda y los resultados se muestran en forma de lista, agrupando los distintos resultados en rectángulos, que a partir de ahora denominaremos fichas.
	Pablo ha optado por formas rectangulares, tanto para las fichas como para el campo de texto, mientras que Irene utiliza formas redondeadas en todos los elementos. 
	Por otro lado, Pablo implementó un diseño orientado a niños por lo que para el color utilizó azules muy suaves y fondos juveniles con lápices de colores, gomas de borrar, etc.  Irene utilizó un color mostaza, siendo un diseño más simple pero intentando abarcar a un usuario de cualquier edad. Por último, los resultados del prototipo de Irene, se muestran en color mostaza indicando de esta forma que se pueden pinchar en ellos, y se realizará la búsqueda del concepto pulsado.
	
	\item El orden a la hora de mostrar los resultados es distinto. En los prototipos de Pablo se muestran primero las metáforas (\textit{Un vehículo es una máquina} o \textit{Un vehículo es un transporte}), después los símiles (\textit{Un vehículo es como un taxi}) y finalmente las analogías (\textit{Un vehículo es rápido como un caballo}). En los prototipos de Irene se muestra únicamente una analogía, una metáfora y un símil para cada significado de la palabra buscada.
	
	\item En el prototipo 3 de Pablo incluye justo debajo de los resultados mostrados un ejemplo siempre visible (``Él necesita un coche para ir a trabajar'') para facilitar al usuario la comprensión del término buscado.
	
	 \item En el prototipo 3 de Irene incluye justo debajo de los resultados mostrados una definición dentro de un botón (``Vehículo motorizado de cuatro ruedas por lo general impulsado por un motor de combustión interna'') dando a elegir al usuario la decisión de poder verla o no.
	
	\item Pablo decidió añadir el título ``Un X es ...'' antes de la lista de los resultados, englobando de esta forma los conceptos que tienen el mismo significado. Si existen varias acepciones del concepto, añade el título ``O también puede ser...'' en los siguientes, dejando claramente divididos los distintos significados que pudieran existir para el concepto buscado. En cambio Irene, añade un único título al principio (``Resultados para la palabra X'') donde queda claro que los resultados que se obtienen son para dicho concepto y además añade delante de cada metáfora: ``Un X es...''  y delante de cada símil y analogía: ``Un X es como...''.

	\item Para la numeración de los resultados obtenidos, Pablo incluye dentro de cada ficha un listado numérico e Irene únicamente añade un número a la ficha.
	
	\item En el prototipo donde se incluyen también los pictogramas, Pablo añade un pictograma que hace referencia al concepto buscado según el contexto en que se utilice. Por ejemplo, un vehículo puede hacer referencia a un coche o puede ser un medio para llegar o lograr un fin, por lo que el añade el pictograma en función de su significado, e Irene además de incluir el mismo pictograma que Pablo, añade pictogramas a las palabras usadas en las comparaciones. Por ejemplo, en la frase ``Un portero es un vigilante'' Irene añadió el pictograma de vigilante.
\end{itemize}



Una vez analizados los prototipos de ambos integrantes nos reunimos con los directores del trabajo y se decidió crear un prototipo con las siguientes características:


\begin{itemize}
	\item Para el diseño de la interfaz se eligió el de Irene por tener una interfaz más atractiva.
	\item Se eligió la palabra Portero para utilizarla como ejemplo en todos los prototipos.
	\item El orden de las resultados se modificó, haciendo que primero aparezcan las metáforas (``Un portero es un vigilante''), después los símiles (``Un portero es como un guardia'') y por último las analogías  (``Un portero es tan ágil como un pájaro'').
	\item Se eliminó la palabra \textit{tan} en la plantilla usada para las analogías.
	\item Se creó un nombre para la aplicación que estuviese en español: ``Aprende Fácil''.
	\item En vez de mostrar un único símil, una metáfora y una analogía. Se deben mostrar todos los resultados obtenidos para dicho concepto.
	\item Se decidió que si la palabra solo tenía un significado se eliminaba el número de la ficha.
	\item Se añadió un pictograma a la característica que relaciona el concepto con el resultado en las analogías. Por ejemplo, en la frase \textit{Un portero es tan fuerte como un gorila}  se añadió el pictograma correspondiente a fuerte.
	\item Dejar el ejemplo del prototipo de Pablo junto con la definición del prototipo de Irene.
	\item Crear un nuevo prototipo donde la definición y el ejemplo se muestren a primera vista, y el usuario no tenga que estar pinchando en ningún botón para que los expertos nos indiquen cual es la mejor opción.
	\item Crear otro prototipo donde la definición y el ejemplo estén separados. Así el usuario puede elegir lo que quiere ver, si solo una cosa o ambas.
\end{itemize} 

Con todas estas decisiones, se crearon los prototipos que se muestran en las Figuras \ref{fig:mockup1irene_vFinal}, \ref{fig:mockup2irene_v1_vFinal}, \ref{fig:mockup2irene_v2_vFinal}, \ref{fig:mockup2irene_v3_vFinal}, \ref{fig:mockup3irene_vFinal}, \ref{fig:mockup4irene_vFinal} y que fueron los que se usaron para en la siguiente iteración donde nos reunimos con los expertos para que nos dieran su opinión sobre la aplicación, y nos ayudarán a decidir sobre los siguientes aspectos:
\begin{itemize}
	\item ¿Se debe mostrar el significado más común o todos los significados de la palabra buscada?
	\item ¿Se debe añadir la definición tradicional y el ejemplo junto con las figuras retóricas o no?  Y en caso de añadirlos, ¿Se deben mostrar juntos en un mismo botón, en distintos botones o sin botón y que aparezcan debajo de los resultados?
	\item ¿Son útiles los pictogramas? En caso de serlo, ¿solamente el pictograma de la palabra buscada o también añadimos los pictogramas de las palabras que se usan en las figuras retóricas?
	\item ¿La forma de mostrar los resultados es correcta, o es mejor otra disposición?
	
\end{itemize} 


 	\figura{Bitmap/Mockups/mockup1_pablo}{width=1.0\textwidth}{fig:mockup1pablo}{Prototipo de Pablo mostrando resultado más común para la palabra vehículo}
	\figura{Bitmap/Mockups/mockup2_pablo}{width=.9\textwidth}{fig:mockup2pablo}{Prototipo de Pablo mostrando resultado más común para la palabra vehículo y un ejemplo} 
	\figura{Bitmap/Mockups/mockup3_pablo}{width=.9\textwidth}{fig:mockup3pablo}{Prototipo de Pablo mostrando todos los resultados para la palabra vehículo} 
 	\figura{Bitmap/Mockups/mockup4_pablo}{width=.9\textwidth}{fig:mockup4pablo}{Prototipo de Pablo mostrando todos los resultados para la palabra vehículo junto con pictogramas} 
 	
 	\figura{Bitmap/Mockups/mockup1_irene_inicial.png}{width=1.2\textwidth}{fig:mockup1irene_vInicial}{Prototipo de Irene mostrando resultado más común para la palabra vehículo}
 	\figura{Bitmap/Mockups/mockup2_irene_inicial.png}{width=1.2\textwidth}{fig:mockup2irene_vInicial}{Prototipo de Irene mostrando resultado más común para la palabra vehículo y una definición}
	\figura{Bitmap/Mockups/mockup3_irene_inicial.png}{width=1.2\textwidth}{fig:mockup3irene_vInicial}{Prototipo de Irene mostrando todos los resultados para la palabra portero} 
	\figura{Bitmap/Mockups/mockup4_irene_inicial.png}{width=1.2\textwidth}{fig:mockup4irene_vInicial}{Prototipo de Irene mostrando todos los resultados para la palabra portero junto con pictogramas} 
	
	\figura{Bitmap/Mockups/mockup1_irene_final.png}{width=1.2\textwidth}{fig:mockup1irene_vFinal}{Prototipo mostrando resultado más común para la palabra portero}
	\figura{Bitmap/Mockups/mockup2_irene_final_v1.png}{width=1.2\textwidth}{fig:mockup2irene_v1_vFinal}{Prototipo Versión 1 mostrando resultado más común para la palabra portero junto con una definición y un ejemplo separados}
	\figura{Bitmap/Mockups/mockup2_irene_final_v2.png}{width=1.2\textwidth}{fig:mockup2irene_v2_vFinal}{Prototipo Versión 2 mostrando resultado más común para la palabra portero junto con una definición y un ejemplo a simple vista}
	\figura{Bitmap/Mockups/mockup2_irene_final_v3.png}{width=1.2\textwidth}{fig:mockup2irene_v3_vFinal}{Prototipo Versión 3 mostrando resultado más común para la palabra portero junto con una definición y un ejemplo en un mismo botón}
	\figura{Bitmap/Mockups/mockup3_irene_final.png}{width=1.2\textwidth}{fig:mockup3irene_vFinal}{Prototipo mostrando todos los resultados para la palabra portero}
	\figura{Bitmap/Mockups/mockup4_irene_final.png}{width=1.2\textwidth}{fig:mockup4irene_vFinal}{Prototipo mostrando todos los resultados para la palabra portero incluyendo los pictogramas}

	
	 
%-------------------------------------------------------------------
\subsection{Segunda Iteración: Evaluación con Expertos}
%-------------------------------------------------------------------
\label{cap:subsec:evaluacionExpertos}

El día 26 de Marzo de 2019 a las 09:00h nos reunimos con la directora y los profesores del colegio Estudio3 Afanias\footnote{https://afanias.org/que-hacemos/educacion/colegio-estudio-3/} situado en la Comunidad de Madrid. Este colegio atiende a niños y jóvenes con discapacidad intelectual entre los 3 y 21 años de edad.

Una vez expuestos los distintos prototipos, nos dieron la enhorabuena y nos hicieron participes de la gran ayuda que supondría esta herramienta. A continuación nos dieron su opinión sobre distintos aspectos que se podrían modificar de los prototipos:

\begin{itemize} 
	\item Modificar el color amarillo-mostaza de los textos por un color más oscuro que contraste más con el fondo blanco.
	\item El tipo de letra debe ser Arial o Script, ya que son las letras con las que los alumnos están familiarizados y las que mejor entienden.
	\item Añadir un reproductor que lea la frase, haciendo así que en caso de no poder leer los resultados, pueda ayudarles la voz.
	\item Incluir la opción de poder ver un video por si los pictos no ayudan al usuario a entender el concepto.
	\item Introducción de distintas personalizaciones:
	\begin{itemize}
		\item Búsqueda en tres niveles: sencillo, medio y amplio. El nivel sencillo sería realizando la búsqueda de las palabras fáciles en las 1.000 palabras más usadas de la RAE, el medio con las 5.000 palabras más usadas de la RAE y el amplio con las 10.000 palabras más usadas de la RAE. 
		\item Uso de mayúsculas en todos los textos para hacer la aplicación más accesible a aquellos alumnos que no entienden los textos en minúsculas.
		\item Que el usuario elija si quiere que aparezca la definición y el ejemplo o no.
		\item Poder elegir si aparecen los pictos o no. En caso de que el usuario decida que si deben aparecer, estos deben situarse debajo de la palabra y no al lado, ya que los expertos comentaron que esto puede llevar a confusión a los alumnos pensando que sería otra palabra más para leer. 
	\end{itemize}
\end{itemize}

Teniendo en cuenta las observaciones de los expertos se creo el diseño final de la aplicación.
