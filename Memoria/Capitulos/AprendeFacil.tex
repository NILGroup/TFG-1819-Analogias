\chapter{Aprende Fácil}
\label{cap:aprendeFacil}




%-------------------------------------------------------------------
\section{Diseño de la Interfaz}
%-------------------------------------------------------------------
\label{cap:sec:disenioInterfaz}
Como ya se ha explicado en capítulos anteriores, está aplicación está creada por y para personas que tienen alguna discapacidad cognitiva, por lo que el diseño de la Interfaz debe estar centrado para este tipo de usuarios, es decir, debe ser lo más sencilla posible, su uso no debe llevar a confusión ni debe hacer sentir al usuario frustrado al usarla. En definitiva, debe cumplir con las Ocho Reglas de Oro del diseño de interfaces, pero simplificándola aún más de lo que podría ser un diseño para otro tipo de aplicación destinada para un usuario distinto. 
Las Ocho Reglas de Oro del diseño de interfaces consisten en que el diseño tenga:
\begin{itemize} 
	\item Consistencia: La funcionalidad debe ser similar a otras aplicaciones que el usuario está acostumbrado a utilizar. En cuanto a la interfaz debe tener los mismos colores, iconos, formas, botones, mensajes de aviso... Por ejemplo, si el usuario está acostumbrado a que el botón de eliminar o cancelar sea rojo, no debemos añadirle uno de color verde.
	\item Usabilidad Universal: Debemos tener en cuenta las necesidades de los distintos tipos de usuario, como por ejemplo, atajos de teclado para un usuario experto o filtros de color para usuarios con deficiencias visuales.
	\item Retroalimentación activa: Por cada acción debe existir una retroalimentación legible y razonable por parte de la aplicación. Por ejemplo, si el usuario quiere guardar los datos obtenidos de la búsqueda, la aplicación debe informarle de si han sido guardados o no.
	\item Diálogos para conducir la finalización: El usuario debe saber en que paso se encuentra en cada momento. Por ejemplo, en un proceso de compra que conlleva varios pasos hasta la finalización de la misma, se le debe informar donde se encuentra y cuánto le queda para terminar.
	\item Prevención de errores: La interfaz debe ayudar al usuario a no cometer errores serios, y en caso de cometerlos se le debe dar una solución lo más clara y sencilla posible. Por ejemplo, deshabilitando opciones o indicando en un formulario el campo en el cual se ha producido el error sin perder la información ya escrita.
	\item Deshacer acciones fácilmente: Se debe dar al usuario la capacidad de poder deshacer o revertir acciones de una manera sencilla. 
	\item Sensación de control: Hay que dar al usuario la sensación de que tiene en todo momento el control de la aplicación, añadiendo contenidos fáciles de encontrar y de esta forma no causarle ansiedad o frustración por utilizar nuestra aplicación.
	\item Reducir la carga de memoria a corto plazo: La interfaz debe ser lo más sencilla posible y con una jerarquía de información evidente, es decir, hay que minimizar la cantidad de elementos a memorizar por el usuario.
\end{itemize}
Es por ello, que para la creación del diseño de la Interfaz se ha realizado en dos iteracciones distintas. Una primera iteracción competitiva entre los integrantes de dicho trabajo y una segunda iteracción con expertos del Colegio Estudio3 Afanias situado en Madrid. Este colegio se basa en la educación especial para la discapacidad intelectual, trastornos generalizados del desarrollo Autismo EBO Infantil Transición a la Vida Adulta.

%-------------------------------------------------------------------
\subsection{Primera Iteración: Iteración Competitiva}
%-------------------------------------------------------------------
\label{cap:subsec:iteracionCompetitiva}

Esta primera iteración se trata de realizar los distintos prototipos de diseño de la aplicación. Por ello, cada integrante del grupo ha realizado cuatro diseños distintos. En la realización de estos, los integrantes no podían hablar entre ellos ni comentar las diferentes ideas que tenían para la  implementación, de esta forma se consigue que los diseños sean totalmente dispares, que las ideas de uno no provoquen la modificación del diseño de otro.
Una vez que los prototipos estaban terminados, los integrantes de este trabajo se juntaron para hacer una puesta en común y analizar los resultados. A continuación, se explicarán los resultados de los análisis indicando las semejanzas y diferencias que se encontraron.

Como se ha comentado anteriormente, cada integrante realizó cuatro prototipos distintos:
\begin{itemize}
	\item Un primer diseño mostrando únicamente el resultado que más se utiliza.
	\item Un segundo diseño mostrando únicamente el resultado más utilizado pero añadiendo una definición del concepto más un ejemplo.
	\item Un tercer diseño mostrando todos los resultados, pero esta vez sin añadir la definición ni el ejemplo.
	\item Y por útimo, un cuarto diseño mostrando todos los resultados y añadiendo pictos que facilitan aún más la comprensión del concepto.
	
\end{itemize}

Por lo general, los cuatro prototipos son bastantes similares, como se puede ver en la Figura \ref{fig:mockup1pablo} y en la Figura ...., ambos tienen un campo de texto para introducir una palabra y un botón de buscar que realiza la búsqueda correspondiente. Y ambos muestran los resultados en forma de lista. La única diferencia que existen entre ellos, es el diseño de colores y formas.

En la Figura \ref{fig:mockup2pablo} se puede ver el diseño para mostrar el resultado más común junto con un ejemplo y en la Figura ....... se muestra el resultado más común junto con la definición. Está sería la única diferencia entre ellos junto con el diseño de colores y formas, ya que como pasa en el anterior protototipo los elementos que disponen la vista son iguales, existe un buscador del concepto, un botón y los resultados en formato de lista.


 	\figura{Bitmap/Mockups/mockup1_pablo}{width=.9\textwidth}{fig:mockup1pablo}{Prototipo de Pablo mostrando resultado más común para la palabra vehículo}
	\figura{Bitmap/Mockups/mockup2_pablo}{width=.9\textwidth}{fig:mockup2pablo}{Prototipo de Pablo mostrando resultado más común para la palabra vehículo y un ejemplo} 
%-------------------------------------------------------------------
\subsection{Segunda Iteración: Evaluación con Expertos}
%-------------------------------------------------------------------
\label{cap:subsec:evaluacionExpertos}