\chapter{Aprende Fácil}
\label{cap:aprendeFacil}




%-------------------------------------------------------------------
\section{Diseño de la Interfaz}
%-------------------------------------------------------------------
\label{cap:sec:disenioInterfaz}
Como ya se ha explicado en capítulos anteriores, está aplicación está creada por y para personas que tienen alguna discapacidad cognitiva, por lo que el diseño de la Interfaz debe estar centrado para este tipo de usuarios, es decir, debe ser lo más sencilla posible, su uso no debe llevar a confusión ni debe hacer sentir al usuario frustrado al usarla. En definitiva, debe cumplir con las Ocho Reglas de Oro del diseño de interfaces, pero simplificándola aún más de lo que podría ser un diseño para otro tipo de aplicación destinada para otro tipo de usuario. 
Las Ocho Reglas de Oro del diseño de interfaces consisten en que el diseño tenga:
\begin{itemize} 
	\item Consistencia: La funcionalidad debe ser similar a otras aplicaciones que el usuario está acostumbrado a utilizar. En cuanto a la interfaz debe tener los mismos colores, iconos, formas, botones, mensajes de aviso... Por ejemplo, si el usuario está acostumbrado a que el botón de eliminar o cancelar sea rojo, no debemos añadirle uno de color verde.
	\item Usabilidad Universal: Debemos tener en cuenta las necesidades de los distintos tipos de usuario, como por ejemplo, atajos de teclado para un usuario experto o filtros de color para usuarios con deficiencias visuales.
	\item Retroalimentación activa: Por cada acción debe existir una retroalimentación legible y razonable por parte de la aplicación. Por ejemplo, si el usuario quiere guardar los datos obtenidos de la búsqueda, la aplicación debe informarle de si han sido guardados o no.
	\item Diálogos para conducir la finalización: El usuario debe saber en que paso se encuentra en cada momento. Por ejemplo, en un proceso de compra que conlleva varios pasos hasta la finalización de la misma, se le debe informar donde se encuentra y cuánto le queda para terminar.
	\item Prevención de errores: La interfaz debe ayudar al usuario a no cometer errores serios, y en caso de cometerlos se le debe dar una solución lo más clara y sencilla posible. Por ejemplo, deshabilitando opciones o indicando en un formulario el campo en el cual se ha producido el error sin perder la información ya escrita.
	\item Deshacer acciones fácilmente: Se debe dar al usuario la capacidad de poder deshacer o revertir acciones de una manera sencilla. 
	\item Sensación de control: Hay que dar al usuario la sensación de que tiene en todo momento el control de la aplicación, añadiendo contenidos fáciles de encontrar y de esta forma no causarle ansiedad o frustración por utilizar nuestra aplicación.
	\item Reducir la carga de memoria a corto plazo: La interfaz debe ser lo más sencilla posible y con una jerarquía de información evidente, es decir, hay que minimizar la cantidad de elementos a memorizar por el usuario.
\end{itemize}
Teniendo en cuenta estas ocho reglas, la creación del diseño de la Interfaz se ha realizado en dos iteracciones distintas. Una primera iteracción competitiva entre los integrantes de dicho trabajo y una segunda iteracción con expertos del Colegio Estudio3 Afanias. 

%-------------------------------------------------------------------
\subsection{Primera Iteración: Iteración Competitiva}
%-------------------------------------------------------------------
\label{cap:subsec:iteracionCompetitiva}

Esta primera iteración se trata de realizar los distintos prototipos de diseño de la aplicación. Por ello, cada integrante del grupo ha realizado cuatro diseños distintos. En la realización de estos, los integrantes no podían hablar entre ellos ni comentar las diferentes ideas que tenían para la implementación, de esta forma se consigue que los diseños sean totalmente dispares, que las ideas de uno no provoquen la modificación del diseño de otro.
Una vez que los prototipos estaban terminados, los integrantes de este trabajo se juntaron para hacer una puesta en común y analizar los resultados. A continuación, se explicarán los resultados de los análisis indicando las semejanzas y diferencias que se encontraron.

Como se ha comentado anteriormente, cada integrante realizó cuatro prototipos distintos:
\begin{itemize}
	\item Un primer diseño mostrando únicamente el resultado que más se suele utilizar en la sociedad.
	\item Un segundo diseño mostrando el resultado más utilizado pero añadiendo una definición del concepto y/o un ejemplo.
	\item Un tercer diseño mostrando todos los resultados, pero esta vez sin añadir la definición ni el ejemplo.
	\item Y por útimo, un cuarto diseño mostrando todos los resultados y añadiendo pictos que facilitan aún más la comprensión del concepto.
	
\end{itemize}

Por lo general, los cuatro prototipos son bastantes similares, como se puede ver en la Figura \ref{fig:mockup1pablo} los resultados para la palabra vehículo se muestran en el siguiente orden: primero varios sinónimos e hiperónimos (\textit{Un vehículo es una máquina} o \textit{Un vehículo es un transporte}), después los hipónimos (\textit{Un vehículo es como un taxi}) y finalmente los hipónimos pero añadiendo un adjetivo que representa una característica del resultado devuelto (\textit{Un vehículo es rápido como un caballo}). Por otro lado, en la Figura \ref{fig:mockup1irene} los resultados para la palabra portero se devuelven en un orden distinto y no muestra varios sinónimos, si no que en primer lugar aparecen los hipónimos junto con el adjetivo que le caracteriza (\textit{Un portero es fuerte como un gorila}), después se muestran los sinónimos e hiperónimos (\textit{Un portero es un vigilante}) y por último aparecen los hipónimos sin el adjetivo (\textit{Un portero es como un guardia}). Otra diferencia que existen entre ellos, es el diseño de colores y formas, y en cuánto a las similitudes ambos disponen de un campo de texto para introducir una palabra, un botón de búsqueda y se muestran los resultados en forma de lista.

En cuanto al segundo diseño, en la Figura \ref{fig:mockup2pablo} se puede ver que los resultados para la palabra vehículo aparecen de la misma forma que en el diseño anterior incluyendo un ejemplo (\textit{Él necesita un coche para ir a trabajar}) y en la Figura \ref{fig:mockup2irene} se muestra el resultado más común para la palabra portero pero añadiendo una definición (\textit{Un portero es alguien que protege una entrada}). Esta sería la gran diferencia entre ambos prototipos, proporcionando al usuario más opciones para elegir, el diseño de colores y formas también son distintos pero los elementos que disponen la vista son iguales, es decir, existe un buscador del concepto, un botón y los resultados se muestran en forma de lista.

Respecto al tercer diseño, donde se muestran todos los resultados pero sin añadir definición ni ejemplo, se puede ver en la Figura \ref{fig:mockup3pablo} que se muestran en dos listas distintas añadiendo un título aclaratorio (\textit{Un vehículo es...} y \textit{O también puede ser...}) y en cambio como se puede ver en la Figura \ref{fig:mockup3irene} se muestran en listas identificando cada resultado con un número delante.

Y por último, para el cuarto diseño, en la Figura \ref{fig:mockup4pablo} se muestra los resultados como en los anteriores diseños pero añadiendo un picto que hace referencia al concepto según en que contexto se utilice. Por ejemplo, un vehículo puede hacer referencia a un coche o puede ser un medio para llegar o lograr un fin, en cambio, en la Figura \ref{fig:mockup4irene} se han añadido los pictos del concepto portero según el contexto que se utilice (portero de discoteca o portero de jugador) y además se han añadido pictos a cada resultado. Por ejemplo, en el resultado \textit{Un portero es fuerte como un gorila}, se añade un picto a la palabra gorila. 
Tanto en el prototipo diseñado por Pablo como en el de Irene se han utilizado los pictos de ARASAAC\footnote{http://www.arasaac.org/}.

Finalmente, tras realizar dicho análisis, se decidió que para la evaluación con los expertos, que se explicará en el siguiente apartado, utilizaríamos el diseño de colores y formas de los prototipos de Irene pero modificando ciertos aspectos y añadiendo ideas de los prototipos de Pablo. Por ejemplo, se hicieron tres versiones del segundo diseño para añadir tanto la definición como el ejemplo, en la Figura \ref{fig:mockup2_v1_irene} podemos ver que al prototipo original se le añadió otro botón para poder ver el ejemplo, de esta forma el usuario puede elegir entre ver una cosa u otra, pero no ambas a la vez. En la Figura \ref{fig:mockup2_v2_irene} se quitarón los botones y tanto la definición como el ejemplo aparecen justo debajo de los resultados y por último en la Figura \ref{fig:mockup2_v3_irene} existe un único botón en el cuál si el usuario pincha aparecen la definición y el ejemplo.
Otra modificación que se realizó fue que los resultados deben aparecer en el siguiente orden:

\begin{itemize} 
	\item Primero: los sinónimos e hiperónimos, como por ejemplo \textit{Un portero es un vigilante}.
	\item Segundo: los hipónimos, como por ejemplo \textit{Un portero es como un guardia}.
	\item Tercero: añadiendo un adjetivo que permita realizar una comparación con un hipónimo, por ejemplo \textit{Un portero es grande como un oso}.
\end{itemize}

Y por último, como se puede ver en la Figura \ref{fig:mockup4_vFinal_irene} se añadieron pictos también a los adjetivos que caracterizan a la palabra, por ejemplo en la frase \textit{Un portero es rápido como un caballo}, se han añadido pictos a rápido y a caballo.



 	\figura{Bitmap/Mockups/mockup1_pablo}{width=.9\textwidth}{fig:mockup1pablo}{Prototipo de Pablo mostrando resultado más común para la palabra vehículo}
	\figura{Bitmap/Mockups/mockup1_irene}{width=.9\textwidth}{fig:mockup1irene}{Prototipo de Irene mostrando resultado más común para la palabra portero}
	\figura{Bitmap/Mockups/mockup2_pablo}{width=.9\textwidth}{fig:mockup2pablo}{Prototipo de Pablo mostrando resultado más común para la palabra vehículo y un ejemplo} 
	\figura{Bitmap/Mockups/mockup2_irene}{width=.9\textwidth}{fig:mockup2irene}{Prototipo de Irene mostrando resultado más común para la palabra vehículo y una definición}
	\figura{Bitmap/Mockups/mockup3_pablo}{width=.9\textwidth}{fig:mockup3pablo}{Prototipo de Pablo mostrando todos los resultados para la palabra vehículo} 
	\figura{Bitmap/Mockups/mockup3_irene}{width=.9\textwidth}{fig:mockup3irene}{Prototipo de Irene mostrando todos los resultados para la palabra portero} 
	\figura{Bitmap/Mockups/mockup4_pablo}{width=.9\textwidth}{fig:mockup4pablo}{Prototipo de Pablo mostrando todos los resultados para la palabra vehículo junto con pictos} 
	\figura{Bitmap/Mockups/mockup4_irene}{width=.9\textwidth}{fig:mockup4irene}{Prototipo de Irene mostrando todos los resultados para la palabra portero junto con pictos} 
	
	
	\figura{Bitmap/Mockups/mockup2_v1_irene}{width=.9\textwidth}{fig:mockup2_v1_irene}{Prototipo Versión 1 mostrando resultado más común para la palabra vehículo junto con un ejemplo y una definición}
	\figura{Bitmap/Mockups/mockup2_v2_irene}{width=.9\textwidth}{fig:mockup2_v2_irene}{Prototipo Versión 2 mostrando resultado más común para la palabra vehículo junto con un ejemplo y una definición}
	\figura{Bitmap/Mockups/mockup2_v3_irene}{width=.9\textwidth}{fig:mockup2_v3_irene}{Prototipo Versión 3 mostrando resultado más común para la palabra vehículo junto con un ejemplo y una definición}
	\figura{Bitmap/Mockups/mockup4_vFinal_irene}{width=.9\textwidth}{fig:mockup4_vFinal_irene}{Prototipo Final  mostrando todos los resultados para la palabra portero junto con pictos}
	
	
	 
%-------------------------------------------------------------------
\subsection{Segunda Iteración: Evaluación con Expertos}
%-------------------------------------------------------------------
\label{cap:subsec:evaluacionExpertos}

Una vez realizadas las pequeñas modificaciones de los prototipos que nuestros directores consideraron que serían mejoras, tuvimos una reunión con la directora y los profesores del colegio Estudio3 Afanias situado en la Comunidad de Madrid. Este colegio se basa en la educación especial para la discapacidad intelectual, trastornos generalizados del desarrollo, autismo, EBO Infantil y ayuda con la transición a la vida adulta.
La reunión tuvo lugar el día 26 de Marzo de 2019 y acudimos los dos integrantes de este trabajo junto con los directores.
Una vez expuestos los distintos diseños, la primera impresión que tuvieron fue bastante optimista y nos dieron su opinión de lo que ellos creían que se debería añadir, eliminar o modificar para que su funcionalidad sea lo más amplia y sencilla posible tanto para su uso al impartir clase, como para el uso propio de los alumnos. 
Las conclusiones principales que obtuvimos de está evaluación han sido:
\begin{itemize} 
	\item Modificar el color amarillo-mostaza por un color más oscuro que contraste más con el fondo blanco.
	\item El tipo de letra debe ser Arial o Script, ya que son las letras con las que los alumnos están familiarizados y las que mejor entienden.
	\item Añadir un reproductor que lea la frase, haciendo así que en caso de no entender el resultado escrito, pueda ayudarles la voz.
	\item Incluir la opción de poder ver un video por si los pictos no aclaran al usuario a entender el concepto.
	\item Una parte configurable, donde se tenga en cuenta:
	\begin{itemize}
		\item La búsqueda puede realizarse en tres niveles: sencillo, medio y amplio. El nivel sencillo sería realizando la búsqueda de las palabras fáciles con las 1000 palabras de la RAE, el medio con las 5000 palabras de la RAE y el amplio con las 10000 palabras de la RAE. 
		\item Dar la opción de poder introducir mayúsculas para realizar la búsqueda.
		\item Que el usuario elija si quiere que aparezca la definición y el ejemplo o no.
		\item Poder elegir si deben aparecer los pictos o no. En caso de que el usuario decida que si deben aparecer, estos deben situarse debajo de la palabra y no al lado, ya que los expertos comentaron que esto puede llevar a confusión a los alumnos pensando que sería otra palabra más para leer. Y además, debería aparecer toda la frase traducida a pictos.
	\end{itemize}
\end{itemize}
