\chapter{Conclusions and Future Work}
\label{cap:conclusions_futureWork}

\section{Conclusions}
\label{sec:conclusions}

After completing the project, some conclusions have been obtained in relation to the objectives initially proposed.

The main goal of this work was to create a web application, based on web services, in which difficult words could be converted into easier words by using  rhetorical figures (metaphors, similes and analogies).

It is difficult to define what an easy concept means because is not the same for everyone. To do so, we have taken the RAE's 1.000, 5.000 and 10.000 most used words in the Spanish language.

The implementation of each web service has been made in a modular way and they are available in a public API, so they can be used by other developers at any point.

When we created our application interface, we used dark colours that stand out on plain backgrounds in order to make reading easy for the users. We consulted with professionals with vast experience in this field so they could give us advice on how to display the results for the words searched.

To make our application more useful we added some customizable settings for the user, such as:

\begin{itemize}
	\item To be able to convert the whole text into capital letter: this makes reading easier because that is how they learn to read and write.
	\item To be able to show pictograms if needed: using pictograms makes it easy to understand certain concepts.
	\item To be able to show definitons and examples for the results: if showing pictograms is not enough to understand a concept, the application can show them an example and definition.
\end{itemize}

We built this application from scratch, not using any external templates. This web application was tested by target users so we could prove its functionality and its interface. As we concluded these tests, we realized the application is useful for people with mild disabilities but it needs to be improved in order to make it useful for people with severe disabilities.

This project has been an opportunity to apply all the knowledge acquired
in different subjects taken during the degree to a large project with real impact. Among all of these subjects, we can highlight:

\begin{itemize}
	\item Data Structures and Algorithms, that helped us to acquire a structured and efficient way of thinking when programming.
	\item Foundations of the Programming and Technology of the Programming, which helped us understand the internal logic of the applications.
	\item Web Applications and User Interfaces, that helped us to acquire enough knowledge about html, css, Bootstrap, jQuery and javascript to develop the application and the basic rules about interface design.
	\item Databases and Extension of Databases to manage big volumes of data.
	\item Software Project Management, for the correct management of our application.
	\item System and Networks Management, that helped us to the configuration of our server.
	\item Software Engineering and Software Modeling, that helped us to design the application.
	\item Ethics, Legislation and Profession, where we learned about software licenses, both to protect our code and to know how to correctly use free software developed by third parties.
\end{itemize}

We have also acquired knowledge about some new things such us Django, Python, Latex and Natural Laguage Processing. We have learned about cognitive disabilities and we have got to spend some time with target users, which made this experience a really valuable one. We have realized the problems they face and the way they surpass them.

We consider we have achieved most of our main goals (section \ref{sec:goals}). The ones which could not be achieve will be explained in the next section.


\section{Future Work}
\label{sec:future_work}

We couldn't achieve every goal we aimed to at the beginning of the project. The main reason is the lack of time and the fact that we needed to cover so many fields. The items that can be developed as future work are:

\begin{itemize}
	\item To show results with analogies: Our application shows results with metaphors and similes. We think it would be useful for the user to show them analogies too. For that purpose, we would need to search information about how to link the concept and the result and which properties they share.
	
	\item To add voice synthesis: We should add an audio description of the selected item to make the application easier for the user.
	
	\item Explanatory video of each pictogram: a button could be added that, when clicked, shows an explanatory video.
	
	\item To show the most relevant result first: When the user has searched for a word, the most significant results should be shown first. This has been difficult to achieve, since the external resources used to mine for words do not provide any information that allows to decide what concepts are more relevant.
	
	\item The pictograms should be displayed with the correct meaning: in case of polysemic words, the pictogram should be shown with the correct relation between the result and the searched concept.
	
	\item We should be able to look for easier words and easier results: we need to find a way to obtain easier results rather than using RAE's 1.000, 5.000 and 10.000 lists of easy words.
	
	\item Develop a smartphone application: It would be useful for the user to be able to have this application in their smartphone instead of as a web application.
	
\end{itemize}

In the near future, we would like to keep working in this project, as during this year we have been involved in its development at a personal level, so we would like it to be an application that could be used by everyone.




