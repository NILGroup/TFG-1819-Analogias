\chapter{Conclusions and Future Work}
\label{cap:conclusions_futureWork}

\section{Conclusions}
\label{sec:conclusions}

After completing the project, some conclusions have been obtained in relation to the objectives initially proposed.

The main goal of this work was to create a web application based on web services, in which difficult words could be converted into easier words by using  retoric figures such as metaphores and similes.

It is difficult to define what an easy concept means because is not the same for everyone. To do so, we took the RAE's 1.000, 5.000 and 10.000 most used words in the spanish language.

The implementation of each web service has been made in a modular way and they are attainable in a public API, so they can be used by other developers at any point.

When we created our application interface, we used dark colours that stand out on plain backgrounds in order to make reading easy for the user. We consulted with professionals with vast experience in this field so they could give us advice on how to display the results for the words searched.

To make our application more useful we added some customable settings for the user, such as:

\begin{itemize}
	\item To be able to convert the whole text into capital letter: this makes reading easier because that is how they learn to read and write.
	\item To be able to show pictograms if needed: using pictograms makes easy to understand certain concepts.
	\item To be able to show definitons an examples for the results: If showing pictograms is not enough to understand a concept the application can show them an example and definition.
\end{itemize}

We built this application from zero, not using any external templates. This web application was tested by targeted users so we could prove its functionality and its interface. As we concluded this tests we realized the application is useful for people with mild dissability but it needs to be improven in order to make it useful for people with severe dissability.

This project has been an opportunity to apply all the knowledge acquired
in different subjects during the degree to a large project with real impact.
Among all of these subjects we can highlight:

\begin{itemize}
	\item Structure of data and Algorithms, that helped us to that helped us to
	acquire a structured and efficient way of thinking when programming.
	\item Foundations of the Programming and Technology of the Programming, which helped us understand internal logic of the application.
	\item Web applications and users interfaces that helped us to acquire enough knowledge about html, css, Bootstrap, jQuery and javascript to develop the application and basic rules about an interface.
	\item Databases and Extension of databases to manage multiple data.
	\item Software Project Management, for the correct management of our application.
	\item System and Networks Management, that helped us to the configuration of our server.
	\item Software Engineering and Software Modeling, that helped us to design the diagrams.
	\item Ethics, Legislation and Profession we learned everything necessary about licenses, both to protect our code
	and to know how to correctly use free software developed by third
	parties.
\end{itemize}

We also acquired knowledge about some new things such us Django, Python, Latex and Natural Laguage Processing. We learned about Cognitive disabilities and we got to spend some time with a target users which made this experience very good experience. We realize the problems they face and we way they surpass them.

We think we achieved most of our main goals (section \ref{sec:goals}). The ones which couldn't be achieve will be explained in the next section.

\section{Future Work}
\label{sec:future_work}

We couldn't achieve every goal we wanted. The main reason is the lack of time and we need to cover so many fields. The items that can be developed as future work are:

\begin{itemize}
	\item To show results with analogies: Our application shows results with metaphors and similes. We think It would be useful for the user to show them analogies too. To that purpose, we would need to search information about how to link the concept and the result and which properties they share.
	
	\item To add voice recognition: We should add an audio description of the selected item to make the application easier for the user.
	
	\item Explanatory video of each pictogram: we should add a button that when clicked shows an explanatory video.
	
	\item To show the most important result first: When the user has searched for the word, the most significant result should be shown first in each tab.
	
	\item The pictograms should be displayed with the correct sense: the pictogram should be shown with the relation between the result and the searched concept.
	
	\item We should de able to load for easier words and easier results: We need to find a way to obtain easier results rather than using RAE's 1.000, 5.000 and 10.000 lists of easy words.
	
	\item Develop a smartphone application: It would be useful for the user, to be able to have this application in their smartphone.
	
\end{itemize}

In the future, we'd to keep working in this project. We would like to be an application that could be used by everyone.




