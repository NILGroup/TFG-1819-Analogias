\chapter{Herramientas Utilizadas}
\label{cap:herramientas}

En este capítulo se van a explicar las herramientas utilizadas para el desarrollo de este trabajo, en el apartado 3.1 se explicará Django que es el \textit{framework} utilizado para el desarrollo del servicio web y en el apartado 3.2 se explicará SpaCy, que es la herramienta que se utilizó para la clasificación de palabras. Las herramientas y recursos que se utilizaron para la gestión del proyecto se explicarán en un capítulo aparte.


%-------------------------------------------------------------------
\section{Django}
%-------------------------------------------------------------------
\label{cap:sec:django}
Django es un \textit{framework} de alto nivel que permite el desarrollo rápido de sitios web seguros y mantenibles y se basa en el patrón MVC \citep{TFGEmociones}. Fue desarrollado entre los años 2003 y 2005 por un grupo de programadores que se encargaban de crear y mantener sitios web de periódicos. 
Es gratuito y de código abierto y dispone de una gran documentación actualizada así como muchas opciones de soporte gratuito y de pago. Gracias a la utilización de Django se puede crear un software\footnote{https://developer.mozilla.org/es/docs/Learn/Server-side/Django/Introducción}:
\begin{itemize}
	\item Completo: provee de casi todo lo que los desarrolladores esperan utilizar, sigue principios de diseño consistentes y dispone de una amplia y actualizada documentación.
	\item Versátil: puede funcionar con cualquier \textit{framework} en el lado del cliente, y puede devolver contenido en casi cualquier formato incluyendo HTML, RSS feeds, JSON y XML. Internamente, ofrece opciones para casi cualquier funcionalidad, como por ejemplo: distintos motores de base de datos , motores de plantillas, etc...
	\item Seguro: Django proporciona una manera segura de administrar cuentas de usuario y contraseñas, por lo que las \textit{cookies} solo contienen una clave y los datos se almacenan en la base de datos, o se almacenan directamente las contraseñas en un hash de contraseñas.
	\item Escalable: usa un componente basado en la arquitectura \textit{shared-nothing}, es decir, cada parte de la arquitectura es independiente de las otras, y por lo tanto puede ser reemplazado o cambiado si es necesario.
	\item Mantenible: el código de Django está escrito usando principios y patrones de diseño para fomentar la creación de código mantenible y reutilizable, como por ejemplo el patrón MVC \textit{(Model View Controller)}, el cuál agrupa código relacionado en módulos.  Por otro lado, utiliza el principio No te repitas \textit{Don't Repeat Yourself} (DRY) para que no exista una duplicación innecesaria, reduciendo la cantidad de código. 
	\item Portable: Django está escrito en Python, el cual se ejecuta en muchas plataformas. Lo que significa que no está sujeto a ninguna plataforma en particular, y puede ejecutar sus aplicaciones en muchas distribuciones de Linux, Windows y Mac OS X.
\end{itemize}

\section{SpaCy}
\label{cap:sec:spacy}
SpaCy es una biblioteca de código abierto para el Procesamiento del Lenguaje Natural en Python, soporta más de 34 idiomas entre ellos el español y tiene el mejor índice de acierto como analizador sintáctico\footnote{https://spacy.io/}, por estos motivos se decidió utilizar esta herramienta para la clasificación sintáctica de palabras.
Para este proyecto se utilizó SpaCy para la clasificación sintáctica de palabras, más concretamente de las 1000 formas más comunes del castellano según la RAE y posteriormente solo utilizar aquellas que son útiles(verbos, pronombres, sustantivos y adverbios).

Para su utilización, hay que importar la biblioteca de idioma correspondiente(en este caso español) e invocar al POS tagger pasando las palabras que se deseen clasificar, el resultado será una etiqueta según la siguiente tabla:



