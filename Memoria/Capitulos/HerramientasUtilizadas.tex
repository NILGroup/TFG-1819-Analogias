\chapter{Herramientas Utilizadas}
\label{cap:herramientas}

En este capítulo se van a explicar las herramientas utilizadas para el desarrollo de este trabajo, en el apartado 3.1 se explicará Django que es el \textit{framework} utilizado para el desarrollo del servicio web y en el apartado 3.2 se explicará SpaCy, que es la herramienta que se utilizó para la clasificación de palabras. Las herramientas y recursos que se utilizaron para la gestión del proyecto se explicarán en un capítulo aparte.


%-------------------------------------------------------------------
\section{Django}
%-------------------------------------------------------------------
\label{cap:sec:django}
Django es un \textit{framework} de alto nivel que permite el desarrollo rápido de sitios web seguros y mantenibles y se basa en el patrón MVC \citep{TFGEmociones}. Fue desarrollado entre los años 2003 y 2005 por un grupo de programadores que se encargaban de crear y mantener sitios web de periódicos. 
Es gratuito y de código abierto y dispone de una gran documentación actualizada así como muchas opciones de soporte gratuito y de pago. Gracias a la utilización de Django se puede crear un software\footnote{https://developer.mozilla.org/es/docs/Learn/Server-side/Django/Introducción}:
\begin{itemize}
	\item Completo: provee de casi todo lo que los desarrolladores esperan utilizar, sigue principios de diseño consistentes y dispone de una amplia y actualizada documentación.
	\item Versátil: puede funcionar con cualquier \textit{framework} en el lado del cliente, y puede devolver contenido en casi cualquier formato incluyendo HTML, RSS feeds, JSON y XML. Internamente, ofrece opciones para casi cualquier funcionalidad, como por ejemplo: distintos motores de base de datos , motores de plantillas, etc...
	\item Seguro: Django proporciona una manera segura de administrar cuentas de usuario y contraseñas, por lo que las \textit{cookies} solo contienen una clave y los datos se almacenan en la base de datos, o se almacenan directamente las contraseñas en un hash de contraseñas.
	\item Escalable: usa un componente basado en la arquitectura \textit{shared-nothing}, es decir, cada parte de la arquitectura es independiente de las otras, y por lo tanto puede ser reemplazado o cambiado si es necesario.
	\item Mantenible: el código de Django está escrito usando principios y patrones de diseño para fomentar la creación de código mantenible y reutilizable, como por ejemplo el patrón MVC \textit{(Model View Controller)}, el cuál agrupa código relacionado en módulos.  Por otro lado, utiliza el principio No te repitas \textit{Don't Repeat Yourself} (DRY) para que no exista una duplicación innecesaria, reduciendo la cantidad de código. 
	\item Portable: Django está escrito en Python, el cual se ejecuta en muchas plataformas. Lo que significa que no está sujeto a ninguna plataforma en particular, y puede ejecutar sus aplicaciones en muchas distribuciones de Linux, Windows y Mac OS X.
\end{itemize}

\section{SpaCy}
\label{cap:sec:spacy}
SpaCy es una biblioteca de código abierto para el Procesamiento del Lenguaje Natural en Python, soporta más de 34 idiomas entre ellos el español. Se requería una herramienta que clasificara según su categoría gramatical las 1000 palabras más utilizadas del castellano, ya que solo se necesitaban los verbos, pronombres sustantivos y adverbios. Se pensó inicialmente en la biblioteca NLTK de Python, pero tras una prueba inicial se vio que la clasificación no era una fiable por lo que se descartó su uso y se buscaron otras opciones, se probó con SpaCy y el resultado fue muy satisfactorio, por lo se decidió utilizar en el proyecto para el fin anteriormente descrito.



\figura{Bitmap/Capitulo3/spacy}{width=.4\textwidth}{fig:spacy}{Ejemplo de clasificación de palabras}

SpaCy según su página web \footnote{https://spacy.io/}, tiene el mejor índice de acierto como analizador sintáctico, para su utilización hay que importar la biblioteca de idioma correspondiente (en este caso español: ``es\_core\_news\_sm'') y pasar como parámetro las palabras que se deseen clasificar, el resultado será una serie de etiquetas tal y como se pueden apreciar en la tabla \ref{fig:spacy}, que es un ejemplo de las etiquetas generadas para una frase en lengua inglesa y que se explica a continuación:
\begin{itemize}
	\item La columna \textit{text} indica cual es la palabra que se ha procesado. En el caso de la primera fila es la palabra \textit{Apple}.
	\item La columna \textit{lemma} indica la forma base de la palabra procesada. En la primera fila es la palabra \textit{apple} en minúsculas.
	\item La columna \textit{pos} es la etiqueta asignada a dicha palabra. Indicando lo siguiente:
		\begin{itemize}
			\item PROPN: Nombre propio. Por ejemplo \textit{Apple}
			\item VERB: Verbo. Como \textit{is} que es una forma del verbo \textit{to be}.
			\item ADP: Preposicion. Por ejemplo \textit{at}.
		\end{itemize}
	
	\item La columna \textit{tag} indica cual es la palabra que se ha procesado con más detalle.
	\item La columna \textit{dep} indica la dependencia sintáctica de la palabra en la frase. Por ejemplo en la segunda frase \textit{is} es un verbo auxiliar.
	\item La columna \textit{shape} indica la apariencia de la palabra procesada, es decir, si está en mayúsculas o si tiene algún signo de puntuación. Por ejemplo en la última columna, la palabra \textit{U.K.} tiene un \textit{shape} X.X. ya que está formado por dos letras mayúsculas y dos puntos.
	\item La columna \textit{alpha} es un valor booleano que tendrá el valor \textit{True} si la palabra es un carácter alfanumérico y \textit{False} si no lo és. Por ejemplo: \textit{buying} tiene la etiqueta \textit{alpha} a \textit{True} ya que está formada por caracteres alfanuméricos y \textit{U.K.} la tiene a \textit{False} ya que los ``.'' no se consideran caracteres alfanuméricos.
	\item La columna \textit{stop} es un valor booleano que tendrá el valor \textit{True} si la palabra forma parte de las palabras más comunes del lenguaje en el que se encuentra \textit{False} si no lo és. Por ejemplo: \textit{is} que es una forma del verbo \textit{to be} tiene esta etiqueta a \textit{True} ya que es muy utilizado en el lenguaje inglés.
\end{itemize}



