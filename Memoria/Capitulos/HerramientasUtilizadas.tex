\chapter{Herramientas Utilizadas}
\label{cap:herramientas}

Como base para poder trabajar en el proyecto de la manera más ordenada y limpia posible, ambos compañeros hemos utilizado ciertas tecnologías que nos facilitan dicho trabajo y nos ayudan a saber en qué momento exacto se encuentra cada uno en la realización del mismo.
Por ello, en este capítulo se va a explicar todas las herramientas utilizadas con este fin, desde las más básicas que se explicarán en el apartado 3.1, como en el apartado 3.2 se explicarán las herramientas utilizadas para la gestión de tareas, ya que ambos debemos dividirnos el trabajo en tareas, siendo éstas lo más simples e independientes unas de las otras para que no nos inmiscuiyamos en sendos trabajos. Y en el capítulo 3.3 explicaremos los pasos iniciales para instalar Django correctamente.


%-------------------------------------------------------------------
\section{Herramientas básicas}
%-------------------------------------------------------------------
\label{cap:sec:herramientasBasicas}
	
	\begin{itemize}
	\item Repositorio: como herramientas básicas para poder comunicarnos tanto entre nosotros como con los directores, e ir teniendo  guardadas todas las versiones de nuestro código así como de la memoria, hacemos uso de un repositorio común en \textit{GitHub}. 
	La dirección del mismo sería https://github.com/NILGroup/TFG-1819-Analogias, donde se pueden ver todos los cambios realizados desde el principio del proyecto.

	\item Pruebas Unitarias: 
	\item Pruebas de Integración: 
	\end{itemize}

%-------------------------------------------------------------------
\section{Herramientas para la Gestión de Tareas}
%-------------------------------------------------------------------
\label{cap:sec:herramientasGestionTareas}
Nuestro equipo no se rige por una metodología en especial, ya que al ser únicamente dos personas no podemos utilizar Scrum o cualquier otra metodología ágil. Pero si que hemos adoptado ciertas características de ese tipo de metodologías para realizar nuestro trabajo.
Hemos hecho uso de un gestor de tareas con el fin de ayudarnos a saber cuál es la próxima tarea por hacer, cuáles están en proceso y cuáles se encuentran ya finalizadas. Existen varios gestores de tareas, pero nos hemos decantado por Trello, ya que dispone de una interfaz simple, amigable y que no lleva a confusión a la hora de crear nuevas tareas o moverse por el tablero.
Se han añadido tres columnas:
\begin{itemize}
	\item Lista de tareas: en dónde se ven representadas todas las tareas a realizar, desgranadas al mayor detalle posible e intentando que éstas sean lo más independientes las unas de las otras. De esta forma, nos aseguramos que cada integrante del equipo trabaja en una tarea específica que no influye en el trabajo del compañero.
	\item En proceso: en el momento en el que un integrante del grupo se asigna una tarea, ésta se traslada a la siguiente columna. Lo que indica que se encuentra en proceso de realización y que ningún otro compañero puede realizarla. 
	\item Hecho: última columna de nuestro tablero. Cuando una tarea se encuentra en dicha columna implica que la tarea ha sido terminada y validada, esta validación la deben dar los directores y nunca los propios integrantes del grupo.
	
\end{itemize}

Para que se pueda entender con mayor claridad, podemos ver la Figura \ref{fig:trello} dónde en cada columna se ve claramente la descripción de la tarea y quién la tiene asignada.
\figura{Bitmap/Capitulo3/trello}{width=.9\textwidth}{fig:trello}{Ejemplo Gestor de Tareas en Trello}

%-------------------------------------------------------------------
\section{Django}
%-------------------------------------------------------------------
\label{cap:sec:django}
Django es un \textit{framework} de alto nivel que permite el desarrollo rápido de sitios web seguros y mantenibles y se basa en el patrón MVC \citep{TFGEmociones}. Fue desarrollado entre los años 2003 y 2005 por un grupo de programadores que se encargaban de crear y mantener sitios web de periódicos. 
Es gratuito y de código abierto y dispone de una gran documentación actualizada así como muchas opciones de soporte gratuito y de pago. Gracias a utilizar Django se puede crear un software\footnote{https://developer.mozilla.org/es/docs/Learn/Server-side/Django/Introducción}:
\begin{itemize}
	\item Completo: provee de casi todo lo que los desarrolladores esperan utilizar, sigue principios de diseño consistentes y dispone de una amplia y actualizada documentación.
	\item Versátil: puede funcionar con cualquier \textit{framework} en el lado del cliente, y puede devolver contenido en casi cualquier formato incluyendo HTML, RSS feeds, JSON y XML. Internamente, ofrece opciones para casi cualquier funcionalidad, como por ejemplo: distintos motores de base de datos , motores de plantillas, etc...
	\item Seguro: Django proporciona una manera segura de administrar cuentas de usuario y contraseñas, por lo que las \textit{cookies} solo contienen una clave y los datos se almacenan en la base de datos, o se almacenan directamente las contraseñas en un hash de contraseñas.
	\item Escalable: usa un componente basado en la arquitectura \textit{shared-nothing}, es decir, cada parte de la arquitectura es independiente de las otras, y por lo tanto puede ser reemplazado o cambiado si es necesario.
	\item Mantenible: el código de Django está escrito usando principios y patrones de diseño para fomentar la creación de código mantenible y reutilizable, como por ejemplo el patrón MVC \textit{(Model View Controller)}, el cuál agrupa código relacionado en módulos.  Por otro lado, utiliza el principio No te repitas \textit{Don't Repeat Yourself} (DRY) para que no exista una duplicación innecesaria, reduciendo la cantidad de código. 
	\item Portable: Django está escrito en Python, el cual se ejecuta en muchas plataformas. Lo que significa que no está sujeto a ninguna plataforma en particular, y puede ejecutar sus aplicaciones en muchas distribuciones de Linux, Windows y Mac OS X.
\end{itemize}

