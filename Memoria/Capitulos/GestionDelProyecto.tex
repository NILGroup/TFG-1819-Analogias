\chapter{Metodología de gestión del Proyecto}
\label{cap:gestionProyecto}

Desde el inicio del proyecto se ha buscado la eficacia y la mejora continua, es por ello que se han tenido reuniones asiduamente con los directores de este trabajo, cada dos o tres semanas en donde se corregían los fallos, se indicaban las siguientes tareas por hacer y se buscaban soluciones a los problemas y dudas que pudieran surgir o plantearse.
Por otro lado, ha habido comunicación con ambos directores vía email para consultar dudas o para concretar tutorías. En relación a la gestión de configuración se ha utilizado la plataforma de desarrollo online GitHub para llevar el control de versiones del proyecto.

Se ha hecho uso de un gestor de tareas para emplearlo como radiador de información y que así todos los integrantes del proyecto puedan conocer en cada momento el estado de este. Para ello, se ha elegido Trello, ya que dispone de una interfaz simple, amigable y que no lleva a confusión a la hora de crear nuevas tareas.
Existen dos tipos de tareas en el tablero: las relacionadas con código y las relacionadas con la memoria. Se ha realizado una distinción entre ambas, ya que la forma de cambiar su estado en el tablero varía significativamente. Para hacer esta distinción se ha escrito la palabra CÓDIGO o la palabra MEMORIA según corresponda delante de la descripción de la misma. El tablero de tareas tiene tres columnas:
\begin{itemize}
	\item TO DO: Tareas a realizar, desgranadas al mayor detalle posible e intentando que éstas sean lo más independientes las unas de las otras. De esta forma, se asegura que cada integrante del equipo trabaja en una tarea específica que no influye en el trabajo del otro compañero.
	\item En proceso: En el momento en el que un integrante del grupo se asigna una tarea, esta se pasa de la columna TO DO a la columna ``En proceso'', lo que indica que se encuentra en proceso de realización y que ningún otro compañero puede ponerse a trabajar en ella. 
	\item Hecho: En esta columna se encuentran las tareas terminadas y validadas. La validación depende del tipo de tarea: si la tarea es de código, la validación la deben hacer los miembros del equipo, y si es de memoria la deben hacer los directores.
	
\end{itemize}

En la Figura \ref{fig:trello} se muestra el estado del tablero al inicio del proyecto.
\figura{Bitmap/Capitulo3/trello}{width=.9\textwidth}{fig:trello}{Tablero de tareas al inicio del proyecto}

