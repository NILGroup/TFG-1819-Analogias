\chapter{Gestión del Proyecto}
\label{cap:gestionProyecto}


%-------------------------------------------------------------------
\section{Procedimientos para la Gestión del Proyecto}
%-------------------------------------------------------------------
\label{cap:sec:procedimientosGestionProyecto}

Desde el inicio de la realización del proyecto, se han seguido ciertas pautas para que el funcionamiento de este fuese lo más eficaz posible, es por ello que se han tenido reuniones asiduamente con los directores de este trabajo, más o menos cada dos/tres semanas en las  corregían los fallos, indicaban las siguientes tareas por hacer (tanto de código como de memoria) y  ayudaban con todos los problemas y dudas que pudieran surgir o plantearse.
Por otro lado, ha habido una comunicación con ambos directores vía email para pequeñas dudas o para concretar citas de tutorías, no siendo estas las reuniones programadas, sino como un plus a la hora de realizar el proyecto.


%-------------------------------------------------------------------
\section{Herramientas para la Gestión de Tareas}
%-------------------------------------------------------------------
\label{cap:sec:herramientasGestionTareas}

Nuestro equipo no se rige por una metodología en especial, ya que al ser únicamente dos personas no podemos utilizar Scrum o cualquier otra metodología ágil. Pero si que hemos adoptado ciertas características de ese tipo de metodologías para realizar nuestro trabajo.
Hemos hecho uso de un gestor de tareas para emplearlo como radiador de información y que así todos los integrantes del proyecto puedan conocer en cada momento el estado de este. Existen varios gestores de tareas, pero nos hemos decantado por Trello, ya que dispone de una interfaz simple, amigable y que no lleva a confusión a la hora de crear nuevas tareas o moverse por el tablero.
Existen dos tipos de tareas: las relacionadas con código y las relacionadas con la memoria. Se ha realizado una distinción entre ambas, ya que la forma de cambiar su estado en el tablero varía significativamente. La forma de poder distinguir estas, es que delante de la descripción de la tarea aparecerá la palabra CÓDIGO o la palabra MEMORIA. Por otro lado se han añadido tres columnas:
\begin{itemize}
	\item Lista de tareas: en dónde se ven representadas todas las tareas a realizar, desgranadas al mayor detalle posible e intentando que éstas sean lo más independientes las unas de las otras. De esta forma, nos aseguramos que cada integrante del equipo trabaja en una tarea específica que no influye en el trabajo del compañero.
	\item En proceso: en el momento en el que un integrante del grupo se asigna una tarea, esta se pasa de la columna anterior a la siguiente. Lo que indica que se encuentra en proceso de realización y que ningún otro compañero puede realizarla. 
	\item Hecho: cuando una tarea se encuentra en dicha columna implica que la tarea  ha sido terminada y validada, esta validación depende del tipo de tarea que sea, si es de tipo código la deben hacer los integrantes del proyecto y si es de tipo memoria la deben dar los directores y nunca los propios integrantes del grupo.
	
\end{itemize}

Para que se pueda entender con mayor claridad, podemos ver la Figura \ref{fig:trello} dónde en cada columna se ve claramente la descripción de la tarea y quién la tiene asignada.
\figura{Bitmap/Capitulo3/trello}{width=.9\textwidth}{fig:trello}{Ejemplo Gestor de Tareas en Trello}
