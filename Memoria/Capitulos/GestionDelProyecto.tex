\chapter{Gestión del Proyecto}
\label{cap:gestionProyecto}

Desde el inicio de la realización del proyecto, se han seguido ciertas pautas para que el funcionamiento de este fuese lo más eficaz posible, es por ello que se han tenido reuniones asiduamente con los directores de este trabajo, cada dos/tres semanas en donde se corregían los fallos, se indicaban las siguientes tareas por hacer (tanto de código como de memoria) y se buscaban soluciones a los problemas y dudas que pudieran surgir o plantearse.
Por otro lado, ha habido una comunicación con ambos directores vía email para pequeñas dudas o para concretar citas de tutorías, no siendo estas las reuniones programadas, sino como un plus a la hora de realizar el proyecto.
En relación con el código de este trabajo, para poder llevar un control de las versiones del mismo y de la versión del código de cada integrante se ha utilizado la plataforma de desarrollo online GitHub.
GitHub 



GitHub es una plataforma de desarrollo colaborativo online que utiliza el sistema de control de versiones distribuido Git\footnote{https://github.com/} en el que cada miembro del equipo de desarrollo tiene su propia copia del repositorio online en local. Los cambios que realice cada usuario se guardan en esta copia local y cuando desee se suben al repositorio online mediante la acción \textit{commit} y \textit{push}. En caso de que haya algún conflicto, el sistema te permite decidir como gestionarlos. 
Las características principales a destacar de GitHub son:
\begin{itemize}
\item Licencias: cuando creamos un proyecto, GitHub nos permite añadir un archivo de texto en el cuál se puede explicar el tipo de licencia que dispone el proyecto. Es por ello, que GitHub tiene diferentes plantillas de licencias como por ejemplo para una licencia tipo GPL, MIT o Apache. Estas licencias determinarán el uso que el usuario puede hacer de los proyectos y si se tiene que realizar diferentes acciones como mencionar el proyecto original o al desarrollador del proyecto original.
\item Gráficas: GitHub dispone de una pestaña \textit{Insights}, donde se puede ver en forma de gráfica las contribuciones (\textit{commits}) que ha realizado cada integrante, es decir, las líneas de código tanto añadido como eliminado. También se puede saber que día se realizaron los \textit{commits} y cuantos.
\item Red Social: el usuario dispone de perfil y se pueden buscar a otros usuarios, dando la posibilidad de poder seguirse.
\item Wiki: cada proyecto puede tener su propia wiki con manuales e información relativas a éste.
\end{itemize}

GitHub también dispone de una pestaña \textit{Issues}, donde los usuarios pueden escribir las tareas que hay que realizar, haciendo éste el símil de una pizarra o tablero, y de esta forma poder tener una gestión de tareas y saber cada integrante del proyecto en que proceso se encuentra cada tarea. 
Nuestro equipo no se rige por una metodología concreta pero si que hemos adoptado ciertas características de las metodologías ágiles para realizar nuestro trabajo.
Hemos hecho uso de un gestor de tareas para emplearlo como radiador de información y que así todos los integrantes del proyecto puedan conocer en cada momento el estado de este. Podríamos hacer utilizado, como hemos comentado anteriormente el tablero de GitHub, pero nos hemos decantado por Trello, ya que dispone de una interfaz simple, amigable y que no lleva a confusión a la hora de crear nuevas tareas o moverse por el tablero.
Existen dos tipos de tareas: las relacionadas con código y las relacionadas con la memoria. Se ha realizado una distinción entre ambas, ya que la forma de cambiar su estado en el tablero varía significativamente. La forma de poder distinguir estas, es que delante de la descripción de la tarea aparecerá la palabra CÓDIGO o la palabra MEMORIA. Por otro lado se han añadido tres columnas:
\begin{itemize}
	\item Lista de tareas: en dónde se ven representadas todas las tareas a realizar, desgranadas al mayor detalle posible e intentando que éstas sean lo más independientes las unas de las otras. De esta forma, nos aseguramos que cada integrante del equipo trabaja en una tarea específica que no influye en el trabajo del compañero.
	\item En proceso: en el momento en el que un integrante del grupo se asigna una tarea, esta se pasa de la columna Lista de Tareas a esta. Lo que indica que se encuentra en proceso de realización y que ningún otro compañero puede realizarla. 
	\item Hecho: cuando una tarea se encuentra en dicha columna implica que la tarea  ha sido terminada y validada, esta validación depende del tipo de tarea que sea, si es de tipo código la validación la deben hacer los desarrolladores y si es de tipo memoria la deben hacer los directores.
	
\end{itemize}

Para que se pueda entender con mayor claridad, podemos ver la Figura \ref{fig:trello} dónde en cada columna se ve claramente la descripción de la tarea y quién la tiene asignada.
\figura{Bitmap/Capitulo3/trello}{width=.9\textwidth}{fig:trello}{Ejemplo Gestor de Tareas en Trello}

