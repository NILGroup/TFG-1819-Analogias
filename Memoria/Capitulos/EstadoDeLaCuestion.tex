\chapter{Estado de la Cuestión}
\label{cap:estadoDeLaCuestion}


Para que un contenido ilustrativo o en formato de texto sea sencillo de entender existe una adaptación llamada lectura fácil cuyo objetivo es facilitar la accesibilidad al mismo. En la sección 2.1 de este capítulo se explicará qué es la lectura fácil y algunas pautas que se pueden seguir para escribir correctamente un texto en lectura fácil. Por otro lado, para que las definiciones de las palabras difíciles sean fácilmente comprensibles para el usuario, se utilizarán figuras retóricas con las que se compararán conceptos complejos con otros más sencillos. De esta manera el usuario se podrá hacer una idea de sus características principales. En la sección 2.2 se hablará de las figuras retóricas y se explicarán los tres tipos fundamentales que se van a utilizar para la realización de este trabajo. 
Pero para poder plasmar todo esto en el trabajo realizado y conseguir una funcionalidad, se deberán usar servicios web, los cuáles se han convertido en una tecnología cotidiana para la mayoría de los usuarios sin ser estos mismos conscientes de su uso. En la sección 2.3 se explicará detalladamente que es un servicio web, los tipos que existen, sus características principales, su arquitectura y las ventajas de ser utilizados así como sus desventajas. Por último, las comparaciones entre conceptos más complejos con otros más sencillos, se construirán a través de conceptos relacionados con la palabra que el usuario busque, es por ello que en la sección 2.4 se explicará en qué consiste el Procesamiento del Lenguaje Natural, los tipos de redes semánticas que existen y las diversas aplicaciones que actúan como redes semánticas y que son capaces de procesar el Lenguaje Natural.

%-------------------------------------------------------------------
\section{Lectura Fácil}
%-------------------------------------------------------------------
\label{cap:sec:lecturafacil}

Se llama lectura fácil\footnote{https://www.discapnet.es/areas-tematicas/diseno-para-todos/accesibilidad-de-comunicacion/lectura-facil} a aquellos contenidos que han sido resumidos y reescritos con lenguaje sencillo y claro, de forma que puedan ser entendidos por personas con discapacidad cognitiva o discapacidad intelectual. Es decir, es la adaptación de textos, ilustraciones y maquetaciones que permite una mejor lectura y comprensión.
Este trabajo se va a centrar en la lectura fácil aplicada a textos.

La lectura fácil surgió en Suecia en el año 1968, donde se editó el primer libro en la Agencia de Educación en el marco de un proyecto experimental. A continuación, en 1976, se creó en el Ministerio de Justicia un grupo de trabajo para conseguir textos legales más claros.
En 1984 nació el primer periódico en lectura fácil, titulado "8 páginas", que tres años más tarde, en 1987, se publicó de forma permanente en papel hasta que empezó a editarse en la web. 
En el año 2013, en México se produce la primera sentencia judicial en lectura fácil\footnote{https://dilofacil.wordpress.com/2013/12/04/el-origen-de-la-lectura-facil/}. En la actualidad, podemos distinguir los documentos en lectura fácil gracias al logo de la Figura \ref{fig:lecturafacil}.

	\figura{Bitmap/Capitulo2/lecturaFacil}{width=.3\textwidth}{fig:lecturafacil}{Logo Lectura Fácil} 
	
	
Los documentos escritos en Lectura Fácil \citep{lecturafacil} son documentos de todo tipo que siguen las directrices internacionales de la IFLA\footnote{International Federation of Library Associations and Institutions} y de Inclusion Europe\footnote{Una asociación de personas con discapacidad intelectual y sus familias en Europa} en cuanto al contenido y la forma.
Algunas pautas a seguir para escribir correctamente un texto en Lectura Fácil son \citep{GarciaMunoz2012LecturaFacil}:


\begin{itemize}
	\item Evitar mayúsculas fuera de la norma, es decir, escribir en mayúsculas sólo cuando lo dicten las reglas ortográficas, como por ejemplo, después de un punto o la primera letra de los nombres propios.
	\item Deben evitarse el punto y seguido, el punto y coma y los puntos suspensivos. El punto y aparte hará la función del punto y seguido.
	\item Evitar corchetes y signos ortográficos poco habituales, como por ejemplo: \%, \& y /.
	\item Evitar frases superiores a 60 caracteres y utilizar oraciones simples. Por ejemplo, la oración \textit{Caperucita ha ido a casa de su abuela y ha desayunado con ella} es mejor dividirla en dos oraciones simples:\textit{ Caperucita ha ido a casa de su abuela} y  \textit{Caperucita ha desayunado con ella}.
	\item Evitar tiempos verbales como: futuro, subjuntivo, condicional y formas compuestas.
	\item Utilizar palabras cortas y de sílabas poco complejas. 
	Por ejemplo: casa, gato, comer o mano.
	\item Evitar abreviaturas, acrónimos y siglas.
	\item Alinear el texto a la izquierda.
	\item Incluir imágenes y pictogramas a la izquierda y su texto vinculado a la derecha.
	\item Evitar la saturación de texto e imágenes.
	\item Utilizar uno o dos tipos de letra como mucho.
	\item Tamaño de letra entre 12 y 16 puntos.
	\item Si el documento está paginado, incluir la paginación claramente y reforzar el mensaje de que la información continúa en la página siguiente.
\end{itemize}

Se debe también hacer hincapié en la distinción entre palabras fáciles y complejas \citep{GarciaMunoz2012LecturaFacil}, puesto que son de gran importancia para la lectura fácil. 
Las palabras complejas son aquellas que no se utilizan a menudo en nuestra sociedad, como por ejemplo: melifluo o inefable. Es por ello que este tipo de palabras deben estar totalmente descartadas en la lectura fácil, y en su lugar debemos introducir palabras fáciles, que son aquellas que se utilizan asiduamente. La RAE (Real Academia Española) dispone de un documento con las mil palabras más usadas\footnote{http://corpus.rae.es/lfrecuencias.html}.

%-------------------------------------------------------------------



%-------------------------------------------------------------------
\section{Figuras retóricas}
%-------------------------------------------------------------------
\label{cap:sec:figurasretoricas}

Las figuras literarias (o retóricas) se podrían definir como formas no convencionales de utilizar las palabras, de manera que, aunque se emplean con sus acepciones habituales, se acompañan de algunas particularidades fónicas, gramaticales o semánticas, que las alejan de ese uso habitual, por lo que terminan por resultar especialmente expresivas. 
Según la RAE (Real Academia Española)\footnote{https://dle.rae.es/?id=WISC3uX},\textit{``la retórica es el arte de bien decir, de dar al lenguaje escrito o hablado eficacia bastante para deleitar, persuadir o conmover"}.
La metáfora, el símil y la analogía se basan en la comparación de dos conceptos: el origen (o tenor), que es el término literal (al que la metáfora se refiere) y el de destino (o vehículo), que es el término figurado. La relación que hay entre el tenor y el vehículo se denomina fundamento. Por ejemplo, en la metáfora \textit{Tus ojos son dos luceros}, \textit{ojos} es el tenor, \textit{luceros} es el vehículo y el fundamento es la belleza de los ojos \citep{GalianaYCasas1994}.


En este trabajo se van a utilizar tres tipos de figuras retóricas \citep{TFMPaloma}: 
\begin{itemize}
	\item Metáfora: Utiliza el desplazamiento de características similares entre dos conceptos con fines estéticos o retóricos. Por ejemplo, cuando el tiempo de una persona es muy preciado se dice: ``Mi tiempo es oro''.
	
	\item Símil: Realiza una comparación entre dos términos usando conectores (por ejemplo, como, cual, que, o verbos).
	Por ejemplo, cuando nos referimos a una persona que es muy corpulenta, se dice: ``Es como un oso'', ya que los osos son muy grandes.
	
	\item Analogía: Es la comparación entre varios conceptos, indicando las características que permiten dicha relación. En la retórica, una analogía es una comparación textual que resalta alguna de las similitudes semánticas entre los conceptos protagonistas de dicha comparación. Por ejemplo: ``Sus ojos son azules como el mar''.
	
	
	
\end{itemize}

%-------------------------------------------------------------------
\section{Servicios Web}
%-------------------------------------------------------------------
\label{cap:sec:serviciosweb}

Para definir el concepto de servicio web de la forma más simple posible, se podría decir que es una tecnología que utiliza un conjunto de protocolos para intercambiar datos entre aplicaciones, sin importar el lenguaje de programación en el cual estén programadas o ejecutadas en cualquier tipo de plataforma\footnote{http://www.jtech.ua.es/j2ee/publico/servc-web-2012-13/sesion01-apuntes.html}. Según el W3C (\textit{World Wide Web Consortium})\footnote{https://www.w3.org/},\textit{``un servicio web es un sistema software diseñado para soportar la interacción máquina-a-máquina, a través de una red, de forma interoperable''}.




Las principales características de un servicio web son \citep{TorresJoaquin2017SC}:



\begin{itemize}
	\item Es accesible a través de la Web. Para ello debe utilizar protocolos de transporte estándares como HTTP, y codificar los mensajes en un lenguaje estándar que pueda ser accesible por cualquier cliente que quiera utilizar el servicio. 
	\item Contiene una descripción de sí mismo. De esta forma, una aplicación web podrá saber cual es la función de un determinado Servicio Web, y cuál es su interfaz, de manera que pueda ser utilizado de forma automática por cualquier aplicación, sin la intervención del usuario.
	\item Debe ser localizado. Debe tener algún mecanismo que permita encontrarle. De esta forma tendremos la posibilidad de que una aplicación localice el servicio que necesite de forma automática, sin tener que conocerlo previamente el usuario.
\end{itemize}

\subsection{Tipos de Servicios Web}
\label{cap:subsec:tiposserviciosweb}

Los servicios web pueden definirse tanto a nivel conceptual como a nivel técnico. A nivel técnico se pueden diferenciar dos tipos de servicios web \citep{TorresJoaquin2017SC}:
\begin{itemize}
	\item Servicios web SOAP \textit({Simple Object Access Protocol}): SOAP es un protocolo basado en XML para el intercambio de información entre ordenadores. Normalmente utilizaremos SOAP para conectarnos a un servicio e invocar métodos remotos\footnote{https://www.ibm.com/support/knowledgecenter/es}. Los mensajes SOAP tienen el formato representado en el Listado \ref{lst:SOAP}, donde podemos ver un ejemplo para reservar un vuelo y está formado por los siguientes campos:
	
		\begin{itemize}
		\item <Envelope>: elemento raíz de cada mensaje SOAP y contiene dos elementos: 
			\begin{itemize}
			\item <Header>: es un elemento opcional que se utiliza para indicar información acerca de los mensajes SOAP. Por ejemplo, como se puede ver en el Listado \ref{lst:SOAP} dentro del campo Header estarían los campos de reservas y pasajeros.
			\item <Body>: es un elemento obligatorio que contiene información dirigida al destinatario del mensaje. Haciendo referencia al Listado \ref{lst:SOAP} se puede ver los campos asociados a un itinerario, teniendo este el lugar de partida, de llegada, la fecha de llegada y la preferencia de asiento.
		\end{itemize}
		\item <Fault>: es un elemento opcional para notificar errores. En el Listado \ref{lst:SOAP} podemos ver que no se encuentra presente, pero en caso de ser utilizado deberá aparecer dentro del elemento \textit{Body} y no puede aparecer más de una vez .
		
	\end{itemize}
	
	\definecolor{gray}{rgb}{0.4,0.4,0.4}
	\definecolor{darkblue}{rgb}{0.0,0.0,0.6}
	\definecolor{cyan}{rgb}{0.0,0.6,0.6}
	
	
	
	
	\lstdefinelanguage{XML}
	{
		morestring=[b]",
		morestring=[s]{>}{<},
		morecomment=[s]{<?}{?>},
		stringstyle=\color{black},
		identifierstyle=\color{darkblue},
		keywordstyle=\color{purple},
		morekeywords={xmlns,version,type}% list your attributes here
	}
	
	
	\lstset{language=XML}
	\begin{lstlisting}[caption= Estructura de un mensaje SOAP, label={lst:SOAP}, frame=single]
	<?xml version='1.0' Encoding='UTF-8' ?>
	<env:Envelope xmlns:env="http://www.w3.org/2003/05/soap-envelope"> 
		<env:Header>
				<m:reservation xmlns:m="http://travelcompany.example.org/reservation" 
							env:role="http://www.w3.org/2003/05/soap-envelope/role/next">
					<m:reference>uuid:093a2da1-q345-739r-ba5d-pqff98fe8j7d</m:reference>
					<m:dateAndTime>2007-11-29T13:20:00.000-05:00</m:dateAndTime>
				</m:reservation>
				<n:passenger xmlns:n="http://mycompany.example.com/employees" 
						env:role="http://www.w3.org/2003/05/soap-envelope/role/next">
					<n:name>Fred Bloggs</n:name>
				</n:passenger>
		</env:Header>
		<env:Body>
			<p:itinerary xmlns:p="http://travelcompany.example.org/reservation/travel">
				<p:departure>
					<p:departing>New York</p:departing>
					<p:arriving>Los Angeles</p:arriving>
					<p:departureDate>2007-12-14</p:departureDate>
					<p:departureTime>late afternoon</p:departureTime>
					<p:seatPreference>aisle</p:seatPreference>
				</p:departure>
				<p:return>
					<p:departing>Los Angeles</p:departing>
					<p:arriving>New York</p:arriving>
					<p:departureDate>2007-12-20</p:departureDate>
					<p:departureTime>mid-morning</p:departureTime>
					<p:seatPreference></p:seatPreference>
				</p:return>
			</p:itinerary>
		</env:Body>
	</env:Envelope>
	\end{lstlisting}
	

	\item Servicios Web RESTful: RESTful es un protocolo que suele integrar mejor con HTTP que los servicios basado en SOAP, ya que no requieren mensajes XML. Cada petición del cliente debe contener toda la información necesaria para entender la petición, y no puede aprovecharse de ningún contexto almacenado en el servidor.
	
\end{itemize}



\subsection{Arquitectura Servicios Web}
\label{cap:subsec:arquitecturaserviciosweb}

Hay que distinguir tres partes fundamentales en los servicios web \citep{TorresJoaquin2017SC}:
\begin{itemize}
	\item El proveedor: es la aplicación que implementa el servicio y lo hace accesible desde Internet.
	\item El solicitante: cualquier cliente que necesite utilizar el servicio web.
	\item El publicador: se refiere al repositorio centralizado en el que se encuentra la información de la funcionalidad disponible y como se utiliza.
	
\end{itemize}
Por otro lado, los servicios web se componen de varias capas\footnote{https://diego.com.es/introduccion-a-los-web-services}:
\begin{itemize}
	\item Descubrimiento del Servicio: responsable de centralizar los servicios web en un directorio común, de esta forma es más sencillo buscar y publicar.
	\item Descripción del Servicio: como ya hemos comentado con anterioridad, los servicios web se pueden definir a sí mismos, por lo que una vez que los localicemos el Service Description nos dará la información para saber que operaciones soporta y como activarlo.
	\item Invocación del Servicio: invocar a un Servicio Web implica pasar mensajes entre el cliente y el servidor. Por ejemplo, si utilizamos SOAP  \textit({Simple Object Access Protocol}), el Service Invocation especifica cómo deberíamos formatear los mensajes request para el servidor, y cómo el servidor debería formatear sus mensajes de respuesta.
	\item Transporte: todos los mensajes han de ser transmitidos de alguna forma entre el servidor y el cliente. El protocolo elegido para ello es HTTP \textit{(HyperText Transfer Protocol)}. 
\end{itemize}



\subsection{Ventajas de los  Servicios Web}
\label{cap:subsec:ventajasserviciosweb}

	Las principales ventajas del uso de los servicios web son las siguientes \citep{doctorado2005}:
\begin{itemize}
	\item Permiten la integración “justo-a-tiempo”:  esto significa que los solicitantes, los proveedores y los agentes actúan en conjunto para crear sistemas que son auto-configurables, adaptativos y robustos.
	\item Reducen la complejidad por medio del encapsulamiento: un solicitante de servicio no sabe cómo fue implementado el servicio por parte del proveedor, y éste, a su vez, no sabe cómo utiliza el cliente el servicio. Estos detalles se encapsulan en los solicitantes y proveedores. El encapsulamiento es crucial para reducir la complejidad.
	\item Promueven la interoperabilidad: la interacción entre un proveedor y un solicitante de servicio está diseñada para que sea completamente independiente de la plataforma y el lenguaje. 
	\item Abren la puerta a nuevas oportunidades de negocio: los servicios web facilitan la interacción con socios de negocios, al poder compartir servicios internos con un alto grado de integración.
	\item Disminuyen el tiempo de desarrollo de las aplicaciones: gracias a la filosofía de orientación a objetos que utilizan, el desarrollo se convierte más bien en una labor de composición.
	\item Fomentan los estándares y protocolos basados en texto, que hacen más fácil acceder a su contenido y entender su funcionamiento.
\end{itemize}


\subsection{Desventajas de los  Servicios Web}
\label{cap:subsec:desventajasserviciosweb}
	El uso de servicios web también tiene algunas desventajas\footnote{http://fabioalfarocc.blogspot.com/2012/08/ventajas-y-desventajas-del-soap.html}:
\begin{itemize}
	\item Al apoyarse en HTTP, pueden esquivar medidas de seguridad basadas en firewall cuyas reglas tratan de bloquear.
	\item Existe poca información de servicios web para algunos lenguajes de programación.
	\item Dependen de la disponibilidad de servidores y comunicaciones.
\end{itemize}

%-------------------------------------------------------------------
\section{Procesamiento del Lenguaje Natural}
%-------------------------------------------------------------------
\label{cap:sec:lenguajenatural}
El Procesamiento del Lenguaje Natural (PLN) es una rama de la Inteligencia Artificial que se encarga de investigar la manera de comunicar máquinas con personas mediante el uso del lenguaje natural (entendiendo como lenguaje natural el idioma usado con fines de comunicación por humanos, ya sea hablado o escrito, como pueden ser el español, el ruso o el inglés). 
Para ello, una de las tareas principales en el Procesamiento del Lenguaje Natural es interpretar un texto escrito en lenguaje natural y entender su significado, entendiendo como significado la relación entre una palabra o una frase con el mundo. Para realizar dicha acción no solo es necesario el conocimiento del propio lenguaje en que está escrito el texto sino que también es necesario un conocimiento del mundo. Por tanto, uno de los grandes retos del Procesamiento del Lenguaje Natural es la representación del conocimiento. Se deben de buscar técnicas que permitan representar conceptos y relaciones semánticas entre ellos. 
Una de las principales técnicas de representación en el Procesamiento del Lenguaje Natural son las redes semánticas, en ellas los conceptos que componen el mundo y sus relaciones se representan mediante un grafo. Se utilizan para representar mapas conceptuales y mentales \citep{redSemantica2018}.
Los nodos están representados por el elemento lingüístico, y la relación entre los nodos sería la arista. Se puede ver un ejemplo en la Figura \ref{fig:red}, donde el nodo \textit{Oso} representa un concepto, en este caso un sustantivo que identifica a un tipo de animal,  y otro nodo \textit{Pelo}. Su relación se ve representada por la arista con valor \textit{tiene}, dando lugar a una característica de este animal: \textit{Oso tiene pelo}.
 
\figura{Bitmap/Capitulo2/redSemantica}{width=.9\textwidth}{fig:red}{Ejemplo Red Semántica}
\figura{Bitmap/Capitulo2/ejemploRedMarco}{width=.9\textwidth}{fig:ejemploMarco}{Ejemplo Red de Marco}
\figura{Bitmap/Capitulo2/ejemploRedIsa}{width=.9\textwidth}{fig:ejemploIsa}{Ejemplo Red IS-A}
Existen principalmente tres tipos de redes semánticas \citep{tiposRedesSemanticas}:
\begin{itemize}
	\item Redes de Marcos: los enlaces de unión de los nodos son parte del propio nodo, es decir, se encuentran organizados jerárquicamente, según un número de criterios estrictos, como por ejemplo la similitud entre nodos. Haciendo referencia a la Figura \ref{fig:ejemploMarco}, los conceptos cama, silla y mesa tienen en común que son objetos.
	\item Redes IS-A: los enlaces entre los nodos están etiquetados con una relación entre ambos. Es el tipo que habitualmente se utiliza junto con las Redes de Marcos. Un ejemplo de esto, haciendo referencia a la Figura \ref{fig:ejemploIsa} sería: el pájaro tiene alas, que es una característica específica de los pájaros.
	\item Grafos Conceptuales: existen dos tipos de nodos: nodos de conceptos, los cuáles representan una entidad, un estado o un proceso y los nodos de relaciones, que indican como se relacionan los nodos de concepto. En este tipo de red semántica no existen enlaces entre los nodos con una etiqueta, sino que son los propios nodos los que tienen el significado. Se puede ver un ejemplo en la Figura \ref{fig:grafoConceptual} \citep{osti_5673179} en la cual la frase \textit{``Man biting dog"} quedaría representada. Los cuadrados implican el concepto y el círculo la relación entre ambos, por lo que en el caso de \textit{man} y \textit{bite}, la acción de morder la realiza \textit{man} siendo éste el agente, y la relación entre \textit{bite} y \textit{dog} sería el objeto.
\end{itemize}
\figura{Bitmap/Capitulo2/grafoConceptual}{width=.9\textwidth}{fig:grafoConceptual}{Ejemplo Grafo Conceptual}
Para el trabajo que queremos realizar, existen varias aplicaciones web que actúan como redes semánticas y son capaces de procesar el Lenguaje Natural.  A continuación, hablaremos de algunas de ellas.

\subsection{ConceptNet} 
\label{cap:subsec:concepnet}

Es una red semántica creada por el MIT \textit{(Massachusetts Institute of Technology)} en 1999, diseñada para ayudar a los ordenadores a entender el significado de las palabras. Está disponible en múltiples idiomas, como el español, el inglés o el chino. ConceptNet ofrece la posibilidad de obtener de una palabra un listado de sinónimos, términos relacionados, términos derivados, el contexto de la palabra, resultados etimológicamente relacionados, símbolos, etc... Y dispone de una aplicación web\footnote{http://conceptnet.io/}, donde seleccionas el idioma deseado y añades la palabra a buscar. En la Figura  \ref{fig:busquedaConcepnet}, se puede ver que si se añade la palabra chaqueta se devuelven sus sinónimos, términos relacionados, términos relacionados etimológicamente y formas de la palabra.
\figura{Bitmap/Capitulo2/busquedaConcepnet}{width=.9\textwidth}{fig:busquedaConcepnet}{Resultados de ConcepNet para la palabra chaqueta} 
Por otro lado, ConcepNet dispone de un servicio web\footnote{http://api.conceptnet.io} que devuelve los resultados en formato JSON. Siguiendo con el mismo ejemplo anterior, se puede ver en el Listado \ref{lst:json} los resultados en dicho formato para la palabra chaqueta. Este consta de cuatro campos principales\footnote{https://github.com/commonsense/conceptnet5/wiki/AP}:
\begin{itemize}
	\item @context: URL enlazada a un archivo de información del JSON para comprender la API. También puede contener comentarios que pueden ser útiles para el usuario.
	\item @id: concepto que se ha buscado y su idioma. En nuestro caso, aparece de la siguiente manera: \textit{/c/es/chaqueta}, donde  \textit{c} significa que es un concepto o término,  \textit{es} indica el lenguaje, en este caso, el español y por último \textit{chaqueta} que es la palabra buscada.
	\item edges: representa una estructura de datos devueltos por Conceptnet compuesta por:
	\begin{itemize}
		\item @id: describe el tipo de relación que existe entre la palabra introducida y la devuelta.
		
		 \textit{/a/[/r/Synonym/,/c/es/chaqueta/n/,/c/es/americana/]},
		 nos indica que la palabra \textit{americana} es un sinónimo de \textit{chaqueta}.
		\item @type: define el tipo del id, es decir, si es una relación (edge) o un término (nodo).
		\item dataset: URI que representa el conjunto de datos creado.
		\item end: nodo destino, que a su vez se compone de:	
		\begin{itemize}
			\item @id: coincide con la palabra del id anterior.
			\item @type: define el tipo de id, como se ha explicado anteriormente.
			\item label: puede ser la misma palabra buscada o una frase más completa, donde adquiera significado la palabra obtenida.
			\item language: lenguaje en el que está la palabra devuelta de la consulta.
			\item term: enlace a una versión mas general del propio término. Normalmente, suele coincidir con la URI.			
		\end{itemize}
		\item license: aporta información sobre como debe usarse la información proporcionada por conceptnet.
		\item rel: describe la relación que hay entre la palabra origen y destino, dentro del cual hay tres campos: @id, @type y label, descritos anteriormente.
		\item sources: indica por qué ConceptNet guarda esa información, este campo como los anteriores, es un objeto que tiene su propio id y un campo @type, A parte, hay un campo \textit{contributor}, en el que aparece la fuente por la que se ha obtenido ese resultado y por último un campo \textit{process} indicando si la palabra se ha añadido mediante un proceso automático.
		\item start: describe el nodo origen, es decir, la palabra que hemos introducido en ConceptNet para que haga la consulta, este campo esta compuesto por elementos ya descritos como son: @id, @type, label, language y term.
		\item surfaceText: algunos datos de ConceptNet se extraen de texto en lenguaje natural. El valor de surface text muestra lo que era este texto, puede que este campo tenga valor nulo.
		\item weight: indica la fiabilidad de la información guardada en conceptnet, siendo normal que su valor sea 1.0. Cuanto mayor sea este valor, más fiables serán.
	\end{itemize}
	\item view: describe la longitud de la lista de paginación, es un objeto con un id propio, y además, aparecen los campos \textit{firstPage} que tiene como valor un enlace a la primera pagina de los resultados obtenidos, y \textit{nextPage} que tiene un enlace a la siguiente página de la lista.
\end{itemize}

\lstset{
	string=[s]{"}{"},
	stringstyle=\color{blue},
	comment=[l]{:},
	commentstyle=\color{black},
}

\begin{lstlisting} [caption=JSON devuelto por la API de ConceptNet para la palabra chaqueta, label={lst:json}, frame=single]
{
	"@context": [
		"http://api.conceptnet.io/ld/conceptnet5.6/context.ld.json"
],
	"@id": "/c/es/chaqueta",
	"edges": [
{
	"@id": "/a/[/r/Synonym/,/c/es/chaqueta/n/,/c/es/americana/]",
	"@type": "Edge",
	"dataset": "/d/wiktionary/fr",
	"end": {
	"@id": "/c/es/americana",
	"@type": "Node",
	"label": "americana",
	"language": "es",
	"term": "/c/es/americana"
},
	"license": "cc:by-sa/4.0",
	"rel": {
	"@id": "/r/Synonym",
	"@type": "Relation",
	"label": "Synonym"
},
	"sources": [
{
	"@id": "/and/[/s/process/wikiparsec/1/,/s/resource/wiktionary/fr/]",
	"@type": "Source",
	"contributor": "/s/resource/wiktionary/fr",
	"process": "/s/process/wikiparsec/1"
}
],
	"start": {
	"@id": "/c/es/chaqueta/n",
	"@type": "Node",
	"label": "chaqueta",
	"language": "es",
	"sense_label": "n",
	"term": "/c/es/chaqueta"
},
	"surfaceText": null,
	"weight": 1.0
},

]

	"view": {
	"@id": "/c/es/chaqueta?offset=0&limit=20",
	"@type": "PartialCollectionView",
	"comment": "There are more results. Follow the 'nextPage' link for more.",
	"firstPage": "/c/es/chaqueta?offset=0&limit=20",
	"nextPage": "/c/es/chaqueta?offset=20&limit=20",
	"paginatedProperty": "edges"
	}

}
\end{lstlisting} 


\subsection{Thesaurus}
\label{cap:subsec:thesaurus}
Es una aplicación web\footnote{https://www.thesaurus.com/} que se autodefine como el principal diccionario de sinónimos de la web. Esta página ofrece la posibilidad de introducir una palabra para poder conocer sus sinónimos, pero solamente devuelve resultados en inglés. Aparte del listado de sinónimos, Thesaurus indica que tipo de palabra es y una definición de la misma así como un listado de antónimos y un listado de palabras relacionadas con dicho concepto. 
Por otro lado esta aplicación proporciona una API\footnote{http://thesaurus.altervista.org/} tipo RESTful que obtiene los sinónimos de una palabra mediante una petición HTTP GET a la url \textit{http://thesaurus.altervista.org/thesaurus/v1}. 
Este devuelve los resultados en formato XML o JSON. El contenido de la respuesta es una lista y cada elemento de esta lista contiene un par de elementos: categoría y sinónimos. Este último a su vez contiene una lista de sinónimos separados por el carácter |. 
Se puede ver en el Listiado \ref{lst:xmlthesaurus} un ejemplo de como sería el resultado de una petición en formato XML y en el Listado \ref{lst:jsonthesaurus} en formato JSON.
Ambos son muy similares, por ejemplo en formato XML se puede ver que devuelve el tipo de categoría de las palabras, en este caso son sustantivos y a continuación aparcen los sinónimos. En caso de que alguna palabra sea un antónimo aparecera entre paréntesis al lado de la misma, como ocurre con la palabra \textit{war}. Por otro lado, el formato JSON devuelve dentro del campo \textit{category / categoría} todos los sinónimos, y en caso de ser un antónimo aparecerá de la misma forma que en el formato XML.



\definecolor{gray}{rgb}{0.4,0.4,0.4}
\definecolor{darkblue}{rgb}{0.0,0.0,0.6}
\definecolor{cyan}{rgb}{0.0,0.6,0.6}



\lstdefinelanguage{XML}
{
	morestring=[b]",
	morestring=[s]{>}{<},
	morecomment=[s]{<?}{?>},
	stringstyle=\color{black},
	identifierstyle=\color{darkblue},
	keywordstyle=\color{cyan},
	morekeywords={xmlns,version,type}% list your attributes here
}



\lstset{language=XML}
\begin{lstlisting}[caption= Ejemplo de salida de Thesaurus en formato XML para la palabra \textit{peace}, label={lst:xmlthesaurus}, frame=single]
<response> 
	<list>
		<category>(noun)</category> 
		<synonyms> order|war (antonym) </synonyms>
	</list>
	<list>
		<category>(noun)</category> 
		<synonyms> harmony|concord|concordance </synonyms>
	</list>
	<list>
		<category>(noun)</category> 
		<synonyms> public security|security </synonyms>
	</list>
	<list>
		<category>(noun)</category> 
		<synonyms> peace treaty|pacification|treaty|pact|accord </synonyms>
	</list>
\end{lstlisting}





\lstset{
	string=[s]{"}{"},
	stringstyle=\color{blue},
	comment=[l]{:},
	commentstyle=\color{black},
}

\begin{lstlisting} [caption=Ejemplo de salida de Thesaurus en formato JSON para la palabra \textit{peace}, label={lst:jsonthesaurus}, frame=single]
{
	"response":
	[
		{ "list": 
			{ "category": "(noun)", "synonyms": "order|war (antonym)" }
		},
		{ "list": 
			{ "category": "(noun)", "synonyms": "harmony|concord|concordance" }
		},
		{ "lista": 
			{ "categoria": "(noun)", "synonyms": "public security|security" }
		},
		{ "lista": 
			{ "categoria": "(noun)", "synonyms": "peace treaty|pacification|treaty|pact|accord" }
		}
	
	]
}
\end{lstlisting}




\subsection{Thesaurus Rex}
\label{cap:subsec:thesaurusrex}

Thesaurus Rex\footnote{http://ngrams.ucd.ie/therex3/} es una red semántica que solo admite palabras en inglés y que permite obtener las palabras relacionadas, con una palabra o las categorías que comparten dos palabras, por ejemplo si se introducen las palabras \textit{coffe y cola}, las categorías que comparten dichos conceptos son \textit{cold-beverage, dark-beverage, stimulating-beverage, etc...}
Si por el contrario, únicamente se ha introducido una palabra como se puede ver en la Figura \ref{fig:thesaurusrex} para la palabra  \textit{house}. La aplicación devuelve un listado de las categorías más utilizadas por los hablantes de dicha lengua, como por ejemplo \textit{permanent-structure, inanimate-object, everyday-object, etc...},
 otro listado de las categorías matizadas de dicho concepto, como por ejemplo \textit{permanent, fixed, wooden, etc...} y por último un listado de categorías simples del concepto, como por ejemplo \textit{structure, object, item, building, etc...}

Por otro lado, la aplicación devuelve estos resultados en formato XML, y este como se puede ver en el Listado \ref{lst:xmlthesaurusrex} para la palabra \textit{house}, los divide en distintos campos: \textit{Categories}, \textit{Modifiers} y \textit{CategoryHeads}. Como se puede ver todos los resultados tienen un peso asignado, esto significa que cuanto mayor sea el peso mayor es la similitud con el concepto dado, y en la página aparecerá dicha palabra en un tamaño superior al resto.
Los campos que se encuentran dentro del apartado \textit{categories} son los resultados más utilizados en ese momento por los hablantes y que Thesaurus Rex ha encontrado, los que se encuentran dentro de  \textit{modifiers} son atributos del concepto a buscar y por último los que se encuentran en \textit{categoryHeads} son las categorías más simples que se han para dicho concepto.
Thesaurus Rex utiliza la Web para generar sus resultados, con lo cual la información disponible no es fija, sino que varía según los datos actuales de la web.
La ventaja de utilizar esta herramienta es que se encuentra en continua actualización, pero el inconveniente es que en algunos casos la información puede resultar un poco extraña dado que se crea semiautomáticamente desde contenido de la web \citep{TFMPaloma}.

\lstset{language=XML}
\begin{lstlisting}[caption= Ejemplo formatos XML Thesaurus Rex, label={lst:xmlthesaurusrex}, frame=single]
<MemberData>
	<Categories kw="house">
		<Category weight="91"> large:object </Category>
		<Category weight="307"> inanimate:object </Category>
		<Category weight="261"> everyday:object </Category>
		<Category weight="154"> sensitive:area </Category>
		<Category weight="318"> permanent:structure </Category>
		<Category weight="194"> permanent:construction </Category>
		<Category weight="148"> permanent:installation </Category>
		<Category weight="98"> fixed:object </Category>
	</Categories>
	
	<Modifiers kw="house">
		<Modifier weight="8"> recognizable </Modifier>
		<Modifier weight="9"> relevant </Modifier>
		<Modifier weight="863"> permanent </Modifier>
		<Modifier weight="15"> moderate </Modifier>
		<Modifier weight="477"> fixed </Modifier>
		<Modifier weight="5"> odd </Modifier>
		<Modifier weight="7"> archaeological </Modifier>
		<Modifier weight="5"> electrical </Modifier>
	</Modifiers>
	
	<CategoryHeads kw="house">
		<CategoryHead weight="6"> protection </CategoryHead>
		<CategoryHead weight="40"> obstruction </CategoryHead>
		<CategoryHead weight="5"> whole </CategoryHead>
		<CategoryHead weight="3320"> object </CategoryHead>
		<CategoryHead weight="2340"> structure </CategoryHead>
		<CategoryHead weight="98"> commodity </CategoryHead>
		<CategoryHead weight="713"> thing </CategoryHead>
		<CategoryHead weight="2"> theatre </CategoryHead>
	</CategoryHeads>
</MemberData>

\end{lstlisting}

\figura{Bitmap/Capitulo2/thesaurusrex}{width=.6\textwidth}{fig:thesaurusrex}{Resultados búsqueda Thesaurus Rex con la palabra \textit{house}}


\subsection{Metaphor Magnet}
\label{cap:subsec:metaphormagnet}
Aplicación web\footnote{http://ngrams.ucd.ie/metaphor-magnet-acl/}, que dada una palabra crea metáforas. El objetivo \citep{VealeT2012} de dicha aplicación es encontrar metáforas comunes en los n-gramas de Google. Se llama n-grama\footnote{https://en.wikipedia.org/wiki/N-gram} a una subsecuencia de n elementos consecutivos en una secuencia dada, es decir, contar la cantidad de veces que aparece una palabra, la cantidad de veces que aparecen dos palabras juntas, tres palabras, etc... Google utiliza estos mapeos para interpretar metáforas. Esta aplicación está limitada, ya que como se puede observar en la Figura \ref{fig:metaphormagnet} introduciendo la palabra \textit{house} solo está disponible para el inglés.
Esta consulta devuelve un fichero XML como el expuesto en el Listado \ref{lst:xmlmetaphormagnet}, que puede ser utilizado para otras aplicaciones de Procesamiento de Lenguaje Natural.


\lstset{language=XML}
\begin{lstlisting}[caption= Ejemplo formatos XML Metaphor Magnet, label={lst:xmlmetaphormagnet}, frame=single]
<Metaphor>
	<Source Name="house">
		<Text> towering:mountain </Text>
		<Score> 88 </Score>
	</Source>
	<Source Name="house">
		<Text> protecting:home </Text>
		<Score> 86 </Score>
	</Source>
	<Source Name="house">
		<Text> tall:building </Text>
		<Score> 86 </Score>
	</Source>
	<Source Name="house">
		<Text> charming:castle </Text>
		<Score> 85 </Score>
	</Source>
	<Source Name="house">
		<Text> beautiful:tree </Text>
		<Score> 84 </Score>
	</Source>
	<Source Name="house">
		<Text> charming:mansion </Text>
		<Score> 83 </Score>
	</Source>
	<Source Name="house">
		<Text> strong:rock </Text>
		<Score> 80 </Score>
	</Source>
	<Source Name="house">
		<Text> strong:elephant </Text>
		<Score> 80 </Score>
	</Source>
</Metaphor>
\end{lstlisting}
\figura{Bitmap/Capitulo2/metaphormagnet}{width=.6\textwidth}{fig:metaphormagnet}{Resultados búsqueda Metaphor Magnet con la palabra \textit{house}}




\subsection{WordNet}
\label{cap:subsec:wordnet}
WordNet es un \textit{corpus}\footnote{Colección de documentos de texto} perteneciente a NLTK \textit{(Natural Languaje Toolkit)}\footnote{Conjunto de bibliotecas y programas para el Procesamiento del Lenguaje Natural} que almacena distintos tipos de palabras como sustantivos, verbos, adjetivos y adverbios ignorando preposiciones, determinantes y otras palabras funcionales en varios idiomas como el español, el inglés o el francés. Los conceptos se agrupan en conjuntos de sinónimos cognitivos llamados \textit{synsets}. Cada \textit{synset} contiene:
\begin{itemize}
	\item Una serie de sinónimos llamados \textit{lemmas}.
	\item Hiperónimos, los cuales son palabras cuyo significado está incluido en el de otras\footnote{https://dle.rae.es/?id=KRW1qe2}, por ejemplo: mamífero es hiperónimo de gato y de perro ya que los gatos y los perros pertenecen al conjunto de los mamíferos.
	\item Hipónimos, que son los antónimos del hiperónimo, es decir, son palabras cuyo significado incluyen el de otra\footnote{https://dle.rae.es/?id=KU5UAn5}, por ejemplo: gato es hipónimo de mamífero ya que está incluido dentro del conjunto de los mamíferos.
	\item Holónimos, son palabras que representan el todo respecto a una parte, por ejemplo: coche es el holónimo de rueda, volante y acelerador ya que forman parte de un todo, que es el coche.
	\item Por último, contiene una breve definición y en muchas ocasiones, oraciones cortas que explican su significado.		
\end{itemize} 
Por cada uno de los \textit{lemmas} que contenga el \textit{synset}, se podrán consultar sus antónimos. En caso de que la palabra tenga distintos significados, aparecerá un \textit{synset} por cada una\footnote{http://www.nltk.org/howto/wordnet.html}.

Hay múltiples interfaces web que implementan esta biblioteca, una de las más completas es eurowordnet, ya que está disponible en varios idiomas y permite extraer sinónimos, antónimos, hiperónimos, hipónimos y holónimos. Además de oraciones de ejemplo y definiciones de cada una de las acepciones obtenidas de la palabra que se haya buscado.


\figura{Bitmap/Capitulo2/eurowordnet2}{width=.8\textwidth}{fig:eurowordnet2}{Resultados búsqueda en Eurowordnet con la palabra \textit{casa}}

Como se puede observar en la figura \ref{fig:eurowordnet2} aparecen una serie de sinónimos que corresponden a una determinada acepción de la palabra buscada, con una frase de ejemplo y una definición explicándolo. Además, en el desplegable de la izquierda se muestra lo que se puede obtener de dicha palabra.


