\chapter{Estado de la Cuestión}
\label{cap:estadoDeLaCuestion}
En este capítulo se va a tratar aspectos importantes dentro del ámbito de la retórica así como una explicación detallada de Lectura Fácil, y sin olvidar aquellas herramientas y tecnologías que se van a utilizar.
En la sección 2.1 quedará detalladamente explicado el concepto de Servicio Web, su funcionalidad, ventajas y desventajas, tipos que existen....
En la sección 2.2 se definirá las Figuras retóricas...



%-------------------------------------------------------------------
\section{Servicios Web}
%-------------------------------------------------------------------
\label{cap:sec:servicios_web}

Para definir el concepto de servicio Web de la forma más simple posible, se podría decir que es una tecnología que utiliza un conjunto de protocolos para intercambiar datos entre aplicaciones, sin importar el lenguaje de programación en el cual estén programadas o ejecutadas en cualquier tipo de plataforma. Otra definición más técnica según la W3C (\textit{World Wide Web Consortium})\footnote{https://www.w3.org/},  un servicio web es un sistema software diseñado para soportar la interacción máquina-a-máquina, a través de una red, de forma interoperable. 

\subsection{Características de  Servicios Web}
\label{cap:subsec:serviciosweb}

\begin{itemize}
\item Un servicio debe poder ser accesible a través de la Web. Para ello debe utilizar protocolos de transporte estándares como HTTP, y codificar los mensajes en un lenguaje estándar que pueda ser accesible por cualquier cliente que quiera utilizar el servicio. 

\item Un servicio debe contener una descripción de sí mismo. De esta forma, una aplicación podrá saber cual es la función de un determinado Servicio Web, y cuál es su interfaz, de manera que pueda ser utilizado de forma automática por cualquier aplicación, sin la intervención del usuario.
\item Debe poder ser localizado. Deberemos tener algún mecanismo que nos permita encontrar un Servicio Web que realice una determinada función. De esta forma tendremos la posibilidad de que una aplicación localice el servicio que necesite de forma automática, sin tener que conocerlo previamente el usuario.
\end{itemize}


\subsection{Tipos de Servicios Web}
\label{cap:subsec:serviciosweb}

Los servicios web pueden definirse tanto a nivel conceptual como a nivel técnico, es por ello que mediante este último podemos diferenciar dos tipos distintos de servicio web:
\begin{itemize}
	\item Servicios Web RESTful: no tienen estado. Cada petición del cliente al servidor debe contener toda la información necesaria para entender la petición, y no puede aprovecharse de ningún contexto almacenado en el servidor.
	\item Servicios web SOAP  \textit({Simple Object Access Protocol}): es un protocolo basado en XML para el intercambio de información entre ordenadores. Normalmente utilizaremos SOAP para conectarnos a un servicio e invocar métodos remotos.
\end{itemize}

\subsection{Arquitectura Servicios Web}
\label{cap:subsec:serviciosweb}
Para que los servicios web funcionen correctamente, hay que implementar distintos "roles", como por ejemplo, el proveedor del servicio es
la aplicación que implementa el servicio y lo hace accesible desde Internet. El soliciante, es decir, cualquier cliente que necesite
utilizar un servicio web, y por último, el registro y publicación del servicio, el cual se refiere a que debería existir algún
repositorio centralizado que nos proporcione información sobre los servicios que tenemos disponibles y cómo se utilizan. A continuación,
explicamos más detalladamente las distintas capas que tiene el servicio web:
\begin{itemize}
 \item Service Discovery. Es el responsable de centralizar los servicios web en un directorio común de desta forma es mas sencillo buscar y publicar.
 \item Service Description. Como ya hemos comentado con anterioridad, los servicios web se pueden definir así mismos, por lo que una vez que los localicemos nos darán suficiente información para saber que operaciones soporta y como activarlo.
 \item Service Invocation. Invocar a un Web Service implica pasar mensajes entre el cliente y el servidor. SOAP  \textit({Simple Object Access Protocol}) especifica cómo deberíamos formatear los mensajes request para el servidor, y cómo el servidor debería formatear sus mensajes de respuesta.
 \item Transport. Todos estos mensajes han de ser transmitidos de alguna forma entre el servidor y el cliente. El protocolo elegido para ello es HTTP  \textit({(HyperText Transfer Protocol)}). 
\end{itemize}

\subsection{Ventajas y Desventajas de los  Servicios Web}
\label{cap:subsec:serviciosweb}
Estos tienen multitud de ventajas pero las principales son las siguientes:
\begin{itemize}
	 \item Permiten la integración “justo-a-tiempo”:  Esto significa que los solicitantes, los proveedores y los agentes actúan en conjunto para crear sistemas que son auto-configurables, adaptativos y robustos.
	\item Reducen la complejidad por medio del encapsulamiento: Los solicitantes y los proveedores del servicio se preocupan por las interfaces necesarias para interactuar. Como resultado, un solicitante de servicio no sabe cómo fue implementado el servicio por parte del proveedor, y éste a su vez, no sabe cómo utiliza el cliente el servicio. Estos detalles se encapsulan en los solicitantes y proveedores. El encapsulamiento es crucial para reducir la complejidad.
	\item Promueven la interoperabilidad: La interacción entre un proveedor y un solicitante de servicio está diseñada para que sea completamente independiente de la plataforma y el lenguaje. 
	\item Abren la puerta a nuevas oportunidades de negocio: Los servicios web facilitan la interacción con socios de negocios, al poder compartir servicios internos con un alto grado de integración.
	\item Disminuyen el tiempo de desarrollo de las aplicaciones: Pues gracias a la filosofía de orientación a objetos que utilizan, el desarrollo se convierte más bien en una labor de composición.
	\item Fomentan los estándares y protocolos basados en texto, que hacen más fácil acceder a su contenido y entender su funcionamiento.
\end{itemize}
Pero también tiene desventajas, las cuales tenemos que nombrar:
\begin{itemize}
	\item Al apoyarse en HTTP, pueden esquivar medidas de seguridad basadas en firewall cuyas reglas tratan de bloquear.
	\item Existe poca información de servicios web para algunos lenguajes de programación .
	\item Dependencia de la disponibilidad de servidores y comunicaciones.
\end{itemize}
\section{Figuras retóricas}

 Las figuras literarias(o retóricas) son formas no convencionales de utilizar las palabras, de manera que, aunque se emplean con sus acepciones habituales, se acompañan de algunas particularidades fónicas, gramaticales o semánticas, que las alejan de ese uso habitual, por lo que terminan por resultar especialmente expresivas. (\textit {Introducción al análisis retórico: tropos, figuras y sintaxis del estilo})\footnote{Azaustre Galiana, Antonio y Juan Casas Rigall, Introducción al análisis retórico: tropos, figuras y sintaxis del estilo, Universidad de Santiago de Compostela, Santiago de Compostela, 1994.}

En este trabajo vamos a trabajar con tres tipos de figuras retóricas: Analogía, metáfora y símil.
\begin{itemize}
	\item Una metáfora: Su estudio se remonta a Aristóteles y su Retórica. Una metáfora se refiere a una cosa mencionando otra, utiliza el desplazamiento de características similares entre dos conceptos con fines estéticos o retóricos. Por ejemplo, cuando una persona tiene muy buena memoria, se la dice que tiene memoria de elefante.
	
	\item Un símil: Realiza una comparación entre términos. A pesar de que los símiles y las metáforas son similares, los símiles utilizan explícitamente, aunque no necesariamente, conectores (por ejemplo, como, cual, que, o varios verbos tales como se asemejan).
	Por ejemplo: Cuando nos referimos a una persona que es muy corpulenta, se dice que es como un oso.
	
	\item Una analogía: Es la comparación entre varios conceptos, indicando las características que permiten dicha relación. En la retórica, una analogía es una comparación textual que resalta alguna de las similitudes semánticas entre los conceptos protagonistas en dicha comparación. Por ejemplo, sus ojos son azules como el mar.
	
(\footnote {GENERACIÓN DE RECURSOS LINGÜÍSTICOS MEDIANTE LA EXTRACCIÓN DE RELACIONES ENTRE CONCEPTOS})	
	
\end{itemize}

\begin{itemize}
\item Analogía: Según la RAE, es la relación de semejanza entre dos cosas distintas.
\item Símil: Según la RAE, es la comparación, semejanza entre dos cosas.
\item Metáfora: Una metáfora es una Figura retórica de pensamiento por medio de la cual una realidad o concepto se expresan por medio de una realidad o concepto diferentes con los que lo representado guarda cierta relación de semejanza.
\end{itemize}



\section{Lectura Fácil}

// por que era de lectura facil
La lectura fácil surgió en Suecia. En 1968 se editó el primer libro en la Agencia de Educación en el marco de un proyecto experimental. A continuación, en 1976 se creó en el Ministerio de Justicia un grupo de trabajo para conseguir textos legales más claros.

La rama de educación, comenzó en 1984 la edición de un libro llamado  "8 Sidor", que tres años más tarde,en 1987, se publica de forma permanente en papel hasta que empezó a editarse en la web. Este mismo año la Fundación Lectura Fácil asume la publicación del semanario y de los libros escritos en lectura fácil.


\footnote{https://dilofacil.wordpress.com/2013/12/04/el-origen-de-la-lectura-facil/}

Los documentos escritos en Lectura Fácil son documentos de todo tipo que siguen las directrices internacionales de la IFLA (International Federation of Library Associations and Institutions) y de Inclusion Europe en cuanto al contenido y la forma.

Está dirigida a colectivos que tengan dificultades lectoras como inmigrantes o personas con trastornos de aprendizaje o problemas cognitivos.
Algunos ejemplos de como se debe de escribir un texto en Lectura Fácil son:
\begin{itemize}
	\item Evitar mayúsculas fuera de la norma, es decir, escribir en mayúsculas sólo cuando toca según las reglas ortográficas
	\item Limitar el uso de la coma y evitar signos menos usados de puntuación
	\item Evitar tiempos verbales complejos
	\item Utilizar oraciones simples
	\item Utilizar palabras cortas y de sílabas poco complejas
	\item Incluir imágenes y pictogramas a la izquierda y su texto vinculado a la derecha
	\item Utilizar uno o dos tipos de letra como mucho
	\item Tamaño de letra entre 12 y 16 puntos
	\item Evitar frases superiores a 60 caracteres
	\item Transmitir una idea por línea y cada línea se rompe en un punto natural del discurso
	\item Alineación a la izquierda
	\item Evitar la saturación de texto e imágenes
	\item Incluir el logo de lectura fácil para que se reconozca este tipo de obras. 
	\item Si el documento está paginado, incluir la paginación claramente y reforzar el mensaje de que la información continúa en la página siguiente.
	(\textit {Lectura fácil: Métodos de redacción y evaluación.  Óscar García Muñoz.})	
\end{itemize}

\section{ConceptNet} 

Es una red semántica diseñada para ayudar a los ordenadores a entender los significados de las palabras utilizadas por la gente lanzada por el MIT en 1999.
Tiene un buscador de palabras en el que al introducir una palabra, se selecciona el idioma y devuelve sinónimos y términos relacionados.

