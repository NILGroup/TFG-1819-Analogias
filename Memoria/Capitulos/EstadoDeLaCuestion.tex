\chapter{Estado de la Cuestión}
\label{cap:estadoDeLaCuestion}

\begin{resumen}
	En este capítulo se van a tratar aspectos importantes dentro del ámbito de la retórica así como una explicación detallada de Lectura Fácil, sin olvidar aquellas herramientas y tecnologías que se van a utilizar.
	En la sección 2.1 se explica el concepto de lectura fácil, su historia, patrones a seguir y ejemplos para que se pueda entender aún mejor. 
	En la sección 2.2
	En la sección 2.3 figuras retóricas
	En la sección 2.4 se explica el concepto de Servicio Web, su funcionalidad, ventajas y desventajas y tipos que existentes.
	
\end{resumen}



\section{Lectura Fácil}

Se llama lectura fácil a aquellos contenidos que han sido resumidos y realizados con lenguaje sencillo y claro, de forma que puedan ser entendidos por personas con discapacidad cognitiva o discapacidad intelectual. Es la adaptación de textos, ilustraciones y maquetaciones que permite una mejor lectura y comprensión.
Nosotros nos vamos a centrar en la lectura fácil aplicada a textos.

Surgió en Suecia. En 1968 se editó el primer libro en la Agencia de Educación en el marco de un proyecto experimental. A continuación, en 1976 se creó en el Ministerio de Justicia un grupo de trabajo para conseguir textos legales más claros.

El Ministerio de Educación comenzó en 1984 la edición de un libro llamado  "8 Sidor", que tres años más tarde,en 1987, se publica de forma permanente en papel hasta que empezó a editarse en la web. Este mismo año la Fundación Lectura Fácil asume la publicación del semanario y de los libros escritos en lectura fácil.


\footnote{https://dilofacil.wordpress.com/2013/12/04/el-origen-de-la-lectura-facil/}

Los documentos escritos en Lectura Fácil son documentos de todo tipo que siguen las directrices internacionales de la IFLA (International Federation of Library Associations and Institutions) y de Inclusion Europe en cuanto al contenido y la forma.

Está dirigida a colectivos que tengan dificultades lectoras como inmigrantes o personas con trastornos de aprendizaje o problemas cognitivos.
Algunos ejemplos de como se debe de escribir un texto en Lectura Fácil son:
\begin{itemize}
	\item Evitar mayúsculas fuera de la norma, es decir, escribir en mayúsculas sólo cuando toca según las reglas ortográficas
	\item Limitar el uso de la coma y evitar signos menos usados de puntuación
	\item Evitar tiempos verbales complejos
	\item Utilizar oraciones simples
	\item Utilizar palabras cortas y de sílabas poco complejas
	\item Incluir imágenes y pictogramas a la izquierda y su texto vinculado a la derecha
	\item Utilizar uno o dos tipos de letra como mucho
	\item Tamaño de letra entre 12 y 16 puntos
	\item Evitar frases superiores a 60 caracteres
	\item Transmitir una idea por línea y cada línea se rompe en un punto natural del discurso
	\item Alineación a la izquierda
	\item Evitar la saturación de texto e imágenes
	\item Incluir el logo de lectura fácil para que se reconozca este tipo de obras. 
	\item Si el documento está paginado, incluir la paginación claramente y reforzar el mensaje de que la información continúa en la página siguiente.
	(\textit {Lectura fácil: Métodos de redacción y evaluación.  Óscar García Muñoz.})	
\end{itemize}

A continuación vamos a hablar del lenguaje natural, y de que relación tiene con la lectura fácil:

El lenguaje natural es la forma en la que los humanos nos comunicamos día a día ya sea de forma escrita u oral
\footnote{https://sistemas.com/lenguaje-natural.php/}

Actualmente, el Procesamiento del lenguaje natural, que es un campo de la Inteligencia Artificial que investiga la manera de comunicar a las máquinas con el ser humano, se está aplicando a la lectura fácil. Por ejemplo, el proyecto "Simplext" desarrollado por una empresa de la Fundación ONCE que tiene como objetivo conseguir un adaptador automático de texto normal a texto en lectural fácil
\footnote{http://www.iic.uam.es/inteligencia/que-es-procesamiento-del-lenguaje-natural/}
\footnote{https://dilofacil.wordpress.com/2013/11/07/procesamiento-del-lenguaje-natural-el-futuro-para-la-lectura-facil/}
\footnote{https://www.fundaciononce.es/es/noticia/el-proyecto-simplext-de-technosite-finalista-de-los-premios-bdigital-la-innovacion/}


\section{Procesamiento del lenguaje natural}
El procesamiento del lenguaje natural es una rama de la Inteligencia Artificial
que se encarga de investigar la manera de comunicar máquinas con personas mediante el uso de lenguajes naturales, como pueden ser el español o el inglés...

\subsection{ConceptNet} 

Es una red semántica diseñada para ayudar a los ordenadores a entender el significado de las palabras y fue creada por el MIT en 1999. Esta disponible en múltiples idiomas, como el español, el inglés o el chino.
Dada una palabra y el idioma, devuelve sinónimos y términos relacionados.
\footnote{http://conceptnet.io/}

Por ejemplo, si introducimos la palabra "dinero", especificando que el idioma es el español, ConceptNet te devuelve como sinónimos: lana, pasta, billete, plata. y como términos relacionados: dineral y moneda.

Dispone de una API \textit({http://api.conceptnet.io/}) que devuelve los datos en formato JSON, que utilizaremos para obtener los términos relacionados y sinónimos que necesitamos.

\section{Figuras retóricas}
Las figuras literarias(o retóricas) son formas no convencionales de utilizar las palabras, de manera que, aunque se emplean con sus acepciones habituales, se acompañan de algunas particularidades fónicas, gramaticales o semánticas, que las alejan de ese uso habitual, por lo que terminan por resultar especialmente expresivas. 
La metáfora, el símil y la analogía se basan en la comparación entre dos conceptos, el de origen, que es el término literal (al que la metáfora se refiere) es el llamado tenor. El de destino, que es el término figurado es el vehículo. La relación que hay entre el tenor y el vehículo se denomina fundamento. Por ejemplo, en la metáfora \textit{Tus ojos son dos luceros}, \textit{ojos} es el tenor, \textit{luceros} es el vehículo y el fundamentos es la belleza de los ojos.
(\textit {Introducción al análisis retórico: tropos, figuras y sintaxis del estilo})\footnote{Azaustre Galiana, Antonio y Juan Casas Rigall, Introducción al análisis retórico: tropos, figuras y sintaxis del estilo, Universidad de Santiago de Compostela, Santiago de Compostela, 1994.}

En este trabajo vamos a trabajar con tres tipos de figuras retóricas: Analogía, metáfora y símil.
\begin{itemize}
	\item Metáfora: Se refiere a una cosa mencionando otra, utiliza el desplazamiento de características similares entre dos conceptos con fines estéticos o retóricos. Por ejemplo, cuando una persona tiene muy buena memoria, se dice que tiene memoria de elefante. Ya que una de las características de los elefantes es que tienen buena memoria.
	
	\item Símil: Realiza una comparación entre dos términos. A pesar de que los símiles y las metáforas son similares, los símiles utilizan explícitamente, aunque no necesariamente, conectores (por ejemplo, como, cual, que, o varios verbos tales como se asemejan).
	Por ejemplo, cuando nos referimos a una persona que es muy corpulenta, se dice que es como un oso, ya que los osos son muy grandes
	
	\item Analogía: Es la comparación entre varios conceptos, indicando las características que permiten dicha relación. En la retórica, una analogía es una comparación textual que resalta alguna de las similitudes semánticas entre los conceptos protagonistas en dicha comparación. Por ejemplo, sus ojos son azules como el mar, comparándolos con el color del mar
	
	(\footnote {GENERACIÓN DE RECURSOS LINGÜÍSTICOS MEDIANTE LA EXTRACCIÓN DE RELACIONES ENTRE CONCEPTOS})	
	
\end{itemize}

%-------------------------------------------------------------------
\section{Servicios Web}
%-------------------------------------------------------------------
\label{cap:sec:servicios_web}

Para definir el concepto de servicio Web de la forma más simple posible, se podría decir que es una tecnología que utiliza un conjunto de protocolos para intercambiar datos entre aplicaciones, sin importar el lenguaje de programación en el cual estén programadas o ejecutadas en cualquier tipo de plataforma. Según el W3C(\textit{World Wide Web Consortium})\footnote{https://www.w3.org/}, un servicio web es un sistema software diseñado para soportar la interacción máquina-a-máquina, a través de una red, de forma interoperable. \\



\textbf{Las principales características de un servicio web son:}


\begin{itemize}
\item Debe poder ser accesible a través de la Web. Para ello debe utilizar protocolos de transporte estándares como HTTP, y codificar los mensajes en un lenguaje estándar que pueda ser accesible por cualquier cliente que quiera utilizar el servicio. 

\item Debe contener una descripción de sí mismo. De esta forma, una aplicación web podrá saber cual es la función de un determinado Servicio Web, y cuál es su interfaz, de manera que pueda ser utilizado de forma automática por cualquier aplicación, sin la intervención del usuario.
\item Debe poder ser localizado. Deberemos tener algún mecanismo que nos permita encontrar un Servicio Web que realice una determinada función. De esta forma tendremos la posibilidad de que una aplicación localice el servicio que necesite de forma automática, sin tener que conocerlo previamente el usuario.
\end{itemize}


Los servicios web pueden definirse tanto a nivel conceptual como a nivel técnico, es por ello que mediante este último podemos diferenciar dos tipos distintos de servicio web:
\begin{itemize}
	\item Servicios Web RESTful: no tienen estado. Cada petición del cliente al servidor debe contener toda la información necesaria para entender la petición, y no puede aprovecharse de ningún contexto almacenado en el servidor.
	\item Servicios web SOAP  \textit({Simple Object Access Protocol}): es un protocolo basado en XML para el intercambio de información entre ordenadores. Normalmente utilizaremos SOAP para conectarnos a un servicio e invocar métodos remotos.
\end{itemize}

\subsection{Arquitectura Servicios Web}
\label{cap:subsec:arquitecturaserviciosweb}
Los servicios web se componen fundamentalmente de tres partes:
\begin{itemize}
	\item Proveedor: Es la aplicación que implementa el servicio y lo hace accesible mediante Internet
	\item Solicitante: Cualquier persona que necesite utilizar el servicio web
	\item Publicador: Se refiere al repositorio centralizado en el que se encuentra la información de la funcionalidad disponible y como se utiliza
	
\end{itemize}
 A continuación,
explicamos más detalladamente las distintas capas que tiene el servicio web:
\begin{itemize}
 \item Service Discovery. Es el responsable de centralizar los servicios web en un directorio común de esta forma es mas sencillo buscar y publicar.
 \item Service Description. Como ya hemos comentado con anterioridad, los servicios web se pueden definir así mismos, por lo que una vez que los localicemos nos darán suficiente información para saber que operaciones soporta y como activarlo.
 \item Service Invocation. Invocar a un Web Service implica pasar mensajes entre el cliente y el servidor. SOAP  \textit({Simple Object Access Protocol}) especifica cómo deberíamos formatear los mensajes request para el servidor, y cómo el servidor debería formatear sus mensajes de respuesta.
 \item Transport. Todos estos mensajes han de ser transmitidos de alguna forma entre el servidor y el cliente. El protocolo elegido para ello es HTTP  \textit({(HyperText Transfer Protocol)}). 
\end{itemize}

\subsection{Ventajas de los  Servicios Web}
\label{cap:subsec:ventajasserviciosweb}
	Las principales ventajas de los servicios web son las siguientes:
\begin{itemize}
	\item Permiten la integración “justo-a-tiempo”:  Esto significa que los solicitantes, los proveedores y los agentes actúan en conjunto para crear sistemas que son auto-configurables, adaptativos y robustos.
	\item Reducen la complejidad por medio del encapsulamiento: Un solicitante de servicio no sabe cómo fue implementado el servicio por parte del proveedor, y éste a su vez, no sabe cómo utiliza el cliente el servicio. Estos detalles se encapsulan en los solicitantes y proveedores. El encapsulamiento es crucial para reducir la complejidad.
	\item Promueven la interoperabilidad: La interacción entre un proveedor y un solicitante de servicio está diseñada para que sea completamente independiente de la plataforma y el lenguaje. 
	\item Abren la puerta a nuevas oportunidades de negocio: Los servicios web facilitan la interacción con socios de negocios, al poder compartir servicios internos con un alto grado de integración.
	\item Disminuyen el tiempo de desarrollo de las aplicaciones: Gracias a la filosofía de orientación a objetos que utilizan, el desarrollo se convierte más bien en una labor de composición.
	\item Fomentan los estándares y protocolos basados en texto, que hacen más fácil acceder a su contenido y entender su funcionamiento.
\end{itemize}

\subsection{Desventajas de los  Servicios Web}
\label{cap:subsec:desventajasserviciosweb}
	El uso de servicios web también tiene desventajas:
\begin{itemize}
	\item Al apoyarse en HTTP, pueden esquivar medidas de seguridad basadas en firewall cuyas reglas tratan de bloquear.
	\item Existe poca información de servicios web para algunos lenguajes de programación.
	\item Dependen de la disponibilidad de servidores y comunicaciones.
\end{itemize}




