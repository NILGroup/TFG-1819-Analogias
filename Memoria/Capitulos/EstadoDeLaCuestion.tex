\chapter{Estado de la Cuestión}
\label{cap:estadoDeLaCuestion}



Actualmente, los servicios web son una tecnología utilizada a diario por la mayoría de la población ya sea para la traducción de texto o la consulta del tiempo.

Debido a esto, para el desarrollo de la aplicación utilizaremos esta tecnología. Además, las definiciones generadas deben de ser símiles, metáforas o analogías ya que mediante estas figuras retóricas, se puede comparar el término que se quiere definir con otro concepto más cotidiano para facilitar su comprensión. Estas definiciones deben de estar en un formato que sea comprensible por el mayor número de gente posible para ello se empleará una adaptación llamada ``lectura fácil'' que tiene como objetivo que los texto estén en un formato fácil de leer. Para crear estas definiciones se utilizará Inteligencia Artificial para buscar términos relacionados o sinónimos más sencillos de la palabra que se quiere definir, concretamente la rama del Procesamiento del Lenguaje Natural.

%-------------------------------------------------------------------
\section{Lectura Fácil}
%-------------------------------------------------------------------
\label{cap:sec:lecturafacil}

Se llama lectura fácil a aquellos contenidos que han sido resumidos y reescritos con lenguaje sencillo y claro, de forma que puedan ser entendidos por personas con discapacidad cognitiva o discapacidad intelectual. Es decir, es la adaptación de textos, ilustraciones y maquetaciones que permite una mejor lectura y comprensión.
Este trabajo se va a centrar en la lectura fácil aplicada a textos.

La lectura fácil surgió en Suecia en el año 1968, donde se editó el primer libro en la Agencia de Educación en el marco de un proyecto experimental. A continuación, en 1976, se creó en el Ministerio de Justicia un grupo de trabajo para conseguir textos legales más claros.
En 1984 nació el primer periódico en lectura fácil, titulado "8 páginas", que tres años más tarde, en 1987, se publicó de forma permanente en papel hasta que empezó a editarse en la web. 
En el año 2013, en México se produce la primera sentencia judicial en lectura fácil\footnote{https://dilofacil.wordpress.com/2013/12/04/el-origen-de-la-lectura-facil/}. En la actualidad, podemos distinguir los documentos en lectura fácil gracias al logo de la Figura \ref{fig:lecturafacil}.

	\figura{Bitmap/Capitulo2/lecturaFacil}{width=.3\textwidth}{fig:lecturafacil}{Logo Lectura Fácil} 
	
	
Los documentos escritos en Lectura Fácil \citep{wiki:lecturafacil} son documentos de todo tipo que siguen las directrices internacionales de la IFLA\footnote{International Federation of Library Associations and Institutions} y de la Inclusion Europe\footnote{Una asociación de personas con discapacidad intelectual y sus familias en Europa} en cuanto al contenido y la forma.
Algunas pautas a seguir para escribir correctamente un texto en Lectura Fácil son \citep{GarciaMunoz2012LecturaFacil}:


\begin{itemize}
	\item Evitar mayúsculas fuera de la norma, es decir, escribir en mayúsculas sólo cuando lo dicten las reglas ortográficas, como por ejemplo, después de un punto o la primera letra de los nombres propios.
	\item Deben evitarse el punto y seguido, el punto y coma y los puntos suspensivos. El punto y aparte hará la función del punto y seguido.
	\item Evitar corchetes y signos ortográficos poco habituales, como por ejemplo: \%, \& y /.
	\item Evitar frases superiores a 60 carácteres y utilizar oraciones simples. Por ejemplo, la oración \textit{Caperucita ha ido a casa de su abuela y ha desayunado con ella} es mejor dividirla en dos oraciones simples:\textit{ Caperucita ha ido a casa de su abuela} y  \textit{Caperucita ha desayunado con ella}.
	\item Evitar tiempos verbales como: futuro, subjuntivo, condicional y formas compuestas.
	\item Utilizar palabras cortas y de sílabas poco complejas. 
	Por ejemplo: casa, gato, comer o mano.
	\item Evitar abreviaturas, acrónimos y siglas.
	\item Alinear el texto a la izquierda.
	\item Incluir imágenes y pictogramas a la izquierda y su texto vinculado a la derecha.
	\item Evitar la saturación de texto e imágenes.
	\item Utilizar uno o dos tipos de letra como mucho.
	\item Tamaño de letra entre 12 y 16 puntos.
	\item Si el documento está paginado, incluir la paginación claramente y reforzar el mensaje de que la información continúa en la página siguiente.
\end{itemize}

Debemos también hacer hincapié en la distinción entre palabras fáciles y complejas, puesto que son de gran importancia para la lectura fácil. 
Las palabras complejas son aquellas que no se utilizan a menudo en nuestra sociedad, como por ejemplo: melifluo o inefable. Es por ello que este tipo de palabras deben estar totalmente descartadas en la lectura fácil, y en su lugar debemos introducir palabras fáciles, que son aquellas que se utilizan asiduamente. La RAE dispone de un documento con las mil palabras más usadas\footnote{http://corpus.rae.es/lfrecuencias.html}.

%-------------------------------------------------------------------



%-------------------------------------------------------------------
\section{Figuras retóricas}
%-------------------------------------------------------------------
\label{cap:sec:figurasretoricas}

Las figuras literarias (o retóricas) son formas no convencionales de utilizar las palabras, de manera que, aunque se emplean con sus acepciones habituales, se acompañan de algunas particularidades fónicas, gramaticales o semánticas, que las alejan de ese uso habitual, por lo que terminan por resultar especialmente expresivas. 
La metáfora, el símil y la analogía se basan en la comparación de dos conceptos: de origen (o tenor), que es el término literal (al que la metáfora se refiere) y el de destino (o vehículo), que es el término figurado. La relación que hay entre el tenor y el vehículo se denomina fundamento. Por ejemplo, en la metáfora \textit{Tus ojos son dos luceros}, \textit{ojos} es el tenor, \textit{luceros} es el vehículo y el fundamento es la belleza de los ojos \citep{GalianaYCasas1994}.


En este trabajo vamos a trabajar con tres tipos de figuras retóricas \citep{TFMPaloma2017}: 
\begin{itemize}
	\item Metáfora: Utiliza el desplazamiento de características similares entre dos conceptos con fines estéticos o retóricos. Por ejemplo, cuando una tarea es muy fácil de realizar, se dice que es un regalo.
	
	\item Símil: Realiza una comparación entre dos términos usando conectores (por ejemplo, como, cual, que, o verbos).
	Por ejemplo, cuando nos referimos a una persona que es muy corpulenta, se dice: ``es como un oso'', ya que los osos son muy grandes.
	
	\item Analogía: Es la comparación entre varios conceptos, indicando las características que permiten dicha relación. En la retórica, una analogía es una comparación textual que resalta alguna de las similitudes semánticas entre los conceptos protagonistas de dicha comparación. Por ejemplo: \textit{Sus ojos son azules como el mar}.
	
	
	
\end{itemize}

%-------------------------------------------------------------------
\section{Servicios Web}
%-------------------------------------------------------------------
\label{cap:sec:serviciosweb}

Para definir el concepto de servicio Web de la forma más simple posible, se podría decir que es una tecnología que utiliza un conjunto de protocolos para intercambiar datos entre aplicaciones, sin importar el lenguaje de programación en el cual estén programadas o ejecutadas en cualquier tipo de plataforma \citep{wiki:w3c2004}. Según el W3C(\textit{World Wide Web Consortium})\footnote{https://www.w3.org/}, un servicio web es un sistema software diseñado para soportar la interacción máquina-a-máquina, a través de una red, de forma interoperable.




Las principales características de un servicio web son \citep{TorresJoaquin2017SC}:



\begin{itemize}
	\item Accesible a través de la Web. Para ello debe utilizar protocolos de transporte estándares como HTTP, y codificar los mensajes en un lenguaje estándar que pueda ser accesible por cualquier cliente que quiera utilizar el servicio. 
	\item Contiene una descripción de sí mismo. De esta forma, una aplicación web podrá saber cual es la función de un determinado Servicio Web, y cuál es su interfaz, de manera que pueda ser utilizado de forma automática por cualquier aplicación, sin la intervención del usuario.
	\item Debe ser localizado. Deberemos tener algún mecanismo que nos permita encontrar un Servicio Web que realice una determinada función. De esta forma tendremos la posibilidad de que una aplicación localice el servicio que necesite de forma automática, sin tener que conocerlo previamente el usuario.
\end{itemize}

Por otro lado, los servicios web pueden definirse tanto a nivel conceptual como a nivel técnico, es por ello que mediante este último podemos diferenciar dos tipos distintos de servicio web \citep{TorresJoaquin2017SC}:
\begin{itemize}
	\item Servicios web SOAP  \textit({Simple Object Access Protocol}): es un protocolo basado en XML para el intercambio de información entre ordenadores. Normalmente utilizaremos SOAP para conectarnos a un servicio e invocar métodos remotos 
	\footnote{https://www.ibm.com/support/knowledgecenter/es}. Los mensajes SOAP tienen el siguiente formato representado en la figura \ref{fig:SOAP}:

	
	\begin{itemize}
		\item <Envelope>: Elemento raíz de cada mensaje SOAP y contiene dos elementos: <Header> que es opcional y <Body> que es obligatorio, ambos elementos los describiremos a continuación.
		\item <Header>: Es un elemento que se utiliza para indicar información acerca de los mensajes SOAP.
		\item <Body>: Elemento que contiene información dirigida al destinatario del mensaje.
		\item <Fault>: Elemento en el que se notifican los errores.
			
	\end{itemize}

	\item Servicios Web RESTful: es un protocolo que suele integrar mejor con HTTP que los servicios basado en SOAP, ya que no requieren mensajes XML. Cada petición del cliente debe contener toda la información necesaria para entender la petición, y no puede aprovecharse de ningún contexto almacenado en el servidor.
	
\end{itemize}
\figura{Bitmap/Capitulo2/SOAP}{width=.9\textwidth}{fig:SOAP}{Estructura de un mensaje SOAP}


\subsection{Arquitectura Servicios Web}
\label{cap:subsec:arquitecturaserviciosweb}

Hay que distinguir tres partes fundamentales en los servicios web \citep{TorresJoaquin2017SC}:
\begin{itemize}
	\item El proveedor: Es la aplicación que implementa el servicio y lo hace accesible desde Internet.
	\item El solicitante: Cualquier cliente que necesite utilizar el servicio web.
	\item El publicador: Se refiere al repositorio centralizado en el que se encuentra la información de la funcionalidad disponible y como se utiliza.
	
\end{itemize}
Por otro lado, los servicios web se componen de varias capas \footnote{https://diego.com.es/introduccion-a-los-web-services}:
\begin{itemize}
	\item Service Discovery: Responsable de centralizar los servicios web en un directorio común, de esta forma es más sencillo buscar y publicar.
	\item Service Description: Como ya hemos comentado con anterioridad, los servicios web se pueden definir así mismos, por lo que una vez que los localicemos el Service Description nos dará la información para saber que operaciones soporta y como activarlo.
	\item Service Invocatio: Invocar a un Web Service implica pasar mensajes entre el cliente y el servidor. Por ejemplo, si utilizamos SOAP  \textit({Simple Object Access Protocol}), el Service Invocation especifica cómo deberíamos formatear los mensajes request para el servidor, y cómo el servidor debería formatear sus mensajes de respuesta.
	\item Transport: Todos los mensajes han de ser transmitidos de alguna forma entre el servidor y el cliente. El protocolo elegido para ello es HTTP  \textit({(HyperText Transfer Protocol)}). 
\end{itemize}



\subsection{Ventajas de los  Servicios Web}
\label{cap:subsec:ventajasserviciosweb}

	Las principales ventajas del uso de los servicios web son las siguientes \citep{doctorado2005}:
\begin{itemize}
	\item Permiten la integración “justo-a-tiempo”:  Esto significa que los solicitantes, los proveedores y los agentes actúan en conjunto para crear sistemas que son auto-configurables, adaptativos y robustos.
	\item Reducen la complejidad por medio del encapsulamiento: Un solicitante de servicio no sabe cómo fue implementado el servicio por parte del proveedor, y éste a su vez, no sabe cómo utiliza el cliente el servicio. Estos detalles se encapsulan en los solicitantes y proveedores. El encapsulamiento es crucial para reducir la complejidad.
	\item Promueven la interoperabilidad: La interacción entre un proveedor y un solicitante de servicio está diseñada para que sea completamente independiente de la plataforma y el lenguaje. 
	\item Abren la puerta a nuevas oportunidades de negocio: Los servicios web facilitan la interacción con socios de negocios, al poder compartir servicios internos con un alto grado de integración.
	\item Disminuyen el tiempo de desarrollo de las aplicaciones: Gracias a la filosofía de orientación a objetos que utilizan, el desarrollo se convierte más bien en una labor de composición.
	\item Fomentan los estándares y protocolos basados en texto, que hacen más fácil acceder a su contenido y entender su funcionamiento.
\end{itemize}


\subsection{Desventajas de los  Servicios Web}
\label{cap:subsec:desventajasserviciosweb}
	El uso de servicios web también tiene algunas desventajas \footnote{http://fabioalfarocc.blogspot.com/2012/08/ventajas-y-desventajas-del-soap.html}:
\begin{itemize}
	\item Al apoyarse en HTTP, pueden esquivar medidas de seguridad basadas en firewall cuyas reglas tratan de bloquear.
	\item Existe poca información de servicios web para algunos lenguajes de programación.
	\item Dependen de la disponibilidad de servidores y comunicaciones.
\end{itemize}

\section{Procesamiento del Lenguaje Natural}
%-------------------------------------------------------------------
\label{cap:sec:lenguajenatural}
El Procesamiento del Lenguaje Natural (PLN) es una rama de la Inteligencia Artificial que se encarga de investigar la manera de comunicar máquinas con personas mediante el uso del lenguaje natural (entendiendo como lenguaje natural el idioma usado con fines de comunicación por humanos, ya sea hablado o escrito, como pueden ser el español, el ruso o el inglés). 
El Procesamiento del Lenguaje Natural se ayuda de las redes semánticas, puesto que estas son una forma de representación del conocimiento en la que los conceptos que componen el mundo y sus relaciones se representan mediante un grafo. Se utilizan para representar mapas conceptuales y mentales \citep{wiki:redSemantica2018}.
Los nodos están representados por el elemento lingüístico, y la relación entre los nodos sería la arista. Podemos ver un ejemplo en la Figura \ref{fig:red}, donde el nodo \textit{Oso} representa un concepto, en este caso un sustantivo que identifíca a un tipo de animal,  y otro nodo \textit{Pelo}. Su relación se ve representada por la arista con valor \textit{tiene}, dando lugar a una característica de este animal: \textit{Oso tiene pelo}.
 
\figura{Bitmap/Capitulo2/redSemantica}{width=.9\textwidth}{fig:red}{Ejemplo Red Semántica}
Existen principalmente tres tipos de redes semánticas\footnote{http://elies.rediris.es/elies9/4-3-2.htm}:
\begin{itemize}
	\item Redes de Marcos: los enlaces de unión de los nodos son parte del propio nodo, es decir, se encuentran organizados jerárquicamente, según un número de criterios estrictos, como por ejemplo la similitud entre nodos.
	\item Redes IS-A: los enlaces entre los nodos están etiquetados con una relación entre ambos. Es el tipo que habitualmente se utiliza junto con las Redes de Marcos. Por ejemplo, \textit{la ballena vive en el agua}. 
	\item Grafos Conceptuales: existen dos tipos de nodos: nodos de conceptos, los cuáles representan una entidad, un estado o un proceso y los nodos de relaciones, que indican como se relacionan los nodos de concepto. En este tipo de red semántica no existen enlaces entre los nodos con una etiqueta, sino que son los propios nodos los que tienen el significado. Por ejemplo......
\end{itemize}

Para el trabajo que queremos realizar, existen varias aplicaciones web que actúan como redes semánticas y son capaces de procesar el Lenguaje Natural.  A continuación, hablaremos de algunas de ellas.

\subsection{ConceptNet} 
\label{cap:subsec:concepnet}

Es una red semántica creada por el MIT \textit{(Massachusetts Institute of Technology)} en 1999, diseñada para ayudar a los ordenadores a entender el significado de las palabras. Está disponible en múltiples idiomas, como el español, el inglés o el chino. ConceptNet dispone de una aplicación web\footnote{http://conceptnet.io/}, donde seleccionas el idioma deseado y añades la palabra a buscar. En la Figura  \ref{fig:busquedaConcepnet}, podemos ver que si añadimos la palabra chaqueta nos devuelve sus sinónimos, términos relacionados, símbolos, etc...
\figura{Bitmap/Capitulo2/busquedaConcepnet}{width=.9\textwidth}{fig:busquedaConcepnet}{Resultados de ConcepNet para la palabra chaqueta} 
Por otro lado, ConcepNet dispone de un servicio web\footnote{http://api.conceptnet.io} que devuelve los resultados en formato JSON. Siguiendo con el mismo ejemplo anterior, podemos ver en la Figura  \ref{lst:json} los resultados en dicho formato para la palabra chaqueta. Este consta de cuatro campos\footnote{principales https://github.com/commonsense/conceptnet5/wiki/AP}:
\begin{itemize}
	\item @context: URL enlazada a un archivo de información del JSON para comprender la API. También puede contener comentarios que pueden ser útiles para el usuario.
	\item @id: concepto que se ha buscado y su idioma. En nuestro caso, aparece de la siguiente manera: \textit{/c/es/chaqueta}, donde  \textit{c} significa que es un concepto o término,  \textit{es} indica el lenguaje, en este caso, el español y por último \textit{chaqueta} que es la palabra buscada.
	\item edges: representa una estructura de datos devueltos por Conceptnet compuesta por:
	\begin{itemize}
		\item @id: describe el tipo de relación que existe entre la palabra introducida y la devuelta, en nuestro caso \textit{/a/[/r/Synonym/,/c/es/chaqueta/n/,/c/es/americana/]}, nos indica que la palabra \textit{americana} es un sinónimo de \textit{chaqueta}.
		\item @type: define el tipo del id, es decir, si es una relación (edge) o un término (nodo).
		\item dataset: URI que representa el conjunto de datos creado.
		\item end: nodo destino, que a su vez se compone de:	
		\begin{itemize}
			\item @id: coincide con la palabra del id anterior.
			\item @type: define el tipo de id, como se ha explicado anteriormente.
			\item label: frase más completa, donde adquiera significado la palabra obtenida.
			\item language: lenguaje en el que está la palabra devuelta de la consulta.
			\item term: enlace a una versión mas general del propio término. Normalmente, suele coincidir con la URI.			
		\end{itemize}
		\item license: aporta información sobre como debe usarse la información proporcionada por conceptnet.
		\item rel: describe la relación que hay entre la palabra origen y destino, dentro del cual hay tres campos: @id, @type y label, descritos anteriormente.
		\item sources: indica por qué ConceptNet guarda esa información, este campo como los anteriores, es un objeto que tiene su propio id y un campo @type, A parte, hay un campo \textit{contributor}, en el que aparece la fuente por la que se ha obtenido ese resultado y por último un campo \textit{process} indicando si la palabra se ha añadido mediante un proceso automático.
		\item start: describe el nodo origen, es decir, la palabra que hemos introducido en ConceptNet para que haga la consulta, este campo esta compuesto por elementos ya descritos como son: @id, @type, label, language y term.
		\item surfaceText: indica de que frase del lenguaje natural se han extraido los datos que estan guardados en conceptnet. En nuestro caso es nulo.
		\item weight: indica la fiabilidad de la información guardada en conceptnet, siendo normal que su valor sea 1.0. Cuanto mayor sea este valor, más fiables serán.
	\end{itemize}
	\item view: describe la longitud de la lista de paginación, es un objeto con un id propio, y además, aparecen los campos \textit{firstPage} que tiene como valor un enlace a la primera pagina de los resultados obtenidos, y \textit{nextPage} que tiene un enlace a la siguiente página de la lista.
\end{itemize}

\lstset{
	string=[s]{"}{"},
	stringstyle=\color{blue},
	comment=[l]{:},
	commentstyle=\color{black},
}

\begin{lstlisting} [title=JSON devuelto por la API de ConceptNet para la palabra chaqueta, label={lst:json}, frame=single]
{
	"@context": [
		"http://api.conceptnet.io/ld/conceptnet5.6/context.ld.json"
],
	"@id": "/c/es/chaqueta",
	"edges": [
{
	"@id": "/a/[/r/Synonym/,/c/es/chaqueta/n/,/c/es/americana/]",
	"@type": "Edge",
	"dataset": "/d/wiktionary/fr",
	"end": {
	"@id": "/c/es/americana",
	"@type": "Node",
	"label": "americana",
	"language": "es",
	"term": "/c/es/americana"
},
	"license": "cc:by-sa/4.0",
	"rel": {
	"@id": "/r/Synonym",
	"@type": "Relation",
	"label": "Synonym"
},
	"sources": [
{
	"@id": "/and/[/s/process/wikiparsec/1/,/s/resource/wiktionary/fr/]",
	"@type": "Source",
	"contributor": "/s/resource/wiktionary/fr",
	"process": "/s/process/wikiparsec/1"
}
],
	"start": {
	"@id": "/c/es/chaqueta/n",
	"@type": "Node",
	"label": "chaqueta",
	"language": "es",
	"sense_label": "n",
	"term": "/c/es/chaqueta"
},
	"surfaceText": null,
	"weight": 1.0
},

]

	"view": {
	"@id": "/c/es/chaqueta?offset=0&limit=20",
	"@type": "PartialCollectionView",
	"comment": "There are more results. Follow the 'nextPage' link for more.",
	"firstPage": "/c/es/chaqueta?offset=0&limit=20",
	"nextPage": "/c/es/chaqueta?offset=20&limit=20",
	"paginatedProperty": "edges"
	}

}
\end{lstlisting} 


\subsection{Thesaurus}
\label{cap:subsec:thesaurus}
Es una aplicación web\footnote{https://www.thesaurus.com/} que se autodefine como principal diccionario de sinónimos de la web. Esta página ofrece la posibilidad de introducir una palabra para poder conocer sus sinónimos, pero solamente devuelve resultados en inglés. Aparte del listado de sinónimos,  Thesaurus te dice que tipo de palabra es y una definición de la misma.
Esta aplicación proporciona una API tipo RESTful\footnote{http://thesaurus.altervista.org/} que obtiene los sinónimos de una palabra y devuelve los resultados en formato XML o JSON. El contenido de la respuesta es una lista y cada elemento de esta lista contiene un par de elementos: categoría y sinónimos. Este último a su vez contiene una lista de sinónimos separados por el carácter |. 
Podemos ver en la Figura \ref{lst:xmlthesaurus} un ejemplo de como sería el resultado de una petición en formato XML  y en la Figura \ref{lst:jsonthesaurus} en formato JSON.
Ambos son muy similares, por ejemplo en formato XML podemos ver que devuelve el tipo de categoría de las palabras, en este caso son sustantivos y a continuación aparcen los sinónimos. En caso de que alguna palabra sea un antónimo aparecera entre paréntesis al lado de la misma, como ocurre con la palabra \textit{war}. Por otro lado, el formato JSON devuelve dentro del campo \textit{category / categoría} todos los sinónimos, y en caso de ser un antónimo aparecerá de la misma forma que en el formato XML.



\definecolor{gray}{rgb}{0.4,0.4,0.4}
\definecolor{darkblue}{rgb}{0.0,0.0,0.6}
\definecolor{cyan}{rgb}{0.0,0.6,0.6}



\lstdefinelanguage{XML}
{
	morestring=[b]",
	morestring=[s]{>}{<},
	morecomment=[s]{<?}{?>},
	stringstyle=\color{black},
	identifierstyle=\color{darkblue},
	keywordstyle=\color{cyan},
	morekeywords={xmlns,version,type}% list your attributes here
}



\lstset{language=XML}
\begin{lstlisting}[caption= Ejemplo de salida de Thesaurus en formato XML, label={lst:xmlthesaurus}, frame=single]
<response> 
	<list>
		<category>(noun)</category> 
		<synonyms> order | war (antonym) </synonyms>
	</list>
	<list>
		<category>(noun)</category> 
		<synonyms> harmony | concord | concordance </synonyms>
	</list>
	<list>
		<category>(noun)</category> 
		<synonyms> public security | security </synonyms>
	</list>
	<list>
		<category>(noun)</category> 
		<synonyms> tratado de paz | pacificacion | acuerdo | pact | accord </synonyms>
	</list>
\end{lstlisting}





\lstset{
	string=[s]{"}{"},
	stringstyle=\color{blue},
	comment=[l]{:},
	commentstyle=\color{black},
}

\begin{lstlisting} [caption=Ejemplo de salida de Thesaurus en formato JSON, label={lst:jsonthesaurus}, frame=single]
{
	"response":
	[
		{ "list": 
			{ "category": "(noun)", "sinonimos": "order | war (antonym)" }
		},
		{ "list": 
			{ "category": "(noun)", "sinonimos": "armonia | concordia | concordancia" }
		},
		{ "lista": 
			{ "categoria": "(sustantivo)", "sinonimos": "seguridad publica | seguridad" }
		},
		{ "lista": 
			{ "categoria": "(sustantivo)", "sinonimos": "tratado de paz | pacificacion | tratado | pacto | acuerdo" }
		}
	
	]
}
\end{lstlisting}




\subsection{Thesaurus Rex}
\label{cap:subsec:thesaurusrex}

Thesaurus Rex\footnote{http://ngrams.ucd.ie/therex3/} es una red semántica. La aplicación permite introducir una palabra y devuelve palabras relacionados, o bien puedes introducir un par de palabras (separadas por el operador  \& ) y devolverá las categorías que comparten ambos términos.
Thesaurus Rex solo admite palabras en inglés. En la Figura \ref{fig:thesaurusrex} podemos ver los resultados para la palabra  \textit{house}.
La aplicación devuelve estos resultados en formato XML, y este como se puede ver en la Figura \ref{fig:xmlthesaurusrex} los divide en distintos campos: \textit{Categories}, \textit{Modifiers} y \textit{CategoryHeads}.


\figura{Bitmap/Capitulo2/thesaurusrex}{width=.9\textwidth}{fig:thesaurusrex}{Resultados búsqueda Thesaurus Rex con la palabra \textit{house}}
\figura{Bitmap/Capitulo2/xmlthesaurusrex}{width=.9\textwidth}{fig:xmlthesaurusrex}{Ejemplos formato XML Thesaurus Rex}

\subsection{Metaphor Magnet}
\label{cap:subsec:metaphormagnet}
Otra aplicación web \footnote{http://ngrams.ucd.ie/metaphor-magnet-acl/}, en la cuál introduces una palabra pero esta vez devuelve metáforas. El objetivo \citep{VealeT2012} de dicha aplicación es encontrar metáforas comunes en los n-gramas de Google y utilizar estos mapeos para interpretar metáforas, pero esta aplicación está limitada ya que solamente se puede buscar palalabras en inglés como podemos observar en la imagen \ref{fig:metaphormagnet} introduciendo la palabra  \textit{house}.
Esta consulta devuelve un fichero XML como el expuesto en la figura \ref{fig:xmlmetaphormagnet} que puede ser utilizado para otras aplicaciones de Procesamiento de Lenguaje Natural.
\figura{Bitmap/Capitulo2/metaphormagnet}{width=.9\textwidth}{fig:metaphormagnet}{Resultados búsqueda Metaphor Magnet con la palabra \textit{house}}
\figura{Bitmap/Capitulo2/xmlmetaphormagnet}{width=.9\textwidth}{fig:xmlmetaphormagnet}{Ejemplos formato XML Metaphor Magnet}



\subsection{Wordnet}
\label{cap:subsec:wordnet}
Es una base de datos léxica solamente en inglés donde agrupa las palabras en función de su significado \citep{wordnet2010}. Utiliza distintos tipos de palabras tales como sustantivos, verbos, adjetivos y adverbios ignorando preposiciones, determinantes y otras palabras funcionales. Agrupando estos tipos en conjuntos de sinónimos cognitivos o también llamados \textit{synsets}. Estos, aparte contienen una breve definición y en muchas ocasiones oraciones cortas que explican el significado. En el caso de que la palabra tenga distintos significados se representará en tantos \textit{synsets} como significados haya. Podemos ver un ejemplo en la figura \ref{fig:wordnet} realizando una búsqueda con la palabra \textit{house}.
WordNet dispone de una API HTTP donde permite buscar información en la base de datos de WordNet mediante métodos GET y POST a través de la URL raíz \textit{/graph/v1}.
\figura{Bitmap/Capitulo2/wordnet}{width=.9\textwidth}{fig:wordnet}{Resultados búsqueda Wordnet con la palabra \textit{house}}


\subsection{EuroWordnet}
\label{cap:subsec:eurowordnet}
Es una base de datos multilingüe que se basa en relaciones semánticas básicas entre palabras. Es muy parecido a WordNet pero en vez de usar \textit{synsets} ellos lo llaman \textit{WordNet}. Cada \textit{WordNet} representa un sistema único de lexicalizaión y ádemas estos están conectados con el Índice Inter-Lingual (ILI) \citep{wiki:eurowordnet2017}.
Podemos ver en la figura \ref{fig:eurowordnet} un ejemplo de búsqueda con la palabra  \textit{casa}, donde la aplicación devuelve los sinónimos, una pequeña definición así como un ejemplo para que el usuario pueda entender mejor el concepto.
\figura{Bitmap/Capitulo2/eurowordnet}{width=.9\textwidth}{fig:eurowordnet}{Resultados búsqueda EuroWordnet con la palabra \textit{casa}}

\section{Conclusiones}


