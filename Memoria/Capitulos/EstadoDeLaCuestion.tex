\chapter{Estado de la Cuestión}
\label{cap:estadoDeLaCuestion}
En este capítulo se va a tratar aspectos importantes dentro del ámbito de la retórica así como una explicación detallada de Lectura Fácil, y sin olvidar aquellas herramientas y tecnologías que se van a utilizar.
En la sección 2.1 quedará detalladamente explicado el concepto de Servicio Web, su funcionalidad, ventajas y desventajas, tipos que existen....
En la sección 2.2 se definirá las Figuras retóricas...



%-------------------------------------------------------------------
\section{Servicios Web}
%-------------------------------------------------------------------
\label{cap:sec:servicios_web}

Para definir el concepto de servicio Web de la forma más simple posible, se podría decir que es una tecnología que utiliza un conjunto de protocolos para intercambiar datos entre aplicaciones, sin importar el lenguaje de programación en el cual estén programadas o ejecutadas en cualquier tipo de plataforma. Otra definición más técnica según la W3C (\textit{World Wide Web Consortium})\footnote{https://www.w3.org/},  un servicio web es un sistema software diseñado para soportar la interacción máquina-a-máquina, a través de una red, de forma interoperable. 

\subsection{Características de  Servicios Web}
\label{cap:subsec:serviciosweb}

\item Un servicio debe poder ser accesible a travñes de la Web. Para ello debe utilizar protocolos de transporte estándares como HTTP, y condificar los mensajes en un lenguaje estándar que pueda conocder cualquier cliente que quiera utilizar el servicio. 

\item Un servicio debe contener una descripción de sí mismo. De esta forma, una aplicación podrá saber cual es la función de un determinado Servicio Web, y cuál es su interfaz, de manera que pueda ser utilizado de forma autómatica por cualquier aplicación, sin la interveción del usuario.
\item Debe poder ser localizado. Deberemos tener algún mecanismo que nos permita encontrar un Servicio Web que realice una determinada función. De esta forma tendremos la posibilidad de que una aplicación localice el servicio que necesite de forma automática, sin tener que conocerlo previamente el usuario.

\subsection{Tipos de Servicios Web}
\label{cap:subsec:serviciosweb}
Los servicios web pueden definirse tanto a nivel conceptual como a nivel técnico, es por ello que mediante este último podemos diferenciar dos tipos distintos de servicio web:
\begin{itemize}
\item Servicios web RESTful: no tienen estado. Cada petición del cliente al servidor debe contener toda la información necesaria para entender la petición, y no puede aprovecharse de ningún contexto almacenado en el servidor.
\item Servicios web SOAP  \textit({Simple Object Access Protocol}): es un protocolo basado en XML para el intercambio de información entre ordenadores. Normalmente utilizaremos SOAP para conectarnos a un servicio e invocar métodos remotos.



\section{Figuras retóricas}
Se entiende por figura a cualquier tipo de recurso o manipulación del lenguaje con fines retóricos.
Las figuras retóricas son recursos del lenguaje literario utilizados para dar más belleza y una mejor expresión a sus palabras.

Nuestro servicio web se va a basar en el uso de figuras retóricas (especialmente metáforas, símiles y analogías) para la ayuda al aprendizaje de palabras más complejas, como se ha ejemplificado anteriormente.

\begin{itemize}
\item Analogía: Según la RAE, es la relación de semejanza entre dos cosas distintas.
\item Símil: Según la RAE, es la comparación, semejanza entre dos cosas.
\item Metáfora: Una metáfora es una Figura retórica de pensamiento por medio de la cual una realidad o concepto se expresan por medio de una realidad o concepto diferentes con los que lo representado guarda cierta relación de semejanza.
\end{itemize}



\section{Lectura Fácil}

Son documentos de todo tipo que siguen las directrices internacionales de la IFLA (International Federation of Library Associations and Institutions) y de Inclusion Europe en cuanto al contenido y la forma.

Está dirigida a colectivos que tengan dificultades lectoras como inmigrantes o personas con trastornos de aprendizaje o problemas cognitivos.

\section{ConceptNet} 

Es una red semántica diseñada para ayudar a los ordenadores a entender los significados de las palabras utilizadas por la gente lanzada por el MIT en 1999.
Tiene un buscador de palabras en el que al introducir una palabra, se selecciona el idioma y devuelve sinónimos y términos relacionados.

