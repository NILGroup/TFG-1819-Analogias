\chapter{Estado de la Cuestión}
\label{cap:estadoDeLaCuestion}

\begin{resumen}
	En este capítulo se van a tratar aspectos importantes dentro del ámbito de la retórica así como una explicación detallada de Lectura Fácil, sin olvidar aquellas herramientas y tecnologías que se van a utilizar.
	En la sección 2.1 se explica el concepto de lectura fácil, su historia, pautas a seguir y ejemplos para que se pueda entender aún mejor. 
	En la sección 2.2
	En la sección 2.3 figuras retóricas
	En la sección 2.4 se explica el concepto de Servicio Web, su funcionalidad, ventajas y desventajas y tipos que existentes.
	
\end{resumen}



\section{Lectura Fácil}

Se llama lectura fácil a aquellos contenidos que han sido resumidos y realizados con lenguaje sencillo y claro, de forma que puedan ser entendidos por personas con discapacidad cognitiva o discapacidad intelectual. Es la adaptación de textos, ilustraciones y maquetaciones que permite una mejor lectura y comprensión.
Nosotros nos vamos a centrar en la lectura fácil aplicada a textos.

La lectura fácil surgió en Suecia en el año 1968, donde se editó el primer libro en la Agencia de Educación en el marco de un proyecto experimental. A continuación, en 1976 se creó en el Ministerio de Justicia un grupo de trabajo para conseguir textos legales más claros.
En 1984 nació el primer periódico en lectura fácil, titulado "8 páginas", que tres años más tarde,en 1987, se publica de forma permanente en papel hasta que empezó a editarse en la web. 
En el año 2013, en México se produce la primera sentencia judicial en lectura fácil.
\footnote{https://dilofacil.wordpress.com/2013/12/04/el-origen-de-la-lectura-facil/}

En la actualidad, podemos distinguir los documentos en lectura fácil gracias a este logo:
	\figura{Bitmap/Capitulo2/lecturaFacil}{width=.3\textwidth}{fig:lectura}{Logo Lectura Fácil} 
	
	
Los documentos escritos en Lectura Fácil son documentos de todo tipo que siguen las directrices internacionales de la IFLA (International Federation of Library Associations and Institutions) y de Inclusion Europe en cuanto al contenido y la forma.
Algunas pautas a seguir para escribir correctamente un texto en Lectura Fácil son:
\citep{GarcíaMuñoz2012LecturaFacil}

\begin{itemize}
	\item Evitar mayúsculas fuera de la norma, es decir, escribir en mayúsculas sólo cuando toca según las reglas ortográficas, como por ejemplo, después de un punto o la primera letra de los nombres propios.
	\item Debe evitarse el punto y seguido, el punto y coma y los puntos suspensivos.
	\item El punto y aparte hará la función del punto y seguido.
	\item Evitar corchetes y signos ortográficos poco habituales, como por ejemplo: \%, \& y /.
	\item Utilizar oraciones simples.
	\item Evitar tiempos verbales como: futuro, subjuntivo, condicional y formas compuestas.
	\item Utilizar palabras cortas y de sílabas poco complejas.
	\item Evitar abreviaturas, acrónimos y siglas.
	\item Alineación a la izquierda
	\item Incluir imágenes y pictogramas a la izquierda y su texto vinculado a la derecha.
	\item Evitar la saturación de texto e imágenes
	\item Utilizar uno o dos tipos de letra como mucho.
	\item Tamaño de letra entre 12 y 16 puntos y evitar frases superiores a 60 carácteres.s
	\item Si el documento está paginado, incluir la paginación claramente y reforzar el mensaje de que la información continúa en la página siguiente.
\end{itemize}

Estos son solo algunos de los pasos que se deben seguir para realizar correctamente un documento en lectura fácil. 
Debemos también hacer hincapié en la distinción entre palabras fáciles y complejas, puesto que son de gran importancia para la lectura fácil. 
Las palabras complejas son aquellas que no se utilizan a menudo en nuestra sociedad, como por ejemplo: "Melifluo, que quiere decir sonido excesivamente suave, dulce o delicado". O "Inefable, que es algo tan increíble que no se puede explicar con palabras".
\footnote{http://historiasmaravillosas122.blogspot.com/2015/07/las-20-palabras-mas-bonitas-del-idioma.html}
Es por ello que este tipo de palabras deben estar totalmente descartadas en la lectura fácil, y en su lugar debemos introducir palabras fáciles, que son aquellas que se utilizan asiduamente. La RAE dispone de un documento con las mil palabras más usadas.


\section{Procesamiento del lenguaje natural}
 
El procesamiento del lenguaje natural(PLN) es una rama de la Inteligencia Artificial que se encarga de investigar la manera de comunicar máquinas con personas mediante el uso de lenguajes naturales (entendiendo el lenguaje natural al idioma ya sea hablado o escrito por humanos con fines de comunicación), como pueden ser el español, el ruso o el inglés. Gracias a esto podemos identificar mediante un programa informático que palabras son las más utilizadas en un idioma determinado(por lo que podemos deducir que esas palabras van a ser las más sencillas de aprender) e incluirlas en textos para que sea más fácil su comprensión.

\subsection{ConceptNet} 

Es una red semántica diseñada para ayudar a los ordenadores a entender el significado de las palabras y fue creada por el MIT en 1999. Esta disponible en múltiples idiomas, como el español, el inglés o el chino. ConceptNet dispone tanto de una aplicación web \textit{http://conceptnet.io/} 
	\figura{Bitmap/Capitulo2/concepnet}{width=.9\textwidth}{fig:concepnet}{Sinonimos y Términos relacionados} 
	 que dispone de una interfaz en la que puedes introducir una palabra y seleccionar el idioma en el que está, a su vez, dispone de un servicio web\textit{http://api.conceptnet.io/ld/conceptnet5.6/context.ld.json}, es decir, una API a la que se puede acceder para realizar peticiones de palabras y que devuelve la información en formato JSON.

Este consta principalmente de tres campos:
\begin{itemize}
	\item @context: que muestra una URL hacia la documentación de la API.
	\item @id: Indica el concepto que se quiere buscar y su idioma. Por ejemplo: si queremos buscar gato en español, aparece de la siguiente manera: \textit{/c/es/gato}, siendo "es" el id del idioma en el que hemos buscado la palabra, en este caso indica español y siendo "gato" la palabra buscada.
	\item edges: Que constituyen los datos devueltos por conceptnet y está formado por los siguientes campos:
	\begin{itemize}
		\item @id: Describe como están conectados dos nodos(palabras), es decir, la relación que hay entre ellos Por ejemplo, como podemos ver en la figura 2.3, la relación entre gato y animal, es que el gato pertenece al grupo de los animales.
		\item @type:Describe el tipo del id, es decir, si es una relación (edge) o un término (nodo).
		\item dataset: No es muy importante, es un valor acerca de como está construido conceptnet.
		\item end: Describe el nodo destino, es decir, si hemos introducido la palabra "gato" un nodo destino sería por ejemplo "animal", ya que es un término relacionado. Dentro de end hay varios campos ya explicados anteriormente, como @id o @type, sin embargo hay otros nuevos como "language", en el que se indica el idioma en el que esta la palabra destino y "term" que en este caso tiene el mismo valor que el @id
		\item license: Aporta información sobre como debe usarse la información proporcionada por conceptnet.
		\item rel: Describe la relación que hay entre la palabra origen y destino, dentro del cual hay tres campos: @id, @type y label, descritos anteriormente.
		\item sources: Indica por qué conceptnet guarda esa información, este campo como los anteriores, es un objeto que tiene su propio id, además de un campo "activity" indicando la tarea del contribuidor que ha añadido esta palabra o si ha sido añadida mediante n proceso automático. Por último hay un campo "contributor", en el que aparece el nombre de la persona que ha añadido esta palabra.
		\item start: Describe el nodo origen, es decir, la palabra que hemos introducido en conceptnet para que haga la consulta, este campo esta compuesto por elementos ya descritos como son: @id, @type, label, language y term.
		\item surfaceText: Indica de que frase del lenguaje natural se han extraido los datos que estan guardados en conceptnet
		\item weight: Indica la fiabilidad de la información guardada en conceptnet, siendo normal que su valor sea 1.0. Dicho valor es mayor cuanto más filedigna sean los datos.
	\end{itemize}
	\item view: Describe la longitud de la lista de paginación, es un objeto con un id propio,y además, aparecen los campos "firstPage" que tiene como valor un enlace a la primera pagina de los resultados obtenidos, "nextPage" que tiene un enlace a la siguiente página de la lista
\end{itemize}

	\figura{Bitmap/Capitulo2/json}{width=.9\textwidth}{fig:json}{Vista de JSON} 

\footnote{http://conceptnet.io/}

Por ejemplo, si introducimos la palabra "dinero", especificando que el idioma es el español, ConceptNet te devuelve como sinónimos: lana, pasta, billete, plata. y como términos relacionados: dineral y moneda.

Dispone de una API \textit({http://api.conceptnet.io/}) que devuelve los datos en formato JSON, que utilizaremos para obtener los términos relacionados y sinónimos que necesitamos.
\subsection{Alternativas a conceptnet}
Una de las alternativas que hemos encontrado a conceptnet es Thesaurus \textit{https://www.thesaurus.com } que es una aplicación web en la que introduces una palabra y obtienes términos relacionados y sinónimos en varios idiomas, principalmente en inglés, también te dice que tipo de palabra es (sustantivo, verbo... etc), esta opción en nuestro caso es inviable ya que la base de datos de palabras en castellano es bastante pobre.

Otra página relacionada que hemos encontrado ha sido Metaphor Magnet, que es otra aplicación web en la que al introducir una palabra, genera metáforas o palabras relacionadas, este servicio, unicamente esta implementado en inglés.
\section{Figuras retóricas}
Las figuras literarias(o retóricas) son formas no convencionales de utilizar las palabras, de manera que, aunque se emplean con sus acepciones habituales, se acompañan de algunas particularidades fónicas, gramaticales o semánticas, que las alejan de ese uso habitual, por lo que terminan por resultar especialmente expresivas. 
La metáfora, el símil y la analogía se basan en la comparación entre dos conceptos, el de origen, que es el término literal (al que la metáfora se refiere) es el llamado tenor. El de destino, que es el término figurado es el vehículo. La relación que hay entre el tenor y el vehículo se denomina fundamento. Por ejemplo, en la metáfora \textit{Tus ojos son dos luceros}, \textit{ojos} es el tenor, \textit{luceros} es el vehículo y el fundamento es la belleza de los ojos.
(\textit {Introducción al análisis retórico: tropos, figuras y sintaxis del estilo})\footnote{Azaustre Galiana, Antonio y Juan Casas Rigall, Introducción al análisis retórico: tropos, figuras y sintaxis del estilo, Universidad de Santiago de Compostela, Santiago de Compostela, 1994.}

En este trabajo vamos a trabajar con tres tipos de figuras retóricas: Analogía, metáfora y símil.
\begin{itemize}
	\item Metáfora: Se refiere a una cosa mencionando otra, utiliza el desplazamiento de características similares entre dos conceptos con fines estéticos o retóricos. Por ejemplo, cuando una persona tiene muy buena memoria, se dice que tiene memoria de elefante. Ya que una de las características de los elefantes es que tienen buena memoria.
	
	\item Símil: Realiza una comparación entre dos términos. A pesar de que los símiles y las metáforas son similares, los símiles utilizan explícitamente, aunque no necesariamente, conectores (por ejemplo, como, cual, que, o varios verbos tales como se asemejan).
	Por ejemplo, cuando nos referimos a una persona que es muy corpulenta, se dice que es como un oso, ya que los osos son muy grandes
	
	\item Analogía: Es la comparación entre varios conceptos, indicando las características que permiten dicha relación. En la retórica, una analogía es una comparación textual que resalta alguna de las similitudes semánticas entre los conceptos protagonistas en dicha comparación. Por ejemplo, sus ojos son azules como el mar, comparándolos con el color del mar
	
	(\footnote {GENERACIÓN DE RECURSOS LINGÜÍSTICOS MEDIANTE LA EXTRACCIÓN DE RELACIONES ENTRE CONCEPTOS})	
	
\end{itemize}

%-------------------------------------------------------------------
\section{Servicios Web}
%-------------------------------------------------------------------
\label{cap:sec:servicios_web}

Para definir el concepto de servicio Web de la forma más simple posible, se podría decir que es una tecnología que utiliza un conjunto de protocolos para intercambiar datos entre aplicaciones, sin importar el lenguaje de programación en el cual estén programadas o ejecutadas en cualquier tipo de plataforma.\citep{wiki:w3c2004} Según el W3C(\textit{World Wide Web Consortium})\footnote{https://www.w3.org/}, un servicio web es un sistema software diseñado para soportar la interacción máquina-a-máquina, a través de una red, de forma interoperable. 
\newline



Las principales características de un servicio web son:
\citep{TorresJoaquin2017SC}


\begin{itemize}
\item Debe poder ser accesible a través de la Web. Para ello debe utilizar protocolos de transporte estándares como HTTP, y codificar los mensajes en un lenguaje estándar que pueda ser accesible por cualquier cliente que quiera utilizar el servicio. 
\item Debe contener una descripción de sí mismo. De esta forma, una aplicación web podrá saber cual es la función de un determinado Servicio Web, y cuál es su interfaz, de manera que pueda ser utilizado de forma automática por cualquier aplicación, sin la intervención del usuario.
\item Debe poder ser localizado. Deberemos tener algún mecanismo que nos permita encontrar un Servicio Web que realice una determinada función. De esta forma tendremos la posibilidad de que una aplicación localice el servicio que necesite de forma automática, sin tener que conocerlo previamente el usuario.
\end{itemize}

Por otro lado, los servicios web pueden definirse tanto a nivel conceptual como a nivel técnico, es por ello que mediante este último podemos diferenciar dos tipos distintos de servicio web:
\begin{itemize}
	\item Servicios web SOAP  \textit({Simple Object Access Protocol}): es un protocolo basado en XML para el intercambio de información entre ordenadores. Normalmente utilizaremos SOAP para conectarnos a un servicio e invocar métodos remotos.
	\item Servicios Web RESTful: es un protocolo que suele integrar mejor con HTTP que los servicios basado en SOAP, ya que no requieren mensajes XML. Cada petición del cliente debe contener toda la información necesaria para entender la petición, y no puede aprovecharse de ningún contexto almacenado en el servidor.
	
\end{itemize}

\subsection{Arquitectura Servicios Web}
\label{cap:subsec:arquitecturaserviciosweb}
Hay que distinguir tres partes fundamentales en los servicios web:
\begin{itemize}
	\item El proveedor: Es la aplicación que implementa el servicio y lo hace accesible desde Internet.
	\item El solicitante: Cualquier clienteque necesite utilizar el servicio web.
	\item El publicador: Se refiere al repositorio centralizado en el que se encuentra la información de la funcionalidad disponible y como se utiliza.

\end{itemize}
 Por otro lado, los servicios web se componen de varias capas:
\begin{itemize}
 \item Service Discovery. Es el responsable de centralizar los servicios web en un directorio común de esta forma es mas sencillo buscar y publicar.
 \item Service Description. Como ya hemos comentado con anterioridad, los servicios web se pueden definir así mismos, por lo que una vez que los localicemos el Service Description nos dará suficiente información para saber que operaciones soporta y como activarlo.
 \item Service Invocation. Invocar a un Web Service implica pasar mensajes entre el cliente y el servidor. Por ejemplo, si utilizamos SOAP  \textit({Simple Object Access Protocol}) , el Service Invocation especifica cómo deberíamos formatear los mensajes request para el servidor, y cómo el servidor debería formatear sus mensajes de respuesta.
 \item Transport. Todos estos mensajes han de ser transmitidos de alguna forma entre el servidor y el cliente. El protocolo elegido para ello es HTTP  \textit({(HyperText Transfer Protocol)}). 
\end{itemize}

\subsection{Ventajas de los  Servicios Web}
\label{cap:subsec:ventajasserviciosweb}
	Las principales ventajas del uso de los servicios web son las siguientes:
\begin{itemize}
	\item Permiten la integración “justo-a-tiempo”:  Esto significa que los solicitantes, los proveedores y los agentes actúan en conjunto para crear sistemas que son auto-configurables, adaptativos y robustos.
	\item Reducen la complejidad por medio del encapsulamiento: Un solicitante de servicio no sabe cómo fue implementado el servicio por parte del proveedor, y éste a su vez, no sabe cómo utiliza el cliente el servicio. Estos detalles se encapsulan en los solicitantes y proveedores. El encapsulamiento es crucial para reducir la complejidad.
	\item Promueven la interoperabilidad: La interacción entre un proveedor y un solicitante de servicio está diseñada para que sea completamente independiente de la plataforma y el lenguaje. 
	\item Abren la puerta a nuevas oportunidades de negocio: Los servicios web facilitan la interacción con socios de negocios, al poder compartir servicios internos con un alto grado de integración.
	\item Disminuyen el tiempo de desarrollo de las aplicaciones: Gracias a la filosofía de orientación a objetos que utilizan, el desarrollo se convierte más bien en una labor de composición.
	\item Fomentan los estándares y protocolos basados en texto, que hacen más fácil acceder a su contenido y entender su funcionamiento.
\end{itemize}

\subsection{Desventajas de los  Servicios Web}
\label{cap:subsec:desventajasserviciosweb}
	El uso de servicios web también tiene algunas desventajas:
\begin{itemize}
	\item Al apoyarse en HTTP, pueden esquivar medidas de seguridad basadas en firewall cuyas reglas tratan de bloquear.
	\item Existe poca información de servicios web para algunos lenguajes de programación.
	\item Dependen de la disponibilidad de servidores y comunicaciones.
\end{itemize}




