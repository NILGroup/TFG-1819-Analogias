\chapter{Servicios Web para la obtención de conceptos fáciles relacionados}
\label{cap:serviciosWebImplementados}

Como se ha comentado en anteriores capítulos existen varias aplicaciones web que son redes semánticas y facilitan sinónimos, términos relacionados, metáforas, hipéronimos, etc... de un concepto dado por el usuario. Para este trabajo, algunas de ellas son de gran utilidad ya que de esta forma podemos obtener las palabras fáciles para un concepto más complicado, pero hay que saber exactamente cuál es la que mejor conviene y la que mejores resultados ofrece. Hay que corroborar que los conceptos devueltos son correctos y que disponen de un significado claro y parecido respecto al concepto buscado. Para ello se han implementado dos servicios web, uno utilizando la aplicación de ConceptNet y el otro utilizando WordNet, en la sección 5.1 se explicará el servicio web para ConceptNet con los métodos creados y su explicación y en la sección 5.2 aparece lo mismo pero utilizando WordNet. Por último, una vez implementados dichos servicios, se han realizado una serie de pruebas para comprobar que aplicación era más necesaria para la aplicación final, en la sección 5.3 queda reflejado dichas pruebas, como se han realizado, con que cantidad de palabras se han hecho las pruebas y cuales han sido los resultados obtenidos.


%-------------------------------------------------------------------
\section{Diseño de la evaluación}
%-------------------------------------------------------------------
\label{cap:sec:disenoEval}

Para decidir que red semántica se iba a usar se decidió hacer una prueba con palabras que fuesen lo más heterogéneas posible, para ello se eligieron una serie de artículos periodísticos de distintos temas: medioambiental, tecnológico, deportivo y político. Estos artículos se filtraron para utilizar únicamente los verbos, artículos, sustantivos y adverbios. Para cada una de estas palabras se obtuvieron en WordNet y ConceptNet sus sinónimos y términos relacionados (en WordNet se entiende como término relacionado a los hiperónimos e hipónimos) y se analizaron, tanto la cantidad de términos relacionados y sinónimos que generaban cada una de las redes semánticas que coincidían con alguna de las palabras fáciles de la RAE (prueba cuantitativa), como la calidad de los mismos, es decir, si las palabras que generaban tenían alguna relación aceptable con la palabra origen(prueba cualitativa).

%-------------------------------------------------------------------
\section{Resultados cuantitativos}
%-------------------------------------------------------------------
\label{cap:sec:pruebaCuantitativa}

\begin{table}
\centering
\begin{tabular}{|l|l}
	
	\multicolumn{2}{c} {1000 palabras}\\ 
	
\end{tabular}
\end{table}


%-------------------------------------------------------------------
\section{Servidor de Base de Datos}
%-------------------------------------------------------------------
Para este proyecto, se ha utilizado una base de datos para la persistencia de los datos. El sistema encargado de la gestión de la base de datos corresponde con MariaDB y esta constituida por varias tablas:
\begin{itemize}
	\item Se ha hecho uso de MCR (\textit{Multilingual Central Repository}), utilizando la versión 3.0. MCR es una base de datos de código abierto que integra distintas versiones de WordNet para seis lenguajes diferentes: Inglés, Español, Catalán, Vasco, Gallego y Portugués. 
	\item Tres tablas que guardan tanto las 1.000, 5.000 y 10.000 palabras más usadas de la RAE.
	\item Otra tabla que almacena los pictogramas de ARASAAC. 
	
	
	
	TERMINAR!!! 
\end{itemize}


%-------------------------------------------------------------------
\section{Servicio Web  para obtener sinónimos fáciles}
%-------------------------------------------------------------------

La implementación de dicho servicio web se basa en que introduciendo un offset y un nivel de profundidad, este devuelve todos los sinónimos fáciles. Se entiende como sinónimos fáciles aquellos que se encuentren en las 1.000 palabras más usadas de la RAE (nivel de profundidad sencillo), las 5.000 palabras más usadas de la RAE (nivel de profundidad medio) o en las 10.000 palabras más usadas de la RAE (nivel de profundidad avanzado).

Una vez introducido el offset, se buscan todos los resultados coincidentes en la base de datos de MCR 3.0. Con estos resultados, mediante un cursor se realiza una consulta a una de las tres tablas de las palabras más usadas de la RAE. En función del nivel de dificultad introducido se buscará en:
\begin{itemize}
	\item Nivel 1 (Nivel sencillo): Se realizará la búsqueda en las 1.000 palabras más usadas de la RAE.
	\item Nivel 2 (Nivel medio): Se realizará la búsqueda en las 5.000 palabras más usadas de la RAE.
	\item Nivel 3 (Nivel avanzado): Se realizará la búsqueda en las 10.000 palabras más usadas de la RAE.
\end{itemize}

Si tras realizar dicha consulta, se obtiene algún resultado, estos se guardaran en formato JSON como se puede ver en el Listado  \ref{lst:JSONsinonimofacil}.



\colorlet{punct}{red!60!black}
\definecolor{background}{HTML}{EEEEEE}
\definecolor{delim}{RGB}{20,105,176}
\colorlet{numb}{magenta!60!black}

\lstdefinelanguage{json}{
	basicstyle=\normalfont\ttfamily,
	numbersep=8pt,
	showstringspaces=false,
	breaklines=true,
	frame=lines,
	literate=
	*{0}{{{\color{numb}0}}}{1}
	{1}{{{\color{numb}1}}}{1}
	{2}{{{\color{numb}2}}}{1}
	{3}{{{\color{numb}3}}}{1}
	{4}{{{\color{numb}4}}}{1}
	{5}{{{\color{numb}5}}}{1}
	{6}{{{\color{numb}6}}}{1}
	{7}{{{\color{numb}7}}}{1}
	{8}{{{\color{numb}8}}}{1}
	{9}{{{\color{numb}9}}}{1}
	{:}{{{\color{punct}{:}}}}{1}
	{,}{{{\color{punct}{,}}}}{1}
	{\{}{{{\color{delim}{\{}}}}{1}
	{\}}{{{\color{delim}{\}}}}}{1}
	{[}{{{\color{delim}{[}}}}{1}
	{]}{{{\color{delim}{]}}}}{1},
}

	
	\begin{lstlisting}[language=json, caption= Estructura JSON para sinónimos fáciles, label={lst:JSONsinonimofacil}, frame=single]
		{ 
		   "offset": "", 
	      "easySynonyms": [ ], 
	      "definition": "", 
	      "example": "", 
	      "picto": ""
	    }
	\end{lstlisting}
	
Donde el campo offset será el mismo offset que recibió en la entrada,  un array de sinonimos donde se guardarán todos los resultados obtenidos, la definición y el ejemplo en caso de tenerlos, si no se quedarán vacíos. Y por último el campo picto, el cual realiza una petición GET al servidor (http://127.0.0.1:8000/imagenByPalabra/\textit{word}) y devolverá la URL del pictograma almacenado. En caso de no tener ningún pictograma asociado, se añadirá al campo PICTO el string ``NOT FOUND".
En la Figura ..... podemos ver un ejemplo de la petición con el offset ....... y el nivel de dificultad 2, es decir, nivel medio. Obteniendo el siguiente resultado:



TERMINAR


%-------------------------------------------------------------------
\section{Servicio Web  para obtener hipónimos fáciles}
%-------------------------------------------------------------------
Para obtener los hipónimos fáciles, su implementación es similar al servicio web para sinónimos fáciles explicado con anterioridad pero con la diferencia de que en el JSON se guarda un campo más, el offset propio de los hipónimos.
En este caso, el offset recibido en la entrada no coincide con el offset de los hipónimos obtenidos por lo que se deben guardar ambos resultados, tanto el recibido en la entrada como el de los hipónimos.
Por lo que la estructura del JSON se puede ver en el Listado \ref{lst:JSONhiponimofacil}.



\colorlet{punct}{red!60!black}
\definecolor{background}{HTML}{EEEEEE}
\definecolor{delim}{RGB}{20,105,176}
\colorlet{numb}{magenta!60!black}

\lstdefinelanguage{json}{
	basicstyle=\normalfont\ttfamily,
	numbersep=8pt,
	showstringspaces=false,
	breaklines=true,
	frame=lines,
	literate=
	*{0}{{{\color{numb}0}}}{1}
	{1}{{{\color{numb}1}}}{1}
	{2}{{{\color{numb}2}}}{1}
	{3}{{{\color{numb}3}}}{1}
	{4}{{{\color{numb}4}}}{1}
	{5}{{{\color{numb}5}}}{1}
	{6}{{{\color{numb}6}}}{1}
	{7}{{{\color{numb}7}}}{1}
	{8}{{{\color{numb}8}}}{1}
	{9}{{{\color{numb}9}}}{1}
	{:}{{{\color{punct}{:}}}}{1}
	{,}{{{\color{punct}{,}}}}{1}
	{\{}{{{\color{delim}{\{}}}}{1}
	{\}}{{{\color{delim}{\}}}}}{1}
	{[}{{{\color{delim}{[}}}}{1}
	{]}{{{\color{delim}{]}}}}{1},
}


\begin{lstlisting}[language=json, caption= Estructura JSON para hipónimos fáciles, label={lst:JSONhiponimofacil}, frame=single]
{ 
	"offsetFather: "",
	"offset": "", 
	"easyHyponyms": [ ], 
	"definition": "", 
	"example": "", 
	"picto": ""
}
\end{lstlisting}


A continuación, se puede ver en la Figura ..... un ejemplo introduciendo el offset ...... correspondiente a la palabra ......  y un nivel de dificultad 2 (nivel medio) y los resultados de dicha consulta.


TERMINAR!!!!!
%-------------------------------------------------------------------
\section{Servicio Web  para obtener hiperónimos fáciles}
%-------------------------------------------------------------------

%-------------------------------------------------------------------
\section{Servicio Web  para obtener una metáfora}
%-------------------------------------------------------------------
Este servicio web dado un offset y una profundidad obtiene las metáforas de un concepto.
Para ello, una vez que se obtienen todos los sinónimos fáciles del offset de entrada, se llama a otro servicio web que crea la metáfora. Este servicio web utiliza Spacy para definir si el sinónimo fácil que recibe es singular o plural y si es femenino o masculino y a partir de estos datos forma la metáfora correcta.
A continuación, se guardan los resultados en formato JSON. En el Listado   se puede ver un ejemplo de su estructura.


\colorlet{punct}{red!60!black}
\definecolor{background}{HTML}{EEEEEE}
\definecolor{delim}{RGB}{20,105,176}
\colorlet{numb}{magenta!60!black}

\lstdefinelanguage{json}{
	basicstyle=\normalfont\ttfamily,
	numbersep=8pt,
	showstringspaces=false,
	breaklines=true,
	frame=lines,
	literate=
	*{0}{{{\color{numb}0}}}{1}
	{1}{{{\color{numb}1}}}{1}
	{2}{{{\color{numb}2}}}{1}
	{3}{{{\color{numb}3}}}{1}
	{4}{{{\color{numb}4}}}{1}
	{5}{{{\color{numb}5}}}{1}
	{6}{{{\color{numb}6}}}{1}
	{7}{{{\color{numb}7}}}{1}
	{8}{{{\color{numb}8}}}{1}
	{9}{{{\color{numb}9}}}{1}
	{:}{{{\color{punct}{:}}}}{1}
	{,}{{{\color{punct}{,}}}}{1}
	{\{}{{{\color{delim}{\{}}}}{1}
	{\}}{{{\color{delim}{\}}}}}{1}
	{[}{{{\color{delim}{[}}}}{1}
	{]}{{{\color{delim}{]}}}}{1},
}


\begin{lstlisting}[language=json, caption= Estructura JSON para las metáforas, label={lst:JSONhiponimofacil}, frame=single]
	{ 
	"offset": "", 
	"phraseSynonyms": [ ], 
	"definition": "", 
	"example": "", 
	"picto": ""
	}
\end{lstlisting}

En la Figura ..... se puede ver un ejemplo introduciendo el offset..... que corresponde a la palabra ...... y el nivel de profundidad 2 (nivel medio).



TERMINAR!!!!!!!
%-------------------------------------------------------------------
\section{Servicio Web  para obtener un símil}
%-------------------------------------------------------------------


%-------------------------------------------------------------------
\section{Servicio Web  para obtener una analogía}
%-------------------------------------------------------------------

%-------------------------------------------------------------------
\section{Servicio Web Para ConceptNet}
%-------------------------------------------------------------------
\label{cap:sec:servicioConceptnet}

El primer Servicio Web implementado obtiene los resultados de ConceptNet, una red semántica que al introducir una palabra devuelve sinónimos y términos relacionados de la misma (dicha red semántica ha sido descrita detalladamente en el apartado \ref{cap:subsec:concepnet}).

\figura{Bitmap/Capitulo5/ejemploConceptnet}{width=1.0\textwidth}{fig:conceptnet}{Interfaz del Servicio Web para ConceptNet}

Este servicio web consta de tres servicios, como se puede ver en la Figura \ref{fig:conceptnet}. El primero, cuando se introduce en el campo de texto una palabra en castellano, se realiza una consulta a la API de ConceptNet y se muestran únicamente los sinónimos obtenidos para dicha palabra. El segundo, realiza otra consulta a la API de ConceptNet pero esta vez en lugar de obtener los sinónimos, devuelve solamente los términos relacionados.

Y por último, el tercer servicio hace uso de los servicios 1 y 2 para obtener tanto los sinónimos como los términos relacionados del concepto introducido. Con los resultados obtenidos, se realiza una nueva búsqueda en la lista de ``las 1000 palabras más utilizadas del castellano según la RAE'' y en caso de que algún resultado se encuentre en dicha lista, se le mostrará al usuario y si no, se mostrará un mensaje informativo indicando que no hay resultados. Aparte, este servicio consta de un selector numérico donde se indica la profundidad de la búsqueda que por defecto, su valor es 1. Esto quiere decir que, por ejemplo, con un nivel de profundidad de nivel 1 se realizará el proceso descrito anteriormente una única vez. Sin embargo, si se introduce una profundidad mayor, por ejemplo 2, se repetirá el proceso dos veces pero esta vez buscando los sinónimos y los términos relacionados de los resultados y no de la palabra introducida en un primer momento. Por ejemplo, si el usuario introduce la palabra Gato, ConcepNet devuelve como sinónimos: feria, madrileño y felino pero ninguna de estas palabras se encuentran en las lista de ``las 1000 palabras más utilizadas del castellano según la RAE'', por lo que el servicio web implementado buscará los sinónimos y términos relacionados de cada una de las palabras, es decir, de feria, madrileño y felino. Si alguno de los resultados se encuentra en la lista, entonces se mostrarán y si no, aparecerá el mensaje informativo indicando que no hay resultados.

Si por otro lado, se añade un nivel de profundidad aún mayor pero la coincidencia de los resultados con la lista se encuentra en un nivel anterior, la búsqueda se detendrá y se indicará al usuario en que nivel ha sido obtenido con éxito dicho resultado.
Ahora bien, dependiendo de qué tipo de palabra haya coincidido y en que nivel, se generará el resultado utilizando un tipo de comparación distinta. Estas pueden ser:
\begin{itemize}
	\item Si la palabra que coincide es un sinónimo de la palabra original  y se ha encontrado en un nivel de profundidad 1, entonces se entiende que existe una similitud alta entre ambos conceptos por lo que el resultado que se mostrará tendrá el siguiente aspecto: ``A \textbf{es} B'' . Por ejemplo, si se ha introducido ``hogar'' y el resultado obtenido con éxito es ``casa'' , se mostrará: ``hogar \textbf{es} casa''.
	
	\item En cualquier otro caso, se entiende que la similitud entre los conceptos es más baja por lo que el mensaje que aparecerá será:  ``A \textbf{es como} B''. Por ejemplo, si se ha introducido ``casa'' y se ha encontrado como término relacionado en el nivel 1 de profundidad o en cualquier otro nivel la palabra ``edificio'', la comparación que se mostrará será: ``casa \textbf{es como} edificio''.
\end{itemize}


%-------------------------------------------------------------------
\section{Servicio Web para WordNet}
%-------------------------------------------------------------------
\label{cap:sec:servicioWordnet}

Para la implementación y desarrollo del servicio web, se ha hecho uso de MCR (\textit{Multilingual Central Repository}), el cuál es una base de datos de código abierto que integra distintas versiones de WordNet para seis lenguajes diferentes: Inglés, Español, Catalán, Vasco, Gallego y Portugués. 
Lo que se consigue al utilizar MCR es poder acceder a la base de datos de WordNet y obtener los resultados que dicha aplicación.
Como se puede ver en la Figura \ref{fig:textoWordnet}, el usuario dispone de un campo de texto donde puede introducir únicamente una palabra. Cuando se pulsa el botón de enviar, se capta dicho concepto mediante una petición POST a través de un formulario.
Al tener la base de datos de WordNet, podemos realizar consultas directamente para obtener los resultados en función de si se buscan sinónimos, hiperónimos o hipónimos, y aunque la base de datos es mysql, al estar utilizando el framework Django este no admite la realización de \textit{querys} si no que utiliza \textit{querysets}.  Las \textit{querysets} son listas de objetos de un modelo determinado y permiten leer los datos de la base de datos, filtrarlos y ordenarlos.
Por lo que el siguiente paso, es realizar una \textit{queryset} para obtener todas las filas filtrando por el nombre, posteriormente se busca por separado los sinónimos, los hiperónimos y los hipónimos. Para ello, se vuelve a realizar una \textit{queryset} donde el filtro de búsqueda es que tengan el mismo \textit{offset}, el \textit{offset} es un ``identificador'' del \textit{synset} que lo tienen todas las palabras que lo forman.
Una vez obtenidos los sinónimos, hiperónimos e hipónimos de la base de datos, se realiza una búsqueda en un fichero csv que tiene las mil palabras de la RAE quedándose con las palabras que se encuentren en dicho fichero.
Los resultados se guardan en un diccionario cuyas claves son``sinónimos'',``hipónimos'' e ``hiperónimos'' y se convierten a formato JSON. Una vez convertidos se pasan al HTML, donde se recorre y se muestra como en la Figura \ref{fig:resultadoWordnet} donde los resultados se han dividido en 6 campos: Sinónimos que coinciden con las palabras que forman las mil palabras de la RAE, Sinónimos que devuelve la aplicación de WordNet, y de la misma manera para los hipónimos y los hiperónimos.
Por último, el usuario dispone de un botón como el de la Figura \ref{fig:resultadoJSON} para ver los resultados que coinciden con las mil palabras de la RAE en formato JSON.
\figura{Bitmap/Capitulo5/jsonWordnet}{width=1.2\textwidth}{fig:resultadoJSON}{Resultados en formato JSON para el Servicio Web de WordNet}
\figura{Bitmap/Capitulo5/campoTextoWordnet}{width=1.0\textwidth}{fig:textoWordnet}{Campo de texto para introducir una palabra}
\figura{Bitmap/Capitulo5/WordNet}{width=1.0\textwidth}{fig:resultadoWordnet}{Resultados obtenidos para el Servicio Web de WordNet}


%-------------------------------------------------------------------
\section{Pruebas Estadísticas}
%-------------------------------------------------------------------
\label{cap:sec:pruebasEstadisticas}
Una vez implementados los distintos servicios web, los integrantes del grupo deben elegir cuál de ellos devuelve más resultados fiables y cuantos se pueden utilizar, ya que puede ser que los resultados obtenidos difieran mucho del concepto inicial a buscar.
Es por ello que se realizaron varias pruebas introduciendo una cantidad de palabras para medir cuantos de los sinónimos y términos relacionados generados, coincidían con las listas de ``las 1000 palabras más utilizadas del castellano según la RAE'', ``las 5000 palabras más utilizadas del castellano según la RAE'' y ``las 10000 palabras más utilizadas del castellano según la RAE''. Lo primero que se buscó fue un artículo periodístico que constaba de un total de 2.542 palabras, pero que tras un proceso de filtrado se eliminaron aquellas palabras que no fueran verbos, sustantivos, adverbios o adjetivos así como las palabras que apareciesen repetidas, quedando finalmente con un total de 717 palabras válidas para realizar la prueba. A continuación, se introdujeron en el prototipo de ConceptNet y de WordNet generando unos resultados que se describirán y se analizarán a continuación.

\subsection{Estadística para ConceptNet}
\label{cap:subsec:pruebaConceptnet}

Para esta prueba se tuvieron en cuenta principalmente tres parámetros:
\begin{itemize}
	\item Cantidad de palabras que tuviesen al menos un sinónimo.
	\item Cantidad de palabras que tuviesen al menos un término relacionado.
	\item Cantidad de palabras de las que no se obtuvo ningún sinónimo ni ningún término relacionado.
\end{itemize}


Como se puede observar en la Figura \ref{fig:resultadosConceptnet} no se obtuvo ningún resultado para un total de 565 palabras, es decir, un 78,8\% de las palabras introducidas no disponen de ningún sinónimos ni término relacionado que aparezca en la lista de las 1000 palabras de la RAE. Del 21,20\% restante que se atribuye a un total de 173 coincidencias que si disponen de resultados coincidentes en dicha lista, 51 palabras son sinónimos y 122 términos relacionados, es decir, el 6,25\% pertenece a los sinónimos y el 14,95\% a términos relacionados.

\figura{Bitmap/Capitulo5/resultadosConceptnet}{width=1.0\textwidth}{fig:resultadosConceptnet}{Resultados obtenidos para el Servicio Web de ConceptNet con la lista de 1000 palabras}
Se realizó una segunda prueba pero comparando esta vez con la lista de las 5000 palabras de la RAE, los resultados mejoran con respecto a los resultados anteriores, pasando de 173 a 362 las palabras que tienen al menos un sinónimos o un término relacionado en la lista, como se puede comprobar en la Figura \ref{fig:resultadosConceptnet5000}, lo que supone un aumento del 109\%. Los sinónimos encontrados son 94 (aumento del 84\% respecto a los resultados anteriores) y los términos relacionados son 268 (aumento del 120\%), por otra parte las palabras de las que no se obtuvo ninguna coincidencia pasaron a ser 428 (una reducción del 24\%).

\figura{Bitmap/Capitulo5/resultadosConceptnet5000}{width=1.0\textwidth}{fig:resultadosConceptnet5000}{Resultados obtenidos para el Servicio Web de ConceptNet con la lista de 5000 palabras}

Por último, se probó con la lista de 10.000 palabras obteniendo unos resultados de 423 coincidencias totales como se puede comprobar en la Figura \ref{fig:resultadosConceptnet10000} (aumento del 16\% con respecto a la lista de 5.000 palabras), de estas 423 coincidencias 112 corresponden a sinónimos (aumento del 19\%) y 311 términos relacionados (aumento del 16\%). Las palabras que no obtuvieron ningún resultado fueron 387 (disminución del 10\%).

\figura{Bitmap/Capitulo5/resultadosConceptnet10000}{width=1.0\textwidth}{fig:resultadosConceptnet10000}{Resultados obtenidos para el Servicio Web de ConceptNet con la lista de 10000 palabras}


\subsection{Estadística para WordNet}
\label{cap:subsec:pruebaWordnet}

Para esta prueba se valoraron los siguientes parámetros:
\begin{itemize}
	\item Cantidad de palabras que tuviesen al menos un sinónimo.
	\item Cantidad de palabras que tuviesen al menos un hipónimo.
	\item Cantidad de palabras que tuviesen al menos un hiperónimo.
	\item Cantidad de palabras de las que no se obtuvo ningún sinónimo, ni hipónimo ni hiperónimo.
\end{itemize}

Al contrastar los resultados de las consultas a WordNet con la lista de las 1.000 palabras, se obtuvieron como se puede observar en la Figura \ref{fig:resultadosWordnet} un total de 430 coincidencias (131 sinónimos, 112 hipónimos y 187 hiperónimos) y un total de 488 palabras no tienen ningún resultado, lo que supone un 68\% sobre el total de palabras introducidas.

\figura{Bitmap/Capitulo5/resultadosWordnet}{width=1.0\textwidth}{fig:resultadosWordnet}{Resultados obtenidos para el Servicio Web de WordNet con la lista de 1.000 palabras}

Al aumentar la lista de palabras a 5.000(Figura \ref{fig:resultadosWordnet5000}), se obtuvieron un total 631 coincidencias(aumento del 47\%) de las cuales 234 eran sinónimos (aumento del 78\%), 164 hipónimos (46\% más) y 233 hiperónimos (aumento del 25\%). El número de palabras sin coincidencia, fue de 427(disminución del 13\%).

\figura{Bitmap/Capitulo5/resultadosWordnet5000}{width=1.0\textwidth}{fig:resultadosWordnet5000}{Resultados obtenidos para el Servicio Web de WordNet con la lista de 5.000 palabras}

Para finalizar la prueba, se probó con la lista de las 10.000 palabras fáciles y los resultados como se pueden apreciar en la Figura \ref{fig:resultadosWordnet10000} fueron de un total de 685 coincidencias, lo que supone un aumento del 9\% respecto a los resultados obtenidos con la lista de 5.000 palabras, de las cuales 262 eran sinónimos (aumento del 12\%), 177 hipónimos (aumento del 8\%) y 246 hiperónimos(aumento del 6\%). Por último el número de palabras sin coincidencia fue de 413 sobre el total de 717, lo que supone una disminución del 3\% con respecto a la lista anterior.

\figura{Bitmap/Capitulo5/resultadosWordnet10000}{width=1.0\textwidth}{fig:resultadosWordnet10000}{Resultados obtenidos para el Servicio Web de WordNet con la lista de 10.000 palabras}

\subsection{Conclusiones}
\label{cap:subsec:conclusionesPruebas}

A continuación, se muestran unas gráficas para poder poder analizar mejor los resultados y obtener una conclusión más clara. En cuánto a los resultados obtenidos utilizando la lista de las 1000 palabras de la RAE podemos observar en la Figura \ref{fig:resultados1000palabrasConcepNet} que un 78,80\% de las palabras introducidas en ConcepNet no tiene ningún sinónimo ni ningún término relacionado frente a un 68,06\% de las palabras introducidas en WordNet que tampoco tienen ningún resultado coincidente en la lista, como se puede ver en la Figura \ref{fig:resultados1000palabrasWordNet} Y los resultados que si disponen de un algún concepto, ya sea un sinónimo, hiperónimo o hipónimo en WordNet es mayor que en ConcepNet, por lo que en un primer momento el servicio web de WordNet sería más acorde a utilizar para nuestra aplicación.

Si nos fijamos en los resultados obtenidos utilizando la lista de las 5000 palabras de la RAE, podemos ver en la Figura \ref{fig:resultados5000palabrasConcepNet} el porcentaje de palabras de ConcepNet que no tienen ningún resultado coincidente con la lista es similar al de la Figura \ref{fig:resultados5000palabrasWordNet}de WordNet, siendo estos un 59,7\% y un 59,55\% respectivamente. Respecto a los sinónimos, podemos ver que WordNet tiene un porcentaje superior al de ConcepNet, siendo este un 13,10\% frente a un 32,63\%. Por lo que teniendo en cuenta tanto estos resultados como los anteriores, seguiriamos decantandonos más por WordNet que por ConcepNet.

Y por último, utilizando la lista de las 10000 palabras de la RAE, podemos ver en la Figura \ref{fig:resultados10000palabrasConcepNet} que el porcentaje de palabras no encontradas por ConcepNet es esta vez menor que en WordNet, como se puede ver en la Figura \ref{fig:resultados10000palabrasWordNet}, siendo estos un 54\% y un 57,60\% respectivamente. Incluso el porcentaje de términos relacionados encontrados en ConcepNet es también superior a los hiperónimos e hipónimos de WordNet, pero en cambio el porcentaje de sinónimos encontrados en WordNet con un 36,54\% es el doble de los encontrados en ConcepNet, con un 15,62\%.

Teniendo también en cuenta la calidad\footnote{Entendiendo por calidad la relación de las palabras devueltas por las dos aplicaciones con la palabra buscada} de los sinónimos y términos relacionados generados por ambos servicios decidimos utilizar WordNet para la aplicación definitiva ya los resultados son más satisfactorios y las palabras no difieren demasiado del concepto buscado.


\figura{Bitmap/Graficas/resultados1000palabrasConcepNet}{width=.6\textwidth}{fig:resultados1000palabrasConcepNet}{Gráfica de los resultados obtenidos para las 1000 palabras con ConcepNet}
\figura{Bitmap/Graficas/resultados1000palabrasWordNet}{width=.6\textwidth}{fig:resultados1000palabrasWordNet}{Gráfica de los resultados obtenidos para las 1000 palabras con WordNet}

\figura{Bitmap/Graficas/resultados5000palabrasConcepNet}{width=.6\textwidth}{fig:resultados5000palabrasConcepNet}{Gráfica de los resultados obtenidos para las 5000 palabras con ConcepNet}
\figura{Bitmap/Graficas/resultados1000palabrasWordNet}{width=.6\textwidth}{fig:resultados5000palabrasWordNet}{Gráfica de los resultados obtenidos para las 5000 palabras con WordNet}

\figura{Bitmap/Graficas/resultados10000palabrasConcepNet}{width=.6\textwidth}{fig:resultados10000palabrasConcepNet}{Gráfica de los resultados obtenidos para las 10000 palabras con ConcepNet}
\figura{Bitmap/Graficas/resultados10000palabrasWordNet}{width=.6\textwidth}{fig:resultados10000palabrasWordNet}{Gráfica de los resultados obtenidos para las 10000 palabras con WordNet}