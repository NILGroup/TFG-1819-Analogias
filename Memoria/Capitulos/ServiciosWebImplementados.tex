\chapter{Servicios Web para la obtención de conceptos fáciles relacionados}
\label{cap:serviciosWebImplementados}

Como se ha comentado en anteriores capítulos existen varias aplicaciones web que son redes semánticas y facilitan sinónimos, términos relacionados, metáforas, hiperónimos, etc... de un concepto dado por el usuario. Para este trabajo, algunas de ellas son de gran utilidad ya que de esta forma podemos obtener las palabras fáciles para un concepto más complicado, pero hay que saber exactamente cuál es la que mejor conviene y la que mejores resultados ofrece. Hay que corroborar que los conceptos devueltos son correctos y que disponen de un significado claro y parecido respecto al concepto buscado. Para ello se han implementado dos servicios web, uno utilizando la aplicación de ConceptNet y el otro utilizando WordNet, en la sección 5.1 se explicará el diseño de la prueba que se ha realizado para ver que red semántica se utilizará finalmente en el proyecto, en la sección 5.2 se analizarán los resultados cuantitativos de la prueba realizada, en el punto 5.3, se analizarán los resultados obtenidos a nivel cualitativo y en el 5.4 se describirán las conclusiones finales de la prueba. Por otra parte, en los apartados posteriores, se detallarán los distintos servicios web utilizados para este proyecto, en la sección 5.5 se hablará de la base de datos utilizada y de su estructura, en los apartados 5.6, 5.7 y 5.8 se describirán los servicios web que se utilizan para la obtención de sinónimos, hipónimos e hiperónimos fáciles respectivamente. Por último, se hablará en los puntos 5.9 y 5.10 sobre los servicios web utilizados para crear las metáforas y los símiles que finalmente leerá el usuario.


%-------------------------------------------------------------------
\section{Diseño de la evaluación}
%-------------------------------------------------------------------
\label{cap:sec:disenoEval}

Para decidir que red semántica se iba a usar se decidió hacer una prueba con palabras que fuesen lo más heterogéneas posible, para ello se eligieron una serie de artículos periodísticos de distintos temas: medioambiental, tecnológico, deportivo y político. Estos artículos se filtraron para utilizar únicamente los verbos, artículos, sustantivos y adverbios. Para cada una de estas palabras se obtuvieron en WordNet y ConceptNet sus sinónimos y términos relacionados (en WordNet se entiende como término relacionado a los hiperónimos e hipónimos) y se analizaron, tanto la cantidad de términos relacionados y sinónimos que generaban cada una de las redes semánticas que coincidían con alguna de las palabras fáciles de la RAE (prueba cuantitativa) probando en cada una de las tres listas(1.000 palabras fáciles, 5.000 palabras fáciles y 10.000 palabras fáciles), como la calidad de los mismos, es decir, si las palabras que generaban tenían alguna relación aceptable con la palabra origen(prueba cualitativa).

%-------------------------------------------------------------------
\section{Resultados cuantitativos}
%-------------------------------------------------------------------
\label{cap:sec:pruebaCuantitativa}
\figura{Bitmap/Capitulo5/tabla}{width=1.0\textwidth}{fig:tabla}{Tabla de resultados de las pruebas} 

Como se puede observar en la Tabla \ref{fig:tabla} en términos generales, WordNet ofrece mejores resultados que ConceptNet ya que el porcentaje de palabras sin términos relacionados, sinónimos o ambas, es mayor en esta última, siendo especialmente notorio en el caso de los sinónimos con más de 60 puntos de diferencia en todos los casos en favor de WordNet. Los únicos casos en los que ConceptNet supera a WordNet son con las listas de palabras fáciles de 5.000 y 10.000 palabras, ConceptNet deja menos palabras sin coincidencia que WordNet, sin embargo, encuentra mas términos relacionados y muchos más sinónimos que ConceptNet.


\begin{comment}
\begin{table}
\centering
\begin{tabular}{|l|l}
	
	\multicolumn{2}{c} {1000 palabras}\\ 
	
\end{tabular}
\end{table}
\begin{table}
	\begin{tabular}{|l|l|l|l|l|l|l|l|l|l|}
		\hline
		\multirow{2}{*}{Resultados} &
		\multicolumn{3}{c}{Lista 1000 palabras} &
		\multicolumn{3}{c}{Lista 5000 palabras} &
		\multicolumn{3}{c|}{Lista 10000 palabras} \\
		& \% nada & \% noTerm & \% noSin &\% nada & \% noTerm & \% noSin & \% nada & \% noTerm & \% noSin \\
		\hline
		D1 & 2.1\% & 2.1\% & 2.1\% & 2.1\% & 2.1\% & 2.1\% \\
		\hline
		D2 & 11.6\% & 11.6\% & 11.6\% & 11.6\% & 11.6\% & 11.6\% \\
		\hline
		D3 & 5.5\% & 5.5\% & 5.5\% & 5.5\% & 5.5\% & 5.5\% \\
		\hline
	\end{tabular}
\end{table}

\end{comment}

%-------------------------------------------------------------------
\section{Análisis cualitativo}

Pendiente de terminar...
%-------------------------------------------------------------------
\label{cap:sec:pruebaCualitativa}

%-------------------------------------------------------------------
\section{Conclusiones}
%-------------------------------------------------------------------
\label{cap:sec:conclusionPruebas}

Pendiente de terminar...

%-------------------------------------------------------------------
\section{Servidor de Base de Datos}
%-------------------------------------------------------------------
Para este proyecto, se ha utilizado una base de datos para la persistencia de los datos. El sistema encargado de la gestión de la base de datos corresponde con MariaDB y esta constituida por varias tablas:
\begin{itemize}
	\item Se ha hecho uso de MCR (\textit{Multilingual Central Repository}), utilizando la versión 3.0. 
	MCR es una base de datos de código abierto que integra distintas versiones de WordNet para seis lenguajes diferentes: Inglés, Español, Catalán, Vasco, Gallego y Portugués. Se ha utilizado principalmente la tabla llamada \textbf{wei\_spa-30\_variant} y \textbf{wei\_spa-30\_relation}. De la tabla \textit{Variant} las columnas con las que hemos trabajado han sido:
	\begin{itemize}
		\item \textit{Word}: Contiene la palabra. De esta forma se pueden realizar búsquedas cuando el usuario introduce el concepto para de esta forma saber si dicha palabra se encuentra en la base de datos.
		\item \textit{Offset}: Es el identificador de la palabra, aunque una misma palabra puede tener distintos \textit{offsets}. Con dicha columna se pueden obtener los sinónimos, así como el identificador para posteriormente obtener los hipónimos e hiperónimos.
	\end{itemize}
	\item Tres tablas llamadas \textbf{1000\_palabras\_faciles}, \textbf{5000\_palabras\_faciles} y \textbf{10000\_palabras\_faciles} que guardan tanto las 1.000, 5.000 y 10.000 palabras más usadas de la RAE. Las tres se componen únicamente de una columna llamada \textit{word} y que almacena la palabra.
	\item Una tabla llamada \textbf{datos\_picto}, que está formada por las siguientes columnas:
	\begin{itemize}
		\item Palabra: Contiene la palabra en cuestión. 
		\item Offset31: Es el identificador del \textit{synset} en la versión 3.1 de WordNet.
		\item Offset30: Es el identificador del \textit{synset} en la versión 3.0 de WordNet.
		\item id\_picto: Es el identificador del pictograma.
	\end{itemize}
	\item Una última tabla llamada \textbf{pictogramas} que almacena los pictogramas de ARASAAC y que contienen dos columnas: 
	\begin{itemize}
		\item id\_picto: Es el identificador del pictograma.
		\item imagen: Contiene el pictograma.
	\end{itemize}
 
\end{itemize}



%-------------------------------------------------------------------
\section{Servicio Web  para obtener sinónimos fáciles}
%-------------------------------------------------------------------

La implementación de dicho servicio web se basa en que introduciendo una palabra y un nivel de búsqueda, este devuelve todos los sinónimos fáciles correspondientes en formato JSON. Para ello se realiza la siguiente petición GET:

http://127.0.0.1:8000/easySynonym/json/word=\textit{palabra}\&level=\textit{nivel}

Donde \textit{palabra} es el concepto a buscar y \textit{nivel} es el grado de búsqueda, el cual puede tomar los siguientes valores:
\begin{itemize}
	\item Nivel 1 (Nivel sencillo): Se realizará la búsqueda en las 1.000 palabras más usadas de la RAE.
	\item Nivel 2 (Nivel medio): Se realizará la búsqueda en las 5.000 palabras más usadas de la RAE.
	\item Nivel 3 (Nivel avanzado): Se realizará la búsqueda en las 10.000 palabras más usadas de la RAE.
\end{itemize}

Una vez introducida la palabra y el nivel, se realiza una consulta a la base de datos de MCR 3.0 a través de una \textit{queryset} para obtener todos los \textit{offsets} cuya palabra sea igual que la introducida.
Por cada \textit{offset} volveremos a buscar en la tabla \textit{Variant} de la base de datos de MCR 3.0 para obtener las palabras que compartan dicho identificador y posteriormente, mediante un cursor se realizará una búsqueda en una de las tres tablas de la RAE (en función del nivel introducido) y buscará si alguna de estas palabras se encuentra en dicha tabla.
Si el resultado es positivo se añadirá al JSON.

Por ejemplo, si se realizará una búsqueda para la palabra inmueble con un nivel de búsqueda 2, la petición GET sería de la siguiente manera:

http://127.0.0.1:8000/easySynonym/json/word=inmueble\&level=2

Y en la Figura \ref{fig:peticionGetEasySynonym} se puede ver el resultado obtenido en formato JSON. En primer lugar aparece la palabra buscada, en este caso inmueble y a continuación un objeto que incluye tanto el \textit{offset} como la lista de sinónimos fáciles .
 

\figura{Bitmap/Capitulo5/peticionGetEasySynonym}{width=1.0\textwidth}{fig:peticionGetEasySynonym}{JSON devuelto al buscar los sinónimos fáciles de inmueble}



%-------------------------------------------------------------------
\section{Servicio Web  para obtener hipónimos fáciles}
%-------------------------------------------------------------------

La implementación de dicho servicio web se basa en que introduciendo una palabra y un nivel de búsqueda, este devuelve todos los hipónimos fáciles correspondientes en formato JSON. Para ello se realiza la siguiente petición GET:

http://127.0.0.1:8000/easyHyponym/json/word=\textit{palabra}\&level=\textit{nivel}

Donde \textit{palabra} es el concepto a buscar y \textit{nivel} es el grado de búsqueda, igual que en el caso explicado anteriormente para buscar sinónimos fáciles.

Una vez introducida la palabra y el nivel, se realiza una consulta a la base de datos de MCR 3.0 a través de una \textit{queryset} para obtener todos los \textit{offsets} cuya palabra sea igual que la introducida.
Por cada \textit{offset} volveremos a buscar en la tabla \textit{Relation} de la base de datos de MCR 3.0 para obtener los offsets que se encuentran en la columna \textit{targetSynset}. Después, con cada \textit{offset} obtenido de esta \textit{queryset}, se realizará la búsqueda en la tabla \textit{Variant} para obtener las palabras cuyo identificador sea igual que el offset de \textit{targetSynset}.
Mediante un cursor se realizará una búsqueda en una de las tres tablas de la RAE (en función del nivel introducido) y buscará si alguna de estas palabras se encuentra en dicha tabla.
Si el resultado es positivo se añadirá al JSON.

Por ejemplo, si se realizará una búsqueda para la palabra inmueble con un nivel de búsqueda 2, la petición GET sería de la siguiente manera:

http://127.0.0.1:8000/easyHyponym/json/word=inmueble\&level=2

Y en la Figura \ref{fig:peticionGetEasyHyponym} se puede ver el resultado obtenido en formato JSON. En primer lugar aparece la palabra buscada, en este caso inmueble y a continuación un objeto que incluye el \textit{offset} de hipónimo fácil , la lista de hipónimos fáciles así como el \textit{offsetFather}, este identificador corresponde al \textit{offset} de uno de los \textit{synsets} de inmueble. De esta forma, si una palabra buscada dispone de varios \textit{synsets}, se podrá mostrar sus correspondientes sinónimos e hipónimos.


\figura{Bitmap/Capitulo5/peticionGetEasyHyponym}{width=1.0\textwidth}{fig:peticionGetEasyHyponym}{JSON devuelto al buscar los hipónimos fáciles de inmueble}
%-------------------------------------------------------------------
\section{Servicio Web  para obtener hiperónimos fáciles}
%-------------------------------------------------------------------
La implementación de dicho servicio web se basa en que introduciendo una palabra y un nivel de búsqueda, este devuelve todos los hiperónimos fáciles correspondientes en formato JSON. Para ello se realiza la siguiente petición GET:

http://127.0.0.1:8000/easyHyperonym/json/word=\textit{palabra}\&level=\textit{nivel}

Donde \textit{palabra} es el concepto a buscar y \textit{nivel} es el grado de búsqueda, igual que en los casos explicados anteriormente para buscar sinónimos e hipónimos fáciles.

Una vez introducida la palabra y el nivel, se realiza una consulta a la base de datos de MCR 3.0 a través de una \textit{queryset} para obtener todos los \textit{offsets} cuya palabra sea igual que la introducida.
Por cada \textit{offset} volveremos a buscar en la tabla \textit{Relation} de la base de datos de MCR 3.0 para obtener los offsets que se encuentran esta vez y con diferencia de la búsqueda de hipónimos fáciles, en la columna \textit{sourceSynset}. Después, con cada \textit{offset} obtenido de esta \textit{queryset}, se realizará la búsqueda en la tabla \textit{Variant} para obtener las palabras cuyo identificador sea igual que el offset de \textit{sourcetSynset}.
Mediante un cursor se realizará una búsqueda en una de las tres tablas de la RAE (en función del nivel introducido) y buscará si alguna de estas palabras se encuentra en dicha tabla.
Si el resultado es positivo se añadirá al JSON.

Por ejemplo, si se realizará una búsqueda para la palabra inmueble con un nivel de búsqueda 2, la petición GET sería de la siguiente manera:

http://127.0.0.1:8000/easyHyperonym/json/word=inmueble\&level=2

En la Figura \ref{fig:peticionGetEasyHyperonym} se puede ver el resultado obtenido en formato JSON. En primer lugar aparece la palabra buscada, en este caso inmueble y a continuación un objeto que incluye el \textit{offset} de hiperónimo fácil, la lista de hiperónimos fáciles así como el \textit{offsetFather}, ya explicado en el apartado anterior.


\figura{Bitmap/Capitulo5/peticionGetEasyHyperonym}{width=1.0\textwidth}{fig:peticionGetEasyHyperonym}{JSON devuelto al buscar los hiperónimos fáciles de inmueble}
%-------------------------------------------------------------------
\section{Servicio Web  para obtener una metáfora}
%-------------------------------------------------------------------
Este servicio web dado un offset y una profundidad obtiene las metáforas de un concepto, es decir, se obtienen tanto los sinónimos como los hiperónimos fáciles formando la metáfora y este servicio devuelve dichos resultados en formato JSON.
Para ello se realiza la siguiente petición GET:

http://127.0.0.1:8000/metaphor/json/word=\textit{palabra}\&level=\textit{nivel}

Donde \textit{palabra} es el concepto a buscar y \textit{nivel} es el grado de búsqueda, igual que en los casos explicados anteriormente para buscar sinónimos, hipónimos e hiperónimos fáciles.

La implementación de dicho servicio es idéntico al ya explicado en el servicio web para obtener sinónimos fáciles e hiperónimos fáciles, con la única diferencia que para poder formar la metáfora se llama a otro servicio implementado, el cual utiliza SpaCy para definir si el concepto es masculino o femenino y si está en singular o plural.

Por ejemplo, si se realizará una búsqueda para la palabra inmueble con un nivel de búsqueda 2, la petición GET sería de la siguiente manera:

http://127.0.0.1:8000/metaphor/json/word=inmueble\&level=2

En la Figura \ref{fig:peticionMetaphor} se puede ver el resultado obtenido en formato JSON. En primer lugar aparece la palabra buscada, en este caso inmueble y a continuación un objeto que incluye el \textit{offset} del sinónimo o hiperónimo fácil, el \textit{offsetFather}  ya explicado en apartados anteriores y la lista de metáforas (es una casa, es una construcción, son unos edificios, es un edificio). 
	
	
\figura{Bitmap/Capitulo5/peticionMetaphor}{width=1.0\textwidth}{fig:peticionMetaphor}{JSON devuelto al buscar las metáforas de la palabra inmueble}
%-------------------------------------------------------------------
\section{Servicio Web  para obtener un símil}
%-------------------------------------------------------------------
Este servicio web dado un offset y una profundidad obtiene los símiles de un concepto, es decir, se obtienen los hipónimos fáciles formando así el símil y devolviendo los resultados en formato JSON.
Para ello se realiza la siguiente petición GET:

http://127.0.0.1:8000/simil/json/word=\textit{palabra}\&level=\textit{nivel}

Donde \textit{palabra} es el concepto a buscar y \textit{nivel} es el grado de búsqueda, igual que en los casos explicados anteriormente.
Para la implementación de dicho servicio, se utiliza exactamente el mismo código explicado en el apartado de obtención de hipónimos fáciles pero incluyendo la llamada al servicio de SpaCy para volver a definir si el concepto es masculino o femenino y si es singular o plural.

Por ejemplo, si se realizará una búsqueda para la palabra inmueble con un nivel de búsqueda 2, la petición GET sería de la siguiente manera:

http://127.0.0.1:8000/simil/json/word=inmueble\&level=2

En la Figura \ref{fig:peticionSimil} se puede ver el resultado obtenido en formato JSON. En primer lugar aparece la palabra buscada, en este caso inmueble y a continuación un objeto que incluye el \textit{offset} del hipónimo fácil, el \textit{offsetFather}  ya explicado en apartados anteriores y la lista de símiles (es como una biblioteca es como un colegio, es como un restaurante).


\figura{Bitmap/Capitulo5/peticionSimil}{width=1.0\textwidth}{fig:peticionSimil}{JSON devuelto al buscar los símiles de la palabra inmueble}


