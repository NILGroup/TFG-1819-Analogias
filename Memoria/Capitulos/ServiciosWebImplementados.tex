\chapter{Servicios Web para la obtención de conceptos fáciles relacionados}
\label{cap:serviciosWebImplementados}

Como se ha comentado en anteriores capítulos existen varias aplicaciones web que son redes semánticas y facilitan sinónimos, términos relacionados, metáforas, hiperónimos, etc... de un concepto dado por el usuario. Para este trabajo, algunas de ellas son de gran utilidad ya que de esta forma podemos obtener las palabras fáciles para un concepto más complicado, pero hay que saber exactamente cuál es la que mejor conviene y la que mejores resultados ofrece. Hay que corroborar que los conceptos devueltos son correctos y que disponen de un significado claro y parecido respecto al concepto buscado. Para ello se han implementado dos servicios web, uno utilizando la aplicación de ConceptNet y el otro utilizando WordNet, en la sección 5.1 se explicará el servicio web para ConceptNet con los métodos creados y su explicación y en la sección 5.2 aparece lo mismo pero utilizando WordNet. Por último, una vez implementados dichos servicios, se han realizado una serie de pruebas para comprobar que aplicación era más necesaria para la aplicación final, en la sección 5.3 queda reflejado dichas pruebas, como se han realizado, con que cantidad de palabras se han hecho las pruebas y cuales han sido los resultados obtenidos.


%-------------------------------------------------------------------
\section{Diseño de la evaluación}
%-------------------------------------------------------------------
\label{cap:sec:disenoEval}

Para decidir que red semántica se iba a usar se decidió hacer una prueba con palabras que fuesen lo más heterogéneas posible, para ello se eligieron una serie de artículos periodísticos de distintos temas: medioambiental, tecnológico, deportivo y político. Estos artículos se filtraron para utilizar únicamente los verbos, artículos, sustantivos y adverbios. Para cada una de estas palabras se obtuvieron en WordNet y ConceptNet sus sinónimos y términos relacionados (en WordNet se entiende como término relacionado a los hiperónimos e hipónimos) y se analizaron, tanto la cantidad de términos relacionados y sinónimos que generaban cada una de las redes semánticas que coincidían con alguna de las palabras fáciles de la RAE (prueba cuantitativa) probando en cada una de las tres listas(1.000 palabras fáciles, 5.000 palabras fáciles y 10.000 palabras fáciles), como la calidad de los mismos, es decir, si las palabras que generaban tenían alguna relación aceptable con la palabra origen(prueba cualitativa).

%-------------------------------------------------------------------
\section{Resultados cuantitativos}
%-------------------------------------------------------------------
\label{cap:sec:pruebaCuantitativa}
\figura{Bitmap/Capitulo5/tabla}{width=1.0\textwidth}{fig:tabla}{Tabla de resultados de las pruebas} 

Como se puede observar en la Tabla \ref{fig:tabla} en términos generales, WordNet ofrece mejores resultados que ConceptNet ya que el porcentaje de palabras sin términos relacionados, sinónimos o ambas, es mayor en esta última, siendo especialmente notorio en el caso de los sinónimos con más de 60 puntos de diferencia en todos los casos en favor de WordNet. Los únicos casos en los que ConceptNet supera a WordNet son con las listas de palabras fáciles de 5.000 y 10.000 palabras, ConceptNet deja menos palabras sin coincidencia que WordNet, sin embargo, encuentra mas términos relacionados y muchos más sinónimos que ConceptNet.


\begin{comment}
\begin{table}
\centering
\begin{tabular}{|l|l}
	
	\multicolumn{2}{c} {1000 palabras}\\ 
	
\end{tabular}
\end{table}
\begin{table}
	\begin{tabular}{|l|l|l|l|l|l|l|l|l|l|}
		\hline
		\multirow{2}{*}{Resultados} &
		\multicolumn{3}{c}{Lista 1000 palabras} &
		\multicolumn{3}{c}{Lista 5000 palabras} &
		\multicolumn{3}{c|}{Lista 10000 palabras} \\
		& \% nada & \% noTerm & \% noSin &\% nada & \% noTerm & \% noSin & \% nada & \% noTerm & \% noSin \\
		\hline
		D1 & 2.1\% & 2.1\% & 2.1\% & 2.1\% & 2.1\% & 2.1\% \\
		\hline
		D2 & 11.6\% & 11.6\% & 11.6\% & 11.6\% & 11.6\% & 11.6\% \\
		\hline
		D3 & 5.5\% & 5.5\% & 5.5\% & 5.5\% & 5.5\% & 5.5\% \\
		\hline
	\end{tabular}
\end{table}

\end{comment}

%-------------------------------------------------------------------
\section{Análisis cualitativo}

Pendiente de terminar...
%-------------------------------------------------------------------
\label{cap:sec:pruebaCualitativa}

%-------------------------------------------------------------------
\section{Conclusiones}
%-------------------------------------------------------------------
\label{cap:sec:conclusionPruebas}

Pendiente de terminar...

%-------------------------------------------------------------------
\section{Servidor de Base de Datos}
%-------------------------------------------------------------------
Para este proyecto, se ha utilizado una base de datos para la persistencia de los datos. El sistema encargado de la gestión de la base de datos corresponde con MariaDB y esta constituida por varias tablas:
\begin{itemize}
	\item Se ha hecho uso de MCR (\textit{Multilingual Central Repository}), utilizando la versión 3.0. 
	MCR es una base de datos de código abierto que integra distintas versiones de WordNet para seis lenguajes diferentes: Inglés, Español, Catalán, Vasco, Gallego y Portugués. Se ha utilizado principalmente la tabla llamada \textbf{wei\_spa-30\_variant} y \textbf{wei\_spa-30\_relation}. De la tabla \textit{Variant} las columnas con las que hemos trabajado han sido:
	\begin{itemize}
		\item \textit{Word}: Contiene la palabra. De esta forma se pueden realizar búsquedas cuando el usuario introduce el concepto para de esta forma saber si dicha palabra se encuentra en la base de datos.
		\item \textit{Offset}: Es el identificador de la palabra, aunque una misma palabra puede tener distintos \textit{offsets}. Con dicha columna se pueden obtener los sinónimos, así como el identificador para posteriormente obtener los hipónimos e hiperónimos.
	\end{itemize}
	\item Tres tablas llamadas \textbf{1000\_palabras\_faciles}, \textbf{5000\_palabras\_faciles} y \textbf{10000\_palabras\_faciles} que guardan tanto las 1.000, 5.000 y 10.000 palabras más usadas de la RAE. Las tres se componen únicamente de una columna llamada \textit{word} y que almacena la palabra.
	\item Una tabla llamada \textbf{datos\_picto}, que está formada por las siguientes columnas:
	\begin{itemize}
		\item Palabra: Contiene la palabra en cuestión. 
		\item Offset31: Es el identificador del \textit{synset} en la versión 3.1 de WordNet.
		\item Offset30: Es el identificador del \textit{synset} en la versión 3.0 de WordNet.
		\item id\_picto: Es el identificador del pictograma.
	\end{itemize}
	\item Una última tabla llamada \textbf{pictogramas} que almacena los pictogramas de ARASAAC y que contienen dos columnas: 
	\begin{itemize}
		\item id\_picto: Es el identificador del pictograma.
		\item imagen: Contiene el pictograma.
	\end{itemize}
 
\end{itemize}



%-------------------------------------------------------------------
\section{Servicio Web  para obtener sinónimos fáciles}
%-------------------------------------------------------------------

La implementación de dicho servicio web se basa en que introduciendo una palabra y un nivel de búsqueda, este devuelve todos los sinónimos fáciles correspondientes en formato JSON. Para ello se realiza la siguiente petición GET:

http://127.0.0.1:8000/easySynonym/json/word=\textit{palabra}\&level=\textit{nivel}

Donde \textit{palabra} es el concepto a buscar y \textit{nivel} es el grado de búsqueda, el cual puede tomar los siguientes valores:
\begin{itemize}
	\item Nivel 1 (Nivel sencillo): Se realizará la búsqueda en las 1.000 palabras más usadas de la RAE.
	\item Nivel 2 (Nivel medio): Se realizará la búsqueda en las 5.000 palabras más usadas de la RAE.
	\item Nivel 3 (Nivel avanzado): Se realizará la búsqueda en las 10.000 palabras más usadas de la RAE.
\end{itemize}

Una vez introducida la palabra y el nivel, se realiza una consulta a la base de datos de MCR 3.0 a través de una \textit{queryset} para obtener todos los \textit{offsets} cuya palabra sea igual que la introducida.
Por cada \textit{offset} volveremos a buscar en la tabla \textit{Variant} de la base de datos de MCR 3.0 para obtener las palabras que compartan dicho identificador y posteriormente, mediante un cursor se realizará una búsqueda en una de las tres tablas de la RAE (en función del nivel introducido) y buscará si alguna de estas palabras se encuentra en dicha tabla.
Si el resultado es positivo se añadirá al JSON.

Por ejemplo, si se realizará una búsqueda para la palabra inmueble con un nivel de búsqueda 2, la petición GET sería de la siguiente manera:

http://127.0.0.1:8000/easySynonym/json/word=inmueble\&level=2

Y en la Figura \ref{fig:peticionGetEasySynonym} se puede ver el resultado obtenido en formato JSON. En primer lugar aparece la palabra buscada, en este caso inmueble y a continuación un objeto que incluye tanto el \textit{offset} como la lista de sinónimos fáciles .
 

\figura{Bitmap/Capitulo5/peticionGetEasySynonym}{width=1.0\textwidth}{fig:peticionGetEasySynonym}{JSON devuelto al buscar los sinónimos fáciles de inmueble}



%-------------------------------------------------------------------
\section{Servicio Web  para obtener hipónimos fáciles}
%-------------------------------------------------------------------

La implementación de dicho servicio web se basa en que introduciendo una palabra y un nivel de búsqueda, este devuelve todos los hipónimos fáciles correspondientes en formato JSON. Para ello se realiza la siguiente petición GET:

http://127.0.0.1:8000/easyHyponym/json/word=\textit{palabra}\&level=\textit{nivel}

Donde \textit{palabra} es el concepto a buscar y \textit{nivel} es el grado de búsqueda, igual que en el caso explicado anteriormente para buscar sinónimos fáciles.

Una vez introducida la palabra y el nivel, se realiza una consulta a la base de datos de MCR 3.0 a través de una \textit{queryset} para obtener todos los \textit{offsets} cuya palabra sea igual que la introducida.
Por cada \textit{offset} volveremos a buscar en la tabla \textit{Relation} de la base de datos de MCR 3.0 para obtener los offsets que se encuentran en la columna \textit{targetSynset}. Después, con cada \textit{offset} obtenido de esta \textit{queryset}, se realizará la búsqueda en la tabla \textit{Variant} para obtener las palabras cuyo identificador sea igual que el offset de \textit{targetSynset}.
Mediante un cursor se realizará una búsqueda en una de las tres tablas de la RAE (en función del nivel introducido) y buscará si alguna de estas palabras se encuentra en dicha tabla.
Si el resultado es positivo se añadirá al JSON.

Por ejemplo, si se realizará una búsqueda para la palabra inmueble con un nivel de búsqueda 2, la petición GET sería de la siguiente manera:

http://127.0.0.1:8000/easyHyponym/json/word=inmueble\&level=2

Y en la Figura \ref{fig:peticionGetEasyHyponym} se puede ver el resultado obtenido en formato JSON. En primer lugar aparece la palabra buscada, en este caso inmueble y a continuación un objeto que incluye el \textit{offset} de hipónimo fácil , la lista de hipónimos fáciles así como el \textit{offsetFather}, este identificador corresponde al \textit{offset} de uno de los \textit{synsets} de inmueble. De esta forma, si una palabra buscada dispone de varios \textit{synsets}, se podrá mostrar sus correspondientes sinónimos e hipónimos.


\figura{Bitmap/Capitulo5/peticionGetEasyHyponym}{width=1.0\textwidth}{fig:peticionGetEasyHyponym}{JSON devuelto al buscar los hipónimos fáciles de inmueble}
%-------------------------------------------------------------------
\section{Servicio Web  para obtener hiperónimos fáciles}
%-------------------------------------------------------------------
La implementación de dicho servicio web se basa en que introduciendo una palabra y un nivel de búsqueda, este devuelve todos los hiperónimos fáciles correspondientes en formato JSON. Para ello se realiza la siguiente petición GET:

http://127.0.0.1:8000/easyHyperonym/json/word=\textit{palabra}\&level=\textit{nivel}

Donde \textit{palabra} es el concepto a buscar y \textit{nivel} es el grado de búsqueda, igual que en los casos explicados anteriormente para buscar sinónimos e hipónimos fáciles.

Una vez introducida la palabra y el nivel, se realiza una consulta a la base de datos de MCR 3.0 a través de una \textit{queryset} para obtener todos los \textit{offsets} cuya palabra sea igual que la introducida.
Por cada \textit{offset} volveremos a buscar en la tabla \textit{Relation} de la base de datos de MCR 3.0 para obtener los offsets que se encuentran esta vez y con diferencia de la búsqueda de hipónimos fáciles, en la columna \textit{sourceSynset}. Después, con cada \textit{offset} obtenido de esta \textit{queryset}, se realizará la búsqueda en la tabla \textit{Variant} para obtener las palabras cuyo identificador sea igual que el offset de \textit{sourcetSynset}.
Mediante un cursor se realizará una búsqueda en una de las tres tablas de la RAE (en función del nivel introducido) y buscará si alguna de estas palabras se encuentra en dicha tabla.
Si el resultado es positivo se añadirá al JSON.

Por ejemplo, si se realizará una búsqueda para la palabra inmueble con un nivel de búsqueda 2, la petición GET sería de la siguiente manera:

http://127.0.0.1:8000/easyHyperonym/json/word=inmueble\&level=2

En la Figura \ref{fig:peticionGetEasyHyperonym} se puede ver el resultado obtenido en formato JSON. En primer lugar aparece la palabra buscada, en este caso inmueble y a continuación un objeto que incluye el \textit{offset} de hiperónimo fácil, la lista de hiperónimos fáciles así como el \textit{offsetFather}, ya explicado en el apartado anterior.


\figura{Bitmap/Capitulo5/peticionGetEasyHyperonym}{width=1.0\textwidth}{fig:peticionGetEasyHyperonym}{JSON devuelto al buscar los hiperónimos fáciles de inmueble}
%-------------------------------------------------------------------
\section{Servicio Web  para obtener una metáfora}
%-------------------------------------------------------------------
Este servicio web dado un offset y una profundidad obtiene las metáforas de un concepto, es decir, se obtienen tanto los sinónimos como los hiperónimos fáciles formando la metáfora y este servicio devuelve dichos resultados en formato JSON.
Para ello se realiza la siguiente petición GET:

http://127.0.0.1:8000/metaphor/json/word=\textit{palabra}\&level=\textit{nivel}

Donde \textit{palabra} es el concepto a buscar y \textit{nivel} es el grado de búsqueda, igual que en los casos explicados anteriormente para buscar sinónimos, hipónimos e hiperónimos fáciles.

La implementación de dicho servicio es idéntico al ya explicado en el servicio web para obtener sinónimos fáciles e hiperónimos fáciles, con la única diferencia que para poder formar la metáfora se llama a otro servicio implementado, el cual utiliza SpaCy para definir si el concepto es masculino o femenino y si está en singular o plural.

Por ejemplo, si se realizará una búsqueda para la palabra inmueble con un nivel de búsqueda 2, la petición GET sería de la siguiente manera:

http://127.0.0.1:8000/metaphor/json/word=inmueble\&level=2

En la Figura \ref{fig:peticionMetaphor} se puede ver el resultado obtenido en formato JSON. En primer lugar aparece la palabra buscada, en este caso inmueble y a continuación un objeto que incluye el \textit{offset} del sinónimo o hiperónimo fácil, el \textit{offsetFather}  ya explicado en apartados anteriores y la lista de metáforas (es una casa, es una construcción, son unos edificios, es un edificio). 
	
	
\figura{Bitmap/Capitulo5/peticionMetaphor}{width=1.0\textwidth}{fig:peticionMetaphor}{JSON devuelto al buscar las metáforas de la palabra inmueble}
%-------------------------------------------------------------------
\section{Servicio Web  para obtener un símil}
%-------------------------------------------------------------------
Este servicio web dado un offset y una profundidad obtiene los símiles de un concepto, es decir, se obtienen los hipónimos fáciles formando así el símil y devolviendo los resultados en formato JSON.
Para ello se realiza la siguiente petición GET:

http://127.0.0.1:8000/simil/json/word=\textit{palabra}\&level=\textit{nivel}

Donde \textit{palabra} es el concepto a buscar y \textit{nivel} es el grado de búsqueda, igual que en los casos explicados anteriormente.
Para la implementación de dicho servicio, se utiliza exactamente el mismo código explicado en el apartado de obtención de hipónimos fáciles pero incluyendo la llamada al servicio de SpaCy para volver a definir si el concepto es masculino o femenino y si es singular o plural.

Por ejemplo, si se realizará una búsqueda para la palabra inmueble con un nivel de búsqueda 2, la petición GET sería de la siguiente manera:

http://127.0.0.1:8000/simil/json/word=inmueble\&level=2

En la Figura \ref{fig:peticionSimil} se puede ver el resultado obtenido en formato JSON. En primer lugar aparece la palabra buscada, en este caso inmueble y a continuación un objeto que incluye el \textit{offset} del hipónimo fácil, el \textit{offsetFather}  ya explicado en apartados anteriores y la lista de símiles (es como una biblioteca es como un colegio, es como un restaurante).


\figura{Bitmap/Capitulo5/peticionSimil}{width=1.0\textwidth}{fig:peticionSimil}{JSON devuelto al buscar los símiles de la palabra inmueble}


%-------------------------------------------------------------------
\section{Servicio Web Para ConceptNet}
%-------------------------------------------------------------------
\label{cap:sec:servicioConceptnet}

El primer Servicio Web implementado obtiene los resultados de ConceptNet, una red semántica que al introducir una palabra devuelve sinónimos y términos relacionados de la misma (dicha red semántica ha sido descrita detalladamente en el apartado \ref{cap:subsec:concepnet}).

\figura{Bitmap/Capitulo5/ejemploConceptnet}{width=1.0\textwidth}{fig:conceptnet}{Interfaz del Servicio Web para ConceptNet}

Este servicio web consta de tres servicios, como se puede ver en la Figura \ref{fig:conceptnet}. El primero, cuando se introduce en el campo de texto una palabra en castellano, se realiza una consulta a la API de ConceptNet y se muestran únicamente los sinónimos obtenidos para dicha palabra. El segundo, realiza otra consulta a la API de ConceptNet pero esta vez en lugar de obtener los sinónimos, devuelve solamente los términos relacionados.

Y por último, el tercer servicio hace uso de los servicios 1 y 2 para obtener tanto los sinónimos como los términos relacionados del concepto introducido. Con los resultados obtenidos, se realiza una nueva búsqueda en la lista de ``las 1000 palabras más utilizadas del castellano según la RAE'' y en caso de que algún resultado se encuentre en dicha lista, se le mostrará al usuario y si no, se mostrará un mensaje informativo indicando que no hay resultados. Aparte, este servicio consta de un selector numérico donde se indica la profundidad de la búsqueda que por defecto, su valor es 1. Esto quiere decir que, por ejemplo, con un nivel de profundidad de nivel 1 se realizará el proceso descrito anteriormente una única vez. Sin embargo, si se introduce una profundidad mayor, por ejemplo 2, se repetirá el proceso dos veces pero esta vez buscando los sinónimos y los términos relacionados de los resultados y no de la palabra introducida en un primer momento. Por ejemplo, si el usuario introduce la palabra Gato, ConceptNet devuelve como sinónimos: feria, madrileño y felino pero ninguna de estas palabras se encuentran en las lista de ``las 1000 palabras más utilizadas del castellano según la RAE'', por lo que el servicio web implementado buscará los sinónimos y términos relacionados de cada una de las palabras, es decir, de feria, madrileño y felino. Si alguno de los resultados se encuentra en la lista, entonces se mostrarán y si no, aparecerá el mensaje informativo indicando que no hay resultados.

Si por otro lado, se añade un nivel de profundidad aún mayor pero la coincidencia de los resultados con la lista se encuentra en un nivel anterior, la búsqueda se detendrá y se indicará al usuario en que nivel ha sido obtenido con éxito dicho resultado.
Ahora bien, dependiendo de qué tipo de palabra haya coincidido y en que nivel, se generará el resultado utilizando un tipo de comparación distinta. Estas pueden ser:
\begin{itemize}
	\item Si la palabra que coincide es un sinónimo de la palabra original  y se ha encontrado en un nivel de profundidad 1, entonces se entiende que existe una similitud alta entre ambos conceptos por lo que el resultado que se mostrará tendrá el siguiente aspecto: ``A \textbf{es} B'' . Por ejemplo, si se ha introducido ``hogar'' y el resultado obtenido con éxito es ``casa'' , se mostrará: ``hogar \textbf{es} casa''.
	
	\item En cualquier otro caso, se entiende que la similitud entre los conceptos es más baja por lo que el mensaje que aparecerá será:  ``A \textbf{es como} B''. Por ejemplo, si se ha introducido ``casa'' y se ha encontrado como término relacionado en el nivel 1 de profundidad o en cualquier otro nivel la palabra ``edificio'', la comparación que se mostrará será: ``casa \textbf{es como} edificio''.
\end{itemize}


%-------------------------------------------------------------------
\section{Servicio Web para WordNet}
%-------------------------------------------------------------------
\label{cap:sec:servicioWordnet}

Para la implementación y desarrollo del servicio web, se ha hecho uso de MCR (\textit{Multilingual Central Repository}), el cuál es una base de datos de código abierto que integra distintas versiones de WordNet para seis lenguajes diferentes: Inglés, Español, Catalán, Vasco, Gallego y Portugués. 
Lo que se consigue al utilizar MCR es poder acceder a la base de datos de WordNet y obtener los resultados que dicha aplicación.
Como se puede ver en la Figura \ref{fig:textoWordnet}, el usuario dispone de un campo de texto donde puede introducir únicamente una palabra. Cuando se pulsa el botón de enviar, se capta dicho concepto mediante una petición POST a través de un formulario.
Al tener la base de datos de WordNet, podemos realizar consultas directamente para obtener los resultados en función de si se buscan sinónimos, hiperónimos o hipónimos, y aunque la base de datos es mysql, al estar utilizando el framework Django este no admite la realización de \textit{querys} si no que utiliza \textit{querysets}.  Las \textit{querysets} son listas de objetos de un modelo determinado y permiten leer los datos de la base de datos, filtrarlos y ordenarlos.
Por lo que el siguiente paso, es realizar una \textit{queryset} para obtener todas las filas filtrando por el nombre, posteriormente se busca por separado los sinónimos, los hiperónimos y los hipónimos. Para ello, se vuelve a realizar una \textit{queryset} donde el filtro de búsqueda es que tengan el mismo \textit{offset}, el \textit{offset} es un ``identificador'' del \textit{synset} que lo tienen todas las palabras que lo forman.
Una vez obtenidos los sinónimos, hiperónimos e hipónimos de la base de datos, se realiza una búsqueda en un fichero csv que tiene las mil palabras de la RAE quedándose con las palabras que se encuentren en dicho fichero.
Los resultados se guardan en un diccionario cuyas claves son``sinónimos'',``hipónimos'' e ``hiperónimos'' y se convierten a formato JSON. Una vez convertidos se pasan al HTML, donde se recorre y se muestra como en la Figura \ref{fig:resultadoWordnet} donde los resultados se han dividido en 6 campos: Sinónimos que coinciden con las palabras que forman las mil palabras de la RAE, Sinónimos que devuelve la aplicación de WordNet, y de la misma manera para los hipónimos y los hiperónimos.
Por último, el usuario dispone de un botón como el de la Figura \ref{fig:resultadoJSON} para ver los resultados que coinciden con las mil palabras de la RAE en formato JSON.
\figura{Bitmap/Capitulo5/jsonWordnet}{width=1.2\textwidth}{fig:resultadoJSON}{Resultados en formato JSON para el Servicio Web de WordNet}
\figura{Bitmap/Capitulo5/campoTextoWordnet}{width=1.0\textwidth}{fig:textoWordnet}{Campo de texto para introducir una palabra}
\figura{Bitmap/Capitulo5/WordNet}{width=1.0\textwidth}{fig:resultadoWordnet}{Resultados obtenidos para el Servicio Web de WordNet}


