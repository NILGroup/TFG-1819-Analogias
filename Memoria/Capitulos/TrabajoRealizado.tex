\chapter{Trabajo Realizado}
\label{cap:TrabajoRealizado}


En este capítulo vamos a describir que trabajo hemos hecho cada uno


\section{Trabajo realizado por Irene}
\label{cap:sec:trabajo_Irene}

Primero investigué las bibliotecas que utilizaremos para el procesado de las palabras, al principio encontramos una biblioteca para el procesado de texto en Python, que es la ntlk pero vimos que las etiquetas que ponía a las palabras no eran del todo correctas por lo que buscamos otra biblioteca y encontramos Spacy, con esta ya pudimos etiquetar bien todas las palabras diseñando un programa inicialmente en el Jupyter. A continuación, investigué que tecnologías utilizar para la realización del prototipo tecnológico, encontramos como entorno de desarrollo Pycharm y como framework Django. Una vez seleccionadas las tecnologías, investigamos como se utilizaban y nos pusimos a trabajar en el prototipo tecnológico.

Yo me encargué de conectar las vistas html con la lógica en Python, a continuación vimos como se implementaba un formulario y como se hacia una redirección a vista. Cuando supimos como se hacia todo esto, integramos el código desarrollado en Jupyter en nuestro servicio web finalizando el prototipo tecnológico. \newline

En cuanto a la memoria me la dividí a partes iguales con Pablo, intentando que los dos toquemos todo, por lo que ambos redactamos tanto una parte de la introducción (en la que redacté la motivación) como el estado de la cuestión (yo hice el apartado de lectura fácil y Procesamiento del Lenguaje Natural). \newline

La investigación de como funcionaba Conceptnet y su API la hicimos conjuntamente.


\section{Trabajo realizado por Pablo}
\label{cap:sec:trabajo_Pablo}

Al igual que mi compañera, lo primero que hicimos fue investigar como podíamos etiquetar las palabras, encontramos la librería ntlk de Python para hacerlo, pero tras un primer intento nos dimos cuenta de que muchas palabras no estaban etiquetadas como deberían por lo que decidimos buscar alternativas, indagando un poco encontramos Spacy, la probamos y obtuvimos unos resultados mucho mejores que con ntlk por lo que decidimos utilizar esta última (todo esto lo hicimos desde el Jupyter). 

Cuando terminamos de etiquetar las palabras nos pusimos a investigar herramientas para el desarrollo del prototipo tecnológico y nos decantamos por utilizar Django como framework integrado en Pycharm, que es el entorno de desarrollo.

A continuación empezamos el desarrollo del prototipo tecnológico primero investigando como se utilizaban estas tecnologías(implementar formularios, hacer las redirecciones a vista...). Para finalizar migramos lo hecho desde Jupyter a nuestro servicio web. \newline

Irene y yo nos dividimos la redacción de la memoria de tal manera que los dos hicimos tanto la parte de la introducción como del estado de la cuestión, de la introducción a mí me tocó la parte de los objetivos y del estado de la cuestión la parte de figuras retóricas y servicios web. \newline

La investigación de como funcionaba Conceptnet y su API la hicimos de manera conjunta.

