\chapter{Trabajo Realizado}
\label{cap:TrabajoRealizado}

\begin{resumen}
En este capítulo vamos a describir que trabajo hemos hecho cada uno
\end{resumen}

\section{Trabajo realizado por Irene}
\label{cap:sec:trabajo_Irene}

Primero investigamos las bibliotecas que utilizaremos para el procesado de las palabras, al principio encontramos una biblioteca para el procesado de texto en Python, que es la ntlk pero vimos que las etiquetas que ponía a las palabras no eran del todo correctas por lo que buscamos otra biblioteca y encontramos Spacy, con esta ya pudimos etiquetar bien todas las palabras diseñando un programa inicialmente en el Jupyter . A continuación, investigamos que tecnologías utilizar para la realización del prototipo(un servicio web) y encontramos como entorno de desarrollo el Pycharm y como framework el Django. Una vez seleccionadas las tecnologías, investigamos como se utilizaban y nos pusimos a trabajar en el prototipo.

Yo me encargué de conectar las vistas html con la lógica en Python, a continuación vimos como se implementaba un formulario y como se hacia una redirección a vista. Cuando supimos como se hacia todo esto, integramos el código desarrollado en Jupyter a nuestro servicio web finalizando el prototipo.



\section{Trabajo realizado por Pablo}
\label{cap:sec:trabajo_Pablo}

Al igual que mi compañera, lo primero que hicimos fue investigar como podíamos etiquetar las palabras, encontramos la librería ntlk de Python para hacerlo, pero tras un primer intento nos dimos cuenta de que muchas palabras no estaban etiquetadas como deberían por lo que decidimos buscar alternativas, indagando un poco encontramos Spacy, la probamos y obtuvimos unos resultados mucho mejores que con ntlk por lo que decidimos utilizar esta última (todo esto lo hicimos desde el Jupyter). 

Cuando terminamos de etiquetar las palabras nos pusimos a investigar herramientas para el desarrollo del servicio web prototipo y nos decantamos por utilizar Django como framework integrado en Pycharm, que es el entorno de desarrollo.

A continuación empezamos el desarrollo del servicio web primero investigando como se utilizaban estas tecnologías(implementar formularios, hacer las redirecciones a vista...). Para finalizar migramos lo hecho desde el Jupyter a nuestro servicio web.