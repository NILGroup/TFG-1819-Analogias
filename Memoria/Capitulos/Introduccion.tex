\chapter{Introducción}
\label{cap:introduccion}

\chapterquote{Frase célebre dicha por alguien inteligente}{Autor}

\begin{resumen}
	En el capítulo 1 vamos a realizar una introducción a nuestro TFG. Dicha introducción la dividiremos en varias secciones. En el apartado 1.1 hablaremos de la razón de ser nuestro trabajo y en el 1.2 de los objetivos que queremos alcanzar con el mismo
	
	
\end{resumen}


\section{Motivación}
\label{cap:sec:motivacion}
En nuestra sociedad actual, existen ciertos colectivos como pueden ser personas con algún tipo de trastorno cognitivo, inmigrantes, ancianos, analfabetos funcionales, niños, etc... que tienen dificultad para aprender conceptos complejos o no tan complejos. Para ellos, esto supone una serie de limitaciones importantes en su vida cotidiana, como en la forma de relacionarse con otros individuos, en su vida profesional e incluso su vida personal. Por ejemplo en el colectivo de los inmigrantes que tienen que realizar gestiones para la estancia en nuestro país, si no tienen una buena comprensión de nuestro lenguaje pueden ser engañados fácilmente y no pueden defender sus derechos de una forma adecuada. Otro ejemplo de la magnitud de las consecuencias de este problema es lo que ocurrió hace un tiempo con las preferentes que muchas entidades financieras engañaron a personas que pertenecen a este tipo de colectivos (sobretodo personas mayores) para que invirtieran en las preferentes cuando realmente era una estafa.  

Para ayudar principalmente a estas personas a que puedan entender el significado de cualquier palabra, y de esta forma superar algunas de sus limitaciones, vamos a desarrollar una aplicación que permita definir palabras complejas mediante comparaciones con otras más fáciles ya conocidas por los usuarios. 

\section{Objetivos}
\label{cap:sec:objetivos}
Nuestro objetivo final es crear una aplicación que dada una palabra compleja, genere símiles, analogías o metáforas que expliquen dicha palabra utilizando conceptos más simples, preferiblemente que estén entre las 1000 palabras más utilizadas de la RAE.
La aplicación se apoyará en varios servicios web donde el usuario podrá buscar cualquier palabra, y obtendrá una descripción lo más sencilla posible. Por ejemplo, si buscamos la palabra piraña, la aplicación devolverá la siguiente descripción:  \textit{Nada como un pez, es pequeño como un ratón y es agresivo como un león}. De esta manera, cualquier persona que no sepa lo que es una piraña, puede hacerse una idea de como es dicho animal y asimilar el nuevo concepto. Además, utilizaremos técnicas centradas en el usuario para diseñar una interfaz lo más usable posible y que así el usuario tenga una experiencia de uso satisfactoria.

Por otra parte, queremos que nuestro producto se encuentre al alcance de todas aquellas personas que lo necesiten, que puedan sentir que es realmente útil porque obtienen resultados satisfactorios. Para que todo ello ocurra, los objetivos principales a alcanzar son:
\begin{itemize}
	\item Desarrollar una aplicación que esté al alcance de todos los usuarios.	
	\item Aplicaciones del proyecto: Estará enfocado en ser utilizado por personas con algún tipo de dificultad de aprendizaje para la mejora de la comprensión de palabras complejas
	\item Desarrollar servicios web que puedan ser utilizados por cualquier programador
	\item Construir la aplicación de manera incremental, añadiéndole valor poco a poco
	\item Construir una aplicación que esté entrada en el usuario, que solucione sus problemas de manera usable
	\item Realizar un producto que sea eficiente
	\item Que la aplicación cumpla con unos estándares de calidad
	\item Un diseño de Interfaz que cumpla con las Ocho Reglas de Oro del diseño de interfaces:
	\begin{itemize} 
		\item Consistencia: La funcionalidad de la interfaz sea similar a otras aplicaciones que el usuario utiliza normalmente.
		\item Usabilidad Universal: Cualquier usuario sea capaz de utilizar nuestra aplicación.
		\item Retroalimentación activa: Informar al usuario de cada acción que realiza.
		\item Diálogos para conducir la finalización: El usuario sepa en que etapa está cuando quiere realizar una acción que requiera varios pasos.
		\item Prevención de errores: Diseñar la interfaz para que el usuario cometa el mínimos de errores posibles.
		\item Deshacer acciones fácilmente: Cualquier acción pueda ser deshecha
		\item Sensación de control: Dar al usuario la sensación de que tiene el control en todo momento de la aplicación.
		\item Reducir la carga de memoria a corto plazo: Minimizar la cantidad de elementos que tiene que memorizar el usuario.
	\end{itemize}
\end{itemize}
Y no nos tenemos que olvidar de los objetivos académicos. Poner en práctica los conocimientos adquiridos durante el Grado, y ampliar nuestros conocimientos gracias a la utilización de herramientas, lenguajes y metodologías nuevas.