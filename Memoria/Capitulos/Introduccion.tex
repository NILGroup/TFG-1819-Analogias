\chapter{Introducción}
\label{cap:introduccion}

Actualmente, existen diversos colectivos con un entendimiento limitado del castellano, como por ejemplo las personas inmigrantes, y en especial está limitación se ve aumentada en aquellos en los que en su país de origen hablan una lengua totalmente distinta al castellano. Es por ello, que una vez aquí se tienen que enfrentar a situaciones en las cuáles son más vulnerables a sufrir un engaño y/o estafa. Esto ocurre tanto en situaciones cotidianas como comprar cualquier producto, o poder comunicarse con otra persona así como en situaciones más específicas de trámites de papeles oficiales, firma de contratos, etc... Y aunque existen multitud de herramientas online para facilitar la comprensión de textos en distintos idiomas, como pueden ser los servicios de traducción instantánea, en los que al escribir un texto se traduce automáticamente a cualquier lengua que el usuario desee, no existen herramientas que permitan definir palabras difíciles utilizando comparaciones o términos más sencillos para facilitar su comprensión. 
Nuestro objetivo es ofrecer un servicio accesible que defina palabras complejas de una manera clara y sencilla, empleando para ello palabras más sencillas.
 

	


%-------------------------------------------------------------------
\section{Motivación}
%-------------------------------------------------------------------
\label{cap:sec:motivacion}

En nuestra sociedad, existen ciertos colectivos como pueden ser las personas con algún tipo de trastorno cognitivo, inmigrantes, ancianos, analfabetos funcionales, niños, etc... que tienen dificultad para aprender conceptos complejos o no tan complejos. 
Existen multitud de palabras cuyo significado es bastante complicado de explicar de una manera sencilla, por lo que una solución para que cualquier persona los pueda comprender es hacer uso de metáforas o analogías en las que intervengan palabras conocidas para los usuarios. Por ejemplo, para explicar que es un pony, se puede decir que un ``pony es como un caballito". De esta forma, se puede asimilar el concepto de una manera más rápida y sencilla.

La dificultad para entender conceptos complejos supone una serie de limitaciones en la vida cotidiana, en la forma de relacionarse con otros individuos, en la vida profesional e incluso la vida personal. Por ejemplo, si un inmigrante necesita realizar gestiones para legalizar su estancia o firmar un contrato, se enfrenta a ciertas posibles estafas y/o engaños ya que posiblemente no entienda las claúsulas y no sabe a ciencia cierta lo que está firmando lo que conlleva a la firma de claúsulas abusivas, incluso ilegales. Otro ejemplo sería una persona que sufre de dislexia, para la cual algo tan cotidiano como poder leer el periódico, les supone un gran esfuerzo.


Para ayudar principalmente a estas personas a que puedan entender el significado de cualquier palabra, y de esta forma superar algunas de sus limitaciones, se va a desarrollar una aplicación que permita definir palabras complejas mediante comparaciones con otras más fáciles ya conocidas por ellos. Por ejemplo, si se quiere explicar una palabra compleja como puede ser piraña, se puede describir utilizando conceptos más simples de la siguiente manera:\textit{ ``Una piraña nada como un pez y es agresiva como un león''.} Mediante esta comparación, alguien que desconozca completamente el significado de \textit{piraña}, puede hacerse una idea muy aproximada de lo que es.


%-------------------------------------------------------------------
\section{Objetivos}
%-------------------------------------------------------------------
\label{cap:sec:objetivos}

El objetivo principal de este Trabajo de Fin de Grado es crear una aplicación que dada una palabra compleja para el usuario obtenga una definición clara y sencilla mediante símiles, analogías o metáforas que empleen palabras más sencillas y conocidas. 
Se utilizarán técnicas centradas en el usuario para diseñar la interfaz y así conseguir una aplicación usable y que se adapte a las necesidades y limitaciones de los potenciales usuarios finales.
Los objetivos tecnológicos a alcanzar en este Trabajo de Fin de Grado son:
\begin{itemize}
	\item La aplicación estará construida con servicios web que la doten de funcionalidad a la aplicación para que sean reutilizables en otras aplicaciones y se puedan adaptar a las distintas necesidades de los usuarios finales.
	\item Los servicios web desarrollados estarán disponibles en una API pública para que todo el mundo pueda utilizarlos.	
	\item La aplicación se construirá de manera incremental, añadiéndole valor al producto poco a poco.	
	\item Aplicar patrones de diseño.
\end{itemize}

Por último, no se deben de olvidar los objetivos académicos de este trabajo:
\begin{itemize}
	\item Poner en práctica los conocimientos adquiridos durante el Grado y ampliar nuestros conocimientos en distintas áreas.
	\item Usar correctamente Django y su modelo de uso MVT (\textit{model view template}) que es algo nuevo para los integrantes de este Trabajo de Fin de Grado.
	\item Utilizar Python, así como intentar seguir correctamente sus buenas prácticas.
	
\end{itemize}

Alcanzando los objetivos anteriormente descritos, se conseguirá obtener un producto de calidad, con una gran utilidad tanto social como académica, que puede ayudar a mucha gente a aprender ciertos conceptos de nuestro idioma de una manera más sencilla.
	
	
%-------------------------------------------------------------------
\section{Estructura de la memoria}
%-------------------------------------------------------------------
\label{cap:sec:estructuramemoria}


En el \textbf{capítulo dos} se presenta el Estado de la Cuestión, en el que se explicará que es la Lectura Fácil y como se aplica y se introducirán los conceptos de Procesamiento del Lenguaje Natural (PLN) y algunas herramientas que sirven para PLN, además se hablará de figuras retóricas y servicios web, en especial qué son, su arquitectura y las ventajas y desventajas de su uso.\\


En el \textbf{capítulo tres} se describe el trabajo realizado por cada uno de los autores.
