\chapter{Introducción}
\label{cap:introduccion}


El español, hoy en día es la segunda lengua más hablada del mundo y actualmente más de 90000 palabras forman el castellano. 
Se trata de una lengua con multitud de palabras, y que dependiendo del contexto en el que se encuentre, puede tener múltiples significados.
Si esto puede suponer una complicación, por ejemplo para una persona que no tiene ningún trastorno cognitivo, para ciertos colectivos de la sociedad, lo es aún mucho más afectándoles en su vida cotidiana, profesional o personal. 

A esto se suma, que nuestra lengua española, con motivo de los nuevos avances en tecnología y nuevos hábitos que van surgiendo, han hecho que el lenguaje haya tenido que evolucionar adecuándose a la misma evolución de la sociedad.
En la sección 1.1 se explicará con más detalle este problema que afecta a una gran parte de la sociedad, y que no disponen de ninguna herramienta para entender ciertos conceptos complejos. 
Nuestro objetivo es ofrecer un servicio accesible que defina palabras complejas de una manera clara, empleando para ello palabras más sencillas y que en la sección 1.2 se explicarán todos los objetivos tanto tecnológicos como académicos que los integrantes que desarrollan dicho trabajo se han propuesto.
Por útlimo, en la sección 1.3 vendrá explicada la estructura de dicho documento.



	


%-------------------------------------------------------------------
\section{Motivación}
%-------------------------------------------------------------------
\label{cap:sec:motivacion}

En nuestra sociedad, existen ciertos colectivos como pueden ser inmigrantes, personas con algún tipo de trastorno cognitivo, ancianos, analfabetos funcionales, niños, etc... que tienen dificultad para aprender conceptos complejos. 
Existen multitud de palabras cuyo significado es bastante complicado de explicar de una manera sencilla, por lo que una solución para que cualquier persona los pueda comprender es hacer uso de metáforas o analogías en las que intervengan palabras conocidas para los usuarios. Por ejemplo, para explicar que es un selfi, se puede decir que un ``selfi es como una fotografía". De esta forma, se puede asimilar el concepto de una manera más rápida y sencilla haciendo así que la dificultad para entender conceptos complejos no suponga una limitacion en la vida cotidiana, en la forma de relacionarse con otros individuos, en la vida profesional e incluso la vida personal. 
Por ejemplo, una persona que sea analfabeta emocional puede tener limitaciones al ver un programa de televisión, leer un manual técnico para realizar su trabajo, utilizar el teléfono móvil, etc...

Para ayudar principalmente a estas personas a que puedan entender el significado de cualquier palabra, y de esta forma superar algunas de sus limitaciones, se va a desarrollar una aplicación que permita definir palabras complejas mediante comparaciones con otras más fáciles ya conocidas por ellos. Por ejemplo, si se quiere explicar una palabra compleja como puede ser piraña, se puede describir utilizando conceptos más simples de la siguiente manera:\textit{ ``Una piraña nada como un pez y es agresiva como un león''.} Mediante esta comparación, alguien que desconozca completamente el significado de \textit{piraña}, puede hacerse una idea muy aproximada de lo que es.


%-------------------------------------------------------------------
\section{Objetivos}
%-------------------------------------------------------------------
\label{cap:sec:objetivos}

El objetivo principal de este Trabajo de Fin de Grado es crear una aplicación web basada en servicios que dada una palabra compleja para el usuario devuelva una definición  de dicha palabra mediante símiles, analogías o metáforas que empleen palabras más sencillas y conocidas para el usuario. 
Se utilizarán técnicas centradas en el usuario para diseñar la interfaz y así conseguir una aplicación usable y que se adapte a las necesidades y limitaciones de los potenciales usuarios finales.
Los objetivos tecnológicos a alcanzar en este Trabajo de Fin de Grado son:
\begin{itemize}
	\item La aplicación estará construida con servicios web que la doten de funcionalidad a la aplicación para que sean reutilizables en otras aplicaciones y se puedan adaptar a las distintas necesidades de los usuarios finales.
	\item Los servicios web desarrollados estarán disponibles en una API pública para que todo el mundo pueda utilizarlos.	
	\item La aplicación se construirá de manera incremental, añadiéndole valor al producto poco a poco.	
	
\end{itemize}

Por último, no se deben de olvidar los objetivos académicos de este trabajo:
\begin{itemize}
	\item Poner en práctica los conocimientos adquiridos durante el Grado y ampliar nuestros conocimientos en distintas áreas.

	
\end{itemize}

Alcanzando los objetivos anteriormente descritos, se conseguirá obtener un producto de calidad, con una gran utilidad tanto social como académica, que puede ayudar a mucha gente a aprender ciertos conceptos de nuestro idioma de una manera más sencilla.
	
	
%-------------------------------------------------------------------
\section{Estructura de la memoria}
%-------------------------------------------------------------------
\label{cap:sec:estructuramemoria}


En el \textbf{capítulo dos} se presenta el Estado de la Cuestión, en el que se explicará que es la Lectura Fácil y como se aplica y se introducirán los conceptos de Procesamiento del Lenguaje Natural (PLN) y algunas herramientas que sirven para PLN, además se hablará de figuras retóricas y servicios web, en especial qué son, su arquitectura y las ventajas y desventajas de su uso.\\


En el \textbf{capítulo tres} se describe el trabajo realizado por cada uno de los autores.
