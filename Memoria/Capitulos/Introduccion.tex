\chapter{Introducción}
\label{cap:introduccion}

\chapterquote{Frase célebre dicha por alguien inteligente}{Autor}

Existen ciertos colectivos como pueden ser personas con algún tipo de trastorno cognitivo o inmigrantes que tienen dificultad para aprender conceptos complejos, para ayudar a estas personas con su aprendizaje, vamos a desarrollar un servicio web que mediante analogías con conceptos más sencillos describa las características de estos conceptos complejos, para facilitar su entendimiento.
\section{Motivación}
Existen ciertos colectivos como pueden ser personas con algún tipo de trastorno cognitivo o inmigrantes que tienen dificultad para aprender conceptos complejos, para ayudar a estas personas con su aprendizaje, vamos a desarrollar un servicio web que mediante analogías con conceptos más sencillos describa las características de estos conceptos complejos, para facilitar su entendimiento.

\subsection{Explicaciones adicionales}
Si quieres cambiar el \textbf{estilo del título} de los capítulos, abre el fichero \verb|TeXiS\TeXiS_pream.tex| y comenta la línea \verb|\usepackage[Lenny]{fncychap}| para dejar el estilo básico de \LaTeX.

Si no te gusta que no haya \textbf{espacios entre párrafos} y quieres dejar un pequeño espacio en blanco, no metas saltos de línea (\verb|\\|) al final de los párrafos. En su lugar, busca el comando  \verb|\setlength{\parskip}{0.2ex}| en \verb|TeXiS\TeXiS_pream.tex| y aumenta el valor de $0.2ex$ a, por ejemplo, $1ex$.

El siguiente texto se genera con el comando \verb|\lipsum[2-20]| que viene a continuación en el fichero .tex. El único propósito es mostrar el aspecto de las páginas usando esta plantilla. Quita este comando y, si quieres, comenta o elimina el paquete \textit{lipsum} al final de \verb|TeXiS\TeXiS_pream.tex|

\subsubsection{Texto de prueba}


\lipsum[2-20]