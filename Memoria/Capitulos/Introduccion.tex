\chapter{Introducción}
\label{cap:introduccion}

Actualmente, si pensamos en el día a día de cualquier persona en nuestra sociedad, creemos que no tiene ningún incoveniente para realizar cualquier acción. Es más, podemos pensar que cada día existen grandes avances que hacen que la vida resulte más fácil.
Pero esto no es del todo cierto, ya que existen ciertos colectivos como pueden ser inmigrantes, personas con alguna discapacidad cognitiva, ancianos, analfabetos funcionales, etc... que para realizar cualquier acción cotidiana les supone un gran esfuerzo o incluso no pueden realizarla y estos avances no están pensados para ellos.

En especial, este trabajo se va a centrar en personas con discapacidad cognitiva y en la sección 1.1 se explicará con mayor detalle los problemas a los que se tienen que enfrentar, que solución queremos aportar y en que ayudaría dicha solución.
Para poder ofrecer la mejor solución posible necesitamos desgranar el problema e ir formando dicha solución en función de sus necesidades y limitaciones cumpliendo pequeños objetivos como puede ser la distinción entre palabras fáciles y palabras difíciles, o como obtener los términos relacionados de un concepto específico. En la sección 1.2 se explicarán cuales han sido estos objetivos así como la forma en la que se cumplirán.

Por último, en la sección 1.3 vendrá explicada la estructura de dicho documento explicando en cada capítulo lo que se contará de una manera breve.


%-------------------------------------------------------------------
\section{Motivación}
%-------------------------------------------------------------------
\label{cap:sec:motivacion}

El español, hoy en día es la segunda lengua más hablada del mundo y actualmente más de 90000 palabras forman el castellano. 
Se trata de una lengua con multitud de términos, y que dependiendo del contexto en el que se encuentren, pueden tener múltiples significados. Por ejemplo, la palabra gato puede hacer referencia a un gato de animal o un gato como herramienta para elevar un coche.
Si esto puede suponer una complicación para cualquier persona, para ciertos colectivos de la sociedad afectados por algún trastorno cognitivo, lo es aún mucho más afectándoles en su vida cotidiana, profesional o personal. Por ejemplo, poder leer un periódico es difícil para ellos ya que muchas palabras no saben lo que significa. Otro ejemplo podría ser leer un manual de instrucciones de una función más técnica, donde se encuentran en la misma situación de no poder entender las palabras.
Una de las soluciones que se podrían pensar en un primer momento, es buscar su significado en un diccionario. Pero esto no les sirve, puesto que las definiciones que aparecen en muchos casos no utilizan términos o frases sencillas. Por ejemplo



En nuestra sociedad, existen ciertos colectivos como pueden ser inmigrantes, personas con algún tipo de trastorno cognitivo, ancianos, analfabetos funcionales, niños, etc... que tienen dificultad para aprender conceptos complejos. 
Existen multitud de palabras cuyo significado es bastante complicado de explicar de una manera sencilla, por lo que una solución para que cualquier persona los pueda comprender es hacer uso de metáforas o analogías en las que intervengan palabras conocidas para los usuarios. Por ejemplo, para explicar que es un selfi, se puede decir que un ``selfi es como una fotografía''. De esta forma, se puede asimilar el concepto de una manera más rápida y sencilla haciendo así que la dificultad para entender conceptos complejos no suponga una limitación en la vida cotidiana, en la forma de relacionarse con otros individuos, en la vida profesional e incluso la vida personal. 
Por ejemplo, una persona que sea analfabeta funcional puede tener limitaciones al ver un programa de televisión, leer un manual técnico para realizar su trabajo, utilizar el teléfono móvil, etc...

Para ayudar principalmente a estas personas a que puedan entender el significado de cualquier palabra, y de esta forma superar algunas de sus limitaciones, se va a desarrollar una aplicación que permita definir palabras complejas mediante comparaciones con otras más fáciles ya conocidas por ellos. Por ejemplo, si se quiere explicar una palabra compleja como puede ser piraña, se puede describir utilizando conceptos más simples de la siguiente manera:\textit{ ``Una piraña nada como un pez y es agresiva como un león''.} Mediante esta comparación, alguien que desconozca completamente el significado de \textit{piraña}, puede hacerse una idea muy aproximada de lo que es.


%-------------------------------------------------------------------
\section{Objetivos}
%-------------------------------------------------------------------
\label{cap:sec:objetivos}

El objetivo principal de este Trabajo de Fin de Grado es crear una aplicación web basada en servicios que dada una palabra compleja para el usuario devuelva una definición de dicha palabra mediante símiles, analogías o metáforas que empleen palabras más sencillas y conocidas para el usuario. 
Para poder obtener esta definición, habrá que estudiar como obtener los términos relacionados de un concepto, así como distinguir de estos resultados cuales son palabras fáciles y dificiles. 
También habrá que saber que tipos de figuras retóricas existen y cuál utilizar según la relación entre el concepto inicial y los conceptos fáciles, de esta forma se podrá facilitar al usuario final un resultado claro y correcto.

La aplicación estará construida con servicios web que doten de funcionalidad a la aplicación y que sean reutilizables en otras aplicaciones, haciendo así que se puedan adaptar a las distintas necesidades de los usuarios finales.
Los servicios web desarrollados estarán disponibles en una API pública para que todo el mundo pueda utilizarlos y puedan servir para que otros desarrolladores integren nuestros servicios en sus aplicaciones.

La aplicación se construirá de manera incremental, añadiéndo valor al producto poco a poco. De este modo se podrá testear las distintas hipótesis de trabajo poco a poco y realizar modificaciones de una manera simple para así conseguir una aplicación que se adecúe a las necesidades de los usuarios.

Una vez que se disponga de lo comentado anteriormente, se deberá crear una interfaz. Siempre es necesario que el diseño de la interfaz esté centrada en el usuario, y para ello se debe obtener la mayor información posible sobre el usuario final. De esta forma se realiza un diseño basado en sus necesidades y se obtiene una mayor satisfacción de este al utilizar la aplicación, reduce el tiempo de desarrollo, etc...
Como dicho trabajo está enfocado para personas con discapacidad cognitiva, se debe realizar un diseño que se adapte aún más a sus necesidades y limitaciones.

Por último, no se deben olvidar los objetivos académicos de este trabajo, como puede ser poner en práctica los conocimientos adquiridos durante el Grado y ampliar nuestros conocimientos en distintas áreas. 

Alcanzando los objetivos anteriormente descritos, se conseguirá obtener un producto de calidad, con una gran utilidad tanto social como académica, que pueda ayudar a muchos usuarios a aprender ciertos conceptos de nuestro idioma de una manera más sencilla.
	
	
%-------------------------------------------------------------------
\section{Estructura de la memoria}
%-------------------------------------------------------------------
\label{cap:sec:estructuramemoria}


En el \textbf{capítulo dos} se presenta el Estado de la Cuestión, en el que se explicará que es la Lectura Fácil y como se aplica y se introducirán los conceptos de Procesamiento del Lenguaje Natural (PLN) y algunas herramientas que sirven para PLN, además se hablará de figuras retóricas y servicios web, en especial qué son, su arquitectura y las ventajas y desventajas de su uso.


En el \textbf{capítulo tres} se explicarán las herramientas utilizadas para la creación de este trabajo, como pueden ser Django para el desarrollo de la aplicación y SpaCy para el etiquetado de palabras, donde se explicará que son ambas herramientas, para que se utilizan y sus características principales.

En el \textbf{capitulo cuatro} queda detallado como ha sido la gestión del proyecto, que herramientas se han utilizado para la asignación de tareas, como se ha realizado la distinción de si una tarea hace referencia al código o a la memoria y el tratamiento de cada una de ellas, las reuniones establecidas con los directores así como la utilización de un repositorio para el control de versiones de dicho trabajo.

En el \textbf{capitulo cinco} se explicarán los Servicios Web implementados por los integrantes del trabajo para dotar de funcionalidad a la aplicación.

El \textbf{capítulo seis} está enfocado en el diseño de la aplicación, se explicará el proceso de diseño de la interfaz de la aplicación, desde la creación de los primeros bocetos hasta la implementación del diseño final pasando por la evaluación de los prototipos por parte de los expertos.
 
En el \textbf{capítulo siete} se describe el trabajo realizado por cada uno de los autores de dicho trabajo.
