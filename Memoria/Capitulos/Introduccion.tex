\chapter{Introducción}
\label{cap:introduccion}

\chapterquote{Frase célebre dicha por alguien inteligente}{Autor}

En nuestra sociedad actual, existen ciertos colectivos como pueden ser personas con algún tipo de trastorno cognitivo, inmigrantes, ancianos,  analfabetos funcionales, niños, etc... que tienen dificultad para aprender conceptos complejos o no tan complejos. Para ellos, esto supone una serie de limitaciones en su vida cotidiana, en la forma de relacionarse con otros individuos, su vida profesional e incluso su vida personal, que puede afectarles de una manera considerada.
Para ayudar a estas personas a que puedan entender el significado de cualquier palabra, y de esta forma sentirse mejor consigo mismo, vamos a desarrollar una aplicación que mediante analogías se pueda definir de la manera más detallada dicha palabra/concepto.

\section{Motivación}
Realizaremos un Servicio Web en el que utilizaremos principalmente \textbf{analogías, símiles y metáforas}, para, como se ha comentado en el apartado anterior, describir un concepto más o menos complejo. 
El usuario podrá buscar cualquier palabra, teniendo a su alcance una descripción lo más sencilla posible, por ejemplo, si buscamos la palabra piraña, la aplicación devolverá una descripción parecida a:
\textit{Nada como un pez, es pequeño como un ratón y es agresivo como un león}
De esta manera, cualquier persona que no sepa lo que es una piraña, puede  hacerse una idea de como es dicho animal y facilitar la asimilación del concepto.
Para ello, las palabras que utilizaremos principalmente serán adverbios, adjetivos, verbos y nombres.
Además, utilizaremos técnicas para mejorar la experiencia del usuario como pueden ser realizar un diseño plano o respetar la ley de proximidad para que la interfaz sea lo más usable posible y el usuario tenga una experiencia satisfactoria utilizando nuestra aplicación.
Actualmente hay servicios web parecidos a lo que queremos hacer nosotros como por ejemplo conceptnet que es una red semántica disponible en varios idiomas, entre ellos el castellano, que fue lanzado en 1999 por el MIT y está diseñado para ayudar a las máquinas a entender las palabras que las personas usamos habitualmente.



\subsection{Explicaciones adicionales}
Si quieres cambiar el \textbf{estilo del título} de los capítulos, abre el fichero \verb|TeXiS\TeXiS_pream.tex| y comenta la línea \verb|\usepackage[Lenny]{fncychap}| para dejar el estilo básico de \LaTeX.

Si no te gusta que no haya \textbf{espacios entre párrafos} y quieres dejar un pequeño espacio en blanco, no metas saltos de línea (\verb|\\|) al final de los párrafos. En su lugar, busca el comando  \verb|\setlength{\parskip}{0.2ex}| en \verb|TeXiS\TeXiS_pream.tex| y aumenta el valor de $0.2ex$ a, por ejemplo, $1ex$.

El siguiente texto se genera con el comando \verb|\lipsum[2-20]| que viene a continuación en el fichero .tex. El único propósito es mostrar el aspecto de las páginas usando esta plantilla. Quita este comando y, si quieres, comenta o elimina el paquete \textit{lipsum} al final de \verb|TeXiS\TeXiS_pream.tex|

\subsubsection{Texto de prueba}


\lipsum[2-20]