\chapter{Introducción}
\label{cap:introduccion}

\chapterquote{Frase célebre dicha por alguien inteligente}{Autor}

En nuestra sociedad actual, existen ciertos colectivos como pueden ser personas con algún tipo de trastorno cognitivo, inmigrantes, ancianos,  analfabetos funcionales, niños, etc... que tienen dificultad para aprender conceptos complejos o no tan complejos. Para ellos, esto supone una serie de limitaciones en su vida cotidiana, en la forma de relacionarse con otros individuos, en su vida profesional e incluso su vida personal.

Para ayudar a estas personas a que puedan entender el significado de cualquier palabra, y de esta forma superar algunas de sus limitaciones, vamos a desarrollar una aplicación que permita definir palabras complejas mediante comparaciones con otras palabras más fáciles ya conocidas por los usuarios. 

\section{Motivación}
Crearemos una serie de servicios Web para describir un concepto más o menos complejo, a partir de comparaciones con otros conceptos más sencillos. Crearemos una aplicación que se apoye en dichos Servicios Web donde el usuario podrá buscar cualquier palabra, y obtendrá una descripción lo más sencilla posible. Por ejemplo, si buscamos la palabra piraña, la aplicación devolverá la siguiente descripción:  \textit{Nada como un pez, es pequeño como un ratón y es agresivo como un león}. De esta manera, cualquier persona que no sepa lo que es una piraña, puede hacerse una idea de como es dicho animal y asimilar el nuevo concepto. Además, utilizaremos técnicas centradas en el usuario para diseñar una interfaz lo más usable posible y que así el usuario tenga una experiencia satisfactoria utilizando nuestra aplicación.

\section{Objetivos}
Ambos autores de dicho documento nos fijamos una serie de objetivos que esperemos queden reflejados con total claridad tanto en la aplicación como en la documentación aquí presente. Por un lado, nuestros objetivos respecto al trabajo es implementar un servicio web  que se encuentre al alcance de todas aquellas personas que lo necesiten, que puedan sentir que es realmente útil por que obtienen resultados satisfactorios. Para que todo ello ocurra, los objetivos principales a alcanzar son:
\begin{itemize}
	\item Desarrollar un servicio web que este al alcance de todos los usuarios.	
	\item Aplicaciones: Nuestro proyecto puede ser utilizad
	\item Un diseño de Interfaz que cumpla con las Ocho Reglas de Oro, es decir:
	\begin{itemize} 
		\item Consistencia
		\item Usabilidad Universal
		\item Retroalimentación activa
		\item Diálogos para conducir la finalización
		\item Prevención de errores
		\item Deshacer acciones fácilmente
		\item Sensación de control
		\item Reducir la carga de memoria a corto plazo
	\end{itemize}
\end{itemize}
Y no nos tenemos que olvidar de los objetivos académicos. Pondremos en práctica los conocimientos adquiridos durante los años de duración del Grado como el uso del lenguaje Python o el maquetado web mediante html y css así como el uso de Bootstrap, además de ampliar nuestros conocimientos en base a herramientas nuevas utilizadas, como el Pycharm, que es el entorno de desarrollo que utilizaremos para la codificación del proyecto.